% intro.tex

\begin{frame}[ctb!]
  \frametitle{The Nuclear Fuel Cycle}
  \begin{figure}
    \includegraphics<1>[height=5cm]{./images/fc.eps}
    \includegraphics<2>[height=5cm]{./images/fc-front.eps}
    \includegraphics<3>[height=5cm]{./images/fc-back.eps}
    \caption{An example Advanced Breeder Reator (ABR) fuel cycle. \cite{lisowski_global_2007}.}
    \label{fig:fc}  
  \end{figure}
\end{frame}

\begin{frame}[ctb!]
  \frametitle{Recycling Fuel}
  Used fuel exiting light water reactors has the following average elemental
  breakdown:

  \begin{table} [h]
    \centering
    \begin{tabular} {|c|c|} 
      \hline
      Element Group & wt \% \\
      \hline
      Uranium           & $\sim$95  \\
      Plutonium         & $\sim$1   \\
      Mixed Actinides   & $\sim$0.1 \\
      Fission Products  & $\sim$4   \\
      \hline
    \end{tabular}
    \caption{Elemental Breakdown of Spent Fuel Exiting a Typical LWR}
    \label{tab:lwr_fuel}
  \end{table}

  Recycled fuel can be used in both thermal and fast reactors.

  Fast reactors are defined by transuranic conversion ratios.
\end{frame}

\begin{frame}[ctb!]
  \frametitle{Fuel Cycle Modeling Complexity}
  Modeling the nuclear fuel cycle is complex for a variety of reasons:

  \begin{enumerate}
    \item recycling
    \item material quantity and \textit{quality}
    \item \textit{fungibility} from both a supplier and consumer perspective
    \item the Department of Energy has identified ``an endless'' number of
      possible fuel cycles, but identifies 40 ``evaluation groups'' of cycles
      \cite{wigeland_evaluation_2013}
  \end{enumerate}
\end{frame}
