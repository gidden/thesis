% simulation.tex

\begin{frame}[ctb!]
  \frametitle{Simulating the Nuclear Fuel Cycle} 

  Fuel cycle simulators are designed to answer policy-related questions
  regarding transitions from one equilibrium state to another.

  \vspace{0.2cm}

  \pause
  A simulator answers the following questions as a function of its 
  parameter space:
  \begin{itemize}
    \item how much material exists
    \item where does that material reside
    \item from/to where and when is material transported
    \item what kinds of facilities are needed
    \item when is each type of facility needed
  \end{itemize}
\end{frame}

\begin{frame}[ctb!]
  \frametitle{Simulating the Nuclear Fuel Cycle}
  The nuclear fuel cycle is simulated via \textit{scenarios}.\vspace{0.2cm}

  A scenario generally defines a demand for nuclear power and the types of
  reactors that can respond to that demand. Supporting facilities can be
  explicitly or implicitly modeled to provide fuel for the reactors, and such
  behavior depends on the model used by the simulator.\vspace{0.2cm}

  Some nuclear fuel cycle simulators (FCS) currently in use today:
  \begin{itemize}
    \item CAFCA (MIT) \cite{busquim_e_silva_system_2008}
    \item COSI (CEA) \cite{boucher_cosi:_2006}
    \item DANESS (ANL) \cite{durpel_daness_2003}
    \item VISION (INL) \cite{yacout_vision_2006}
  \end{itemize}
\end{frame}

\begin{frame}[ctb!]
  \frametitle{FCS Metrics}

  FCSs generally report back metrics about the fuel cycle in question. Many of
  these metrics are functions of facility deployment, material flow, and
  material residency, including:

  \begin{itemize}
    \item economics
    \item waste management
    \item sustainability
    \item nonproliferation
  \end{itemize}
\end{frame}

\begin{frame}[ctb!]
  \frametitle{FCS Design Choices}
  Simulation designers are faced with a number of decisions, including:
  \begin{itemize}
    \item facility deployment
    \item fidelity of physical, chemical, and industrial processes
    \item material transaction modeling
  \end{itemize}
\end{frame}

\begin{frame}[ctb!]
  \frametitle{FCS Design Choices: Deployment}

  Reactor deployment choices in almost all cases are chosen by the user.\vspace{0.2cm}

  Supporting facility deployment can be chosen by the user, but in most present
  cases, they are deployed when needed by a look-ahead function.\vspace{0.2cm}

  In many present cases, deployment, although user defined, is restricted by
  future available material. For example, if an advanced reactor requires
  transuranic (TRU) fuel, and a simulation knows that TRU will not be available,
  simulators instead deploy a non-constrained reactor.\vspace{0.2cm}

  In such cases, a class of reactors, e.g. LWRs, is considered never to be
  fuel constrained.
\end{frame}

\begin{frame}[ctb!]
  \frametitle{FCS Design Choices: Fidelity}

  Physical, chemical, and process fidelity are all design choices for simulators.\vspace{0.2cm}

  Physical fidelity includes the choice to model isotopic decay and whether or
  not to model in-core reactor physics.\vspace{0.2cm}

  Chemical and process fidelity include the modeling of the chemical separations
  and advanced fuel fabrication processes. The separations-fuel fabrication
  interface is generally not treated by the current simulator cadre.
\end{frame}

\begin{frame}[ctb!]
  \frametitle{FCS Design Choices: Material Transactions}

  Determining how material flows, or is transacted, is a critical design
  decision, because of the effect it has on output metrics.\vspace{0.2cm}
  
  Most simulators in use to date use a systems dynamics approach, which models
  aggregate material flows between fleets of reactors.\vspace{0.2cm}

  These flows are \textit{static}, i.e., connections between fleets of
  facilities are known \textit{apriori}. In fact, these connections govern the
  equations used in the systems dynamics modeling architecture. This makes
  extensibility of the codes difficult for any new type of fuel cycle.
\end{frame}

\begin{frame}[ctb!]
  \frametitle{FCS Design Choices: Summary}
  
    \begin{table} [h!]
      \small
      \centering
      \begin{tabular} {|c|c|c|c|c|}
        \hline
        Decision                     & Category & CAFCA & COSI & VISION \\ 
        \hline
        \multirow{2}{*}{Deployment}  & Determined by 
        & user & user & user \\ \cline{2-5}
        & Constrained by\footnote{Generally a look-ahead function determines the associated metric.}   
        & available material & not treated by literature & available material \\ \hline
        \multirow{3}{*}{Fidelity}    & Decay 
        & yes\footnote{Capability is available, but it is not used.} & yes & yes \\ \cline{2-5}
        & Reactor Physics 
        & no & yes & no \\ \cline{2-5}
        & Fuel Matching\footnote{How recycled fuel orders are ``matched'' with available isotopics.} 
        & aggregate mass flows & equivalence method & aggregate mass flows \\ \hline
        \multirow{2}{*}{Connections} & Static/Dynamic 
        & static & static & static \\ \cline{2-5}
        & Fleet/Individual 
        & fleet & fleet & fleet \\
        \hline
      \end{tabular}
      \caption{Simulation-Level Design Decisions as Taken by Each Simulator}
      \label{tab:sim-summary}
    \end{table}

\end{frame}
