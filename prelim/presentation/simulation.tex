% simulation.tex

\begin{frame}[ctb!]
  \frametitle{Simulating the Nuclear Fuel Cycle} 

  Fuel cycle simulators are designed to answer policy-related questions
  regarding transitions from one equilibrium state to another.

  \vspace{0.2cm}

  \pause
  A simulator answers the following questions as a function of its 
  parameter space:
  \begin{itemize}
    \item how much material exists
    \item where does that material reside
    \item from/to where and when is material transported
    \item what kinds of facilities are needed
    \item when is each type of facility needed
  \end{itemize}
\end{frame}

\begin{frame}[ctb!]
  \frametitle{Simulating the Nuclear Fuel Cycle}
  The nuclear fuel cycle is simulated via \textit{scenarios}.

  Some codes currently in use today:
  \begin{itemize}
    \item CAFCA (MIT) \cite{busquim_e_silva_system_2008}
    \item COSI (CEA) \cite{boucher_cosi:_2006}
    \item DANESS (ANL) \cite{durpel_daness_2003}
    \item VISION (INL) \cite{yacout_vision_2006}
  \end{itemize}
\end{frame}

\begin{frame}[ctb!]
  \frametitle{Nuclear Fuel Cycle Simulation Metrics}

  Fuel Cycle Simulators generally report back metrics about the fuel cycle in
  question, including:

  \begin{itemize}
    \item economics
    \item waste management
    \item sustainability
    \item nonproliferation
  \end{itemize}
\end{frame}

\begin{frame}[ctb!]
  \frametitle{Nuclear Fuel Cycle Simulation Design Choices}
  Simulation designers are faced with a number of decisions, including:
  \begin{itemize}
    \item facility deployment
    \item fidelity of physical, chemical, and industrial processes
    \item material transaction modeling
  \end{itemize}
\end{frame}
