\section{\Cyclus To Date}\label{sec:cyclus}

Significant, strategic improvements have been added to the \Cyclus code base.
Interactions with dynamically-loadable libraries and reading XML-input files
have been encapsulated in classes and enhanced. 

\subsection{Dynamic Loading}\label{sec:prev-dynamic}

Software libraries can be accessed in a static manner, i.e., they are connected
to an application during the linking phase of building a program, or can be
accessed in a dynamic manner, i.e., they are connected at run time. Dynamic
linkage, or dynamic loading, is a well-known technique for to support
connectable, modular components, or plug-ins. \Cyclus utilizes a plug-in
approach for its facility, institution, and region agent libraries. This design
decision furthers the \Cyclus development goal of providing an agnostic
framework into which sophisticated users can develop different models of agents
to investigate specific simulation-modeling change.

Work was performed in order to use class-based representation of dynamic
libraries, allowing for easy opening, closing, and access thereof. Each dynamic
library represents an agent type in \Cyclus, providing a constructor and
destructor. The DynamicModule class then provides the \Cyclus core access to
these constructors and destructors to perform the appropriate operations at run
time. Because dynamic loading is treated differently on POSIX-based systems than
it is on Windows-based systems, specialty functions for library access were
provided for each system. The correct header file (UnixHelperFunctions.h or
WindowsHelperFunctions.h) is chosen during compilation. 

These changes simplified the client code that utilizes library access. The
\Cyclus core application can now call appropriate library-related functions in
an agnostic manner. Specific, well-defined time points of module loading
(dynamic library opening) and unloading (dynamic library closing) were defined
in the \Cyclus application. These operations, called by the application,
currently reside in the Model class and could likely be refactored into a
specific handler class designed for this purpose.

\subsection{Input Reading}

A large overhaul to the input-reading code base was performed. \Cyclus currently
only supports XML input files that adhere to the RNG schema defined in the
\Cyclus RNG file. However, it is a well-known best practice to provide an
agnostic application programming interface (API) that can be configured with
specific class instances given some user-defined input. Accordingly, such an API
was constructed which currently supports XML input but can also support future
input types that are tree-based (e.g., JSON, CSV, etc.).

The top-level abstraction is encased in a QueryEngine class. Basic operations
are provided assuming a tree-based input formation, including querying the
number of child elements at the current reading level, the name of each element,
and access to each element. The application code is responsible for configuring
the appropriate input parser and populating an instance of a QueryEngine at its
root. The client code then populates input parameters through the QueryEngine
interface rather than an input-format specific interface.

Support for XML-file reading has been enhanced by separating various concerns
into appropriate classes. Four classes have been constructed with specific
purposes regarding input file reading: file loading (XMLFileLoader), file
validation (RelaxNGValidator), file parsing (XMLParser), and querying
(XMLQueryEngine). The application interfaces with the file loader, initializing
it with a given input file path and then invoking the loading of various
\Cyclus-specific parameters (e.g., simulation control parameters, material
recipes, agent modules, etc.). The loader is responsible for managing its parser
and providing client code with correctly-configured instances of
XMLQueryEngines. The parser is responsible for providing an interface to the
underlying C++ XML parsing library (i.e., libxml++) and invoking the appropriate
validation routines on the parsed file. The validator is responsible for
providing access to the RNG-validation operations through libxml and
libxml++. Finally, the XML-specific derived QueryEngine class is responsible for
implementing XML querying using the generic QueryEngine interface.

Other input file formats can be supported by providing the appropriate
format-specific derived QueryEngine class and adjusting the application code as
necessary. A developer could then choose how to implement the loading and
validation, if any, of the input file. The structure described above is just one
of many ways to achieve such a goal.
