Fuel cycle simulators have a number of possible output metrics of interest. One
of the original reasons that fuel cycle simulators were developed was to
determine the relative cost of different fuel cycles. This is also a key example
of differences between simulators. The notion of fuel cycle cost can either be
calculated in a post-processing step, or it can be calculated on-the-fly during
a simulation. The benefit of maintaining cost information during a simulation is
that the simulation can react to these costs. A prime example of such behavior
is in the DANESS simulation code, described in \S\ref{sec:other-sims}. 

A large number of metrics related to fuel cycle simulation come directly from
the material mass flows. For instance, proliferation based metrics are generally
determined by the amount of separated transuranic material at any point in the
fuel cycle. Metrics that lead to less proliferation resistance include the
amount of signifigant quantities of material, defined by the IAEA to be the
quantity of any given material (e.g. TRU or plutonium) sufficient to make a
nuclear weapon. These are generally defined on the elemental level, which
ignores realities based on isotopic dependencies (e.g. \nucl{240}{Pu} is a
relatively large source of decay heat and is thus reduces the aggregate
plutonium's effectiveness for use as a weapon). Metrics that lead to more
resistance of proliferation activities include the unshielded dose rate of a
given material, which makes handling (and thus diverting) the material more
difficult \cite{yacout_vision_2006}. Another key metric that is important to
many in the fuel cycle simulation community is uranium utilization. At present,
the United States political stance is to continue the once-through fuel cycle
until it is economically viable to begin reprocessing spent fuel for recycle in
reactors \cite{hamilton_blue_2012}. Such a situation can only arise if the
supply of uranium dwindles to a low-enough level which will raise the price of
uranium sufficiently enough to make reprocessing a viable alternative. It is
additionally interesting to note the approximate time at which uranium capture
from seawater, which is in developmental stages, will become economically
viable. These situations all depend on the amount of uranium used in a given
simulation, which is, of course, related to fuel mass flows.

One of the key outputs of simulators is related to repository capacities. The
capacity a repository to store spent fuel could be reached in a number of
ways. Spent fuel isotopics, especially fission products, produce a large amount
of decay heat which can reach the thermal limits of their containers over short
periods of (geologic) time. There are also radiotoxity limits for repositories
in order to limit the risk of radiological releases. Finally, there is a mass
and volume limit on repositories, representing a space storage capacity. It is
not clear at the current time which of these limiting capacities will
ultimiately be the governing factor for repositories. Furthermore these
capacities change based on repository layout and geology. A more in-depth study
is presented by Katy Huff in her thesis work
\cite{huff_integrated_2013}. Furthermore, this parameter was key in Scopatz's
work with static fuel cycle simulation \cite{scopatz_essential_2011}. In all
cases, these capacities can be calculated in terms of isotopic quantities of
waste products. It is possible to calculate the capacity parameters \textit{in
  situ} during the simulation and dynamically alter the simulation based on the
results rather than simply performing a post-processing operation to report the
required number of repositories for the simulation. 
