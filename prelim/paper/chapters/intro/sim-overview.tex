Fuel cycle simulation is a field with a variety of actors, including
governments, universities, and international governance
organizations. Accordingly, a variety of modeling strategies have been applied
to the nuclear fuel cycle. Such strategies span a wide range of fidelity, both
at the facility level and the material level. For instance, some simulators
describe reactors by fleet (or types) and solve material balances for the entire
fleet in aggregate while others instantiate individual (or discrete)
facilities. Similarly, some simulators make detailed calculations of fuel
depletion due to reactor fluence whereas others simply use pre-tabulated values
that depend (generally) on burnup values for thermal reactors and conversion
ratios for fast reactors.

There are, broadly, three decision categories that are of concern to fuel cycle
simulation. The first is facility deployment, specifically, how and why certain
facilities are deployed. There is general consensus regarding reactor deployment
in the community: a user defines an energy growth curve and, for each type of
reactor in the simulation, a percentage of that total energy demand to be met by
the reactor type. However, the nuclear fuel cycle is a special case of
supply-demand modeling where certain facilities (e.g., fast reactors) require
fuel that has been processed by other facilities (e.g., thermal
reactors). Accordingly, simulation developers must make a choice: should one
allow a facility to be built if it may not be able to be fueled? Certain
simulators explicitly disallow this behavior by determining reactor build
decisions based on lookahead algorithms (e.g., CAFCA, VISION), others explicitly
allow it (e.g., COSI), while still others offer a hybrid approach that allow a
lookahead function based on a certain amount of fuel that will eventually be
needed over a reactors lifetime (e.g., DANESS). The eventual choice of this
decision making process greatly affects simulation outcomes in any scenario in
which a lack of fuel exists. Because these simulation tools are built to analyze
the dynamic symbiotic relationship between different reactors in a cyclical
process (e.g., thermal and fast reactors), among other scenarios, this
simulation development decision is arguably very important to simulation
outcomes.

The second major simulation development decision is determining fuel
isotopics. A number of complications are encompassed in this decision. As an
example, consider MOX fuel for thermal reactors. In general, MOX fuel is
composed of oxide forms of plutonium (and some minor actinides, such as
americium) from spent thermal fuel as well as uranium (the source of which can
be depleted enrichment tails, depleted recycled uranium or natural uranium). The
fact that depleted uranium, rather than enriched uranium, can be used stems from
the fact that the separated plutonium is largely comprised of fissile isotopes
(\nucl{239}{Pu} and \nucl{241}{Pu}). The source of uranium already introduces an
isotopic dependency of importance: the \nucl{235}{U} enrichment of the fill
uranium should affect either the quantity of plutonium used, the isotopics of
plutonium used (with higher \nucl{235}{U} enrichment implying a lower
concentration of fissile plutonium isotopes), or both. Further complicating the
issue is that plutonium isotopes are radioactive and decay on (relatively) short
time scales. For instance, the half life of \nucl{241}{Pu} is $\sim$14
years. Accordingly, the quality, or isotopic content, of the separated plutonium
changes on a time scale on the order of the simulation due to decay. There is a
similar issue with other transuranic radioactive isotopes of interest to nuclear
fuel cycles.

Accordingly, simulation developers have two general choices with respect to
input fuel isotopics (and isotopic-level modeling in general). The first is
whether or not to include isotope decay. As might be expected, the simulators
fall into two camps, those that include decay (e.g., VISION and COSI) and those
that do not (e.g., DANESS and CAFCA). Interestingly, the MIT development team
claims that the lack of modeling decay does not affect the simulation as long as
all transuranic isotopes are lumped together \cite{guerin_impact_2009}. Other
codes appear to include isotopic decay in order to inform output metrics such as
repository heat capacity. The second choice involves matching input isotopics
with available separated isotopes. This is an interesting problem because
separations technology work on a elemental scale, whereas input fuel recipes are
defined on an isotopic scale because neutronics properties are functions of
individual isotopes rather than elemental aggregates. In other words, you can't
change the separated plutonium isotopic vector to match a recipe, as one does
with uranium enrichment (which is an isotopic-scale process). A full treatment
of the problem is relatively complicated and requires mixing separated plutonium
vectors to find a ``best match''; this problem has been termed the Winery
Problem or the Recipe Approximation Problem \cite{oliver_geniusv2:_2009}. The
current generation of fuel cycle simulators generally punt on this issue. A
common strategy is to declare a target subset of isotopics, normally a specific
plutonium isotope or the aggregate plutonium isotopes, and match quantities of
that set. For example, if the set is of single cardinality (e.g.,
\nucl{239}{Pu}), then the amount of that isotope is guaranteed to be correct,
but accompanying isotopics are not. On the other hand, if the set is a group of
isotopes (e.g., all plutonium isotopes), that group's quantity is guaranteed to
be correct, but the specific isotopics are not. One can conclude from this
discussion that the level of isotopic detail modeling in a simulation could
greatly affect the outcome of the simulation, especially if decisions are made
mid-simulation regarding the isotopes in question (e.g., whether or not to build
a fast reactor given the available amount of separated isotopes). A full
treatment of how the current generation of simulators tackle this issue is
described in \S\ref{sec:simulators}. An overview of the proposed strategy to
take in the \Cyclus simulation environment is described in \S\ref{sec:rap}.

The third major development decision is how to determine connections between
facilities. At issue here is how servicing facilities (e.g., fuel fabrication
facilities) are connected to serviced facilities (e.g., reactors). For those
simulators that do not model discrete facilities (e.g., CAFCA), the modeling
technique is relatively trivial: a fleet of servicing facilities are directly
connected to their serviced facilities. If more than one type of facility is
being serviced (e.g., TRU-based fuels going to thermal and fast reactors), then
a user must define the percentage of capacity going towards each type of
serviced facility \cite{busquim_e_silva_system_2008}. A similar situation arises
in other systems-dynamics based simulators, because mass flow balance equations
govern the inner workings of the simulations. This situation becomes even more
complicated if regional scenarios are to be modeled. In addition to determining
which serviced facilities will be connected to servicing facilities, one must
also incorporate a notion of the region of serviced and servicing
facilities. DESAE includes a rather simplistic model of this relational nature
that predetermines the yearly intra-regional trading \cite{iaea_nuclear_2010}. A
full treatment of this class of developmental decision making with respect to
the current set of fuel cycle simulators is provided in
\S\ref{sec:simulators}. An overview of the proposed strategy to take in the
\Cyclus simulation environment is described in \S\ref{sec:rap}.
