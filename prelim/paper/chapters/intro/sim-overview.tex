Fuel cycle simulation is a field with a variety of actors, including
governments, universities, and international governance
organizations. Accordingly, a variety of modeling strategies have been applied
to the nuclear fuel cycle. Such strategies span a wide range of fidelity, both
at the facility level and the material level. For instance, some simulators
describe reactors by fleet (or types) and solve material balances for the entire
fleet in aggregate while others instantiate individual (or discrete)
facilities. Similarly, some simulators make detailed calculations of fuel
depletion due to reactor fluence whereas others simply use pre-tabulated values
that depend (generally) on burnup values for thermal reactors and conversion
ratios for fast reactors.

There are, broadly, three decision categories that are of concern to fuel cycle
simulation. The first is facility deployment, specifically, how, why, and when
certain facilities are deployed. In the current simulation development
environment, the most common reactor deployment mechanism is allowing a user to
define an energy growth curve and, for each type of reactor in the simulation, a
percentage of that total energy demand to be met by the reactor type. However,
the nuclear fuel cycle is a special case of supply-demand modeling where certain
facilities (e.g., fast reactors) require fuel that has been processed by other
facilities (e.g., thermal reactors). Accordingly, simulation developers must
make a choice regarding the ability for facilities to be built if fuel may not
be available for their use. Certain simulators explicitly disallow this behavior
by determining reactor build decisions based on lookahead algorithms, others
explicitly allow it, while still others offer a hybrid approach that allow a
lookahead function based on a certain amount of fuel that will eventually be
needed over a reactors lifetime. The eventual choice of this decision making
process greatly affects simulation outcomes in any scenario in which a lack of
fuel exists. Because these simulation tools are built to analyze the dynamic
symbiotic relationship between different reactors in a cyclical process (e.g.,
thermal and fast reactors), among other scenarios, this simulation development
decision is arguably very important to simulation outcomes.

The second simulation design decision category is the level of fidelity with
which to model the physical and chemical processes involved in the nuclear fuel
cycle. Broadly, physical fidelity includes two processes, isotopic decay and
isotopic transmutation due to residency in a reactor. Physical fidelity is an
important concern because fuel cycle simulation measures individual isotopic
masses at each point in the fuel cycle, and the isotopic profiles of those mass
streams change due to physical processes. 

Isotopic decay is important to consider because some isotopes decay on time
scales on the order of or smaller than the simulation time. \nucl{241}{Pu}, for
instance, has a half life of $\sim$14 years. As might be expected, the
simulators fall into two camps, those that include decay and those that do
not. Interestingly, the MIT development team claims that the lack of modeling
decay does not affect the simulation as long as all transuranic isotopes are
lumped together \cite{guerin_impact_2009}. Other codes appear to include
isotopic decay in order to inform output metrics such as repository heat
capacity. 

Reactor physics, i.e., the process by which the transmuted isotopic profile of
fuel due to reactor residency is determined, is also an important physical
consideration. The rigorous solution of reactor physics equations is an entire
field in nuclear science unto itself, and is thus not normally treated by fuel
cycle simulators. In most cases for the current suite of simulators, some amount
of calculation is performed before a simulation is run, and isotopic profiles
are determined via lookup tables. Some simulators, however, choose to perform
transmutation calculations \textit{in situ}, during the simulation.

The chemical process fidelity design decision occurs at the interface between
the elemental separations process in fuel recycling and the usage of recycled
fuel by reactors. This is an interesting problem because separations technology
work on a elemental scale, whereas input fuel recipes are defined on an isotopic
scale because neutronics properties are functions of individual isotopes rather
than elemental aggregates. In other words, you can't change the separated
plutonium isotopic vector to match a recipe, as one does with uranium enrichment
(which is an isotopic-scale process). A full treatment of the problem is
relatively complicated and requires mixing separated plutonium vectors to find a
``best match''; this problem has been termed the Winery Problem or the Recipe
Approximation Problem \cite{oliver_geniusv2:_2009}. The current generation of
fuel cycle simulators generally do not address this issue. A common strategy is
to declare a target subset of isotopics, normally a specific plutonium isotope
or the aggregate plutonium isotopes, and match quantities of that set. For
example, if the set is of single cardinality (e.g.,
\nucl{239}{Pu}), then the amount of that isotope is guaranteed to be correct,
but accompanying isotopics are not. On the other hand, if the set is a group of
isotopes (e.g., all plutonium isotopes), that group's quantity is guaranteed to
be correct, but the specific isotopics are not. One can conclude from this
discussion that the level of isotopic detail modeling in a simulation could
greatly affect the outcome of the simulation, especially if decisions are made
mid-simulation regarding the isotopes in question (e.g., whether or not to build
a fast reactor given the available amount of separated isotopes). A full
treatment of how the current generation of simulators tackle this issue is
described in \S\ref{sec:simulators}. An overview of the proposed strategy to
take in the \Cyclus simulation environment is described in \S\ref{sec:rap}.

The third simulation-level design decision concerns the connections between
facilities and the type of material that flows along those connections. In
general, connections between facilities can either be static or dynamic, and can
either be fleet-based or facility-based. A static connection implies that
material will always flow between two facilities, whereas a dynamic connection
implies that a facility's input or output connection may change.  For those
simulators that model fleets of facilities with static connections, the modeling
technique is relatively trivial: a fleet of servicing facilities are directly
connected to their serviced facilities. If more than one type of facility is
being serviced (e.g., TRU-based fuels going to thermal and fast reactors), then
either a user or the simulation engine must define the percentage of capacity
going towards each type of serviced facility \cite{busquim_e_silva_system_2008}.

Most simulators to date have taken the fleet-based, static connection approach
to modeling fuel cycles which lacks the ability to be easily extended and
improved upon, a key feature of a research code. The \Cyclus fuel cycle
simulator has taken the dynamic connection approach. Dynamic connections between
facilities allow for more complicated scenarios, e.g., scenarios with regional
influences, to be modeled. The dynamic exchange of resources, however,
introduces two complications. The first is that a given need, e.g., for fuel,
can be met by multiple commodities. As an example, consider fuel for thermal
reactors. Such fuel can be either uranium oxide (UOX) or mixed uranium-plutonium
oxide (MOX). Furthermore, MOX fuel is composed of plutonium (and some minor
actinides, such as americium) from spent thermal fuel as well as uranium (the
source of which can be depleted enrichment tails, depleted recycled uranium or
natural uranium). The second is that the isotopics comprising fuel orders
are \textit{fungible}, i.e., the source of isotopes are generally exchangeable. 

The core concept that connects both the design decisions regarding fidelity and
facility connections is the notion of material \textit{quality}. Because of the
nature of nuclear reactors, the simulation of their fuel usage must consider the
isotopic profile of material being produced and consumed. This additional
concern greatly complicates the modeling of fuel cycles and must be taken into
account. Each of the issues addressed by the fidelity decision, decay,
transmutation, and fuel fabrication, are all operations on the isotopic profile
of material. The mechanism by which dynamic connections between facilities are
determined must also take into account the isotopic profiles of the commodities
produced and consumed by those facilities. This work attempts to satisfy the
material quality concerns in fuel cycle simulation.
