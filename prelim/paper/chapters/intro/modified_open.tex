The modified open fuel cycle is effectively a hybrid of the open and closed fuel
cycles. The Blue Ribbon Commission's Reactor and Fuel Cycle
Technology Subcommittee tackled a definition as follows:

\begin{quotation}
We have defined this category to encompass a very wide range of possible fuel
cycles with multiple possible combinations of different reactor, separations,
and fuel fabrication technologies. Our definition includes any fuel cycle in
which some of the spent fuel is processed rather than being directly disposed of
after a single pass through a reactor.~\cite{brc_reactor_2012}
\end{quotation}

They mention, however, that there is no industry-wide agreed-upon
definition. For the purposes of this work I will use the committee's
definition. It should be noted that by the committee's definition, the French
nuclear power program is technically a modified-open cycle because MOX fuel
reprocessing has been demonstrated, but is not used at an industrial scale.
