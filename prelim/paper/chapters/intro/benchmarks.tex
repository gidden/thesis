The effort to benchmark fuel cycle simulation codes has occured (relatively)
recently, and there have been a number of different benchmarking exercises
attempted by governmental and international agencies to date. The most notable
of these are the International Atomic Energy Agency (IAEA) International Project
on Nuclear Reactors and Fuel Cycle's (INPRO) benchmark
exercise \cite{_international_2009}, the Massachusettes Institute of
-Technology's benchmark exercise \cite{guerin_benchmark_2009}, the Nuclear Waste
Technology Review Board's (NWTRB) benchmark
exercise \cite{abkowitz_workshop_2011,_nuclear_2011} and the Organization for
Economic Cooperation and Development's (OECD) benchmark
exercise \cite{boucher_benchmark_2012}.

Each of the benchmarking efforts used a different set of codes by which to
compare agaisnt one another. The codes used in each is described in
Table \ref{tab:benchmark-codes}.

\begin{table} [h!]
\centering
\begin{tabular} {|c|c|} 
\hline
Benchmark & Codes Used \\
\hline
IAEA - INPRO & COSI, DESAE, FAMILY, TAPS, VISTA, VISION \\
MIT          & CAFCA, COSI, DANESS, VISION \\
NWTRB        & AREVA, CAFCA, NUWASTE, VISION \\
OECD         & COSI, EVOLCODE, DESAE, FAMILY, VISION \\
\hline
\end{tabular}
\caption{The Set of Codes Compared in Each Benchmarking Effort}
\label{tab:benchmark-codes}
\end{table}

Each benchmarking effort investigated a different set of scenarios. The actual
scenarios investigate depends heavily on the interest of the organizing
institution. For example, the NWTRB is a congressionally mandated review board,
thus their scenario suite focused on options for the U.S. fleet of nuclear
reactors. The INPRO benchmarks called for scenarios with low, moderate, and high
nuclear power capacity growth. For each growth scenario, three different
deployment scenarios were modeled: PWRs and HWRs; PWRs, HWRs, and advanced PWRs;
and PWRs, HWRs, advanced PWRs, and fast reactors. The actual specifications for
reactor power level, fuel isotopic composition, and any other simulation
parameters (outside of the power level and types of reactors to be used) were
provided in a reference database \cite{_international_2009}. The MIT benchmark
incorporated the growth of nuclear power demand in three categories: 0 \% per
year, 1.5 \% per year, and 3 \% per year, all of which are rates of exponential
demand growth. The deployment schemes include: LWRs fueled with UOX
transitioning to fast reactors with conversion rates of 0.5, LWRs with UOX and
(eventually) MOX transitioning to fast reactors with conversion rates of 0.5,
and LWRs fueled with UOX transitioning to fast reactors with conversion rates of
1.0. The transition period began at year 2040 in each scenario. This set of
specifications is tailored to be easily met by the host institution's simulation
code, CAFCA (see \cite{guerin_benchmark_2009} and \S\ref{sec:cafca}). The NWTRB
benchmark breaks scenarios down into
``phases'' \cite{abkowitz_scenario_2011}. The first phase is a starting point,
representing spent fuel currently in the system. The second phase models a
constant power level held at the 2009 level (100.3 GWe), with new plants coming
online to keep that power level constant when older reactors retire. The third
phase models a scenario in which a repository opens in year 2040 and includes a
constraint on the age of fuel that can enter the repository. The fourth phase
models a suite of scenarios in which a reprocessing facility enters operation at
year 2040. The family of scenarios varies the separations capacity from 1500 to
3000 t/yr and varies the allowable fuel age from 5 to 50 years. The fifth phase
models scenarios that incorporate both reprocessing and a repository. The OECD
benchmark assumes a constant installed power for each scenario with reactor
deployments varying across scenarios. The first scenario assumes an open cycle,
i.e., power is only met by PWRs using UOX fuel. The second scenario assumes that
some percentage of the PWRs can use MOX fuel. The third scenario assumes that
for a finite period of time, MOX is used by some LWRs, and after 80 years, all
retired LWRs are replaced by fast reactors.


discuss how the scenarios were described, citing boucher as the best

It must be noted that the fuel cycle simulation community does not benefit from
benchmarking exercises in the same way that the nuclear physics community
benefits from them. For example, the Monte Carlo N-Particle (MCNP) code
community has a number of criticality safety benchmarks \cite{wagner_mcnp:_1992}
which have been verified by years of experiment and experience. Put another way,
for certain MCNP problems, there is a \textit{verifiably correct answer}. The
community can then build up from those basic principles to understand how
``believable'' their solutions are for larger, more complex problems. This is,
in fact, the reason why benchmarks exist. Fuel cycle simulation does not
necessarily have these characteristics. For instance, there is a fundamental
difference between simulations that model continuous material flow versus those
that model distcrete material transfers. Accordingly there is no ``right
answer'' for even most simple simulations. Instead, there must be a community
consensus, which leads to benchmarking exercises that incorporate some subset of
the simulators in the community. In general, these exercises compare large,
aggregate metrics in order to limit differences caused by fundamental
differences in modeling approaches. For example, benchmarks may compare the
total aggregate flow of used fuel coming out of reactors. However, one can't
really come to a consensus answer regarding the isotopics of this spent fuel due
to the wide variety of simulator treatment of depletion and decay (ranging from
no use of such methods to full integration of depletion codes).

In his conclusions of the MIT benchmarking exercise, Guerin states that
``operation of a fuel cycle model is as much art as science''
\cite{guerin_benchmark_2009}. Understanding just how much of FCS is art and how
much is science is critical to providing reliable results. 
