\subsection{Using Agent-Based Models}

Stepping back a moment from the discussion of nuclear-specific fuel cycles, the
interactions that we wish to model are those of supply and demand. Certain
facilities need some sort or raw or processed material that other facilities
produce. The fuel cycle is more complicated than this simple statement, though,
because facilities are connected through recycled fuel. Even with such a
complication, the notion of a generic fuel cycle, i.e., from the perspective of
facilities that supply and demand material, quickly begins to look like a supply
chain model. There is a growing literature of agent-based supply chain modeling
\cite{swaminathan_modeling_1998,julka_agent-based_2002,van_der_zee_modeling_2005,chatfield_multi-formalism_2007,holmgren_agent_2007}.
The general premise of these types of models is that individual facilities have
a notion of their needs (i.e., their demands) and can express to the system
these needs at the required time. There is heavy use of inventory policy to
determine the correct amount of material inventory that is needed and the
correct time to request a resupply. In general, a facility may be comprised of
many agents, e.g., an ordering agent, a stocking agent, a forecasting agent,
etc. Such an approach has not heretofore been attempted for the nuclear fuel
cycle and opens up a variety of doors. For example, reactor facilities could be
allowed to be fueled by multiple fuel types (e.g., UOX or MOX), and decide which
type to choose based on the simulation environment. However, due to the
fungibility of material in the fuel cycle and its multicommodity nature, many
questions remain about how to describe the materials to be traded and the
markets on which they will be traded. A more fully formed proposal for such a
treatment is presented in \S\ref{sec:agent-interaction} and \S\ref{sec:gfctp}.

\subsection{Input Fuel Isotopic Matching}

One of the major issues with current fuel cycle simulation is the disconnect
between fuel recipes being requested by reactors that need recycled fuel, e.g.,
MOX fuel and the facility that produces MOX. The isotopics that comprise these
materials, especially the transuranic isotopes, are radioactive, and thus the
actual isotopic composition of material change with time due to radioisotope
decay. Accordingly, if a simulator wishes to try to match the requested
isotopics, it must keep track of the decay of isotopes.  Further complicating
the problem is that, for a given fuel fabrication facility, stocks of available
materials may not exactly match the materials being requested, , i.e., there is
a mismatch between output isotopics and input isotopics even without the
presence of isotopic decay.  The majority of fuel cycle simulators to date
ignore this issue and instead choose to match recipes ``exactly'' based on a
subset of isotopes. Further details of how other simulators tackle this issue
are reviewed in \S\ref{sec:simulators}. A full treatment of the recipe-matching
problem would require an approach similar to the pooling-blending problem
\cite{tawarmalani_convexification_2002}. There is a rich literature base for
pooling-blending problems
\cite{glen_mixed_1988,rigby_evolution_1995,mendez_simultaneous_2006,misener_advances_2009}
in the industrial processes literature. However, the majority of the subject
matter concentrates on refinery operations such as mixing various crude oils to
achieve a certain octane level. These are (relatively) simple problems because
the properties that are trying to be matched are extrinsic and linear function
of the mixing variables. The nuclear arena is much more complicated because of
criticality limitations and the want to match aggregate nuclear properties of
materials. Accordingly, a hybrid approach was proposed that involves a notion of
blending via an approximation linear program. The approach is outlined in
\S\ref{sec:rap}.

\subsection{Modeling Global Regions}

The notion of regional modeling in fuel cycle simulation has to date been a
secondary concern. Certain simulators attempt to model it explicitly, e.g.,
DESAE \cite{iaea_nuclear_2010}. Other simulators have attempted to add the
capability at some point after their simulator had already been developed, e.g.,
VISION and COSI. A discussion of these capabilities follows in
\S\ref{sec:simulators}. Some semblance of this capability is needed if one is to
incorporate outside effects on a domestic fuel cycle. Furthermore, a robust
capability is required if one wishes to actually investigate dynamic
interactions amongst regional entities. Again, a full treatment of this sort of
regional interaction would require international relations models, most of which
can be found in the cross-cutting disciplines of economics, political science,
and game theory. The primary solution technique is called Nash Equilibrium, and
effectively describes an optimal solution as follows: given a set of players,
states, preferences, and actions, all players choose an action such that any
single player's deviation from that actions results in a state of lower
preference for that player (thus no player has an incentive to deviate)
\cite{mccarty_political_2007}. There also exists a body of literature that
examine Nash Equilibria in the context of optimal flow models
\cite{mazumdar_fairness_1991,nagurney_supply_2002,song_nash_2002}. However, the
complexity of such models quickly brings them out of the scope of our needs,
i.e., dynamic modeling of multi-lateral scenarios ranging 100+ years in a short
computation time ($\sim 10^0 - 10^1$ minutes). Accordingly, a market resolution mechanism
has been proposed that allows for interaction amongst the various facilities and
managing entities (e.g., their regions). In order to inform a cardinal
preference \cite{strotz_cardinal_1953} relation for a facility over its possible
supply materials. The full approach is outlined in sections
\S\ref{sec:agent-interaction} and \S\ref{sec:gfctp}.
