\chapter{Introduction}\label{ch:intro}

\section{The Nuclear Fuel Cycle}

The nuclear fuel cycle, in general, can be described as a set of facilities that
interact with one another to either provide or consume fuel services. The
overall goal of the nuclear fuel cycle is to provide fuel to nuclear power
plants which, in turn, generate energy. The overall goal of the system is to
produce power at a competitive price while managing externalities of the
process, the chief of which is spent nuclear fuel. There exist a myriad of
strategies to achieve this aim which generally fall along a spectrum of the
degree to which fuel is recycled. In general, fuel cycles on the end of the
spectrum that does not recycle fuel are concerned most with cost, whereas fuel
cycles that fully recycle fuel are concerned most with issues of sustainability
and intergenerational equity.

\subsection{The Open Fuel Cycle}

The open, or once-through, fuel cycle is relatively simplistic and is in place
in most countries in the world currently utilizing nuclear power. In practice,
the primary fuel element used in this type of cycle is uranium; however, a
combination of thorium and uranium, or some other initial source of neutrons,
could be used in theory. The fuel cycle is considered open because fuel that is
used in a reactor is stored indefinitely once its reactivity has dropped below
useful levels due to the presence of neutron poisons, primarily in the form of
fission products.

Uranium ore is initially extracted from the ground using one of a variety of
techniques including open pit mining, underground mining, and in situ
leaching. The uranium ore is then milled to form yellowcake,
$\mathrm{U_3O_8}$. The tailings, or byproducts, of this process are slightly
radioactive and are therefore considered to be low-level waste (LLW) by the
Nuclear Regulatory Commission (NRC) (see \citet{nrc_10_1985}). 

Certain reactors are designed to use naturally enriched uranium. For these
reactors, yellowcake can be directly reduced with oxygen to form naturally
enrich $\mathrm{UO_2}$. For the majority of power reactors, however, the uranium
must be enriched with higher-than-normal levels of uranium-235. In order to do
so, yellowcake is sent to a conversion facility, which converts it from
$\mathrm{U_3O_8}$ to $\mathrm{UF_6}$. The uranium hexaflouride is then enriched
to the required level in an enrichment facility, of which three classes exist:
gaseous diffusion, the original enrichment technology; centrifugal diffusion,
the current enrichment technology; and Atomic Vapor Laser Isotope Separation
(AVLIS), a newer technology not currently in commerical production. The enriched
uranium hexaflouride is then sent to a fuel fabrication facility where it is
returned to yellowcake form before being reduced to uranium oxide, similarly to
the process for natural uranium fuel. The uranium oxide is then sintered into
pellets and loaded into fuel assemblies to be placed in a reactor. This process
in conjunction with uranium mining is termed the 'front end' of the nuclear fuel
cycle.

Making the open fuel cycle unique, once fuel has been processed in a reactor, it
is cooled off in pools for a number of years, and then stored in dry casks
before eventually being sent to a final repository. The physical location of the
fuel may vary during dry cask storage between the reactor site or some other
interim storage site.

Graphically, the open fuel cycle is shown in Figure \ref{fig:open-cycle}.

\begin{figure}[]
  \begin{center}
    \includegraphics[width=6cm]{./chapters/intro/open_cycle.png}
  \caption{The once-through fuel cycle as shown in \cite{cochran1990nuclear}.}
  \label{fig:open-cycle}
  \end{center}
\end{figure}


\subsection{The Closed Fuel Cycle}
The closed fuel cycle is one that includes the recycling of used, or spent, fuel
to be reused in a reactor. Spent fuel that exists the average Light Water
Reactor (LWR) has an elemental makeup shown in Table \ref{tab:lwr_fuel}. The
economics of recycling spent fuel is complicated and depends on many external
factors, but doing so has two overarching benefits that contribute to lowering
the cost of the fuel cycle: increasing repository capacity and increasing fuel
utilization. 

\begin{table} [h]
\centering
\begin{tabular} {|c|c|} 
\hline
Element Group & wt \% \\
\hline
Uranium           & $\mathrm{\sim}$95  \\
Plutonium         & $\mathrm{\sim}$1   \\
Mixed Actinides   & $\mathrm{\sim}$0.1 \\
Fission Products  & $\mathrm{\sim}$4   \\
\hline
\end{tabular}
\caption{Elemental Breakdown of Spent Fuel Exiting a Typical LWR}
\label{tab:lwr_fuel}
\end{table}

Of the elements that comprise used fuel, uranium, plutonium, and the mixed
actinides (MA) are all capable of producing power through the fission
process. The fission products, however, contain isotopes with high neutron
capture cross sections, which therefore act as poisons to the nuclear chain
reaction.  Achieving 100\% fuel utilization would thus require storing
indefinitely only the fission products and any other byproducts of the fuel
cycle. Furthermore, repository capacity is determined not necessarily by total
mass, but rather by heat load and radiotoxicity, making the concentration of
high-activity isotopes the limiting factor in a repository's capacity. Fission
products are generally short-lived (in comparison to transuranic elements, i.e.,
uranium, plutonium, and the MAs). Accordingly, by minimizing the amount of
transuranics in a repository, its capacity can be extended.

The act of reprocessing spent fuel is a requires relatively few steps. Once fuel
has left the reactor core, it is stored in a spent fuel pool for a some number
of years, typically around five. It can then be directly sent to a reprocessing
facility or be sent after some period of dry-cask storage. Reprocessing fuel is
a chemical extraction process and therefore is limited by chemical extraction
techniques. In general, there are two types of such processes: low-temperature
methods using organic solvents (e.g. PUREX), and high-temperature methods using
molten salts and metals, called pyroprocessing. The extraction techniques
separate the spent fuel into chemically-similar groups which can be different
based on the technique used. However, the separated groups are usually some
combination of those shown in Table \ref{tab:lwr_fuel}. The separated streams
are then sent either to a repository as high-level waste (HLW) or to an
appropriate fuel fabrication facility. Graphically, the closed fuel cycle is
shown in Figure \ref{fig:closed-cycle}.

\begin{figure}[]
  \begin{center}
    \includegraphics[height=7.5cm]{./chapters/intro/closed_cycle.png}
  \caption{The closed fuel cycle as shown in \cite{cochran1990nuclear}.}
  \label{fig:closed-cycle}
  \end{center}
\end{figure}


The elemental groups used in fuel fabrication will depend on the fuel cycle that
is developed. The only large-scale industrial reprocessing plants (La Hague in
France, THORP in the U.K., Mayak in Russia, and Rokkasho in Japan) utilize the
PUREX process to extract uranium and plutonium. The plutonium is then oxidized
and mixed with depleted Uranium from the enrichment process to produce what is
known as mixed-oxide fuel (MOX). Other sources of uranium can be used to fill
MOX fuel, as neutronics-related reactivity and safety constraints allow. Other
fuel cycles utilize the mixed actinides elemental group as well. These generally
include fast reactors that convert their transuranics inventory into either more
TRU (i.e., they have a conversion ratio (CR) of greater than 1), less TRU (CR <
1), or they maintain the amount of TRU entering and exiting their system (CR =
1). Fast reactors with CR > 1 are called breeder reactors.

It should be noted that with any reprocessing capability, nonproliferation
issues arise. Nuclear weapons have historically been produced using either
enriched uranium or plutonium; however it is possible to produce one with any
mix of appropriate materials. Accordingly, any fuel cycle that exposes bare
plutonium streams has an inherently higher nonproliferation risk than one that
does not, and such risks must be weighed accordingly. On a technical note,
though, the relatively low content of Pu-239, especially with respect to the
concentration of heat-producing Pu-240, in spent LWR fuel makes diverting such
fuel for the purposes of nuclear weaponry a route with a near-non-existant
probability of success.


\subsection{The Modified-Open Fuel Cycle}
The modified open fuel cycle is effectively a hybrid of the open and closed fuel
cycles. The Blue Ribbon Commission's Reactor and Fuel Cycle
Technology Subcommittee tackled a definition as follows:

\begin{quotation}
We have defined this category to encompass a very wide range of possible fuel
cycles with multiple possible combinations of different reactor, separations,
and fuel fabrication technologies. Our definition includes any fuel cycle in
which some of the spent fuel is processed rather than being directly disposed of
after a single pass through a reactor.~\cite{brc_reactor_2012}
\end{quotation}

They mention, however, that there is no industry-wide agreed-upon
definition. For the purposes of this work I will use the committee's
definition. It should be noted that by the committee's definition, the French
nuclear power program is technically a modified-open cycle because MOX fuel
reprocessing has been demonstrated, but is not used at an industrial scale.


\section{Fuel Cycle Simulation}

\subsection{Simulator Classification}

\subsection{Metrics}

\subsection{Benchmarks}

In his conclusions of the MIT benchmarking exercise, Guerin states that
``operation of a fuel cycle model is as much art as science''
\cite{guerin_benchmark_2009}. Understanding just how much of FCS is art and how
much is science is critical to providing reliable results. 

\section{Open Questions in Fuel Cycle Simulation}


\subsection{Using Agent-Based Models}

\subsection{Input Fuel Isotopic Matching}

\subsection{Modeling Global Regions}
