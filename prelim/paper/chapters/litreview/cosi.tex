Commelini-Sicart (COSI) is a fuel cycle simulation code developed by the French
Commissariat \`{a} l'\'{e}nergie atomique (CEA), written in the Java programming
language. Accordingly, it can be classified as an object oriented simulator. The
stated goal of COSI is to model the amount and isotopics of material at each
stage in the fuel cycle, explicitly modeling reactor parks and their supporting
facilities (i.e., enrichment, separations, and fuel fabrication) in order to
study transition scenarios at a high level of isotopic and physical
fidelity. Its origins date to 1991, and it has been updated with relative
frequency since 2005 \cite{boucher_cosi_2005,boucher_cosi:_2006,meyer_new_2009,
coquelet-pascal_validation_2011}. A time step is never specified in the COSI
literature, so it is possible that COSI is an event-driven simulation.

The main user input for a COSI simulation is the period to be simulated and the
commissioning and decommissioning date for each type of reactor in the
simulation. It does not deploy reactors based on any sort of algorithmic
model. Additional user input includes the processing order of spent fuel
(e.g., first in/first out, as a function of burnup, as a function of Pu-241
content, etc.), and the equivalent fissile enrichment for each fuel type of a
given reactor, a process explained in detail below.

Reactors are assumed to be fueled with material whose isotopic makeup
corresponds to an equilibrium cycle. For each type of reactor, a single fuel
type is assumed to be used. COSI uses a family of methods for determining input
fuel isotopics, the aggregate name of which is the ``Equivalence Model''. The
term refers to an ``equivalent fraction'' of fissile isotopes in fabricated fuel
comprised of fissile and fertile isotopes. In the simplest case, i.e., for
uranium oxide (UOX) fuel for use in thermal reactors, the model is simply a
predefined enrichment.  Similarly, for mixed plutonium-uranium oxide (MOX) fuel
for use in thermal reactors, the equivalence model is simply a predefined
plutonium content. It is not clear, however, if the plutonium content is in
total mass or mass of a specific plutonium isotope \cite{meyer_new_2009,
  coquelet-pascal_validation_2011}. Further, I assume that a similar equivalence
model is used for any other thermal reactor fuel, e.g., thorium-based fuels.

For each type of reactor, this set of models requires a class fertile of
material (e.g., depleted uranium, recycled uranium, natural uranium) and a class
of fissile material (e.g., plutonium, transuranics), and a predefined ratio of
the two. It is unclear how many degrees of freedom a user has in defining these
parameters for each reactor. Enriched UOX is a special case of the equivalence
model, because it has no restrictions on its fabrication, i.e., COSI does not
allow natural uranium availability to be a constraining parameter in a
simulation. In all other cases, the fissile material must come from recycled
fuel, and is therefore restricted based upon material availability in the
simulation.

Thermal reactors and fast reactors are treated differently in COSI. As discussed
previously, thermal reactors simply define a fissile enrichment and assemble the
material accordingly. In order to determine the input isotopics for fast
reactors, COSI uses an old technique developed by Baker and Ross
\cite{baker_comparison_1963}. In fact, this is where the original notion of an
``equivalence method'' originates. The core premise of this method is that a
fast reactor is ideally loaded with Pu-239 and U-238, and deviations from this
ideal composition must be accounted for. Baker and Ross describe an equilibrium
analysis in which the reactivity is maintained zero, given an system initially
just critical and

\begin{equation}
\sum_{i \in I} \left( \nu_{i} \sigma_{f,i} - \sigma_{a,i} \right) N_i = c,
\end{equation}

where $c$ is a constant, and for a given isotope $i$ in the set of heavy fuel
elemental isotopes, $I$, $N_i$ is its average number density, $\nu_{i}$ is the
average number of neutrons resulting from its fission, $\sigma_{f,i}$ is its
microscopic fission cross section, and $\sigma_{a,i}$ is its microscopic
absorption cross section. For convenience, let us (as Baker and Ross do) define
the collection of intrinsic isotopic parameters as

\begin{equation}
x_i = \nu_{i} \sigma_{f,i} - \sigma_{a,i}.
\end{equation}

The authors then note that one can approximate the critical mass of plutonium
required for a given mixture of plutonium and uranium isotopes using an isotopic
worth, $w_i$, that is a function of its deviation from pure plutonium-239,

\begin{equation}
w_i = \frac{x_i - x_{^{238}U}}
           {x_{^{239}Pu} - x_{^{238}U}}.
\end{equation}

Then, for any combination of isotopes that makes a core critical, 

\begin{equation}
\sum_{i \in I} m_i w_i = c,
\end{equation}

will hold, where $m_i$ is the mass of isotope $i$. Herein lies the notion of
``equivalent Pu-239'', which is defined for an isotope $i$ as $m_i w_i$.

The COSI team adapts this approximation method for its use in order to determine
the fuel composition of fast reactors. The formulation requires an ideal
plutonium-239 enrichment, $E_0$, and a set of fertile and fissile isotopes,
$I_{Fe}$ and $I_{Fi}$, respectively. one can compute an equivalent fissile
isotope enrichment, $E$, using the isotopic neutronic weights, $w_i$, and
available isotopic weight fractions, $\xi_i$.

\begin{equation}
E = \frac{E_0 - \sum_{i \in I_{Fe}} \xi_i w_i}
         {\sum_{i \in I_{Fi}} \xi_i w_i - \sum_{i \in I_{Fe}} \xi_i w_i}
\end{equation}

This perturbed fissile enrichment is then used to determine the amount of
material to take from available fissile stockpiles and the amount of material to
take from available fertile stockpiles, i.e., $1 - E$.

It is important to note that the resulting isotopics for either thermal or fast
reactor fuel is a function of the user-defined reprocessing order. The
equivalence model simply determines the quantity of fertile and fissile material
to extract from the available stockpiles. Granted, the equivalence models
are \textit{informed} by the isotopics of those stockpiles, but they are
considered constants in the calculation. 

It is not clear how COSI models fuel batch processing sizes in the separations
and fabrication process; batches of material are referred to in the COSI
literature, but are not defined.  Separations facilities explicitly model waste
streams, thus efficiency parameters must be defined. The facility throughput
capacities fall into one of three categories: fictitious, limited fictitious,
and real. ``Fictitious'' plants simply reprocess as much fuel as is needed by
the simulation, given that the quantity of requested material exists. ``Limited
fictitious'' plants incorporate a limit on maximum mass flow at any given
time. ``Real'' plants reprocess user-specified quantities of fuel. This
distinction is only mentioned in \cite{guerin_benchmark_2009}, and does not
explain further the difference between real and limited fictitious plants.

The real power of COSI, and also the source of some criticism, lies in its
ability to model fuel residence in reactors. In order to perform the reactor
depletion calculations, COSI utilizes the CESAR code \cite{vidal_cesar:_2006}
which solves the coupled ordinary differential equations governing nuclide
populations due to irradiation and decay over time (i.e., Bateman's Equations),
similar to the ORIGEN library used in the United
States \cite{bell_origen_1973}. CESAR estimates the effect of fluence using
one-group cross sections. Accordingly, libraries of cross section data must be
prepared prior to a given scenario based on the reactor and fuel types used in
the scenario. In order to do so, transport calculations are performed using
APOLLO \cite{santamarina_2009_apollo2} for thermal reactors or ERANOS for fast
reactors \cite{ruggieri_2006_eranos}. Both require input of core geometries and
multi-group cross section data. It is not clear in the COSI literature how fuel
isotopics are chosen for these reference cases for fast reactors, but thermal
reactors use an initial U-235 enrichment or plutonium content (again, the
isotopics of this content are not stated) \cite{guerin_benchmark_2009}.

The output of the two transport codes are cross sections of nuclides and fluence
values as functions of energy and burnup. APOGENE then translates the n-group
energy values (where n is normally 33) into a single-group value. The cross
sections are then smoothed as a function of burnup using Lagrangian polynomial
interpolation. The resulting single-group cross section and fluence libraries as
a function of burnup are then stored in an encrypted database for use in the
COSI simulation \cite{atabekjana_2012_analysis}. These cross sections are then
found via table look-up during a simulation using the core geometry. CESAR is
then run to determine the output isotopics of a given input fuel batch using
these cross sections given the input parameters of the initial fuel isotopic
composition, using the equivalence method described above, the total burnup, and
total irradiation time \cite{coquelet-pascal_validation_2011}.

The COSI team recently tested the validity of their use of CESAR with respect to
ERANOS, the full multi-group transport code, and their use of the equivalence
method \cite{coquelet-pascal_validation_2011}. In general, they found good
agreement for aggregate metrics, such as global plutonium and minor actinide
inventories as a function of time in the simulation, between CESAR and
ERANOS. However, the team found that, as a stable equilibrium plutonium
inventory was approached, the equivalence method values for $E_0$ were highly
sensitive to the minor actinide inventory of available fuel. Using extensive
transport calculations with ERANOS could better inform input values for a given
scenario using an extrapolated value for equivalent Pu-239 as a function of
minor actinide content. With these updated values, macroscopic metrics of the
ERANOS-based calculation were more closely matched, as would be expected.
