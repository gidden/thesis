Commelini-Sicart (COSI) is a fuel cycle simulation code developed by the French
Commissariat \`{a} l'\'{e}nergie atomique (CEA), written in the Java programming
language. Accordingly, it can be classified as an object oriented simulator. The
stated goal of COSI is to model the amount and isotopics of material at each
stage in the fuel cycle, explicitly modeling reactor parks and their supporting
facilities (i.e., enrichment, separations, and fuel fabrication) in order to
study transition scenarios at a high level of isotopic and physical
fidelity. Its origins date to 1991, and it has been updated with relative
frequency since 2005 \cite{boucher_cosi_2005,boucher_cosi:_2006,meyer_new_2009,
coquelet-pascal_validation_2011}.

The main user input for a COSI simulation is period to be simulated and the
commissioning and decommissioning date for each type of reactor in the
simulation. It does not deploy reactors based on any sort of algorithmic
model. Users are also allowed to choose the processing order of spent fuel
(e.g. first in/first out, as a function of burnup, as a function of Pu-241
content, etc.).

In general, reactors are assumed to be fueled with material whose makeup
corresponds to an equilibrium cycle. The way that fuel isotopics are determined
are different for different classes of reactors. In all cases, however, the user
parameters for determining the eventual isotopics are constant for each class of
reactor\cite{coquelet-pascal_validation_2011}. The method by which fuel
isotopics are determined is universally called an ``Equivalence Model'' by the
COSI team. The term refers to an ``equivalent fraction'' of fissile isotopes in
fabricated fuel comprised of fissile and fertile isotopes given a target fertile
fraction and available isotopics of separated fuel. 

For LWR reactors, this model is simply a predetermined U-235 enrichment for UOX
fuel or a fixed plutonium content for MOX
fuel \cite{coquelet-pascal_validation_2011}. It is not clear what part of the
plutonium content is fixed (either the total quantity or the quantity of a
specific isotope in the vector).



The actual isotopic makeup of fuel will vary based on the available
separated uranium and transuranics.

