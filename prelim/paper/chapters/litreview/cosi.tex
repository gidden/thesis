Commelini-Sicart (COSI) is a fuel cycle simulation code developed by the French
Commissariat \`{a} l'\'{e}nergie atomique (CEA), written in the Java programming
language. Accordingly, it can be classified as an object oriented simulator. The
stated goal of COSI is to model the amount and isotopics of material at each
stage in the fuel cycle, explicitly modeling reactor parks and their supporting
facilities (i.e., enrichment, separations, and fuel fabrication) in order to
study transition scenarios at a high level of isotopic and physical
fidelity. Its origins date to 1991, and it has been updated with relative
frequency since 2005 \cite{boucher_cosi_2005,boucher_cosi:_2006,meyer_new_2009,
coquelet-pascal_validation_2011}.

The main user input for a COSI simulation is period to be simulated and the
commissioning and decommissioning date for each type of reactor in the
simulation. It does not deploy reactors based on any sort of algorithmic
model. Users are also allowed to choose the processing order of spent fuel
(e.g. first in/first out, as a function of burnup, as a function of Pu-241
content, etc.).

In general, reactors are assumed to be fueled with material whose makeup
corresponds to an equilibrium cycle. For each type of reactor, a single fuel
type is assumed to be used. COSI usese a family of methods for determining input
fuel isotopics, the aggregate name of which is an ``Equivalence Model''. The
term refers to an ``equivalent fraction'' of fissile isotopes in fabricated fuel
comprised of fissile and fertile isotopes given a target fertile fraction and
available isotopics of separated fuel. In the simplest case, i.e., for enriched
uranium oxide (UOX) fuel, the model is simply a predefined enrichment.
Similarly, for mixed plutonium-uranium oxide (MOX) fuel for use in thermal
reactors, the equivalence model is simply a predefined plutonium content. It is
not clear, however, if the plutonium content is in total mass or mass of a
specific plutonium isotope \cite{meyer_new_2009,
coquelet-pascal_validation_2011}. Further, I assume that a similar equivalence
model is used for any other thermal reactor fuel. Specifically, this set of
models requires a class fertile of material (e.g. depleted uranium, recycled
uranium, natural uranium) and a class of fissile material (e.g. plutonium,
transuranics), and a predefined ratio of the two. It is unclear how many degrees
of freedom a user has in defining these parameters for each reactor. In any
case, UOX is a special case of the equivalence model, because it has no
restrictions on its fabrication, i.e., COSI does not allow natural uranium
availability to be a constraining parameter in a simulation.

In order to determine the input isotopics for fast reactors, COSI uses an old
technique developed by Baker and Ross \cite{baker_comparison_1963}. In fact,
this is where the original notion of an ``equivalence method'' originates. The
core premise of this method is that a fast reactor at equilibrium is ideally
loaded with Pu-239 and U-238. Accordingly, deviations from this ideal
composition must be accounted for. Baker and Ross describe an equilibrium
analysis in which ``for a system initially just critical, the reactivity is
maintained zero'', which is ``justified by perturbation theory''. They describe
this condition mathematically as

\begin{equation*}
\sum_{i \in I} \left( \nu_{i} \sigma_{f,i} - \sigma_{a,i} \right) N_i = c,
\end{equation*}

where $c$ is a constant, and for a given isotope $i$ in the set of heavy fuel
elemental isotopes, $I$, $N_i$ is its average number density, $\nu_{i}$ is the
average number of neutrons resulting from its fission, $\sigma_{f,i}$ is its
microscopic fission cross section, and $\sigma_{a,i}$ is its microscopic
absorption cross section. For convenience, let us define the collection of
isotopic parameters as

\begin{equation*}
x_i = nu_{i} \sigma_{f,i} - \sigma_{a,i}.
\end{equation*}

Baker and Ross then note that one can approximate the critical mass of plutonium
required for a given mixture of plutonium isotopes using an isotopic worth,
$w_i$, that is a function of its deviation from pure plutonium-239,

\begin{equation*}
w_i = \frac{x_i - x_{^{238}U}}
           {x_{^{239}Pu} - x_{^{238}U}}.
\end{equation*}

Then, for any combination that makes a core critical, the following holds

\begin{equation*}
\sum_{i \in I} m_i w_i = c,
\end{equation*}

where $m_i$ is the mass of isotope $i$.

The COSI team adapts this approximation method for its use in order to determine
the fuel composition of fast reactors. Given an ideal plutonium-239 enrichment,
$E_0$, and for each isotope in a set of fertile and fissile isotopes, $I_{Fe}$
and $I_{Fi}$, respectively, the isotopic neutronic weights, $w_i$, and available
isotopic weight fractions, $\xi_i$, one can compute an equivalent fissile
isotope enrichment, $E$.

\begin{equation*}
E = \frac{E_0 - \sum_{i \in I_{Fe}} \xi_i w_i}
         {\sum_{i \in I_{Fi}} \xi_i w_i - \sum_{i \in I_{Fe}} \xi_i w_i}
\end{equation*}

This perturbed fissile enrichment is then used to determine the amount of
material to take from available fissile stockpiles and the amount of material to
take from available fertile stockpiles, i.e., $1 - E_0$.

It is important to note that the resulting isotopics for either thermal or fast
reactor fuel is a function of the user-defined reprocessing order. The
equivalence model simply determines the quantity of fertile and fissile material
to extract from the available stockpiles. Granted, the equivalence models
are \textit{informed} by the isotopics of those stockpiles, but they are
considered constants in the calculation. Further, the level of fidelity of
modeling of the separations and fabrications facilities are not explicitly
stated. In the COSI literature, the authors refer to batches of material, but
never define explicitly what a batch is. 
