\chapter{Literature Review}\label{ch:litreview}

\section{State of the Art of Fuel Cycle Simulation}\label{sec:simulators}

This section of the literature review is designed to give an overview of the
current landscape of fuel cycle simulation technology from the perspective of a
fuel cycle simulation developer. Accordingly, I discuss both the actual
simulation aspects of each simulator, e.g., how entities are instatiated and
connected, as well as the more nuts-and-bolts aspects, e.g. how simulation input
and output are handled. The goal of this part of the review is to inform
decisions made on a simulation level as \Cyclus is continued to be developed.

While many fuel cycle simulators exist, all of their source code is effectively
closed source, i.e., not available to the public. Any knowledge, therefore, of
their inner workings must be gleaned from the available literature published by
their respective development teams. From a developers perspective, I make some
assumptions when the literature isn't clear, but I do my best to note what are
my assumptions and what are stated facts in the literature.


%% I'm not sure about this, revise once other sections are written %%
Finally, let me note that there are really two subjects under investigation
here. The first is how to go about simulating the nuclear fuel cycle with
respect to when and how to build facilities in the simulation. The second is how
to govern those facility's interactions. The former subject varies by each
simulator in question whereas the latter is generally handled by simulation
dynamics software in most cases. I highlight a different approach in the section
on COSI who wrote their own interaction mechanisms from the ground up.

\subsection{COSI}\label{sec:cosi}
\subsection{CAFCA}\label{sec:cafca}
\subsection{VISION}\label{sec:vision}
\subsection{DANESS}\label{sec:daness}

\section{Approximation Linear Program}\label{sec:approx}

\subsection{The Simplex Method}

\section{Multicommodity Transport Problem}\label{sec:MCTP}

%%% 
\begin{subequations}\label{eqs:MCTP}
  \begin{align}
    %%
    \min_{z} \:\: & 
    z = \sum_{i \in I}\sum_{j \in J}\sum_{h \in H} c_{i,j}^{h} x_{i,j}^{h}
    & \label{eqs:MCTP_obj} \\
    %%
    \text{s.t.} \:\: &
    \sum_{j \in J} x_{i,j}^{h} = a_{i}^{k}
    &
    \forall \: i \in I, \forall \: h \in H \label{eqs:MCTP_sup} \\
    %%
    &
    \sum_{i \in I} x_{i,j}^{h} = b_{j}^{k}
    & 
    \forall \: j \in J, \forall \: h \in H \label{eqs:MCTP_dem} \\
    %%
    &
    \sum_{h \in H} x_{i,j}^{h} \leq u_{i,j}
    & 
    \forall \: j \in J \label{eqs:MCTP_cap} \\
    %%
    &
    x_{i,j}^{k} \geq 0
    &
    \forall \: i \in I, \forall \: j \in J, \forall \: h \in H \label{eqs:MCTP_x}
    %%
  \end{align}
\end{subequations}
%%% 


\subsection{The Transportation Simplex Model}\label{sec:trans-simplex}

\section{Integer Programming}

\subsection{The Branch and Bound Algorithm}\label{sec:bnb}
