The Code for Advanced Fuel Cycles Assessment - System Dynamics (CAFCA-SD) is
developed at the Massachusetts Institute of Technology (MIT). Originally
developed in MATLAB, it is currently an application written in the commercial
software VENSIM \cite{vensim_2010_ventana} ``with potential interactions with
C++ programs'' \cite{guerin_benchmark_2009}. CAFCA-SD has gone through a number
of developmental iterations. It originally was designed to analyze transuranic
(TRU) recycling in fertile/fertile-free and actinide-burning reactors. It was
then extended to incorporate fast, self-sustaining reactors (i.e., with a
conversion ratio of unity) and to allow constraints on reprocessing facility
capacity factors. The third update allowed for isotope tracking and the decay of
isotopes, however, the CAFCA literature states that it does not incorporate the
use of decay and claims such capability does not greatly affect results
\cite{guerin_impact_2009,guerin_benchmark_2009}. The final iteration
improvements include the ability to model fuels such as mixed-oxide fuels (MOX)
and ``high burnup fuel in thermal reactors'', fast reactors with conversion
ratios other than unity, and the use of multiple fuel technologies that require
recycling in the same scenario (for example, having MOX fuel being recycled
multiple times). Time steps in CAFCA-SD occur every 1.5
months \cite{guerin_impact_2009}, at which point discrete events are executed to
change the system state that depend on the current system state.

The main user-provided parameters for a simulation are the power demand for
nuclear reactors, which is modeled as an exponential cuver, and the fraction of
TRU or minor actinides (MA) to provide for each reactor technology. TRU set
aside for each technology is essentially modeled as a different fuel type
(e.g. TRU for MOX and TRU for Fast Reactors). The simulation framework uses
these fractions to determine the total number of reactors that can be fuel for
each type of reactor. Users also set a number of other parameters, such as
reactor power and other reactor characteristics, but the above two parameter
categories govern the actual course of the simulation with respect to building
reactors and determining material flows. The forcing function for the
simulation, is the availability of recycled fuel. In other words, as one of
CAFCA's basic simulation assumptions, the inventory of used light water reactor
(LWR) fuel is minimized. Thus the number of reactors using this recycled fuel is
maximized agaisnt this constraint, according to the user-defined fuel technology
preference fractions. 

As previously stated, CAFCA-SD determines the evolution of system states using
system dynamics. It should be noted that the CAFCA team states explicitly that
their model should be used for large scoping studies \cite{guerin_impact_2009},
which informs their simulation methodology and inherent assumptions. Their
overall system model is comprised of coupled single-input, single-output
structure-policy diagrams. Each structure-policy diagram is composed of a
system-structure substructure and a policy-structure substructure, which defines
decision rules for the diagram. The beginning of a time step invokes the
application of decision rules which changes the state of the system. The new
state is then updated in the system structure, which informs new decision
rules. Each diagram is connected to others via their input or output. Examples
of such diagrams are: the LWRs structure-policy diagram, the front-end
structure-policy diagram, and the back-end structure-policy diagram. Further,
there are diagrams for each reactor fuel type that requires new technologies
(e.g. MOX, ABR).

In general, CAFCA does not model individual
facilities \cite{guerin_impact_2009}. Furthermore, it only explicitly models
reactors and reprocessing facilities. The remaining entities, which are
conceptually connected to the instantiated entities via markets, are assumed to
belong to completely elastic markets, i.e., demand is always met. Instead of
modeling individual facilities, it models rate changes of fleets of facilities
and monitors the status of those fleets (i.e., how many should currently be in
operation). This methodology is perhaps not intuitively obvious, so let me
briefly follow a similar overview of such a process as provided in CAFCA's
original methodolgy report \cite{busquim_e_silva_system_2008}. First, though, a
note on their notation; CAFCA's methodology denotes many system variables as a
function of time for each class (fleet) of reactors, which are described in
Table \ref{tbl:cafca-rxtr-vars}.

Before wading too deep into CAFCA's methodology formulation, let me take a
moment to reflect on the variables in Table \ref{tbl:cafca-rxtr-vars}. The
actual definition of the various rates and stocks described therein depend on
the type of reactor they are describing. Any reactor that is fueled in some part
by material that is output from reprocessing facilities has the same general
form, which incorporates the respective user-defined percentages previously
discussed. This makes sense, of course, because CAFCA's simulation methodology
only allows advanced reactors to be built if they can be fueled for their entire
lifetimes. Accordingly, their building rate depends on the availability of
recycled fuel. Reactors that use non-recycled fuel, i.e. light water reactors
(LWRs), have different formulations. There is no restriction on their building
based on fuel availability (their input fuel markets are assumed totally liquid
as stated above), and so their rate and state values which are informed by the
number of recycle-fueled reactors in operation. In other words, LWRs ``fill the
holes'' in energy demand left by fast reactors that can't otherwise make them up
due to lack of input fuel. The cycle is connected because LWRs provide the
source of fast reactor fuel.

%%% 
\begin{table} [h!]
\centering
\begin{tabularx}{\textwidth-20pt}{|c|X|} % line wraps second column if too long
\hline
Variable    & Description \\
\hline
$P_r$            & the power rating for reactors of fleet $r$ \\
$CF_r$           & the capacity factor rating for reactors of fleet $r$ \\
$F^r_{Est}(t)$   & the forecasted fleet requirements for the fleet of $r$ at time $t$  \\
$F_r(t)$         & the actual fleet requirements for the fleet of $r$ at time $t$  \\
$ADJ_r(t)$       & the adjustment for the fleet of $r$ at time $t$  \\
$\tau_r(t)$      & the fleet adjustment time for the fleet of $r$ at time $t$  \\
$R^r_{CO}(t)$    & the construction order rate for the fleet of $r$ at time $t$  \\
$R^r_{FO}(t)$    & the fulfilled order rate for the fleet of $r$ at time $t$  \\
$R^r_{FO}(t)$    & the fulfilled order rate for the fleet of $r$ at time $t$  \\
$R^r_{Frac}(t)$  & the fractional order rate for the fleet of $r$ at time $t$  \\
$R^r_{CR}(t)$    & the contruction rate for the fleet of $r$ at time $t$  \\
$R^r_{DR}(t)$    & the decommissioning rate for the fleet of $r$ at time $t$  \\
$I_r(t)$         & the integer number of reactors ready to start commercial operation for the fleet of $r$ at time $t$  \\
$F^r_N(t)$       & the rate of reactors $r$ starting commerical operation at time $t$ \\
$O_r(t)$         & the fulfilled order delay of the fleet of $r$ at time $t$ \\
\hline
\end{tabularx}
\caption{Variables Associated with a Class of Reactor, $r$, in CAFCA's Methodology}
\label{tbl:cafca-rxtr-vars}
\end{table}
%%% 

Finally, let me note that there are similar rates defined for supporting
facilities, i.e. reprocessing facilities for each class of reactor. They are
similar to the above formulation, but have as input the amount of used fuel to
be reprocessed and output the required number of facilities. A full treatment of
their reactor, reprocessing, and material state and rate formulations is
provided in \cite{busquim_e_silva_system_2008}.

CAFCA makes a number of assumptions regarding material isotopic consituencies
and reactor fuel loading. First of all, each class of reactors is assumed to be
loaded with equilibrium-cycle fresh fuel. This implies that all reactors of a
given class of reactors take as input a given makeup of fresh fuel and provide
as output a given makeup of used fuel. Accordingly, there is no notion of fuel
isotopics changed based on the number of cycles it has resided in a
reactor. Furthermore, CAFCA models continuous annual mass flows rather than the
discrete transfer of material amongst facilities. This means that every time
step, the same amount of material is continuously consumed by a reactor, rather
than an actual core reload being modeled. The initial core loading is modeled
similarly, simply with an elevated annual consumption level for its first cycle
period. This assumption leads to an 0.6\% increased overall fuel usage per
reactor \cite{guerin_impact_2009}.

An incredibly important consequence of CAFCA's formulation is that it inherently
bins the isotopics of fuel into categories, e.g., TRU or plutonium and minor
actinides. Accordingly, there is no inherent check on the validity of isotopics
entering or exiting reactors, i.e., their methodology guarantees
correct \textit{aggregate mass flows} at an elemental scope, but does not
consider individual isotopics at a simulation-methodology level. In other words,
CAFCA treats all classes of material as lumped\cite{guerin_impact_2009}. One
aspect of their simulation where this assumption is plays a critical role is in
their determination of input fuel for fast reactors that have a conversion ratio
other than unity. In such scenarios, TRU from LWRs and similar fast reactors are
lumped together and treated as one fuel source commodity after being mixed with
either depleted or recycled uranium \cite{guerin_impact_2009}. Such an
assumption allows CAFCA to still compute larger aggregate metrics such as
uranium utilization. Additionally, the lumped material assumption is perhaps one
reason why the CAFCA team sees minimal impact from isotopic decay.
