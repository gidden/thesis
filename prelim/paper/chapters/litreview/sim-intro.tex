A number of current fuel cycle simulators are reviewed in this section with a
keen focus of simulation-developer related design aspects. There are broadly
three main categories that simulation developers must address: how simulation
entities enter a given simulation, at what level of fidelity the physical and
chemical processes are modeled, and how simulation entities interact, or are
connected, in a simulation. Deployment and process fidelity concerns are
generally handled differently by each simulator. Entity interaction, i.e., the
determining of mass flows, is generally handled by proprietary system dynamics
software in most cases. Peripheral concerns are also addressed, namely
simulation input/output and the associated coding platform(s). Because all
current fuel cycle simulators are effectively closed source, any knowledge of
their inner workings must be gleaned from the available literature published by
their respective development teams. Additionally, summary information can be
gathered from the MIT benchmarking exercise \cite{guerin_benchmark_2009},
because the sections corresponding to each simulator were written by the
simulator developers. From a developers perspective, assumptions are made when
the literature does not provide a clear explanation of a given design decision,
but such assumptions are expressly noted.
