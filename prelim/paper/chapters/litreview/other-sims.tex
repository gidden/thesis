There are many other fuel cycle simulators that have been developed to
date. Many different countries will take on the effort for their personal needs,
such as Japan and Russia. Others are developed by international organizations
like the IAEA. However, all of them, including those outlined in previous
sections, are have code bases that are closed source, i.e., they are private
projects. Accordingly we must rely on their respectively produced literature for
insight into their modeling practice. The previous sections benefited from a
reasonably large literature base (with respect to the norms of the fuel cycle
simulation community). This section overviews other fuel cycle simulators that
are important in the world context because of their user base, but have few
available sources from which to glean developer-level information about their
simulation structure for either proprietary purposes or other reasons. VISTA is
a fuel cycle simulation code developed by the IAEA and distributed to member
states. However, for benchmarking exercises more recently, the IAEA has used
the Russian code DESAE. DESAE is designed with a focus on regional material
flows, making it relatively unique in the fuel cycle simulation community and of
interest to the IAEA. DANESS is a fuel cycle simulation code originally
developed at Argonne National Laboratory, but its ownership has transferred to a
Belgian group, LISTO bvba. It is currently used by many European member states,
including Belgium and Italy.

VISTA is designed to estimate the front end and back end material requirements
for a given fuel cycle. It is not clear what language or platform VISTA is
developed in. Input parameters include the total nuclear energy to be met, the
types of reactors and share of electricity for each, the equilibrium fuel
composition for each type of reactor, and the average discharge burnup for each
type of reactor. Output parameters include the required amount of natural
uranium and enrichment services as well as the nuclide inventory of spent fuel
developed over the course of the simulation. For each type of reactor modeled in
the simulation, there is a fuel pairing with which it is fueled, as well as a
corresponding burnup.

VISTA models the nuclide inventory at a ``low level'', i.e., it keeps track of
individual isotopic quantities. The isotopic composition of spent fuel is
determined by its depletion module, Calculation of Actinide Inventory
(CAIN). Similar to the United State's depletion code, ORIGEN
\cite{bell_origen_1973}, CAIN solves Bateman's Equation for a point assembly for
a number of burnup steps as required by the target burnup and initial fuel
composition. The solver is, however, limited by the input provided to it by
VISTA. For instance, the VISTA team cites the assumptions used for a MOX
calculation in \cite{iaea_nuclear_2007}. First, the plutonium vector for a
MOX-based fuel is assumed to be directly taken from it's UOX variant as a
function of burnup. In other words, the MOX plutonium vector for $45 \frac{GW
  d}{tHM}$ MOX fuel is the result of discharged $45 \frac{GW d}{tHM}$ UOX
fuel. Furthermore, this plutonium vector is assumed to be constant across
generations or ``passes'' of MOX fuel. This implies that MOX fuel sourced from
recycled MOX is isotopically identical to MOX fuel sourced from recycled UOX.

The actual simulation engine methodology, which uses a time step of one year, is
somewhat outlined in Appendix II of \cite{iaea_nuclear_2007}. Again, the
simulation tracks quantities of isotopes as well as quantities of their
elemental aggregate. Facility fleet connections are hard-coded (i.e., known at
the beginning of the simulation), and balance equations are applied at each time
step to determine the quantity of material flowing between facility fleets. The
equations presented in \cite{iaea_nuclear_2007} assume reactors are built
regardless of fuel availability. It appears as if each reactor type has two fuel
types with which it can be fueled. If there is not enough fuel availability for
advanced reactors using recycled fuel (type 2), they are considered to be fueled
with their other fuel type. Such an observation is not explicitly stated by the
VISTA team, but implied by their series of balance equations. Further, it is
unclear how the simulation handles isotopic mismatches. As was described above,
constant plutonium vectors are assumed for fresh fuel, and isotopic-specific
values are determined for used fuel by CAINS. However, some strategy must be
used to match the used fuel isotopics to new fresh fuel isotopics, which are
defined. Such a strategy is not described in the VISTA literature, but some sort
of check (either at the elemental or isotopic level) must be made, because an
alternative fuel type is used "[i]f there is enough reprocessed material in the
stockpile" \cite{iaea_nuclear_2007}. Finally, although again not explicitly
stated, the set of supply-demand equations are solved for each time step using
values from the previous time step and the additional reactor requirements for
the current time step. It does not appear that core reloads are explicitly
modeled, and instead reactor requirements are modeled on simply a per-year
basis.

The Dynamic of Energy System - Atomic Energy (DESAE) code is developed by the
Russian government as a basis for benchmark calculations called for by the
IAEA's International Project on Innovative Nuclear Reactors and Fuel Cycles
(INPRO) \cite{_international_2009,andrianova_desae_2008}. DESAE is another
example of a macro-model of the nuclear fuel cycle, concentrating on
infrastructure requirements as well as material balances. Although not very
descriptive with respect to simulation methodology, one can learn a little of
DESAE's simulation model from their published literature
\cite{andrianova_desae_2008}. Reactors are modeled in order to meet a specified
energy production level, and different types of reactors can be modeled; 20
different reactor types are currently allowed in a given scenario. Assuming
DESAE models different ``types'' of reactors as other simulators do, a different
reactor ``type'' is created if any parameter changes (e.g. a $50 \frac{GW
  d}{tHM}$ UOX-fueled LWR is different from a $60 \frac{GW d}{tHM}$ UOX-fueled
LWR). Characteristic parameters of a given reactor fleet are allowed to change
over time, however. Material pathways are explicitly modeled, with flows going
directly between facilities based on their isotopic content (e.g., plutonium
isotopes from recycling go to fuel fabrication). It is not clear how
isotopic-specific information affects the course of a simulation, nor is it
clear what strategy is taken to model the required advanced fuel to separated
fuel supply gap.

Additional information discussing DESAE's internal modeling methods are
described in the IAEA's INPRO fuel cycle scenario synopsis document
\cite{iaea_nuclear_2010}.  DESAE models the fuel cycle on both the regional and
global level, and its is possible (and indeed likely, according to
\cite{iaea_nuclear_2010}) that the supply-demand regional and global
requirements will be inconsistent, with the region-specific calculation
overestimating the resource requirements. Exactly what resources are being
overestimate is not discussed, and the topic is only breached in a broad
sense. The inter-regional transportation calculation is discussed in slight
detail, however. 

The supply and demand for any given resource is defined by region, where $c_i$
is the consumption (or demand) in a region, $i$, $p_i$ is the production of that
resource in the region. Further, a transport matrix, $T$, is defined, where an
element $t_{i,j}$, defines the fraction of the resource produced in region $i$
that is sent to region $j$. The following matrix equation can then be solved,

\begin{equation*}
\vec{C} = T \cdot \vec{P}.
\end{equation*}

The solution is constrained by total production capacity, i.e., 

\begin{equation*}
\sum_{j \in N} t_{i,j} = 1,
\end{equation*}

where $N$ is the set of regions in the simulation. The production and supply
vectors, $P$ and $C$, are provided as input to the calculation. Accordingly,
they must be consistent, i.e.,

\begin{equation*}
\sum_{i \in N} c_i - d_i = 0,
\end{equation*}

however, it is not stated how such a constraint is enforced \textit{a
  priori}. The unknown variables in this calculation are the transportation
coefficients. An initial value must be provided (``by an expert'') for the
transportation matrix, $T_0$. The solution mechanism then iteratively set to
minimize the aggregate $L_2$ norm of the transportation-matrix elements:

\begin{equation*}
min \sum_{i \in N} \sum_{j \in N} \left( t_{i,j} - t_{i,j}^0 \right)^2,
\end{equation*}

The actual solution algorithm is not described in either
\cite{iaea_nuclear_2010} or \cite{andrianova_desae_2008}.

The Dynamic Analysis of Nuclear Energy System Strategies (DANESS) code was
originally developed at ANL, and its development has subsequently been moved to
Belgium's LISTO bvba, who provides commercial fuel-cycle modeling
services. Accordingly, its simulation methodology is only lightly treated in the
available literature, but information can be gathered from updates provided by
the development team \cite{van_den_durpel_daness_2009} and benchmarking
exercises in which the code participates \cite{guerin_benchmark_2009}. 

DANESS is a system dynamics application written in the system dynamics code
iThink \cite{richmond_ithink_2004}. It is broken into a variety of ``models''
which house the simulation's decision making methods and track the state of
various entities in the simulation. DANESS output is supplied via Microsoft
Excel template files \cite{guerin_benchmark_2009}.

DANESS makes reactor deployment decisions using user-defined deployment
profiles, a minimum-cost deployment algorithm, or some combination thereof. The
user can define either a specific deployment plan or a nuclear power demand with
a set of reactor-type fractions to determine which reactors meet that
demand. The minimum-cost deployment algorithm compares the cost of different
reactor types and other non-nuclear energy sources to determine which technology
to deploy. DANESS also employs a forecasting algorithm to determine the future
fissile material availability for each fuel type. It is able to guarantee that
the initial fuel loading plus some multiplicative constant will be available
when the proposed facility will enter the simulation. For example, a ten-percent
buffer would guarantee that the initial core plus ten-percent of all fuel
required by the reactor over its lifetime is available
\cite{guerin_benchmark_2009}. Importantly, it is not clear how the isotopics of
the input recipe are treated. It is possible that, like most other simulators, a
macro specification of TRU or plutonium is queried.

The facilities are modeled explicitly, with individual facilities being
tracked. As with other simulators, reactors are paired with a specific fuel
type, and DANESS can model at most 10 types of reactors. Supporting facilities
are deployed according to a lookahead function that predicts future demand for
their services. An explicit repository model is also provided, which applies
thermal-loading capacity limitations for both Yucca Mountain (which has a tuff
geology) and a generic clay repository.

The full set of cost factors involved in the minimum cost deployment algorithm
is not explicitly stated in the DANESS literature, but a few are
discussed. Interestingly, DANESS incorporates a cost model for disposing of
waste in the repository model which informs the deployment algorithm,
e.g. making thermal reactors slightly more expensive due to higher spent fuel
volumes. Given that fuel costs are relatively small for thermal reactors (at
least currently), it is not clear what affect these considerations have in the
deployment algorithm.

Reactors in DANESS can have time-varying core compositions as defined by the
user. Reactors order fuel for reloading at a specified time prior to its need
based upon known upstream processing times (i.e., it orders it ``just in
time''). Upon fueling a reactor, downstream buffers are updated with respect to
mass-flows resulting from the eventual ejection of this fuel
\cite{guerin_benchmark_2009}. DANESS uses a table lookup system to determine the
isotopics of spent fuel based on burnup for thermal reactors and conversion
ratio for fast reactors. Isotopics are modeled (mostly) explicitly. DANESS
tracks 68 different isotopes and two aggregate isotopic groups, comprised of
short and long-lived fission products. The full list of isotopes tracked is
provided in \cite{van_den_durpel_daness_2009}; from a review of the DANESS
literature, it appears that the code does not have any sort of isotopic decay
mechanism. Further, it is not stated how DANESS deals with mismatches between
requested fuel recipes and available separated isotopics. However, it is
specifically mentioned that they bin separated material by element
\cite{van_den_durpel_daness_2009}, so it is possible that only bulk elements are
taken into account for such purposes. Finally, If a reactor can not be fueled
due simulation-based material availability, e.g. a lack of fissile material, it
is placed in a ``stand by'' mode and does not generate energy for that period of
time in the simulation.

DANESS has recently become capable of a multi-region fuel cycle simulation.
Beginning with version 4.2, DANESS can model regions separately (as nominally
separate DANESS simulations) and model regional interaction as exogenous input
to supply or demand. The DANESS literature does not describe either the
methodology or implementation of such a interaction structure, only stating that
it has the capability \cite{van_den_durpel_daness_2009}.
