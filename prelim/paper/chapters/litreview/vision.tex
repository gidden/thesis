The Verifiable Fuel Cycle Simulation (VISION) code is developed at the Idaho
National Lab (INL) and is a Powersim \cite{studio_powersim_2003} application
with input and output functionality provided through Excel spreadsheets. Like
VENSIM, Powersim is a systems dynamics coding platform. The original VISION code
base was adapted from DYMOND \cite{moisseytsev_dymond_2001} after internal
memory limits were reached in its application language, iThink/Stella
\cite{richmond_ithink_2004}. VISION explicitly models reactor parks, 
separations, fuel fabrication, storage and repository facilities. ``Front end''
facilities, i.e., mining, conversion, and enrichment facilities, are not
explicitly modeled \cite{guerin_benchmark_2009}. The stated goal of VISION is
two-fold: to investigate infrastructure requirements for various fuel cycles
(termed ``what if'' scenarios) \cite{jacobson_verifiable_2010} and to
investigate upset scenarios, e.g., a loss of a fuel fabrication facility, and
corresponding mitigation strategies \cite{schweitzer_improved_2008}.

VISION, like CAFCA, uses a system dynamics methodology to model the nuclear fuel
cycle. Accordingly, there is a notion of stocks of materials, reactors,
supporting facilities, etc., and flows of materials between those
facilities. Also similar to CAFCA, VISION uses a look-ahead function to assess
the future state of the simulation in order to make decisions at a given
time. Accordingly, reactors are only built if there will be adequate fuel
available for them. The main user input to a VISION calculation is the required
electricity demand to be met by nuclear power. Users also specify a preference
of reactors to meet this power demand (e.g., 80\% fast reactor, 20\% LWR);
however due to the symbiotic nature of the input fuel dependence of fast
reactors on LWRs and the basic VISION assumption that all reactors will be
fueled for their entire lifespan, the user-preferred distribution may not always
be met. Users also must define all input and output material characteristics for
reactors as well as other facility-level parameters (power level, capacity
factor, etc.). Time steps in VISION are 3 months long, breaking a year into four
time steps.

VISION maintains records for 81 different isotopes. The planning algorithm for
recycled reactor fuel, however, concentrates on a small subset of isotopes
termed ``control isotopes''. The control isotopes for any given simulation are
user-defined input, and choices include individual plutonium isotopes, all
plutonium isotopes, and all transuranic isotopes. Interestingly, VISION also
allows material in any storage buffer to decay if it has resided there for
longer than a year. It is not clear how the changing of fuel isotopics affects
the look-ahead calculation. All material-based output metrics operate on these
isotopes. For example, proliferation resistance is measured by the amount of
plutonium isotopes or transuranics at any point in the fuel cycle
\cite{yacout_vision_2006}.

VISION has no transmutation engine to determine output fuel isotopics. Instead,
a look-up table is used, scanning precalculated isotopic vectors and
interpolating based on fuel type and ``key input parameters''
\cite{jacobson_verifiable_2010}. In general, input fuel recipes are defined and
then output fuel recipes are determined based on burnup for LWR UOX fuel and
conversion ratio for fast reactor fuel. Accordingly, input fuel recipes must be
known \textit{a priori} and physics calculations must be run to populate a
database of preformed fuel output recipes. Such an operation has been
implemented for the current predefined cases, but it is not clear how isotopic
decay is involved in determining perturbations to the input fuel recipes, if at
all.

Schweitzer provides a look at the facility ordering methodology in his master's
thesis \cite{schweitzer_improved_2008}. Understanding this methodology is
important to understanding how VISION works, so a short review is presented for
the remainder of this section. There are a few important assumptions worked into
the methodology. First, the only limiting factor on reactor fuel supply is used
LWR fuel, i.e., there is no limiting factor on fresh UOX. Secondly, it is
assumed that reactors have a known fuel usage over their lifetime and that this
usage is divided into fuel of different integral number of passes through the
reactor type. Put another way, fast reactors lifetime inventories are measured
by some fuel isotopic makeup and each successive refueling (pass) is comprised
of a different isotopic makeup, as defined by the user, up to some limit. Any
additional recycles are assumed to have reached an equilibrium and their
isotopics remain at this maximum. In Schweitzer's work and in the VISION
literature, the limit is five passes. In general, the driving force behind
facilities being built in the simulation is the need for separated spent fuel
for fast reactors.

Fast reactors are ordered based on a user-defined preferred distribution,
$N_{FBR,E}$, and fast reactor fuel availability, $N_{FBR,SF}$. Given some
look-ahead parameter, $\Delta t$, the number of fast reactors that will be
ordered at the look-ahead time, $t + \Delta t$, is

\begin{equation}
N_{FBR}\left(t+\Delta t\right) = min \left( N_{FBR,E}\left(t+\Delta t\right), N_{FBR,SF}\left(t+\Delta t\right)\right),
\end{equation}

where the user-defined parameter is based on the user-provided distribution
percentage $p_{FBR}$, the rated power for that type of reactor, $P_{FBR}$, and
an energy ``gap'', i.e., the amount of energy required in the simulation as
defined by the user less the current amount of energy in the simulation at the
look-ahead time,

\begin{equation}
N_{FBR,E}\left(t+\Delta t\right) = 
                        \lceil \frac{p_{FBR} \Delta E (t + \Delta t)}
                                    {P_{FBR}} \rceil .
\end{equation}

The number of fast reactors that can be built based fuel availability is a
function of the amount of available spent fuel, $SF(t)$, and the amount
of material required to fuel a reactor for its lifetime, $F^{LWR_{SF}}_{Total}$,

\begin{equation}
N_{FBR,SF}\left(t+\Delta t\right) = 
                        \lceil \frac{SF(t)}
                                    {F^{LWR_{SF}}_{Total}} \rceil .
\end{equation}

The above equation allows some insight into the inner workings of VISION. First
of all, it keeps a running tally of available spent fuel for a given reactor
type, $SF(t)$ above. This is an aggregate parameter for the fleet of
reactors. Secondly, a lifetime's worth of fuel is effectively set aside whenever
a reactor is introduced in the simulation, because the spent fuel is assessed at
time $t$ rather than $t + \Delta t$. Therefore a fast reactor can only enter the
simulation at the look-ahead time if there is enough fuel for it at the present
time. Furthermore, there is in fact a notion of mortgaged spent fuel in VISION,
i.e., spent fuel set aside for future reactors, and unmortgaged spent fuel in
VISION, i.e., spent fuel produced but not yet mortgaged,

\begin{equation}
SF(t) = SF_u(t) - SF_m(t),
\end{equation}

where the mortgaged spent fuel is, of course, a function of the number of
reactors to be built in the future and how much spent fuel they use,

\begin{equation}
SF_m(t) = SF_m(t - \Delta t_{sim}) + N_{FBR}(t + \Delta t) F^{LWR_{SF}}_{Total},
\end{equation}

with $\Delta t_{sim}$ denoting the simulation time step.

The amount of fuel required for a fast reactor's lifetime is based on VISION's
assumption of fuel behavior of ``passing through'' the reactor multiple
times. For each ``pass'', $p$, the amount of fuel reloaded, $FL^p_{FBR}$, into
the core may be different both at loading and unloading. The amount of total
required fuel is defined as

\begin{equation}
\begin{aligned}
F^{LWR_{SF}}_{Total} & = N_{b,FBR} FL^1_{FBR} + \Delta t_p FL^1_{FBR} w^1_{FBR} \\
                   & + \sum_{p=2}^{5} \Delta t_T (FL^p_{FBR,in} w^p_{FBR,in} - FL^{p-1}_{FBR,out} w^{p-1}_{FBR,out}) \\
                   & + (\Delta t_{FBR,l} - 4 \Delta t_T - \Delta t_p) (FL^5_{FBR,in} w^5_{FBR,in} - FL^{5}_{FBR,out} w^{5}_{FBR,out}),
\end{aligned}
\end{equation}

where $N_{b,FBR}$ is the number of fuel batches, each $w^p$ is the weight
percent of the control isotope(s) in the spent fuel, $\Delta t_p$ is the
pipeline time required for fuel exiting the reactor to be put back in it
(storage, separation and fabrication time), $\Delta t_T$ is the total cycle time
(i.e., pipeline time plus reactor residency time), and $\Delta t_{FBR,l}$ is the
reactor's lifetime. Note that all of these parameters are defined implicitly or
explicitly by user input. Furthermore, note that this amount has units
of \textit{mass of the control isotope(s)}, implying that VISION works on mass
balances of those isotopes. Finally, note that VISION treats all recycled LWR
spent fuel as an aggregate mass, whether it is UOX, MOX, or inert matrix fuel
(IMF).

It is entirely possible that the fast reactor builds in a simulation will be
limited by the available fuel. In such cases, VISION still guarantees that the
``power gap'' will be closed, and does so by building LWRs. In addition to any
LWRs requested by the user to be built, others may be built depending on the
number of FBRs not able to be built,

\begin{equation}
N^-_{FBR}\left(t+\Delta t\right) = 
                        max(0,N_{FBR,E}\left(t+\Delta t\right) - 
                        N_{FBR}\left(t+\Delta t\right)),
\end{equation}

which then informs the number of additional light water reactors to be built,

\begin{equation}
N^+_{LWR}\left(t+\Delta t\right) = \lceil N^-_{FBR}\left(t+\Delta t\right) 
                        \frac{P_{FBR}}{P_{LWR}} \rceil .
\end{equation}

According to the VISION literature \cite{schweitzer_improved_2008}, it only has
the ability to model MOX-fueled reactors in a self-sustaining way. Such reactors
are initially fueled with enriched UOX and then reuse the recycled fuel once it
is available. There is a similar notion of passes through thermal recycling
facilities, where each fuel ``pass'' has its own accompanying isotopic
profile. Reactors are assumed to consume more UOX for the remaining reload
amount not met by their own MOX. The material from the final core ejection is
then added to the available spent fuel pool for fast reactors.

Support facilities are built based on the demand of power reactors. VISION is
able to guarantee that connections between facility fleets are maintained at an
appropriate capacity because the look-ahead calculation uses as its look-ahead
time the maximum time required to build any facility in the
simulation. Therefore, a VISION simulation can not ``run out'' of a servicing
capacity, for additional capacity will be built in time. Although the following
is associated with both products and services, I will use the term ``product''
in this discussion. The demand for a product, $x$, is represented as

\begin{equation}
D_x(t + \Delta t_x) = \sum_{y \in Y} \gamma_{x,y} N_y(t + \Delta t^x) C_y(t + \Delta t^x),
\end{equation}

where $\Delta t_x$ is the maximum time to bring a facility on line, $D_x(t)$ is
the demand for product $x$ at time $t$, $N_y(t)$ is the number of
facilities of type $y$ that require product $x$ at time $t$, $C_y(t)$ is
the capacity factor of the fleet of facilities $y$ at time $t$, and
$\gamma_{x,y}$ is a conversion factor to convert demand rate to product
rate that takes into account the manufacturing, delivery, and storage time for
the product.

The supply of a given product, $x$, is

\begin{equation}
S_x(t + \Delta t_x) = \beta_x N_x(t + \Delta t^x) C_x
\end{equation}

where $S_x(t)$ is the supply of product $x$ at time $t$, and $\beta_x$ is a
conversion factor between the number of facilities and the supply
rate. Additionally, facilities can hold an inventory of a product, defined as

\begin{equation}
I_x(t) = min \left( S_x(t-\Delta t_{sim}) - D_x(t-\Delta t_{sim}) 
       + I_x(t-\Delta t_{sim}), I_{x,max}(t) \right)
\end{equation}

where $I_{x,max}(t)$ is the maximum inventory capacity for product $x$ at time
$t$. The simulation then ``builds'' (i.e., adds product to its stocks and notes
the increased capacity) a new facility of type $x$ at time $t$ if 

\begin{equation}
S_x(t + \Delta t_x) + I_x(t + \Delta t_x) - 
      I_{x,b}(t + \Delta t_x) < D_x(t + \Delta t_x),
\end{equation}

where $I_{x,b}(t)$ is an inventory quantity that is saved or ``banked'' for
``emergency recovery'' \cite{schweitzer_improved_2008}.

The final topic of interest to this discussion is the mechanism by which
facilities send material to one another. Fleets of facilities in VISION are
modeled as modules in Powersim, the connections of which of
hardcoded. Accordingly, interactions are trivially predetermined. Furthermore,
the VISION simulation algorithm guarantees an adequate capacity services are
available, so material simply flows along the predefined paths.
