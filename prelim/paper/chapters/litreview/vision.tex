The Verifiable Fuel Cycle Simulation (VISION) code is developed at the Idaho
National Lab (INL) and is a Powersim \cite{studio_powersim_2003} application
with input and output functionality provided through Excel spreadsheets. Like
VENSIM, Powersim is a systems dynamics coding platform. The original VISION code
base was taken from the DYMOND \cite{moisseytsev_dymond_2001} after internal
memory limits were reached in its application language, Stella
\cite{clauset_stella_1987}. VISION explicitly models reactors, separations, fuel
fabrication, storage and repository facilities. ``Front end'' facilities,
i.e. mining, conversion, and enrichment facilities, are not explicitly modeled
\cite{guerin_benchmark_2009}. The stated goal of VISION is two-fold, to
investigate infrastructure requirements for various fuel cycles (termed ``what
if'' scenarios) \cite{jacobson_verifiable_2010} and to investigate upset
scenarios, e.g. a loss of a fuel fabrication facility, and corresponding
mitigation strategies \cite{schweitzer_improved_2008}.

