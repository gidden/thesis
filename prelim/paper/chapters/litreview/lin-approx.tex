Another application of linear programming involves approximating solutions to
linear systems of equations in the form of Equation \ref{eqs:lin-sys}.

\begin{equation}\label{eqs:lin-sys}
A x = b
\end{equation}

$A$ is an $m x n$-dimensional constraint matrix that can
be \textit{overdetermined}, i.e., include more constraints than decision
variables. Given some approximation of the solution, $\tilde{x}$, one can define
the residual, $r$, associated with that approximation.

\begin{equation}\label{eqs:lin-sys}
A \tilde{x} - b = r
\end{equation}

One can then formulate an optimization problem that seeks to minimize the
$\ell_p$-norm of the residual vector, $r$. The literature treats three types of
norms, $\ell_1$, $\ell_2$, and $\ell_\infty$, where each can be defined as
follows.

\begin{equation}
\ell_1 (r) = \| r \|_1 = \sum_{i = 1}^{m} | r_i |
\end{equation}

\begin{equation}
\ell_2 (r) = \| r \|_2 = \sqrt{ \sum_{i = 1}^{m} | r_i |^2 }
\end{equation}

\begin{equation}
\ell_\infty (r) = \| r \|_\infty = \max_{i} | r_i |
\end{equation}

The linear programming formulation takes as its minimizing function the
residual, for which individual $x_i$ values form the decision
variables. Accordingly, each $\ell_p$ norm formulation is slightly
different. 

The $\ell_\infty$ norm utilizes a maximum-violation variable $\epsilon$, as shown
in Equation \ref{eqs:l-inf}.

%%% 
\begin{subequations}\label{eqs:l-inf}
  \begin{align}
    %%
    \min \:\: & 
    \epsilon  \\
    %%
    \text{s.t.} \:\: &
    - \epsilon \leq A_i x - b_i
    & \forall i \\
    %%
    &
    \epsilon \geq A_i x - b_i
    & \forall i \\
    %%
  \end{align}
\end{subequations}
%%% 

The $\ell_1$ norm uses a summation over all violations, as shown in
Equation \ref{eqs:l-1}, by using a proxy vector, $y$.

%%% 
\begin{subequations}\label{eqs:l-1}
  \begin{align}
    %%
    \min \:\: & 
    \sum_i y
    & \\
    %%
    \text{s.t.} \:\: &
    - y_i \leq A_i x - b_i
    & \forall i \\
    %%
    &
    y_i \geq A_i x - b_i
    & \forall i \\
    %%
  \end{align}
\end{subequations}
%%% 

The core difference between the two formulations is of course the norm used. The
$\ell_1$ norm in this case stores more information about the total violation,
whereas the $\ell_\infty$ norm is concerned with only the largest violating
constraint. These formulations are easily solved by the Simplex Method, as a
starting vertex can always be chosen by setting $x = 0$.

The $\ell_2$ norm is not as straightforward to solve. As shown
in \cite{ferris_linear_2008}, the actual norm can be replaced by its square,
because it is uniformly nonnegative. The resulting optimization problem is shown
in Equation \ref{eqs:l-2}.

%%% 
\begin{subequations}\label{eqs:l-2}
  \begin{align}
    %%
    \min \:\: & 
    y^\top y
    & \\
    %%
    \text{s.t.} \:\: &
    - y_i \leq A_i x - b_i
    & \forall i \\
    %%
    &
    y_i \geq A_i x - b_i
    & \forall i \\
    %%
  \end{align}
\end{subequations}
%%% 

As can be seen in its formulation in Equation \ref{eqs:l-2}, the objective
function is not a linear function of the decision variables. Accordingly, one
must tackle this problem using a different domain of mathematical programming,
namely \textit{quadratic programming}, the discussion of which is outside the
scope of this review due to the lack of its necessity to adequately solve the
Recipe Approximation Problem described in \S\ref{sec:rap}.
