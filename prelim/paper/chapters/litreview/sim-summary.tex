Chapter~\ref{ch:intro} introduced three major simulation-level design decisions
with some specific subcategories for each. Although it is not possible to
analyze these decisions for all reviewed simulators due to a lack of available
literature, it is possible to do so for three: CAFCA, COSI, and VISION. A
summary of such design decisions is provided in Table~\ref{tab:sim-summary}.

\begin{savenotes}
\begin{table} [h!]
\centering
\begin{tabular} {|c|c|c|c|c|}
\hline
Decision                     & Category & CAFCA & COSI & VISION \\ 
\hline
\multirow{2}{*}{Deployment}  & Determined by 
                             & user & user & user \\ \cline{2-5}
                             & Constrained by\footnote{Generally a lookahead function determines the associated metric.}   
                             & available material & not treated by literature & available material \\ \hline
\multirow{3}{*}{Fidelity}    & Decay 
                             & yes\footnote{Capability is available, but it is not used. See \S\ref{sec:cafca}.} & yes & yes \\ \cline{2-5}
                             & Reactor Physics 
                             & no & yes & no \\ \cline{2-5}
                             & Fuel Matching\footnote{How recycled fuel orders are ``matched'' with available isotopics.} 
                             & aggregate mass flows & equivalence method & aggregate mass flows \\ \hline
\multirow{2}{*}{Connections} & Static/Dynamic 
                             & static & static & static \\ \cline{2-5}
                             & Fleet/Individual 
                             & fleet & fleet & fleet \\
\hline
\end{tabular}
\caption{Simulation-Level Design Decisions as Taken by Each Simulator}
\label{tab:sim-summary}
\end{table}
\end{savenotes}

While most of the above information is not available for DANESS, it does
introduce additional functionality by allowing minimum-cost deployments,
constraining deployment based on a percentage of future required material (thus
allowing facilities to idle if not fueled), and treating facilities
individually, rather than explicitly as a fleet. It is not clear from the
literature if the facilities are only tracked individually or if they are
specifically interacted with on an individual basis.
