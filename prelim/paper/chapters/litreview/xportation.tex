\subsubsection{Overview}
There are many canonical problems described in the linear programming
literature. A subset of these are classified as network flow programs, which are
ideal for modeling the flows between entities that comprise a model's
formulation. In computational science's parlance, entities make up a set of
nodes in a graph, and the possible connections between those entities make up
the arcs of a graph. Accordingly, if flow can occur between some node $i$ and
some other node $j$, then it flows along arc $(i, j)$. In general there is a set
of nodes $N$ and a set of arcs $A$. Decision variables in network flow problems
determine the optimal flow between nodes and across arcs.

%% include picture

The actual formulation for a network flow problem as a linear program is, of
course, dependent on the problem being modeled. Take for example a maximum flow
problem. The objective of such a problem is to maximize the flow in a system,
i.e.,

\begin{equation}
\max \sum_{(i, j) \in A} x_{i,j}.
\end{equation}

The constraint matrix in such problems is denoted a \textit{node-arc incidence}
matrix. For a directed graph, an arc leaving node $i$ and entering arc $j$
indicates a value of 1 for element $a_{i,j}$ and a value of -1 for element
$a_{j,i}$. If an arc between two nodes does not exist, then a value of 0 is
assigned for the two corresponding coefficients. In general network flow
problems, there is also a notion of \textit{divergence}. Divergence is the total
amount of flow exiting or entering a node, and is modeled by the $b$ vector in
the matrix formulation of network flow problems. As is perhaps obvious, if $b_i
> 0$ for some node, $i$, then more flow exits the node than enters it and it is
termed a \textit{source node}. If $b_i < 0$, then more flow enters the node than
flows in, thus it is termed a \textit{sink node}. Finally, if $b_i = 0$, then
all material that flows into the node flows out as well, and it is termed a
\textit{transshipment node}. Finally, arcs can have their flow bounded, where a
lower bound for arc $(i, j)$ is given as $l_{i,j}$ and an upper bound is denoted
$u_{i,j}$.

One can now construct the LP formulation for a max-flow network flow problem.

%%% 
\begin{subequations}\label{eqs:max-flow}
  \begin{align}
    %%
    \max_{x} \:\: & 
    \sum_{(i, j) \in A} x_{i,j}
    & \label{eqs:max-flow_obj} \\
    %%
    \text{s.t.} \:\: &
    \sum_{j:(i,j) \in A} x_{i,j} - \sum_{i:(i,j) \in A} x_{i,j} = b_i
    & \forall i \in N \label{eqs:max-flow_sup} \\
    %%
    &
    l_{i,j} \leq x_{i,j} \leq u_{i,j}
    & \forall (i, j) \in A \label{eqs:max-flow_x}
    %%
  \end{align}
\end{subequations}
%%% 

\subsubsection{Transportation Problems}
Transportation problems are a subset of the network flow problems and can
therefore be modeled using linear programming. Transportation problems model the
flow of a commodity between source nodes and sink nodes, i.e. there are no
transshipment nodes in the general transportation problem. Critically, this
simplifications allows for the nodes in the transportation problem to be
categorized into two explicit groups, sources and sinks. In other words, source
nodes and sink nodes comprise two distinct subsets, $N_1, N_2$, the union of
which comprises all nodes in the transportation graph, $N$. These properties can
be described in set notation.

\begin{equation}
  N_1 \subset N
\end{equation}

\begin{equation}
 N_2 \subset N
\end{equation}

\begin{equation}
  N_1 \cup N_2 = N
\end{equation}

From the node-arc graph point of view, this strict subset division allows for
the transportation problem to be modeled as a \textit{bipartite} graph.

%% include picture

The transportation problem is a general platform on which one can model more
general problems. Previously, a maximum-flow problem was described, but the
transportation problem lends itself more generally to the minimum-cost network
problems. In such a formulation, each arc has an associated \textit{unit cost}
associated with the cost of transporting a unit of a commodity along it,
$c_{i,j}$. Additionally, instead of the notion of divergence, supplier and
consumer nodes have an associated supply, $s_i$, or demand, $d_i$, which provide
a notion of \textit{node capacity} rather than arc capacity.

Accordingly, the minimum-cost transportation problem can be formulated as a
linear program in the following manner.

%%% 
\begin{subequations}\label{eqs:xport}
  \begin{align}
    %%
    \min_{x_{i,j}} \:\: & 
    \sum_{(i, j) \in A \subset N_1 \times N_2} c_{i,j} x_{i,j}
    & \label{eqs:xport_obj} \\
    %%
    \text{s.t.} \:\: &
    \sum_{j \in N_2} x_{i,j} \leq s_i
    & \forall i \in N_1  \\
    %%
    &
    \sum_{i \in N_1} x_{i,j} \geq d_i
    & \forall j \in N_2  \\
    %%
    &
    x_{i,j} \geq 0
    & \forall i \in N_1, \: \forall j \in N_2 \label{eqs:xport_x}
    %%
  \end{align}
\end{subequations}
%%% 

As is noted in most every text on the Transportation Problem Linear Program
formulation, an intuitive constraint on the problem to guarantee a feasible
solution is that the total demand in the system must be no greater than the
total supply in the system.

\begin{equation}
  \sum_{j \in N_2} d_j \leq \sum_{i \in N_1} s_i
\end{equation}

Feasibility in this sense can be guaranteed, however, by adding an artificial
supply node. Such a node can have infinite supply capacity but at (effectively)
infinite cost. The problem can then be solved, and any flow leaving the
artificial node in the optimal solution can be dealt with accordingly, e.g. it
can be ignored.

\subsubsection{Multi-Commodity Transportation Problems}
The previously-described transportation problem can more precisely be named the
single-commodity transportation problem. It deals with the flow from sources to
sinks of a single commodity. A more complex model includes more than one
commodity.

Variables and constants in the multi-commodity formulation are generally analogs
of their counterparts in the single-commodity problem. There is a unit cost for
commodity $h$ to traverse arc $(i,j)$ denoted as $c_{i,j}^{h}$. A supplier of
commodity $h$ has a certain supply capacity $s_i^h$ which cannot be surpassed
and demanders of commodity $h$ have a certain demand level which must be met,
$d_i^h$.

%% include picture

In the simplest extension from the single-commodity to multi-commodity
transportation problem, arc constraints for all commodities are combined, i.e.,
for a given arc $(i, j)$, there is a single capacity $u_{i,j}$. A classic
interpretation of this enhanced complexity deals with data networks. Multiple
classifications of data exist, but they all must traverse the same network
infrastructure. Accordingly, the infrastructure can only accommodate a certain
level of total flow among all communication types.

This coupling of commodities via arc capacities changes the structure of the
node-arc incidence matrix. Arc capacity constraints no longer have a single
entry. Instead, many flows contribute to the capacity constraint, requiring
different methods to solve the problem. Of course, the Simplex Method can still
solve the linear program, however other potential reductions in time and
complexity that are available to the single-commodity flow problem are not
available to the multi-commodity flow problem due to this coupling.

The formulation of the multi-commodity flow problem is shown in Equation
\ref{eqs:MCTP}. Note the commodity coupling in Equation \ref{eqs:MCTP_cap}.

%%% 
\begin{subequations}\label{eqs:MCTP}
  \begin{align}
    %%
    \min_{x_{i,j}^{h}} \:\: & 
    \sum_{i \in I}\sum_{j \in J}\sum_{h \in H} c_{i,j}^{h} x_{i,j}^{h}
    & \label{eqs:MCTP_obj} \\
    %%
    \text{s.t.} \:\: &
    \sum_{j \in J} x_{i,j}^{h} \leq s_{i}^{h}
    &
    \forall \: i \in I, \forall \: h \in H \label{eqs:MCTP_sup} \\
    %%
    &
    \sum_{i \in I} x_{i,j}^{h} \geq d_{j}^{h}
    & 
    \forall \: j \in J, \forall \: h \in H \label{eqs:MCTP_dem} \\
    %%
    &
    \sum_{h \in H} x_{i,j}^{h} \leq u_{i,j}
    & 
    \forall \: j \in J \label{eqs:MCTP_cap} \\
    %%
    &
    x_{i,j}^{k} \geq 0
    &
    \forall \: i \in I, \forall \: j \in J, \forall \: h \in H \label{eqs:MCTP_x}
    %%
  \end{align}
\end{subequations}
%%% 

A number of important reductions are possible for the multi-commodity
transportation problem. The most important of all is when there is no arc that
shares multiple commodities. In such a case, the single multi-commodity problem
can be broken into $m$ different single-commodity transportation problems where
$m$ is the cardinality of the set of commodities, $H$.
