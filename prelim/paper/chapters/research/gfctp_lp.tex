
The Generic Fuel Cycle Transportation Problem with Side Constraints (GFCTP-LP)
describes a multi-commodity setting in which demand can be met by multiple
commodities. Consumers denote a cost preference over the possible commodities
they consume and a demand for the set of commodities that must be met. Suppliers
denote one or more production capacities for a given commodity which serve as
the set of supply capacities analgous to the normal MTP
(see \S\ref{sec:MTP}). The GFCTP-LP formulation is as follows:

%%% 
\begin{subequations}\label{eqs:GFCTP-LP}
  \begin{align}
    %%
    \min_{z} \:\: & 
    z = \sum_{h \in H}\sum_{i \in I}\sum_{j \in J}c_{i,j}(h) x_{i,j}(h) 
    & \label{eq:GFCTP-LP_obj} \\
    %%
    \text{s.t.} \:\: &
    \sum_{j \in J}\beta_{i,k}(q_{j}(h)) x_{i,j}(h) \leq s_{i,k} 
    &
    \: \forall \: k \in K_{i}(h),  
    \forall \: i \in I, \forall \: h \in H \label{eq:GFCTP-LP_sup} \\
    %%
    &
    \sum_{i \in I}\sum_{h \in H_{j}} x_{i,j}(h) \geq d_{j}(H_{j}) 
    & 
    \forall \: j \in J \label{eq:GFCTP-LP_dem} \\
    %%
    &
    x_{i,j} \geq 0
    &
    \forall \: x \in X \label{eq:GFCTP-LP_x}
    %%
  \end{align}
\end{subequations}
%%% 

The sets and variables involved are described in Tables \ref{tbl:GFCTP-LP-sets}
and \ref{tbl:GFCTP-LP-vars}.

%%% 
\begin{table} [h!]
\centering
\begin{tabularx}{\textwidth-20pt}{|c|X|} % line wraps second column if too long
\hline
Set         & Description \\
\hline
$H$         & all commodities  \\
$I$         & all producers  \\
$J$         & all consumers  \\
$X$         & the feasible set of flows between producers and consumers  \\
$K_{i}(h)$  & the set of constraining capacities for 
            producer $i$ of commodity $h$  \\
$H_{j}$     & the set of satifying commodities for consumer $j$  \\
\hline
\end{tabularx}
\caption{Sets Appearing in the GFCTP-LP Formulation}
\label{tbl:GFCTP-LP-sets}
\end{table}
%%% 

%%% 
\begin{table} [h!]
\centering
\begin{tabularx}{\textwidth-20pt}{|c|X|} % line wraps second column if too long
\hline
Variable    & Description \\
\hline
$c_{i,j}(h)$             & the unit cost of commodity $h$ 
                         for producer $i$ and consumer $j$  \\
$x_{i,j}(h)$             & a decision variable, the flow of commodity $h$ 
                         for producer $i$ and consumer $j$  \\
$q_{j}(h)$               & the requested quality of commodity $h$ 
                         by consumer $j$  \\
$\beta_{i,k}(q_{j}(h))$  & a capacity translation function for capacity 
                         constraint $k$ of producer $i$ given $q_{j}(h)$ \\
$s_{i,k}$                & a supply capacity of producer $i$ corresponding to 
                         capacity constraint $k$ \\
$d_{j}(H_{j})$           & the total demand of consumer $j$ over the set of 
                         satisfying commidities $H_{j}$ \\
\hline
\end{tabularx}
\caption{Variables Appearing in the GFCTP-LP Formulation}
\label{tbl:GFCTP-LP-vars}
\end{table}
%%% 

This formulation deviates from the normal MTP formulation via the expansion of
capacity constraints (Equation \ref{eq:GFCTP-LP_sup}) and the inclusion of a
constraint allowing multiple commodities that are able to meet the demand of a
producer (Equation \ref{eq:GFCTP-LP_dem}). The former constraint maintains the
multicommodity nature of the formulation. This leads to an important insight: if
Equation \ref{eqs:1demand} holds,

\begin{equation}\label{eqs:1demand}
  \left|{H_{j}}\right| = 1 \: \forall \: j \in J
\end{equation}

then the GFCTP-LP can be transformed into a separable multicommodity
transportation problem as shown in \cite{bertsekas_network_1998}. If the problem
is separable, then the Transportation Problem Simplex Method shown in
\S\ref{sec:trans-simplex} can be applied to a series of smaller subproblems,
reducing overall complexity. Furthermore, if Equations \ref{eqs:1demand} and
\ref{eqs:1constraint} both hold,

\begin{equation}\label{eqs:1constraint}
  \left|{K_{i}(h)}\right| = 1 \: \forall \: i \in I, \: \forall \: h \in H
\end{equation}

then the GFCTP-LP is in fact the a normal Transportation Problem, because the
quality translation function ($\beta_{i,k}(q_{j}(h))$) translates to a constant
at solution time.

\paragraph{Capacity Translation Function and Constraint Example}~\\

The notion of a capacity translation function is something that I have
introduced out of necessity due to the complexity of the GFCTP. Accordingly, an
example will help clarify its purpose. I'll use this time to also to provide an
example of a producer with multiple capacity constraints for a given commodity.

Take, for example, an enrichment facility. Such a facility produces the
commodity ``Enriched Uranium''. This facility has two constraints on its
operation for any given time period: the amount of Separative Work Units (SWU)
that it can process and the total uranium feed it has on hand. Note that neither
of these capacities are measure directly in the units of the commodity it
produces, i.e., kilograms of enriched uranium. We can now state the set the
values for $K_{i}(h)$ for this facility:

\begin{equation}\label{eqs:enr-constr-commods}
  K_{enr}(\mbox{Enriched Uranium}) = \{ \mbox{SWU}, \mbox{Natural Uranium} \}
\end{equation}

Let us now consider that there is a set of requests for enriched uranium that
this facility can possibly meet. Such requests have, in general, two parameters:
$P_{j}$, the total product quantity (in kilograms), and $\varepsilon_{j}$, the
product enrichment (in w/o U-235). To maintain the notation that has previously
been used in \S\ref{sec:prev-enrich}, I will denote $\varepsilon_{j}$ as
$x_{p,j}$ for the remainder of this section. Note that we have provided the
following definition:

\begin{equation}\label{eqs:enr-q-swu}
  q_{j}(\mbox{Enriched Uranium}) \equiv x_{p,j}
\end{equation}

These values are set during a prior phase of the overall matching algorithm, and
can therefore be considered constants. Further, let us note that, in general, an
enrichment facility's operation, or rather its capacity thereof, is goverened by
two parameters: $x_{f,enr}$, the fraction of U-235 in its feed material, and
$x_{t,enr}$, the fraction of U-235 in its tails material. Let us assume both of
these are constants of the facility. Utilizing the equations presented in
\S\ref{sec:prev-enrich}, we can denote the functional forms of this facility's
two capacity constraints.

\begin{align}\label{eqs:enr-swu-constr}
\begin{split}
\beta_{enr,SWU}(x_{p,j}) & = V(x_{p,j}) \\
& + \frac{x_{p,j} - x_{t,enr}}{x_{f,enr} - x_{t,enr}} V(x_{t,enr}) \\
& - \frac{ x_{p,j} - x_{f,enr} } { x_{f,enr} - x_{t,enr} } V(x_{f,enr})
\end{split}
\end{align}

Where $V(x)$ is the value function of Equation \ref{eqs:enr-value}.