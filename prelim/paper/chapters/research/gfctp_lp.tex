The linear program formulation of the GFCTP involves a set of players that act
in an overarching market. The players are the supplier and consumer facilities
involved in the simulation, i.e., the reactors, fabrication facilities,
repositories, etc. In strict mathematical programming parlance, the GFCTP can be
described as a Multicommodity Transportation Problem with Side
Constraints, but there are a number of departures from the classical MTP.

To begin, the multicommodity aspect of the problem is not manifest on arc
capacities. Instead, facility demand constraints, whose equivalent in the MTP
are Equation \ref{eqs:MCTP_dem}, incorporate a set of satisfactory commodities,
rather than a single satisfactory commodity. For example, a reactor may be able
to accept UOX or MOX fuel, but has a demand for total fuel. Additionally,
supplier facilities may have a set of constraints on their ability to supply a
given commodity and they may not be able to directly express those constraints
with the unit of the commodity market, which is generally kilograms of a
commodity. Take an enrichment facility for example. Such a facility has
nominally two constraints: SWU capacity and natural uranium capacity. The former
constraint is a processing constraint, and the latter is an inventory
constraint; however, both are necessary to fully define the
problem. Furthermore, let us note that the output of this facility is kilograms
of enriched uranium. Accordingly, the above capacities must be translated into
units of this output. Because multiple commodities can satisfy demand, consumers
denote a cost preference over the possible commodities they consume. Suppliers
denote one or more production capacities for a given commodity which serve as
the set of supply capacities analogous to the normal MCTP
(Equation \ref{eqs:MCTP_sup}). The GFCTP-LP formulation is as follows:

%%% 
\begin{subequations}\label{eqs:GFCTP-LP}
  \begin{align}
    %%
    \min_{z} \:\: & 
    z = \sum_{h \in H}\sum_{i \in I}\sum_{j \in J}c_{i,j}^{h} x_{i,j}^{h} 
    & \label{eqs:GFCTP-LP_obj} \\
    %%
    \text{s.t.} \:\: &
    \sum_{j \in J}\beta_{i,k}(q_{j}^{h}) x_{i,j}^{h} \leq s_{i,k} 
    &
    \: \forall \: k \in K_{i}^{h},  
    \forall \: i \in I, \forall \: h \in H \label{eqs:GFCTP-LP_sup} \\
    %%
    &
    \sum_{i \in I}\sum_{h \in H_{j}} x_{i,j}^{h} \geq d_{j}(H_{j}) 
    & 
    \forall \: j \in J \label{eqs:GFCTP-LP_dem} \\
    %%
    &
    x^h_{i,j} \geq 0
    &
    \forall \: x \in X \label{eqs:GFCTP-LP_x}
    %%
  \end{align}
\end{subequations}
%%% 

The sets and variables involved are described in Tables \ref{tbl:GFCTP-LP-sets}
and \ref{tbl:GFCTP-LP-vars}. Note that $H_{j}$ is a subset of the commodities:

\begin{equation}
  H_{j} \subseteq H \: \forall \: j \in J
\end{equation}

%%% 
\begin{table} [h!]
\centering
\begin{tabularx}{\columnwidth-10pt}{|c|X|} % line wraps second column if too long
\hline
Set         & Description \\
\hline
$H$         & all commodities  \\
$I$         & all producers  \\
$J$         & all consumers  \\
$X$         & the feasible set of flows between producers and consumers  \\
$K_{i}^{h}$  & the set of constraining capacities for 
            producer $i$ of commodity $h$  \\
$H_{j}$     & the set of satisfying commodities for consumer $j$  \\
\hline
\end{tabularx}
\caption{Sets Appearing in the GFCTP-LP Formulation}
\label{tbl:GFCTP-LP-sets}
\end{table}
%%% 

%%% 
\begin{table} [h!]
\centering
\begin{tabularx}{\columnwidth-10pt}{|c|X|} % line wraps second column if too long
\hline
Variable    & Description \\
\hline
$c_{i,j}^{h}$             & the unit cost of commodity $h$ 
                          for producer $i$ and consumer $j$  \\
$x_{i,j}^{h}$             & a decision variable, the flow of commodity $h$ 
                          for producer $i$ and consumer $j$  \\
$q_{j}^{h}$               & the requested quality of commodity $h$ 
                          by consumer $j$  \\
$\beta_{i,k}(q_{j}^{h})$  & a capacity translation function for capacity 
                          constraint $k$ of producer $i$ given $q_{j}^{h}$ \\
$s_{i,k}^{h}$             & a supply capacity of producer $i$ corresponding to 
                          capacity constraint $k$ of commodity $h$ \\
$d_{j}(H_{j})$            & the total demand of consumer $j$ over the set of 
                          satisfying commodities $H_{j}$ \\
\hline
\end{tabularx}
\caption{Variables Appearing in the GFCTP-LP Formulation}
\label{tbl:GFCTP-LP-vars}
\end{table}
%%%

This formulation deviates from the normal MCTP formulation via the expansion of
capacity constraints (Equation \ref{eqs:GFCTP-LP_sup}) and the inclusion of a
constraint allowing multiple commodities that are able to meet the demand of a
producer (Equation \ref{eqs:GFCTP-LP_dem}). The former constraint maintains the
multi-commodity nature of the formulation. 

Under certain conditions, the GFCTP-LP will result in a simpler problem. The
first possible condition is that each consumer could have its demand met by only
one commodity, i.e.,

\begin{equation}\label{eqs:1demand}
  \left|{H_{j}}\right| = 1 \: \forall \: j \in J.
\end{equation}

In such a situation, the GFCTP-LP can be transformed into an analog of the
separable transportation problem as shown in \cite{bertsekas_network_1998}. Such
a condition will effectively allow one to solve $N$ different instances of a
single-commodity problem, where $N$ is the cardinality of $H$. 

The second simplifying condition is if the constraining capacity set has a
cardinality of unity, i.e., 

\begin{equation}\label{eqs:1constraint}
  \left|{K_{i}^{h}}\right| = 1 \: \forall \: i \in I, \: \forall \: h \in H.
\end{equation}

If both Equation \ref{eqs:1constraint} and \ref{eqs:1demand} hold, then the
GFCTP-LP is in fact the a normal Transportation Problem, because the quality
translation function ($\beta_{i,k}(q_{j}^{h})$) translates to a constant at
solution time. 

These simplifications are important to the computation time required to solve
the resulting problem instance. The general solution technique for LPs is the
Simplex Method, as previously described. Klee and Minty show that in the worst
case, the Simplex Method will execute in exponential time \cite{klee_good_1970},
but in practice it is generally considered very computationally efficient. If
the problem can be simplified to a TP, then the Transportation Simplex Method
can be used \cite{ahuja_network_1993}.

\paragraph{Capacity Translation Function and Constraints Example}

The notion of a capacity translation function is something that has been
introduced out of necessity due to the complexity of the GFCTP. Accordingly, an
example will help clarify its purpose. This time can also be used to provide an
example of a producer with multiple capacity constraints for a given commodity.

Take, for example, an enrichment facility. Such a facility produces the
commodity enriched uranium (EU). This facility has two constraints on its
operation for any given time period: the amount of Separative Work Units (SWU)
that it can process, $s_{enr,SWU}$ and the total natural uranium (NU) feed it
has on hand, $s_{enr,NU}$. Note that neither of these capacities are measure
directly in the units of the commodity it produces, i.e., kilograms of enriched
uranium (EU). The set of values for $K_{i}^{h}$ for this facility are:

\begin{equation}\label{eqs:enr-constr-commods}
  K_{enr}^{EU} = \{ \mbox{SWU}, \mbox{NU} \}
\end{equation}

Consider a set of requests for enriched uranium that this facility can possibly
meet. Such requests have, in general, two parameters: $P_{j}$, the total product
quantity (in kilograms), and $\varepsilon_{j}$, the product enrichment (in w/o
\nucl{235}{U}).\footnote{The notation for enrichment, $\varepsilon_{j}$, is chosen over its
normal form, $x_p$, to limit confusion with the LP notation of material flow,
$x^h_{i,j}$.}  For the purposes of this constraint set, the quality of material
in question is its enrichment, i.e.,

\begin{equation}\label{eqs:enr-q-swu}
  q_{j}^{EU} \equiv \varepsilon_{j}.
\end{equation}

These values are set during a prior phase of the overall matching algorithm, and
can therefore be considered constant. Further, let us note that, in general, an
enrichment facility's operation, or rather its capacity, is governed by two
parameters: $\varepsilon_{f,enr}$, the fraction of \nucl{235}{U} in its feed material, and
$\varepsilon_{t,enr}$, the fraction of \nucl{235}{U} in its tails material. These parameters
determine the amount of SWU required to produce some amount of enriched uranium:

\begin{align}
\begin{split}
\label{eqs:swu}
SWU = & \:\: P ( V(\varepsilon_{j}) 
      + \frac{\varepsilon_{j} - \varepsilon_{f,enr}}
               {\varepsilon_{f,enr} - \varepsilon_{t,enr}} V(\varepsilon_{t,enr}) \\
      & - \frac{\varepsilon_{j} - \varepsilon_{t,enr}}
               {\varepsilon_{f,enr} - \varepsilon_{t,enr}} V(\varepsilon_{f,enr}) )
\end{split}
\end{align}

$P$ in Equation \ref{eqs:swu} is the amount of produced enriched uranium, and
$V(x)$ is the value function,

\begin{equation}\label{eqs:value}
  V(x) = (1-2x) \ln \left(\frac{1-x}{x}\right)
\end{equation}

Utilizing the above equations, one can denote the functional forms of the
arguments of this facility's two capacity constraints.

\begin{align}
\label{eqs:enr-prod-beta}
\beta_{enr,NU}(\varepsilon_{j}) = & \:\: \frac{\varepsilon_{j} - \varepsilon_{t,enr}}
                                      {\varepsilon_{f,enr} - \varepsilon_{t,enr}} \\
\begin{split}
\label{eqs:enr-swu-beta}
\beta_{enr,SWU}(\varepsilon_{j}) = & \:\: V(\varepsilon_{j}) \\
                         & + \frac{\varepsilon_{j} - \varepsilon_{f,enr}}
                                  {\varepsilon_{f,enr} - \varepsilon_{t,enr}} V(\varepsilon_{t,enr}) \\
                         & - \frac{\varepsilon_{j} - \varepsilon_{t,enr}}
                                  {\varepsilon_{f,enr} - \varepsilon_{t,enr}} V(\varepsilon_{f,enr})
\end{split}
\end{align}

These constraints correspond to the per-unit requirements for enriched uranium
of natural uranium feed and SWU. Finally, we can form the set of constraint
equations for the enrichment facility by combining
Equations \ref{eqs:GFCTP-LP_sup}, \ref{eqs:enr-q-swu},
\ref{eqs:enr-prod-beta}, and \ref{eqs:enr-swu-beta}.

\begin{align}
\label{eqs:enr-prod-constr}
\sum_{j \in J}\beta_{enr,NU}(\varepsilon_{j}) \: x_{enr,j}^{EU}  & \leq s_{enr,NU} \\
\label{eqs:enr-swu-constr}
\sum_{j \in J}\beta_{enr,SWU}(\varepsilon_{j}) \: x_{enr,j}^{EU} & \leq s_{enr,SWU}
\end{align}

\paragraph{Satisfying Commodity Set Example}

The other departure the GFCTP-LP takes from the normal MCTP formulation is the
location of its multicommodity dependence. As presented above, the
MCTP formulation includes a multicommodity arc capacity constraint, Equation
\ref{eqs:GFCTP-LP_dem}. This constraint models a
situation in which different commodities can satisfy a consumer's demand. There
is no direct analog in the GFCTP, i.e., transportation arcs are assumed separate
for separate commodities.

Take the enrichment facility example, expanding on the previous discussion. Note
that an enrichment facility takes feed uranium and then enriches its \nucl{235}{U}
content. This feed uranium can come from different sources which have different
feed enrichments. In practice, the most likely sources of feed uranium are
natural uranium (NU) or recycled uranium (RU), a product of reprocessing light
water reactor fuel. Recycled uranium may be advantageous to use if it has a
higher weight percent of \nucl{235}{U} than does natural uranium. We can now state the
set the values for $H_{j}$ for this facility:

\begin{equation}\label{eqs:enr-dem-commods}
  H_{enr} = \{ \mbox{NU}, \mbox{RU} \}
\end{equation}

Of course, the facility must define some preference function over the set of
satisfying commodities. In this example, recycled uranium is more valuable
because of its higher \nucl{235}{U} content, which translates into a (relatively large)
SWU reduction in order to meet identical enrichment requests. This preference
ordering is encapsulated in the cost function in Equation
\ref{eqs:GFCTP-LP_obj}. The nature of the cost function in the \Cyclus
simulation environment is nontrivial and explained further in this section.
