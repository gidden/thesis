
The Generic Fuel Cycle Transportation Problem with Side Constraints (GFCTP-LP)
describes a multi-commodity setting in which demand can be met by multiple
commodities. Consumers denote a cost preference over the possible commodities
they consume and a demand for the set of commodities that must be met. Suppliers
denote one or more production capacities for a given commodity which serve as
the set of supply capacities analogous to the normal MCTP
(see \S\ref{sec:MCTP}). The GFCTP-LP formulation is as follows:

%%% 
\begin{subequations}\label{eqs:GFCTP-LP}
  \begin{align}
    %%
    \min_{z} \:\: & 
    z = \sum_{h \in H}\sum_{i \in I}\sum_{j \in J}c_{i,j}(h) x_{i,j}(h) 
    & \label{eqs:GFCTP-LP_obj} \\
    %%
    \text{s.t.} \:\: &
    \sum_{j \in J}\beta_{i,k}(q_{j}(h)) x_{i,j}(h) \leq s_{i,k} 
    &
    \: \forall \: k \in K_{i}(h),  
    \forall \: i \in I, \forall \: h \in H \label{eqs:GFCTP-LP_sup} \\
    %%
    &
    \sum_{i \in I}\sum_{h \in H_{j}} x_{i,j}(h) \geq d_{j}(H_{j}) 
    & 
    \forall \: j \in J \label{eqs:GFCTP-LP_dem} \\
    %%
    &
    x_{i,j} \geq 0
    &
    \forall \: x \in X \label{eqs:GFCTP-LP_x}
    %%
  \end{align}
\end{subequations}
%%% 

The sets and variables involved are described in Tables \ref{tbl:GFCTP-LP-sets}
and \ref{tbl:GFCTP-LP-vars}. Note that $H_{j}$ is a subset of the commodities:

\begin{equation}
  H_{j} \subseteq H \: \forall \: j \in J
\end{equation}

%%% 
\begin{table} [h!]
\centering
\begin{tabularx}{\textwidth-20pt}{|c|X|} % line wraps second column if too long
\hline
Set         & Description \\
\hline
$H$         & all commodities  \\
$I$         & all producers  \\
$J$         & all consumers  \\
$X$         & the feasible set of flows between producers and consumers  \\
$K_{i}(h)$  & the set of constraining capacities for 
            producer $i$ of commodity $h$  \\
$H_{j}$     & the set of satisfying commodities for consumer $j$  \\
\hline
\end{tabularx}
\caption{Sets Appearing in the GFCTP-LP Formulation}
\label{tbl:GFCTP-LP-sets}
\end{table}
%%% 

%%% 
\begin{table} [h!]
\centering
\begin{tabularx}{\textwidth-20pt}{|c|X|} % line wraps second column if too long
\hline
Variable    & Description \\
\hline
$c_{i,j}(h)$             & the unit cost of commodity $h$ 
                         for producer $i$ and consumer $j$  \\
$x_{i,j}(h)$             & a decision variable, the flow of commodity $h$ 
                         for producer $i$ and consumer $j$  \\
$q_{j}(h)$               & the requested quality of commodity $h$ 
                         by consumer $j$  \\
$\beta_{i,k}(q_{j}(h))$  & a capacity translation function for capacity 
                         constraint $k$ of producer $i$ given $q_{j}(h)$ \\
$s_{i,k}$                & a supply capacity of producer $i$ corresponding to 
                         capacity constraint $k$ \\
$d_{j}(H_{j})$           & the total demand of consumer $j$ over the set of 
                         satisfying commodities $H_{j}$ \\
\hline
\end{tabularx}
\caption{Variables Appearing in the GFCTP-LP Formulation}
\label{tbl:GFCTP-LP-vars}
\end{table}
%%%

This formulation deviates from the normal MCTP formulation via the expansion of
capacity constraints (Equation \ref{eqs:GFCTP-LP_sup}) and the inclusion of a
constraint allowing multiple commodities that are able to meet the demand of a
producer (Equation \ref{eqs:GFCTP-LP_dem}). The former constraint maintains the
multi-commodity nature of the formulation. This leads to an important insight: if
Equation \ref{eqs:1demand} holds,

\begin{equation}\label{eqs:1demand}
  \left|{H_{j}}\right| = 1 \: \forall \: j \in J
\end{equation}

then the GFCTP-LP can be transformed into a separable multi-commodity
transportation problem as shown in \cite{bertsekas_network_1998}. If the problem
is separable, then the Transportation Problem Simplex Method shown in
\S\ref{sec:trans-simplex} can be applied to a series of smaller subproblems,
reducing overall complexity. Furthermore, if Equations \ref{eqs:1demand} and
\ref{eqs:1constraint} both hold,

\begin{equation}\label{eqs:1constraint}
  \left|{K_{i}(h)}\right| = 1 \: \forall \: i \in I, \: \forall \: h \in H
\end{equation}

then the GFCTP-LP is in fact the a normal Transportation Problem, because the
quality translation function ($\beta_{i,k}(q_{j}(h))$) translates to a constant
at solution time.

\paragraph{Capacity Translation Function and Constraints Example}~\\

The notion of a capacity translation function is something that I have
introduced out of necessity due to the complexity of the GFCTP. Accordingly, an
example will help clarify its purpose. I'll use this time to also to provide an
example of a producer with multiple capacity constraints for a given commodity.

Take, for example, an enrichment facility. Such a facility produces the
commodity ``Enriched Uranium''. This facility has two constraints on its
operation for any given time period: the amount of Separative Work Units (SWU)
that it can process, $s_{enr,SWU}$ and the total natural uranium (NU) feed it
has on hand, $s_{enr,NU}$. Note that neither of these capacities are measure
directly in the units of the commodity it produces, i.e., kilograms of enriched
uranium (EU). We can now state the set the values for $K_{i}(h)$ for this
facility:

\begin{equation}\label{eqs:enr-constr-commods}
  K_{enr}(\mbox{EU}) = \{ \mbox{SWU}, \mbox{NU} \}
\end{equation}

Let us now consider that there is a set of requests for enriched uranium that
this facility can possibly meet. Such requests have, in general, two parameters:
$P_{j}$, the total product quantity (in kilograms), and $\varepsilon_{j}$, the
product enrichment (in w/o U-235). Although it would be consistent to use the
notation that has previously been used in \S\ref{sec:prev-enrich}, I will denote
the product weight fraction as $\varepsilon_{j}$ rather than $x_{p,j}$ to reduce
confusion with the similar notation of commodity flow, $x_{i,j}$. In any case,
notice that we have provided the following definition:

\begin{equation}\label{eqs:enr-q-swu}
  q_{j}(\mbox{EU}) \equiv \varepsilon_{j}
\end{equation}

These values are set during a prior phase of the overall matching algorithm, and
can therefore be considered constants. Further, let us note that, in general, an
enrichment facility's operation, or rather its capacity thereof, is governed by
two parameters: $x_{f,enr}$, the fraction of U-235 in its feed material, and
$x_{t,enr}$, the fraction of U-235 in its tails material. Let us assume both of
these are constants of the facility. Utilizing the equations presented in
\S\ref{sec:prev-enrich}, we can denote the functional forms of the arguments of 
this facility's two capacity constraints.

\begin{align}
\label{eqs:enr-prod-beta}
\beta_{enr,NU}(\varepsilon_{j}) = & \:\: \frac{\varepsilon_{j} - x_{t,enr}}
                                      {x_{f,enr} - x_{t,enr}} \\
\begin{split}
\label{eqs:enr-swu-beta}
\beta_{enr,SWU}(\varepsilon_{j}) = & \:\: V(\varepsilon_{j}) \\
                         & + \frac{\varepsilon_{j} - x_{t,enr}}
                                  {x_{f,enr} - x_{t,enr}} V(x_{t,enr}) \\
                         & - \frac{\varepsilon_{j} - x_{f,enr}}
                                  {x_{f,enr} - x_{t,enr}} V(x_{f,enr})
\end{split}
\end{align}

where $V(x)$ is the value function described in Equation
\ref{eqs:enr-value}. These constraints correspond to the per-unit requirements
for enriched uranium of natural uranium feed shown in Equation
\ref{eqs:enr-feed} and SWU shown in Equation \ref{eqs:enr-swu-p}. Finally, we
can form the set of constraint equations for the enrichment facility by
combining Equations \ref{eqs:GFCTP-LP_sup}, \ref{eqs:enr-q-swu},
\ref{eqs:enr-prod-beta}, and \ref{eqs:enr-swu-beta}.

\begin{align}
\label{eqs:enr-prod-constr}
\sum_{j \in J}\beta_{enr,NU}(\varepsilon_{j}) \: x_{enr,j}(\mbox{EU})  & \leq s_{enr,NU} \\
\label{eqs:enr-swu-constr}
\sum_{j \in J}\beta_{enr,SWU}(\varepsilon_{j}) \: x_{enr,j}(\mbox{EU}) & \leq s_{enr,SWU}
\end{align}

\paragraph{Satisfying Commodity Set Example}~\\

The other departure the GFCTP-LP takes from the normal MCTP formulation is the
location of its multicommodity dependence. As presented in \S\ref{sec:MCTP}, the
MCTP formulation includes a multicommodity arc capacity constraint, Equation
\ref{eqs:MCTP_cap}. There is no direct analog in the GFCTP, i.e., transportation
arcs are assumed separate for separate commodities. There is still a notion of
multicommodity dependence, however, via Equation \ref{eqs:GFCTP-LP_dem}. This
constraint models a situation in which different commodities can satisfy a
consumer's demand.

Let us again use the enrichment facility example, expanding on the previous
discussion. Note that an enrichment facility takes feed uranium and then
enriches its U-235 content. This feed uranium can come from different sources
which have different feed enrichments. The most likely case in practice of
multiple sources of feed uranium for such a facility involves the recycling of
uranium where an enrichment facility can use either natural uranium (NU) or
recycled uranium (RU), which has a higher weight percent of U-235 than does
natural uranium. We can now state the set the values for $H_{j}$ for this
facility:

\begin{equation}\label{eqs:enr-dem-commods}
  H_{enr} = \{ \mbox{NU}, \mbox{RU} \}
\end{equation}

Of course, the facility must define some preference function over the set of
satisfying commodities. In this example, recycled uranium is more valuable
because of its higher U-235 content, which translates into a (relatively large)
SWU reduction in order to meet identical enrichment requests. This preference
ordering is encapsulated in the cost function in Equation
\ref{eqs:GFCTP-LP_obj}. The nature of the cost function in the \Cyclus
simulation environment is nontrivial and explained further in
\S\ref{sec:cost-function}.
