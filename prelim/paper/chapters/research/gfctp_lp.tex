
The Generic Fuel Cycle Transportation Problem with Side Constraints (GFCTP-LP)
describes a multi-commodity setting in which demand can be met by multiple
commodities. Consumers denote a cost preference over the possible commodities
they consume and a demand for the set of commodities that must be met. Suppliers
denote one or more production capacities for a given commodity which serve as
the set of supply capacities analgous to the normal MTP
(see \S\ref{sec:MTP}). The GFCTP-LP formulation is as follows:

%%% 
\begin{subequations}\label{eqs:GFCTP-LP}
  \begin{align}
    %%
    \min_{z} \:\: & 
    z = \sum_{h \in H}\sum_{i \in I}\sum_{j \in J}c_{i,j}(h) x_{i,j}(h) 
    & \label{eq:GFCTP-LP_obj} \\
    %%
    \text{s.t.} \:\: &
    \sum_{j \in J}\beta_{i,k}(q_{j}(h)) x_{i,j}(h) \leq s_{i,k} 
    &
    \: \forall \: k \in K_{i}(h),  
    \forall \: i \in I, \forall \: h \in H \label{eq:GFCTP-LP_sup} \\
    %%
    &
    \sum_{i \in I}\sum_{h \in H_{j}} x_{i,j}(h) \geq d_{j}(H_{j}) 
    & 
    \forall \: j \in J \label{eq:GFCTP-LP_dem} \\
    %%
    &
    x_{i,j} \geq 0
    &
    \forall \: x \in X \label{eq:GFCTP-LP_x}
    %%
  \end{align}
\end{subequations}
%%% 

The sets and variables involved are described in Tables \ref{tbl:GFCTP-LP-sets}
and \ref{tbl:GFCTP-LP-vars}.

%%% 
\begin{table} [h!]
\centering
\begin{tabularx}{\textwidth-20pt}{|c|X|} % line wraps second column if too long
\hline
Set         & Description \\
\hline
$H$         & all commodities  \\
$I$         & all producers  \\
$J$         & all consumers  \\
$X$         & the feasible set of flows between producers and consumers  \\
$K_{i}(h)$  & the set of constraining capacities for 
            producer $i$ of commodity $h$  \\
$H_{j}$     & the set of satifying commodities for consumer $j$  \\
\hline
\end{tabularx}
\caption{Sets Appearing in the GFCTP-LP Formulation}
\label{tbl:GFCTP-LP-sets}
\end{table}
%%% 

%%% 
\begin{table} [h!]
\centering
\begin{tabularx}{\textwidth-20pt}{|c|X|} % line wraps second column if too long
\hline
Variable    & Description \\
\hline
$c_{i,j}(h)$             & the unit cost of commodity $h$ 
                         for producer $i$ and consumer $j$  \\
$x_{i,j}(h)$             & a decision variable, the flow of commodity $h$ 
                         for producer $i$ and consumer $j$  \\
$q_{j}(h)$               & the requested quality of commodity $h$ 
                         by consumer $j$  \\
$\beta_{i,k}(q_{j}(h))$  & a capacity translation function for capacity 
                         constraint $k$ of producer $i$ given $q_{j}(h)$ \\
$s_{i,k}$                & a supply capacity of producer $i$ corresponding to 
                         capacity constraint $k$ \\
$d_{j}(H_{j})$           & the total demand of consumer $j$ over the set of 
                         satisfying commidities $H_{j}$ \\
\hline
\end{tabularx}
\caption{Variables Appearing in the GFCTP-LP Formulation}
\label{tbl:GFCTP-LP-vars}
\end{table}
%%% 

This formulation deviates from the normal MTP formulation via the expansion of
capacity constraints (Equation \ref{eq:GFCTP-LP_sup}) and the inclusion of a
constraint allowing multiple commodities that are able to meet the demand of a
producer (Equation \ref{eq:GFCTP-LP_dem}). The former constraint maintains the
multicommodity nature of the formulation. This leads to an important insight: if
$\left|{H_{j}}\right| = 1 \forall j \in J$, then the GFCTP-LP can be transformed
into a separable multicommodity transportation problem as shown in
\cite{bertsekas_network_1998}. If the problem is separable, then the
Transportation Problem Simplex Method shown in \S\ref{sec:trans-simplex} can be
applied to a series of smaller subproblems, reducing overall complexity.

