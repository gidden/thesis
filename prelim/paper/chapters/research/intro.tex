This chapter seeks to lay out a plan by which a fully agent-based simulation can
be implemented for a generic nuclear fuel cycle with a more realistic chemical
separations and fuel-matching model than currently exists in the field. The term
generic implies that the facilities involved are not known \textit{a priori}
and, accordingly, facilities can be coupled together automatically, separating
the concern of fuel facility modeling and connection from the simulation
solution engine. For example, a modeler has the choice to model a separations
facility and advanced fuel fabrication facility as separate entities whose
connected supply and demand are met by a generic engine, or to model the two
facilities as a single combined and coupled entity. Additionally, the solution
framework for this matching engine must be agnostic as to the classes of
commodities and materials involved. Rather than hard-coding in constraints and
capacities for different material classes, they are added dynamically based on
the entities involved in the simulation-based solution.

As fuel cycle simulators have progressed from simple spreadsheet applications,
work advancing the field has focused on including in-simulation dynamic
calculations of important metrics and parameters in order to provide feedback to
the simulation rather than solely post-processing simulation output. A number of
examples exist. COSI uses the CESAR depletion code \cite{vidal_cesar:_2006} to
automate output fuel characteristics in order to reduce voluminous user input
for simulated materials. Scopatz introduced the notion of essential physics
modeling with his Bright simulation engine \cite{scopatz_essential_2011}. Most
recently, Huff has added to this area of work by developing a repository
facility for the \Cyclus simulator that analyzes repository effects due to
different combinations of materials in different repository
geologies~\cite{huff_integrated_2013}.

This work proposes additional advancement of the dynamic simulation of nuclear
fuel cycles using the \Cyclus simulator. First, a simulation framework for
simulation entity interaction is introduced, the primary goal of which is to
encapsulate simulation-level design decisions. The framework allows for market
interactions to be defined, providing a formalism by which information related
to supply and demand of commodities and materials can be represented in a
general sense. Commodities are not treated simply as quantities, instead the
quantity and \textit{quality} of commodities is treated.  Next, a mathematical
programming formulation based on the multi-commodity transport problem is
proposed to solve the generic supply-demand matching problem. A linear program
formulation and mixed integer-linear program formulation is proposed, the latter
of which addresses the trade-off between computation speed and realism. Finally,
the issue of matching separated elemental streams with requested recycled fuel
is addressed via an approximation linear program.
