In formulating the Generic Fuel Cycle Transportation Problem (GFCTP), note that
the ``players'' in the set of commodity ``markets'' are the individual
facilities involved in the simulation, i.e., the reactors, fabrication
facilities, repositories, etc. In strict mathematical programming parlance, the
GFCTP can be described as a Mixed-Integer, Multicommodity Transportation Problem
(MTP) with Side Constraints. Accordingly are a number of departures from the
classical MTP that was described in \S\ref{sec:MTP}.

To begin, the multicommodity aspect of the problem is not manifest on arc
capacities. Instead, facility demand constraints incorporate a set of
satisfactory commodities. For example, a reactor may be able to accept UOX or
MOX fuel, but has a demand for total fuel. Additionally, supplier facilities may
have a set of constraints on their ability to supply a given commodity and they
may not be able to directly express those constraints with the unit of the
commodity market, i.e., kilograms. Take for example an enrichment facility. Such
a facility has nominally two constraints: SWU capacity and natural uranium
capacity. The former constraint is temporal, i.e., it is a processing
constraint. The latter constraint is an inventory constraint. However, both are
necessary to fully define the problem. Furthermore, let us note that the output
of this facility is kilograms of enriched uranium. Accordingly, the above
capacities must be translated into this output. Finally, realism is introduced
through integer variables. For a number of facilities, especially reactors, it
may not be realistic for a given fuel order to be split amongst a variety of
suppliers. The realm of integer programming techniques allow us to introduced
binary variables to enforce this reality constraint.

It should be noted that the addition of integer variables changes both the
complexity of the formulation and the complexity of the solution technique. Such
a change requires a Mixed Integer-Linear Program (MILP) formulation and solution
via the branch-and-bound method (see \S\ref{sec:bnb}) which solves NP-Hard
combinatorial optimization problems whereas the Linear Program (LP) version
requires the transportation simplex method (see \S\ref{sec:trans-simplex})
which is solvable in polynomial time.  Accordingly, I describe the full
formulation in two parts below: \S\ref{sec:GFCTP-LP} describes the linear
program formulation with side constraints which I will denote GFCTP-LP
and \S\ref{sec:GFCTP-E} describes the MILP formulation with side constraints
which I will denote GFCTP-E (E stands for ``exclusive'', i.e., integer variables
denote an exclusive selection of consumers and/or producers).

\subsubsection{Linear Program with Side Constraints Formulation}\label{sec:GFCTP-LP}


The Generic Fuel Cycle Transportation Problem with Side Constraints (GFCTP-LP)
describes a multi-commodity setting in which demand can be met by multiple
commodities. Consumers denote a cost preference over the possible commodities
they consume and a demand for the set of commodities that must be met. Suppliers
denote one or more production capacities for a given commodity which serve as
the set of supply capacities analgous to the normal MTP
(see \S\ref{sec:MTP}). The GFCTP-LP formulation is as follows:

%%% 
\begin{subequations}\label{eqs:GFCTP-LP}
  \begin{align}
    %%
    \min_{z} \:\: & 
    z = \sum_{h \in H}\sum_{i \in I}\sum_{j \in J}c_{i,j}(h) x_{i,j}(h) 
    & \label{eq:GFCTP-LP_obj} \\
    %%
    \text{s.t.} \:\: &
    \sum_{j \in J}\beta_{i,k}(q_{j}(h)) x_{i,j}(h) \leq s_{i,k} 
    &
    \: \forall \: k \in K_{i}(h),  
    \forall \: i \in I, \forall \: h \in H \label{eq:GFCTP-LP_sup} \\
    %%
    &
    \sum_{i \in I}\sum_{h \in H_{j}} x_{i,j}(h) \geq d_{j}(H_{j}) 
    & 
    \forall \: j \in J \label{eq:GFCTP-LP_dem} \\
    %%
    &
    x_{i,j} \geq 0
    &
    \forall \: x \in X \label{eq:GFCTP-LP_x}
    %%
  \end{align}
\end{subequations}
%%% 

The sets and variables involved are described in Tables \ref{tbl:GFCTP-LP-sets}
and \ref{tbl:GFCTP-LP-vars}.

%%% 
\begin{table} [h!]
\centering
\begin{tabularx}{\textwidth-20pt}{|c|X|} % line wraps second column if too long
\hline
Set         & Description \\
\hline
$H$         & all commodities  \\
$I$         & all producers  \\
$J$         & all consumers  \\
$X$         & the feasible set of flows between producers and consumers  \\
$K_{i}(h)$  & the set of constraining capacities for 
            producer $i$ of commodity $h$  \\
$H_{j}$     & the set of satifying commodities for consumer $j$  \\
\hline
\end{tabularx}
\caption{Sets Appearing in the GFCTP-LP Formulation}
\label{tbl:GFCTP-LP-sets}
\end{table}
%%% 

%%% 
\begin{table} [h!]
\centering
\begin{tabularx}{\textwidth-20pt}{|c|X|} % line wraps second column if too long
\hline
Variable    & Description \\
\hline
$c_{i,j}(h)$             & the unit cost of commodity $h$ 
                         for producer $i$ and consumer $j$  \\
$x_{i,j}(h)$             & a decision variable, the flow of commodity $h$ 
                         for producer $i$ and consumer $j$  \\
$q_{j}(h)$               & the requested quality of commodity $h$ 
                         by consumer $j$  \\
$\beta_{i,k}(q_{j}(h))$  & a capacity translation function for capacity 
                         constraint $k$ of producer $i$ given $q_{j}(h)$ \\
$s_{i,k}$                & a supply capacity of producer $i$ corresponding to 
                         capacity constraint $k$ \\
$d_{j}(H_{j})$           & the total demand of consumer $j$ over the set of 
                         satisfying commidities $H_{j}$ \\
\hline
\end{tabularx}
\caption{Variables Appearing in the GFCTP-LP Formulation}
\label{tbl:GFCTP-LP-vars}
\end{table}
%%% 

This formulation deviates from the normal MTP formulation via the expansion of
capacity constraints (Equation \ref{eq:GFCTP-LP_sup}) and the inclusion of a
constraint allowing multiple commodities that are able to meet the demand of a
producer (Equation \ref{eq:GFCTP-LP_dem}). The former constraint maintains the
multicommodity nature of the formulation. This leads to an important insight: if
Equation \ref{eqs:1demand} holds,

\begin{equation}\label{eqs:1demand}
  \left|{H_{j}}\right| = 1 \: \forall \: j \in J
\end{equation}

then the GFCTP-LP can be transformed into a separable multicommodity
transportation problem as shown in \cite{bertsekas_network_1998}. If the problem
is separable, then the Transportation Problem Simplex Method shown in
\S\ref{sec:trans-simplex} can be applied to a series of smaller subproblems,
reducing overall complexity. Furthermore, if Equations \ref{eqs:1demand} and
\ref{eqs:1constraint} both hold,

\begin{equation}\label{eqs:1constraint}
  \left|{K_{i}(h)}\right| = 1 \: \forall \: i \in I, \: \forall \: h \in H
\end{equation}

then the GFCTP-LP is in fact the a normal Transportation Problem, because the
quality translation function ($\beta_{i,k}(q_{j}(h))$) translates to a constant
at solution time.

\paragraph{Capacity Translation Function and Constraint Example}~\\

The notion of a capacity translation function is something that I have
introduced out of necessity due to the complexity of the GFCTP. Accordingly, an
example will help clarify its purpose. I'll use this time to also to provide an
example of a producer with multiple capacity constraints for a given commodity.

Take, for example, an enrichment facility. Such a facility produces the
commodity ``Enriched Uranium''. This facility has two constraints on its
operation for any given time period: the amount of Separative Work Units (SWU)
that it can process and the total uranium feed it has on hand. Note that neither
of these capacities are measure directly in the units of the commodity it
produces, i.e., kilograms of enriched uranium. We can now state the set the
values for $K_{i}(h)$ for this facility:

\begin{equation}\label{eqs:enr-constr-commods}
  K_{enr}(\mbox{Enriched Uranium}) = \{ \mbox{SWU}, \mbox{Natural Uranium} \}
\end{equation}

Let us now consider that there is a set of requests for enriched uranium that
this facility can possibly meet. Such requests have, in general, two parameters:
$P_{j}$, the total product quantity (in kilograms), and $\varepsilon_{j}$, the
product enrichment (in w/o U-235). To maintain the notation that has previously
been used in \S\ref{sec:prev-enrich}, I will denote $\varepsilon_{j}$ as
$x_{p,j}$ for the remainder of this section. Note that we have provided the
following definition:

\begin{equation}\label{eqs:enr-q-swu}
  q_{j}(\mbox{Enriched Uranium}) \equiv x_{p,j}
\end{equation}

These values are set during a prior phase of the overall matching algorithm, and
can therefore be considered constants. Further, let us note that, in general, an
enrichment facility's operation, or rather its capacity thereof, is goverened by
two parameters: $x_{f,enr}$, the fraction of U-235 in its feed material, and
$x_{t,enr}$, the fraction of U-235 in its tails material. Let us assume both of
these are constants of the facility. Utilizing the equations presented in
\S\ref{sec:prev-enrich}, we can denote the functional forms of this facility's
two capacity constraints.

\begin{align}\label{eqs:enr-swu-constr}
\begin{split}
\beta_{enr,SWU}(x_{p,j}) & = V(x_{p,j}) \\
& + \frac{x_{p,j} - x_{t,enr}}{x_{f,enr} - x_{t,enr}} V(x_{t,enr}) \\
& - \frac{ x_{p,j} - x_{f,enr} } { x_{f,enr} - x_{t,enr} } V(x_{f,enr})
\end{split}
\end{align}

Where $V(x)$ is the value function of Equation \ref{eqs:enr-value}.

\subsubsection{Mixed Integer-Linear Program with Side and Exclusivity Constraints Formulation}\label{sec:GFCTP-E}

%The previous linear program (LP) formulation of the Generic Fuel Cycle
Transportation Problem fully describes many of the types of transactions that
arise at any given time step. However, it importantly glosses over the critical
case of reactor fuel orders, which comprise a large amount of material orders
within the simulation context. Specifically, it allows reactor fuel orders to be
met by more than one supplier with an arbitrary amount of the order met by each
supplier. Put another way, the LP formulation does not contain the discrete
material information required to model the transaction of fuel assemblies. Such
detail is not necessary in every simulation, but we wish to allow this advanced
modeling for those that do need it. In order to provide this capability of
quantizing orders, binary decision variables must be introduced and integer
programming techniques must be utilized to solve the resulting mixed
integer-linear program. I present the updated formulation below. The key
difference is the inclusion binary variables $y_{i,j}^{h}$, which are 1 if
producer $i$ trades commodity $h$ with consumer $j$ and constants
$\tilde{x}_{j}^{h}$, which denote the quantity of a quantized order. Further a
new set is introduced, $J_{e}$, the set of consumers who require quantized, or
exclusive, orders. The original set of consumers, those who allow partial
orders, I denote $J_{p}$. These two sets constitute the set of all consumers.

\begin{equation}\label{eqs:consumer-union}
  J = J_{p} \cup J_{e}
\end{equation}

The Generic Fuel Cycle Transportation Problem with Exclusive Orders (GFCTP-E)
formulation follows:

\begin{subequations}\label{eqs:GFCTP-E}
  \begin{align}
    %%
    \label{eq:GRCTP-E_obj}
    \min_{z} \:\: 
    & 
    z = \sum_{h \in H}\sum_{i \in I}\sum_{j \in J_{p}}c_{i,j}^{h} x_{i,j}^{h} 
    + \sum_{h \in H}\sum_{i \in I}\sum_{j \in J_{e}}c_{i,j}^{h} y_{i,j}^{h} \tilde{x}_{j}^{h}
    && \\
    %%
    \label{eq:GRCTP-E_sup}
    \text{s.t.} \:\: 
    &
    \sum_{j \in J_{p}}\beta_{i,k}(q_{j}^{h}) x_{i,j}^{h}
    + \sum_{j \in J_{e}}\beta_{i,k}(q_{j}^{h}) y_{i,j}^{h} \tilde{x}_{j}^{h} \leq s_{i,k}^{h} 
    &&
    \forall \: i \in I, \: \forall \: k \in K_{i}^{h}, \forall \: {h \in H} \\
    %%
    \label{eq:GRCTP-E_dem_p}
    &
    \sum_{i \in I}\sum_{h \in H_{j}} x_{i,j}^{h} \geq d_{j}(H_{j}) 
    & 
    \forall \: j \in J_{o} &\\
    %%
    \label{eq:GRCTP-E_dem_e}
    &
    \sum_{i \in I}\sum_{h \in H_{j}} y_{i,j}^{h} \tilde{x}_{j}^{h} \geq d_{j}(H_{j}) 
    &
    \forall \: j \in J_{e}  &\\
    %%
    \label{eq:GRCTP-E_sumy}
    &
    \sum_{h \in H}\sum_{i \in I} y_{i,j}^{h} = 1
    &
    \forall \: j \in J_{e}  &\\
    %%
    \label{eq:GRCTP-E_x}
    &
    x_{i,j}^{h} \geq 0
    &
    \forall \: x \in X  &\\
    %%
    \label{eq:GRCTP-E_y}
    &
    y_{i,j}^{h} \in \{0,1\}
    &
    \forall \: y \in Y &
    %%
  \end{align}
\end{subequations}

The sets and variables involved are described in Tables \ref{tbl:GFCTP-E-sets}
and \ref{tbl:GFCTP-E-vars}. Note that $H_{j}$ is a subset of the commodities:

\begin{equation}
  H_{j} \subseteq H \: \forall \: j \in J_{p}, \forall \: j \in J_{e}
\end{equation}

%%% 
\begin{table} [h!]
\centering
\begin{tabularx}{\textwidth-20pt}{|c|X|} % line wraps second column if too long
\hline
Set         & Description \\
\hline
$H$         & all commodities  \\
$I$         & all producers  \\
$J_{p}$     & all consumers who accept parital orders  \\
$J_{e}$     & all consumers who accept only exclusive orders  \\
$X$         & the feasible set of flows between producers and consumers  \\
$Y$         & the feasible set of exclusive flows between 
            producers and consumers  \\
$K_{i}^{h}$ & the set of constraining capacities for 
            producer $i$ of commodity $h$  \\
$H_{j}$     & the set of satisfying commodities for consumer $j$  \\
\hline
\end{tabularx}
\caption{Sets Appearing in the GFCTP-E Formulation}
\label{tbl:GFCTP-E-sets}
\end{table}
%%% 

%%% 
\begin{table} [h!]
\centering
\begin{tabularx}{\textwidth-20pt}{|c|X|} % line wraps second column if too long
\hline
Variable    & Description \\
\hline
$c_{i,j}^{h}$             & the unit cost of commodity $h$ 
                          for producer $i$ and consumer $j$  \\
$x_{i,j}^{h}$             & a decision variable, the flow of commodity $h$ 
                          for producer $i$ and consumer $j$  \\
$q_{j}^{h}$               & the requested quality of commodity $h$ 
                          by consumer $j$  \\
$y_{i,j}^{h}$             & a binary decision variable that is equal to 1 if 
                          there is flow from producer $i$ to consumer $j$ of 
                          commodity $h$ \\
$\tilde{x}_{j}^{h}$       & the amount of commodity $h$ requested by 
                          consumer $j$ \\
$\beta_{i,k}(q_{j}^{h})$  & a capacity translation function for capacity 
                          constraint $k$ of producer $i$ given $q_{j}^{h}$ \\
$s_{i,k}^{h}$             & a supply capacity of producer $i$ corresponding to 
                          capacity constraint $k$ of commodity $h$ \\
$d_{j}(H_{j})$            & the total demand of consumer $j$ over the set of 
                          satisfying commodities $H_{j}$ \\
\hline
\end{tabularx}
\caption{Variables Appearing in the GFCTP-E Formulation}
\label{tbl:GFCTP-E-vars}
\end{table}
%%%

The examples of the various constraints from the previous section also apply
here. The only difference is the notion of the binary variables, $y_{i,j}^{h}$s,
which denote a sort of on/off switch as to whether a consumer's entire requested
amount of material is met by a supplier or not.

It should be noted that this advanced formulation brings in signifigant
complexity to the resolution method at every time step. As with other mixed
integer-linear programs, the GFCTP-E is NP-Complete and must be solved using the
Branch-and-Bound technique described in \S\ref{sec:bnb}. However, simple
heuristics exist. The most common of them is to solve a relaxed version of the
problem in the form of a linear program, and to round values to form an integer
solution. The exploration of additional heuristics will be performed based on
the outcome of the implementation and analysis of this formulation in
the \Cyclus simulation environment.

