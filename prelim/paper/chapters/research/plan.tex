The Department of Energy has recently released a document outlining steps forward
for its dynamic fuel cycle analysis campaign entitled the Dynamic Systems
Analysis Report for Nuclear Fuel Recycle \cite{dixon_dynamic_2008}. This
document outlines a recycle scenario including fast reactors and recycled fuel
fabrication. Such a scenario will be used a baseline for implementation of this
proposed work.

As an initial step, the dynamic resource exchange framework proposed in
\S\ref{sec:agent-interaction} will be implemented with a basic, ``greedy''
matching algorithm. This will allow for decision making when facilities have
more than one possible input or output in the \Cyclus simulation framework. The
linear programming version of the GFCTP outlined in \S\ref{sec:gfctp-lp} will
then be implemented in order to confirm similar results to the simplistic
matching algorithm. 

In order to demonstrate the nature of the preferential market exchange, the same
scenario will be run with multiple institutions involved, and cardinal
preference assigned between facilities within those institutions. A successful
implementation will show changing material flows based on varying preferences,
and will demonstrate the ease with which such institutional and regional
modeling can be achieved with such a final product. With the linear programming
formulation complete, the integer programming formulation described in
\S\ref{sec:gfctp-ip} will be implemented and investigated. Key items of concern
is the increased computation time resulting from the MILP formulation and the
degree to which it is required, i.e., the degree to which orders are split in an
actual simulation. 

Finally, the fidelity of recycled fuel orders will be investigated by
implementing the RAP as proposed in \S\ref{sec:RAP}. Comparisons will be made
between this model and the aggregate flow and equivalence models currently
implemented by other simulators as summarized in \S\ref{sec:sim-summary}. The
effect and usefulness of the RAP formulation will be reported.
