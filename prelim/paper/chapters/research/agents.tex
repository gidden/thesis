This sections describes the simulation interaction between agents. Of the
research directions presented in this chapter, it is definitely the most
maleable. Deciding how a simulation is structured from an interactions
standpoint is a delicate balance of known necessity and perceived future
needs. There are basic decisions to make: do you want a system with discrete
material transfers or continous material flows? Discrete transfers more closely
match reality and may provide insights in that regard, however the require more
of their modeling apparatus due to messaging needs and other structures. More
complex decisions include how one wants to determine connections between
facilities. Do we assign supplier-consumer pairs to facilities? Do we allow them
to change? Should the facility make such a decision? Should that decision be
affected by any other entities? Guerin's comment in \S\ref{sec:intro-benchmarks}
stems from this ``freedom''. These simulation-engine decisions comprise the
art-related portion of fuel cycle simulation. The goal is to make these
decisions in as informed a way as possible from our domain-level knowledge with
respect to our known and perceieved requirements. In general, we try to minimize
the sheer number of choices we make in this regard, instead relying on well
known and well documented practices of computer scientists and systems
engineers.

\Cyclus has an additional goal in that we wish our core simulation
infrastructure to be as flexible as possible. Given a few basic tennets of agent
interaction, other developers should be able to create a new agent to ``plug
in'' to the simulation. Accordingly, we must define a minimal set of behaviors
to sufficiently inform the simulation infructure to run the simulation. This
freedom allows us to run the simulation program and attach agents at run time,
effectively separating the simulation engine's functionality from the agents in
the simulation. From an ecosystem point of view, being an open source code and
having such capability allows expansion of the user and developer base into
areas and institutions concerned with security and privacy. Furthermore,
developers could participate both privately and publicly, e.g. adding general
capability to the \Cyclus core that is needed for some functionality without
specifying the internals. Such a community paradigm is shown in Figure
\ref{fig:community}.

\begin{figure}[htbp!]
  \begin{center}
    \includegraphics[height=8cm]{./chapters/research/community.png}
  \end{center}
  \caption{The Cyclus Participation Paradigm} 
  \label{fig:community}
\end{figure}

This open-source ecosystem further provides incentive to develop the agent-based
simulation architecture. Other developers can concentrate their efforts on
individual agent interaction, effectively ecapsulating developer requirements
for learning and interacting with the various simulation systems. Having decided
upon agent-based interactions, one must determine a way to govern these
interactions. We want to minimize agent dependency due to our above discussion,
so using preference-based network flow formulations provide us with a viable
solution technique that provides a consistent interface. The remainder of this
section describes how that market resoluation interface is informed by the
agents. Basic agent simulation intereaction, such as entering and leaving the
simulation are also described.

\subsection{Supply/Demand Resolution Mechanism}

The resolution of supply and demand at any given time step is the result of the
mathmatical program techniques described in \S\ref{sec:gfctp}; however, there
are simulation-engine details that must be described in order to set up that
problem. The proposed resolution mechanism occurs in nominally three steps. The
agent interactions include consumers and producers of a set of commodities.

The first step is for consumers to "post" their consumption needs, i.e. their
"demand". Consumers are allowed to "over post", i.e., request more than they
actually need if a corresponding constraint accompanies that request. For
example, a facility could request fuel as either fire or coal, but must denote
that it will take only one of the two. While their preference between the two
competing sources is known at this time, that state is queried in the third
step. This posting is termed a "request for bid".

The second step allows supplier entities to respond to the board of posted
demands. In general, this step is simple: suppliers know which commodities they
produce and at what maximum capacity they produce them. Accordingly, responses
can be posted to each consumer requesting a resource in their production
category. In the current formulation, capacity restrictions must be linear
functions of the demand, but may depend on specific qualities of the demand
(e.g. uranium enrichment and SWUs), because these are static at solution time
and can thus be evaluated and modeled as a constant. Furthermore, a producer can
provide more than one such capacity restriction. For example, an enriched
uranium provider could have capacities related to both SWU and natural uranium
reserves. This posting is termed a "response for requests for bids" and
effectively defines the possible arcs between consumers and producers. It should
be noted that consumers can request different commodities to meet the same
demand, i.e., this is a multi-commodity problem.

The third step is for consumers and their managers to assign a preference score
to each bid. In our system, managers are institutions and regions, where each is
allowed to affect the overal preference through modifiers. The facilities
themselves set the base preference score, which can be simply a funciton of the
commodity or, in a more advanced way, a funciton of both the commodity and
quality thereof. For example, a facility can say apriori that it prefers wood
over coal, but a region can inform the preference that no coal is allowed. It is
at this step that timeliness can be assessed, i.e., a facility can give a
preference of 0 to a response that will arrive too late. It is also at this
stage that consumers can group bid responses. For example, a facility could say
that it will accept only an order of wood or only an order of coal. Furthermore,
it can denote that it will only accept whole orders, i.e. all of its order must
come from a single supplier. This models the reality of the fuel cycle, but
transforms the problem from a linear program (LP) to a mixed integer/linear
program (MILP). Finally, a cost translation mapping is applied to the set of
preferences in order to assume the form of a minimum cost problem. Both
formulations are described in \S\ref{sec:gfctp}.

\subsection{Facilities}
\subsection{Institutions}
\subsection{Regions}
