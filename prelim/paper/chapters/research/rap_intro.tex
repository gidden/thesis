The Recipe Approximation Problem (RAP) was originally conceived by Kyle Oliver
in \cite{oliver_geniusv2:_2009}. It was the result of conversations with
scientists of the VISION fuel cycle simulation team \cite{vision2009} with what
was then called the Winery Problem, because of the similarity due to mixing
vintages of win to match a given recipe. In the case of the RAP, we wish to
model a separations facility. More specifically, we wish to model the matching
of separated material to a given set of requested recipes. Accordingly, a linear
approximation optimization problem is used that takes into account total mass,
isotopic percents, and neutronics parameters into account. The remainder of this
section shows the formulation that has beene expanded from
\cite{oliver_geniusv2:_2009} and a method of adding additional constraints based
on additional information.
