The previous linear program (LP) formulation of the Generic Fuel Cycle
Transportation Problem fully describes many of the types of transactions that
arise at any given time step. However, it importantly glosses over the critical
case of reactor fuel orders, which comprise a large amount of material orders
within the simulation context. Specifically, it allows reactor fuel orders to be
met by more than one supplier with an arbitrary amount of the order met by each
supplier. Put another way, the LP formulation does not contain the discrete
material information required to model the transaction of fuel assemblies. Such
detail is not necessary in every simulation, but we wish to allow this advanced
modeling for those that do need it. In order to provide this capability of
quantizing orders, binary decision variables must be introduced and integer
programming techniques must be utilized to solve the resulting mixed
integer-linear program. I present the updated formulation below. The key
difference is the inclusion binary variables $y_{i,j}^{h}$, which are 1 if
producer $i$ trades commodity $h$ with consumer $j$ and constants
$\tilde{x}_{j}^{h}$, which denote the quantity of a quantized order. Further a
new set is introduced, $J_{e}$, the set of consumers who require quantized, or
exclusive, orders. The original set of consumers, those who allow partial
orders, I denote $J_{p}$. These two sets constitute the set of all consumers.

\begin{equation}\label{eqs:consumer-union}
  J = J_{p} \cup J_{e}
\end{equation}

The Generic Fuel Cycle Transportation Problem with Exclusive Orders (GFCTP-E)
formulation follows:

\begin{subequations}\label{eqs:GFCTP-E}
  \begin{align}
    %%
    \label{eq:GRCTP-E_obj}
    \min_{z} \:\: 
    & 
    z = \sum_{h \in H}\sum_{i \in I}\sum_{j \in J_{p}}c_{i,j}^{h} x_{i,j}^{h} 
    + \sum_{h \in H}\sum_{i \in I}\sum_{j \in J_{e}}c_{i,j}^{h} y_{i,j}^{h} \tilde{x}_{j}^{h}
    && \\
    %%
    \label{eq:GRCTP-E_sup}
    \text{s.t.} \:\: 
    &
    \sum_{j \in J_{p}}\beta_{i,k}(q_{j}^{h}) x_{i,j}^{h}
    + \sum_{j \in J_{e}}\beta_{i,k}(q_{j}^{h}) y_{i,j}^{h} \tilde{x}_{j}^{h} \leq s_{i,k}^{h} 
    &&
    \forall \: i \in I, \: \forall \: k \in K_{i}^{h}, \forall \: {h \in H} \\
    %%
    \label{eq:GRCTP-E_dem_p}
    &
    \sum_{i \in I}\sum_{h \in H_{j}} x_{i,j}^{h} \geq d_{j}(H_{j}) 
    & 
    \forall \: j \in J_{o} &\\
    %%
    \label{eq:GRCTP-E_dem_e}
    &
    \sum_{i \in I}\sum_{h \in H_{j}} y_{i,j}^{h} \tilde{x}_{j}^{h} \geq d_{j}(H_{j}) 
    &
    \forall \: j \in J_{e}  &\\
    %%
    \label{eq:GRCTP-E_sumy}
    &
    \sum_{h \in H}\sum_{i \in I} y_{i,j}^{h} = 1
    &
    \forall \: j \in J_{e}  &\\
    %%
    \label{eq:GRCTP-E_x}
    &
    x_{i,j}^{h} \geq 0
    &
    \forall \: x \in X  &\\
    %%
    \label{eq:GRCTP-E_y}
    &
    y_{i,j}^{h} \in \{0,1\}
    &
    \forall \: y \in Y &
    %%
  \end{align}
\end{subequations}
