A viable critique of the proposal of the formulation of the RAP is that it does
not capture reactor-specific information in its application of a neutronics
parameter approximation. This initial formulation uses the neutron production
factor ($\eta$) primarily because of precedence and because of the nature of the
\Cyclus simulation engine. As it is currently designed, the \Cyclus engine
treats facilities as black boxes, which is to say that it only concerned with
material flowing into and out of those boxes for which there may or may not be
an isotopic profile accompanying an amount of requested material. This loose
coupling is by design, the \Cyclus development team wishes to maintain the
ease-of-use of connecting new simulation entities at run time, as was described
in the dynamic loading segment of \S\ref{sec:prev-dynamic}. Accordingly,
decisions to include facility-specific information, e.g. reactor assembly
characteristics, must be made with forethought and knowledge as to their
usefulness for inclusion and straining the minimal-connections model. However,
as future work progresses, such a need may arise.

One possible extension of the RAP formulation that includes reactor information
(albeit minimally) would be a use of a $k_{\infty}$ constraint either in
conjunction with or instead of the $\eta$ constraint. If used in conjunction
with the $\eta$ constraint, the new formulation would effectively add loss
information due to the presence of moderator parameters, creating a tighter
feasible solution space than would exist with simply using a $k_{\infty}$
constraint in isolation. 

The definition of $k_{\infty}$ is as follows.

\begin{equation}\label{eqs:kinf_macro}
k_{\infty} = \frac{\sum_{i \in I_{F}} \nu^{i} \Sigma_{f}^{i}}
                  {\Sigma_{a}^{M} + \sum_{i \in I_{F}} \Sigma_{a}^{i}}
\end{equation}

The notation is almost the same as Equation \ref{eqs:eta_macro}, except that the
difference between fuel and moderator cross sections are denoted by $F$ and $M$
superscripts respectively. For the remainder of this section, though, I will
drop the $F$ superscript because it can be assumed that the isotopes under
consideration for this problem are those of the fuel. The moderator-based
neutron losses are also accounted for in the denominator of the expression. It
should be noted again that this expression assumes a homogenization of fuel and
moderator material, i.e., they are considered ``smeared''. The corresponding
constraint would take the following form, again, similarly to Equation
\ref{eqs:eta_nonlin}.

\begin{equation}
\label{eqs:kinf_nonlin}
\epsilon_{k_{\infty}} \geq \left| 
\frac{\sum_{i \in I_{B}} \nu^{i} \sigma_{f}^{i} \sum_{b \in B} N_{b}^{i} x_{b,r}}
     {\Sigma_{a,r}^{M} + \sum_{i \in I_{B}} \sigma_{a}^{i} \sum_{b \in B} N_{b}^{i} x_{b,r}} 
- k_{\infty,r} \right|
\: \forall \: r \in R
\end{equation}

$k_{\infty,r}$ in Equation \ref{eqs:kinf_nonlin} is defined in the normal way.

\begin{equation}
\label{eqs:kinf_r}
k_{\infty,r} = \frac{\sum_{i \in I_{r}} \nu^{i} \Sigma_{f}^{i}}
                    {\Sigma_{a,r}^{M} + \sum_{i \in I_{r}} \Sigma_{a}^{i}}
\end{equation}

Applying similar operations to Equation \ref{eqs:kinf_nonlin} as were applied to
Equation \ref{eqs:eta_nonlin} result in a similarly linearized constraint.

\begin{equation}
\label{eqs:kinf_linear}
\epsilon_{k_{\infty}} \left( \Sigma_{a,r}^{M} + \sum_{b \in B} \eta_{b}^{-} x_{b,r} \right)
\geq
\left| \sum_{b \in B} \eta_{b}^{+} x_{b,r}
- k_{\infty,r}  \left( \Sigma_{a,r}^{M} + \sum_{b \in B} \eta_{b}^{-} x_{b,r} \right) \right|
\: \forall \: r \in R
\end{equation}

As can be seen, increasing the information in given to the problem formulation
increases the complexity of the formulation. Further, to preserve the linear
nature of the formulation, any additional parameters that we wish to provide a
constraint for must be able to be decomposed into, at worst, a ratio of linear
combinations of barrel-specific parameters and barrel fractions. Should this
formulation prove too simplistic in practice, one way forward would be to move
into the nonlinear realm of approximation formulations. This would open up a
much more robust set of parameters that we could constrain and would also allow
usage of tighter norms than the $L_1$ norm, e.g. the $L_2$ norm. However,
solution techniques for nonlinear programs are significantly more complicated
and computationally time consuming than for linear programs. Indeed, they
constitute an entirely new field of study.
