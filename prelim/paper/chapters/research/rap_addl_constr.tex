A viable critique of the proposed RAP formulation is that it does not capture
reactor-specific information in its application of a neutronics parameter
approximation. As stated in the previous section, the neutron reproduction
factor was chosen because it is a function purely of the fuel isotopics present
in a target recipe. The best candidate for an additional neutronics parameter
the \textit{infinite medium neutron multiplication factor}, $k_\infty$. In it's
simplest form, $k_\infty$ is a function of the neutron reproduction factor and
the \textit{fuel utilization factor}, where the fuel utilization factor is
defined as

\begin{equation}
f = \frac{\Sigma_{a, F}}{\Sigma_{a, F} + \Sigma_{a, M}},
\end{equation}

and the neutron reproduction factor is more precisely defined as 

\begin{equation}
\eta = \frac{\nu_F \Sigma_{f, F}}{\Sigma_{a, F}}.
\end{equation}

Note the distinction made between physical properties of the fuel region, $F$,
and the moderator region, $M$, and the use of macroscopic rather than
microscopic cross sections (see Equation \ref{eqs:micro-macro}. $k_\infty$ in
the energy-independent case can then be defined as

\begin{equation}
k_\infty =  \eta f = \frac{\nu_F \Sigma_{f, F}}{\Sigma_{a, F} + \Sigma_{a, M}}.
\end{equation}

Critically, $k_\infty$ requires some information about reactor geometry, where
in this case information regarding the ratio of moderator to fuel and the type
of moderator used is required. Such a requirement breaks the modular nature of
facilities in \Cyclus described in \S \ref{sec:agent-interaction}, and as such
would require some amount of simulation redesign and source code refactor.
Accordingly, decisions to include facility-specific information, e.g., reactor
assembly characteristics, must be made with forethought and knowledge as to
their usefulness for inclusion and straining the minimal-connections
model. However, as future work progresses, a need may arise.

If such a design decision was taken, though, the RAP formulation could be
enhanced by adding an additional neutronics constraint related to
$k_\infty$. Given that a reactor facility provides the information required in
the original RAP formulation in addition to a macroscopic moderator cross
section, $\Sigma^M_{a,t}$, the target's multiplication factor is defined as

\begin{equation}
\label{eqs:kinf_t}
k_{\infty, t} = \frac{\sum_{i \in I_t} \nu^i \sigma_f^i N^i}
                      {\Sigma^M_{a, t} + \sum_{i \in I_t} \sigma_a^i N^i},
\end{equation}

and a proposed mixture's multiplication factor is defined as 

\begin{equation}
\label{eqs:kinf_t*}
k_{\infty, t*} = \frac{\sum_{i \in I_{B}} \nu^{i} \sigma_{f}^{i} \sum_{b \in B} N_{b}^{i} x_{b,t}}
                {\Sigma^M_{a,r} + \sum_{i \in I_{B}} \sigma_{a}^{i} \sum_{b \in B} N_{b}^{i} x_{b,t}}.
\end{equation}

The corresponding violation constraint follows, much like Equation
\ref{eqs:eta-constraint-nonlin}, as

\begin{equation}\label{eqs:kinf-constraint-nonlin}
\epsilon_{k_\infty} \geq \left| k_{\infty, t^*} - k_{\infty, t} \right|.
\end{equation}

Like Equation \ref{eqs:eta-constraint-nonlin}, Equation
\ref{eqs:kinf-constraint-nonlin} is a nonlinear function of the decision
variables, $x_{b, t}$. It can also be linearized, resulting in a similar
constraint to Equation \ref{eqs:eta_linear}.

\begin{equation}
\label{eqs:kinf_linear}
\epsilon_{k_{\infty}} \left( \Sigma_{a,t}^{M} + \sum_{b \in B} \eta_{b}^{-} x_{b,t} \right)
\geq
\left| \sum_{b \in B} \eta_{b}^{+} x_{b,t}
- k_{\infty,t}  \left( \Sigma_{a,t}^{M} + \sum_{b \in B} \eta_{b}^{-} x_{b,t} \right) \right|
\end{equation}

Equation \ref{eqs:kinf_linear} can then be added to Equation \ref{eqs:rap} to
include moderator information in the formulation. As can be seen, increasing the
information given to the problem formulation increases the complexity of the
formulation. A barrier exists, though, if additional or more precise
(e.g. adding energy dependence) neutronics parameters are added to the
formulation. Such additions likely would break the linear-programming nature of
the formulation, which would require advanced, nonlinear programming methods to
solve. A move to nonlinear programming would also provide access to the $\ell_2$
norm, tightening the approximations provided by the RAP formulation..
