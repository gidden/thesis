In formulating the Generic Fuel Cycle Transportation Problem (GFCTP), note that
the ``players'' in the set of commodity ``markets'' are the individual
facilities involved in the simulation, i.e., the reactors, fabrication
facilities, repositories, etc. In strict mathematical programming parlance, the
GFCTP can be described as a Mixed-Integer, Multicommodity Transportation Problem
with Side Constraints. Accordingly are a number of departures from the classical
Multicommodity Transportation Problem that was described in \S\ref{sec:MTP}.

To begin, the multicommodity aspect of the problem is not manifest on arc
capacities. Instead, facility demand constraints incorporate a set of
satisfactory commodities. For example, a reactor may be able to accept UOX or
MOX fuel, but has a demand for total fuel. Additionally, supplier facilities may
have a set of constraints on their ability to supply a given commodity and they
may not be able to directly express those constraints with the unit of the
commodity market, i.e., kilograms. Take for example an enrichment facility. Such
a facility has nominally two constraints: SWU capacity and natural uranium
capacity. The former constraint is temporal, i.e., it is a processing
constraint. The latter constraint is an inventory constraint. However, both are
necessary to fully define the problem. Furthermore, let us note that the output
of this facility is kilograms of enriched uranium. Accordingly, the above
capacities must be translated into this output. Finally, realism is introduced
through integer variables. For a number of facilities, especially reactors, it
may not be realistic for a given fuel order to be split amongst a variety of
suppliers. The realm of integer programming techniques allow us to introduced
binary variables to enforce this reality constraint.

The GFCTP formulation is as follows:

%%% 
\begin{subequations}\label{eqs:GFCTP}
  \begin{align}
    %%
    \min_{z} \:\: & 
    z = \sum_{h \in H}\sum_{i \in I}\sum_{j \in J}c_{i,j}(h) x_{i,j}(h) 
    & \label{eq:GFCTP_obj} \\
    %%
    \text{s.t.} \:\: &
    \sum_{j \in J}\beta_{i,k}(h,q_{j}) x_{i,j}(h) \leq S_{i,k}(h) 
    & 
    \forall \: i \in I, \: \forall \: k \in K_{i}(h) \label{eq:GFCTP_sup} \\
    %%
    &
    \sum_{i \in I}\sum_{h \in H_{j}} x_{i,j}(h) \geq d_{j}(H_{j}) 
    & 
    \forall \: j \in J_{o} \label{eq:GFCTP_dem_o} \\
    %%
    &
    \sum_{i \in I}\sum_{h \in H_{j}} y_{i,j} x_{i,j}(h) \geq d_{j}(H_{j}) 
    &
    \forall \: j \in J_{e} \label{eq:GFCTP_dem_e} \\
    %%
    &
    \sum_{i \in I} y_{i,j} = 1
    &
    \forall \: j \in J_{e} \label{eq:GFCTP_sumy} \\
    %%
    &
    x_{i,j} \geq 0
    &
    \forall \: x \in X \label{eq:GFCTP_x} \\
    %%
    &
    y_{i,j} \in \{0,1\}
    &
    \forall \:* y \in Y \label{eq:GFCTP_y}
    %%
  \end{align}
\end{subequations}
%%% 
