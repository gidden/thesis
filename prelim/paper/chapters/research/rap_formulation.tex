The Recipe Approximation Problem formulation is designed to match a set of
target materials given a set of separated material streams. A \textit{match} is
an approximation of a target material that is \textit{close enough} to the
target material. Accordingly, an approximation linear program as described
in \S\ref{sec:approx} is proposed that attempts to minimize the residual
isotopic vectors associated a series of matches. 

A weighted $\ell_1$ norm is used to measure the size of the residual to maintain the
linear nature of the program. The residual associated with a given target recipe
is defined as

\begin{equation}\label{eqs:norm}
\vec{y_{t}} = \left| M \cdot \vec{x_{t}}  - m_t \vec{I_{t}} \right|.
\end{equation}

Weights are required in the objective function for primarily two reasons. First,
the isotopes that are most important from a nuclear engineering domain knowledge
point of view are known to of relatively low concentration in the final fuel
recipe (e.g., U-235 is only 3-5 w/o of LWR fuel). Second, isotopes may exist in
barrels that are not specified in target recipes. A simple physical explanation
is that any separations process has an efficiency which governs how ``pure'' a
separated elemental stream is -- there are always some
contaminants. Accordingly, I adopt Oliver's definition of this weight from
\cite{oliver_geniusv2:_2009}, which normalizes the importance of isotopes in the
final mixture.

\begin{equation}\label{eqs:weights}
c_{i,t} = 
\begin{cases}
 \frac{1}{m_{i,t}} & \text{if } i \in I_{t} \\
 \frac{1}{m_{t}}   & \text{if } i \not\in I_{t}
\end{cases}
\end{equation}

Combing Equation \ref{eqs:weights} with Equation \ref{eqs:norm} yields the
weighted-norm objective function.

\begin{equation}\label{rap-obj}
\min_{z} \:\: z = \sum_{t \in T} \vec{c_{t}}^{\top} \cdot \vec{y_{t}}
\end{equation}

The minimization objective in and of itself is not necessarily interesting. In
order to enforce physical realities and domain-level knowledge, additional
constraints are added with which any set of extraction vectors must comply. In
the simplest formulation, these include total mass and neutronics
constraints. Each constraint has an violation associated with it, i.e., a number
defining the maximum distance allowable away from the target recipe. 

The mass violation constraint associated with a given target recipe is defined
as the difference between the mass of a proposed recipe matching and a target
recipe mass. The mass of the proposed matching is defined as

\begin{equation}\label{eqs:mass-constraint}
m_{t^*} = \sum_{b \in B} m_{b} x_{b,t}
\end{equation}

which then yields the violation constraint for a given target recipe, $t$,

\begin{equation}\label{eqs:mass-constraint-simple}
\epsilon_{m} \geq \left| m_{t^*} - m_{t} \right|.
\end{equation}

The neutronics parameter that is constrained in this formulation is the average
number of neutrons produced in a fission reaction per absorption by a given
isotope, $\eta$. It is chosen as a proof-of-principle constraining parameter
because it can be calculated purely from target fuel recipes without any reactor
core-specific information. Given a set of target isotopes, $I_t$, and the number
density of each isotope in the mixture, $N^i$, $\eta_t$ for homogenous media is
formulated as

\begin{equation}
\label{eqs:eta_micro}
\eta_t = \frac{\sum_{i \in I_t} \nu^{i} \sigma_{f}^{i} N^{i}}
            {\sum_{i \in I_t} \sigma_{a}^{i} N^{i}},
\end{equation}

with physical constants $\nu^{i}$, the average number of neutrons produced by
fission of isotope $i$, $\sigma_{f}^{i}$, the microscopic fission cross section
for isotope $i$, and $\sigma_{a}^{i}$, the microscopic absorption cross section
for isotope $i$.

One can then define the neutronic parameter associated with a mixture of
isotopes and a target recipe,

\begin{equation}
\label{eqs:eta_fractions_nonlin}
\eta_{t^*} = \frac{\sum_{i \in I_{B}} \nu^{i} \sigma_{f}^{i} \sum_{b \in B} N_{b}^{i} x_{b,t}}
                {\sum_{i \in I_{B}} \sigma_{a}^{i} \sum_{b \in B} N_{b}^{i} x_{b,t}},
\end{equation}

resulting in a constraint similar to Equation \ref{eqs:mass-constraint}.

\begin{equation}\label{eqs:eta-constraint-nonlin}
\epsilon_{\eta} \geq \left| \eta_{t^*} - \eta_t \right|
\end{equation}

Such a constraint, however, is a nonlinear function of the decision variables,
$x_{b,t}$, and some additional algebra is needed to place them into a linear
form. Equation \ref{eqs:eta_fractions_nonlin} can be rearranged as

\begin{equation}
\label{eqs:eta_fractions_lin}
\eta_{t^*} = \frac{\sum_{b \in B} x_{b, t} \sum_{i \in I_{b}} \nu^{i} \sigma_{f}^{i} N_{b}^{i}}
                {\sum_{b \in B} x_{b, t} \sum_{i \in I_{b}} \sigma_{a}^{i} N_{b}^{i}},
\end{equation}

allowing one to define two parameters associated with the production and
absorption of neutrons associated with a given barrel of isotopes, $b$.

\begin{equation}
\label{eqs:eta_+}
\eta_{b}^{+} \equiv \sum_{i \in I_{b}} \nu^{i} \sigma_{f}^{i} N_{b}^{i}
\end{equation}

\begin{equation}
\label{eqs:eta_-}
\eta_{b}^{-} \equiv \sum_{i \in I_{b}} \sigma_{a}^{i} N_{b}^{i}
\end{equation}

Equation \ref{eqs:eta_fractions_lin} can then be rewritten using Equations
\ref{eqs:eta_+} and \ref{eqs:eta_-}.

\begin{equation}
\label{eqs:eta_simple}
\eta_{t^*} = \frac{\sum_{b \in B} \eta_{b}^{+} x_{b, t}}
                {\sum_{b \in B} \eta_{b}^{-} x_{b, t}}
\end{equation}

The nonlinear neutronics constraint can then be rewritten using Equation
\ref{eqs:eta_simple}.

\begin{equation}
\label{eqs:eta_nonlin_simple}
\epsilon_{\eta} \geq \left| 
\frac{\sum_{b \in B} \eta_{b}^{+} x_{b,t}}
     {\sum_{b \in B} \eta_{b}^{-} x_{b,t}}
- \eta_{t} \right|
\end{equation}

Finally, Equation \ref{eqs:eta_nonlin_simple} can be massaged to provide the
constraint as a linear function of the decision variables. Note that no sign
change occurs because the multiplicative constant, $\sum_{b \in B} \eta_{b}^{-}
x_{b,t}$, is guaranteed to be positive.

\begin{equation}
\label{eqs:eta_linear}
\epsilon_{\eta} \sum_{b \in B} \eta_{b}^{-} x_{b,t} \geq
\left| \sum_{b \in B} \eta_{b}^{+} x_{b,t}
- \eta_{t} \sum_{b \in B} \eta_{b}^{-} x_{b,t} \right|
\end{equation}

A final constraint is placed on the decision variables, stating that for any
barrel, $b$, the total fraction of material leaving the barrel must be within
the barrel size.

\begin{equation}
\label{eqs:barrel-fracs}
\sum_{t \in T} x_{b, t} \leq 1
\end{equation}

Combining Equations \ref{eqs:norm}, \ref{rap-obj}, 
\ref{eqs:mass-constraint-simple}, \ref{eqs:eta_linear}, and
\ref{eqs:barrel-fracs} the full RAP linear program can be constructed.

%%% 
\begin{subequations}\label{eqs:rap}
  \begin{align}
    %%
    \min_{z} \:\: & 
    z = \sum_{t \in t} \vec{c_{t}}^{\top} \cdot \vec{y_{t}}
    & \label{eqs:rap_obj} \\
    %%
    \text{s.t.} \:\: &
    \vec{y_{t}} = \left| M \cdot \vec{x_{t}}  - m_t \vec{I_{t}} \right|
    &
    \: \forall \: t \in T \label{eqs:rap_iso} \\
    %%
    &
    \epsilon_{m} \geq \left| \sum_{b \in B} m_{b} x_{b, t} - m_{t} \right|
    & 
    \forall \: t \in T \label{eqs:rap_mass} \\
    %%
    &
    \epsilon_{\eta} \sum_{b \in B} \eta_{b}^{-} x_{b, t} \geq 
    \left| \sum_{b \in B} \eta_{b}^{+} x_{b, t} - 
           \eta_{t} \sum_{b \in B} \eta_{b}^{-} x_{b, t} \right|
    & 
    \forall \: t \in T \label{eqs:rap_eta} \\
    &
    \sum_{t \in T} x_{b, t} \leq 1
    & 
    \forall \: b \in B \label{eqs:rap_conserv} \\
    &
    x_{b, t} \in \left[ 0, 1 \right]
    & 
    \forall \: b \in B, \forall \: t \in T  \label{eqs:rap_x}
    %%
  \end{align}
\end{subequations}
%%% 
