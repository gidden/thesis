The Recipe Approximation Problem formulation is designed to match a set of
target materials given a set of separated material streams. A \textit{match} is
an approximation of a target material that is \textit{close enough} to the
target material. Accordingly, an approximation linear program as described
in \S\ref{sec:approx} is proposed that attempts to minimize the residual
isotopic vectors associated a series of matches. 

A weighted $\ell_1$ norm is used to measure the size of the residual to maintain the
linear nature of the program. The residual associated with a given target recipe
mass is defined as

\begin{equation}\label{eqs:norm}
\vec{y_{t}} = \left| M \cdot \vec{x_{t}}  - m_t \vec{I_{t}} \right|.
\end{equation}

Weights are required in the in the objective function for primarily two
reasons. First, the isotopes that are most important from a nuclear engineering
domain knowledge point of view are known to of relatively low concentration in
the final fuel recipe (e.g., U-235 is only 3-5 w/o of LWR fuel). Second,
isotopes may exist in barrels that are not specified in target recipes. A simple
physical explanation is that any separations process has an efficiency which
governs how ``pure'' a separated elemental stream is -- there are always some
contaminants. Accordingly, I adopt Oliver's definition of this weight
from \cite{oliver_geniusv2:_2009}, which normalizes the importance of isotopes
in the final mixture.

\begin{equation}\label{eqs:weights}
c_{i,t} = 
\begin{cases}
 \frac{1}{m_{i,t}} & \text{if } i \in I_{t} \\
 \frac{1}{m_{t}}   & \text{if } i \not\in I_{t}
\end{cases}
\end{equation}

Combing Equation \ref{eqs:weights} with Equation \ref{eqs:norm} yields the
weighted-norm objective function.

\begin{equation}
\min_{z} \:\: z = \sum_{t \in T} \vec{c_{t}}^{\top} \cdot \vec{y_{t}}
\end{equation}

%% end of work so far %%



The goal of this formuation is to match a set of fuel requests as closely as
possible, where proximity in this sense is defined by the $\ell_1$ norm of the
distance between a mixture attribute and a target attribute. In the following
linear program, the mixture attribute is the dot product of a matrix and a
vector. The matrix is that of all isotopic masses in barrels, $M$, where an
entry, $M_{b,i}$, is the mass of isotope $i$ in barrel $b$. The vector is the
decision variables for a given recipe, $\vec{x_{r}}$, where an entry, $x_{b,r}$,
is the fraction of barrel $b$ being assigned to fuel request $r$. The target
attribute is the target fuel request composition, $\vec{t_{r}}$. where an entry,
$t_{i,r}$, is the mass of isotope $i$ in fuel request $r$.

The minimization objective in and of itself is not necessarily interesting. In
order to enforce physical realities and domain-level knowledge, additional
constraints are added that the result must comply with. These include total
mass, isotopic masses, and neutronics constraints. I present the full
formulation below in Equation \ref{eqs:rap} and explain more fully its
derivation and interpretation in the following paragraphs.

%%% 
\begin{subequations}\label{eqs:rap}
  \begin{align}
    %%
    \min_{z} \:\: & 
    z = \sum_{r \in R} \vec{c_{r}}^{\top} \cdot \vec{y_{r}}
    & \label{eqs:rap_obj} \\
    %%
    \text{s.t.} \:\: &
    \vec{y_{r}} = \left| M \cdot \vec{x_{r}}  - \vec{t_{r}} \right|
    &
    \: \forall \: r \in R \label{eqs:rap_iso} \\
    %%
    &
    \epsilon_{m} \geq \left| \sum_{b \in B} m_{b} x_{b,r} - m_{r} \right|
    & 
    \forall \: r \in R \label{eqs:rap_mass} \\
    %%
    &
    \epsilon_{\eta} \sum_{b \in B} \eta_{b}^{-} x_{b,r} \geq 
    \left| \sum_{b \in B} \eta_{b}^{+} x_{b,r} - 
           \eta_{r} \sum_{b \in B} \eta_{b}^{-} x_{b,r} \right|
    & 
    \forall \: r \in R \label{eqs:rap_eta} \\
    &
    \sum_{r \in R} x_{b,r} \leq 1
    & 
    \forall \: b \in B \label{eqs:rap_conserv} \\
    &
    x_{b,r} \in \left[ 0, 1 \right]
    & 
    \forall \: b \in B, \forall \: r \in R  \label{eqs:rap_x}
    %%
  \end{align}
\end{subequations}
%%% 

Let me begin the discussion of the RAP formulation by noting that the
$\vec{y_{r}}$ variables occur twice, once as an assignment in
Equation \ref{eqs:rap_iso} and again in Equation \ref{eqs:rap_obj}, the
objective function. The $\vec{y_{r}}$ variables are column vectors that
represent the different between the isotopic composition of a proposed mixed
solution for a target fuel request, $r$, and the target fuel request itself. The
goal of this formulation is to minimize that difference. However, a strict
difference minimization is not directly suitable. Because separations plants can
separate materials only on elemental scales, not on isotopic scales, and because
fuel recipes are isotopic-specific, there will inevitively be isotopes present
in the barrels that are not requested in the recipe. This is described formally
in Equation \ref{eqs:barrel-req-iso}.

\begin{equation}
\label{eqs:barrel-req-iso}
I_{B} \supseteq I_{r} \: \forall r \in R
\end{equation}

Additionally, the isotopes that are most important from a nuclear engineering
domain knowledge point of view are known to of relatively low concentration in
the final fuel recipe (e.g., U-235 is only 3-5 w/o of LWR fuel). Accordingly,
some kind of weighting function is required to determine the relative importance
among isotopes in the resulting solution. I adopt Oliver's definition of this
weight from \cite{oliver_geniusv2:_2009}, which normalizes the importance of
isotopes in the final mixture.

\begin{equation}
c_{i,r} = 
\begin{cases}
 \frac{1}{m_{i,r}} & \text{if } i \in I_{r} \\
 \frac{1}{m_{r}}   & \text{if } i \not\in I_{r}
\end{cases}
\end{equation}

Note that the dot product of the vectors $\vec{y_{r}}$ and $\vec{c_{r}}^{\top}$
results in a scalar, and we wish to minimize the aggregate sum of scalars for
all fuel requests.

Moving on to the physical constraints, let us cover the simpler
Equation \ref{eqs:rap_mass} first. This constraint states that the mass of a
feasible matching solution, i.e., $\sum_{b \in B} m_{b} x_{b,r}$, must be within
some range of the mass of the given fuel request, $m_{r}$. This range is defined
by some small value, $\epsilon_{m}$. It is important to realize that the
$\epsilon$ vaules appearing in these constraints critically affect the
feasibility space. Whereas the objective function seeks to minimize the
difference between a mixture attribute and target attribute, the $\epsilon$
values define a maximum allowable difference. Accordingly, the define the bounds
of the feasible solution space. It entirely possible that during the course of a
simulation, an instance of the RAP will not have a feasible solution. In such a
scenario, the various $\epsilon$ values will have to be incrementally increased,
or relaxed, until feasibility is attained.

The neutronics constraint, Equation \ref{eqs:rap_eta}, is more complicated and
deserves its own derivation. First, note its purpose. It is a constraint on the
neutronics parameter $\eta$, i.e., the number of neutrons produced per fission
per absorption in the fuel. It is a material-specific value, which is to say
that it is a function only of material properties of the fuel, rather than the
reactor geometry or the material properties of the fuel or the moderator, as are
most other neutronics-related parameters, e.g., the effective multiplication
factor, $k_{eff}$. This parameter is ideal for an initial attempt at this
formulation because it requires only the minimal information associated with
material requests in \Cyclus, i.e., a quantity and an isotopic profile. The
equation for $\eta$ of a homogenous medium is as follows.

\begin{equation}
\label{eqs:eta_macro}
\eta = \frac{\sum_{i \in I} \nu^{i} \Sigma_{f}^{i}}
            {\sum_{i \in I} \Sigma_{a}^{i}}
\end{equation}

In Equation \ref{eqs:eta_macro}, $I$ is the set of isotopes in the material,
$\nu^{i}$ is the average neutrons produced per fisison of isotope $i$,
$\Sigma_{f}^{i}$ is the macroscopic cross section for fission of isotope $i$,
and $\Sigma_{a}^{i}$ is the macroscopic cross section for absorption of isotope
$i$. The ratio of the two cross sections produces the probability that a neutron
causes fission given that is absorbed by isotope $i$. For our purposes, because
we do not know how much of each material will contribute to a final mixture, it
is more ideal to work with microscopic cross sections and number densities of an
isotope $i$, $N^{i}$, as shown in Equation \ref{eqs:eta_micro}.

\begin{equation}
\label{eqs:eta_micro}
\eta = \frac{\sum_{i \in I} \nu^{i} \sigma_{f}^{i} N^{i}}
            {\sum_{i \in I} \sigma_{a}^{i} N^{i}}
\end{equation}

In order to discuss this in terms of barrel fractions, we must include them as
multiples of the number densities. 

\begin{equation}
\label{eqs:eta_fractions_nonlin}
\eta_{B} = \frac{\sum_{i \in I_{B}} \nu^{i} \sigma_{f}^{i} \sum_{b \in B} N_{b}^{i} x_{b}}
                {\sum_{i \in I_{B}} \sigma_{a}^{i} \sum_{b \in B} N_{b}^{i} x_{b}}
\end{equation}

In Equation \ref{eqs:eta_fractions_nonlin}, $I_{B}$ denotes the set of all
isotopes in barrels, $N_{b}^{i}$ denotes the number density of isotope $i$ in
barrel $b$, and $\eta_{B}$ is the reproduction factor resulting from the
specific values of barrel fractions $x_{b}$. If we were to leave the definition
of $\eta_{B}$ in this form, we could formulate the constraint in
Equation \ref{eqs:rap_eta} in the same way as is done in
Equation \ref{eqs:rap_mass}.

\begin{equation}
\label{eqs:eta_nonlin}
\epsilon_{\eta} \geq \left| 
\frac{\sum_{i \in I_{B}} \nu^{i} \sigma_{f}^{i} \sum_{b \in B} N_{b}^{i} x_{b,r}}
     {\sum_{i \in I_{B}} \sigma_{a}^{i} \sum_{b \in B} N_{b}^{i} x_{b,r}} 
- \eta_{r} \right|
\: \forall \: r \in R
\end{equation}

$\eta_{r}$ in Equation \ref{eqs:eta_nonlin} is defined in the normal way.

\begin{equation}
\label{eqs:eta_r}
\eta_{r} = \frac{\sum_{i \in I_{r}} \nu^{i} \Sigma_{f}^{i}}
                {\sum_{i \in I_{r}} \Sigma_{a}^{i}}
\end{equation}

Furthermore, $\eta_{r}$ is presumed to be predefined and is a constant with
respect to the RAP. Equation \ref{eqs:eta_nonlin} presents a problem; it
introduces a ratio of decision variables, which is a nonlinearity. In general,
nonlinear programs are much more difficult to solve than linear programs and
require much more exotic techniques. Some algebra can relieve this issue,
however. We begin by massaging Equation \ref{eqs:eta_nonlin}, reformulating it
into Equation \ref{eqs:eta_fractions_nonlin}.

\begin{equation}
\label{eqs:eta_fractions_lin}
\eta_{B} = \frac{\sum_{b \in B} x_{b} \sum_{i \in I_{b}} \nu^{i} \sigma_{f}^{i} N_{b}^{i}}
                {\sum_{b \in B} x_{b} \sum_{i \in I_{b}} \sigma_{a}^{i} N_{b}^{i}}
\end{equation}

Note that out of Equation \ref{eqs:eta_fractions_nonlin} fall two parameters
that are defined for each barrel, which I denote $\eta_{b}^{+}$ and
$\eta_{b}^{-}$.

\begin{equation}
\label{eqs:eta_+}
\eta_{b}^{+} \equiv \sum_{i \in I_{b}} \nu^{i} \sigma_{f}^{i} N_{b}^{i}
\end{equation}

\begin{equation}
\label{eqs:eta_-}
\eta_{b}^{-} \equiv \sum_{i \in I_{b}} \sigma_{a}^{i} N_{b}^{i}
\end{equation}

Equations \ref{eqs:eta_+} and \ref{eqs:eta_-} allow us to write
Equation \ref{eqs:eta_fractions_lin} more simply.

\begin{equation}
\label{eqs:eta_simple}
\eta_{B} = \frac{\sum_{b \in B} \eta_{b}^{+} x_{b}}
                {\sum_{b \in B} \eta_{b}^{-} x_{b}}
\end{equation}

We can then rewrite Equation \ref{eqs:eta_nonlin} using
Equation \ref{eqs:eta_simple}.

\begin{equation}
\label{eqs:eta_nonlin_simple}
\epsilon_{\eta} \geq \left| 
\frac{\sum_{b \in B} \eta_{b}^{+} x_{b_r}}
     {\sum_{b \in B} \eta_{b}^{-} x_{b_r}}
- \eta_{r} \right|
\: \forall \: r \in R
\end{equation}

Finally, we can multiply Equation \ref{eqs:eta_nonlin_simple} through by the
denominator of Equation \ref{eqs:eta_simple} to arrive at the form of the
constraint in \ref{eqs:rap_eta}.

\begin{equation}
\label{eqs:eta_linear}
\epsilon_{\eta} \sum_{b \in B} \eta_{b}^{-} x_{b,r} \geq
\left| \sum_{b \in B} \eta_{b}^{+} x_{b,r}
- \eta_{r} \sum_{b \in B} \eta_{b}^{-} x_{b,r} \right|
\: \forall \: r \in R
\end{equation}

The constraint in Equation \ref{eqs:eta_linear} has been linearized, and the
polynomial-time Simplex Method can again be applied as the solution mechanism. 

Finally, let me discuss Equations \ref{eqs:rap_conserv} and \ref{eqs:rap_x}.
Equation \ref{eqs:rap_conserv} simply states that the sum of all barrel
fractions for any given barrel must be less than one. This is simply a check on
reality -- you can't take more things from a container than exist in the
container in the first place. Equation \ref{eqs:rap_x} simply defines the notion of
a fraction, saying that all barrel fractions must lie in the range of 0 and 1,
inclusive.
