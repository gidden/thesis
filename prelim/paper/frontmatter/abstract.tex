The nuclear fuel cycle describes the process of the evolution of nuclear fuel
from mining to eventual interment. It is a complex system comprised of supplier
and consumer facilities which act in a global context. It is cyclical, i.e.,
fuel is not simply used and disposed of in all cases; the notion of the
recycling and reuse of fuel must be taken into account. Further complicating the
analysis of the nuclear fuel cycle are competing metrics, such as overall cost,
repository capacity, support facility capacity, etc. Nuclear fuel cycle
simulators have been developed in order to ascertain the effect of different
fuel cycle options, such as fuel types, reactor types, and technology
availability. Previous simulation technologies have relied primarily on pure
system dynamics approaches. This work describes an agent-based modeling approach
that allows increased modeling flexibility through the use of dynamically
loadable agent actors. A linear programming and mixed integer-linear programming
variation of the multi-commodity transportation problem is proposed to be used
in the simulation framework to facilitate the matching of suppliers and
consumers of commodities. Modeling of recycling scenarios is achieved through a
proposed approximation linear programming formulation with domain-specific
constraints, allowing separated material to be matched to multiple target fuel
requests. The combination of the market resolution mechanism with an automated
recipe matching algorithm placed in the framework of the \Cyclus simulation
provides a first-of-the-kind application of a dynamic, agent-based model of the
nuclear fuel cycle.