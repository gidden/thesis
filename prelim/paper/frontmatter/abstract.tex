%\svnidlong{$LastChangedBy$}{$LastChangedRevision$}{$LastChangedDate$}{$HeadURL: http://freevariable.com/dissertation/branches/diss-template/frontmatter/abstract.tex $}
%\vcinfo{}

The nuclear fuel cycle is a complex system of supplier and consumer facilities
which act in a global context. Furthermore, it is a cycle, i.e., the fuel is not
simply used and disposed of in all cases; the notion of recycling must be taken
into account. Further complicating the analysis of the nuclear fuel cycle are
competing metrics, such as overall cost, repository capacity, support facility
capacity, etc. In order to ascertain the effect of different fuel cycle options,
such as fuel types, reactor types, and technology availability, nuclear fuel
cycle simulators have been developed. Previous simulation technologies have
relied mainly on system dynamics approaches. This work describes an agent-based
modeling approach that allows increased flexibility of modeling through
dynamically loadable agent actors. A mixed-integer programming version of the
multi-commodity transportation problem is proposed to be used in the simulation
framework to facilitate the matching of suppliers and consumers of
commodities. A revised approach for matching separated materials to multiple
target fuel requests is also presented and examples of its use are shown. The
model used is a linear programming approximation technique with dynamically
appendable constraints. The combination of this market resolution mechanism with
an automated recipe matching algorithm placed in the framework of the \Cyclus
simulation provides a first-of-the-kind application of agent-based modeling of
the nuclear fuel cycle.
