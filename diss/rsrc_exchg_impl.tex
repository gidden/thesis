The resource exchange model implemented in Cyclus is designed to allow for the
possibly iterative communication between agents regarding the quality-specific
supply and demand of resources. That supply, demand, and associated cost-like
parameters are then translated into a bipartite graph of supplier nodes,
consumer nodes, and arcs connecting each. A solver is then applied to the graph,
providing flows along arcs as output. Finally, the model back translates the arc
flows into trades to be executed the Cyclus resource exchange framework.

The resource exchange process has four classes of participants and three
distinct interface layers, between each class of participant. The first layer
exists between agents interacting in the exchange and the kernel's exchange
manager. The exchange manager executes each phase of the pre-exchange process,
the request for bids (RFB) phase, response to request for bids (RRFB) phase, and
preference adjustment (PA) phase. The PA phase acts to effectively set costs for
resource exchange over defined arcs. The second layer exists between the
kernel's exchange manager, which is distinct for distinct resource types, and
the associated generic exchange graph. It defines the interface of communication
required for the translation of an instance of a resource-type defined exchange
into an instance of a resource-neutral exchange. The third interface layer
exists the generic resource exchange instance and a given instance of a resource
exchange solver. It defines the interface of communication that allows for the
translation of a resource exchange graph into the solver's problem instance
language and the back translation from the solver's solution into matched
resource flow over arcs on the generic resource exchange graph. Similarly, the
defined flows are back translated from the generic resource exchange into the
resource-specific exchange, which form trades to be executed by the kernel's
exchange manager.

% include execution pipeline image

The resource-specific interface is defined by four primary objects:
\begin{enumerate}
  \item \texttt{Request}
  \item \texttt{RequestPortfolio}
  \item \texttt{Bid}
  \item \texttt{BidPortfolio}
\end{enumerate}


\begin{subequations}\label{eqs:GFCTP-E}
  \begin{align}
    %%
    \label{eq:GRCTP-E_obj}
    \min_{z} \:\: 
    & 
    z = \sum_{h \in H}\sum_{i \in I_{p}}\sum_{j \in J_{p}}c_{i,j}^{h} x_{i,j}^{h} 
    + \sum_{h \in H}\sum_{i \in I_{p}}\sum_{j \in J_{e}}c_{i,j}^{h} y_{i,j}^{h} \tilde{x}_{i,j}^{h} \nonumber \\
    & 
    \qquad\qquad\qquad\qquad
    + \sum_{h \in H}\sum_{i \in I_{e}}\sum_{j \in J_{p}}c_{i,j}^{h} y_{i,j}^{h} \tilde{x}_{i,j}^{h}
    + \sum_{h \in H}\sum_{i \in I_{e}}\sum_{j \in J_{e}}c_{i,j}^{h} y_{i,j}^{h} \tilde{x}_{i,j}^{h}
    && \\
    %%
    \label{eq:GRCTP-E_sup}
    &
    \text{s.t.} \:\: 
    \sum_{j \in J_{p}}\beta_{i,k}(q_{j}^{h}) x_{i,j}^{h}
    + \sum_{j \in J_{e}}\beta_{i,k}(q_{j}^{h}) y_{i,j}^{h} \tilde{x}_{i, j}^{h} \leq s_{i,k}^{h} \nonumber \\
    &
    \qquad\qquad\qquad\qquad
    \forall \: i \in I, \: \forall \: k \in K_{i}^{h}, \forall \: {h \in H}\\
    %%
    \label{eq:GRCTP-E_dem}
    &
    \sum_{h \in H}\sum_{i \in I_{p}}\beta_{j,k}(q_{j}^{h}) x_{i,j}^{h}
    + \sum_{h \in H}\sum_{i \in I_{e}}\beta_{j,k}(q_{j}^{h}) y_{i,j}^{h} \tilde{x}_{i, j}^{h} \leq d_{j,k} \nonumber \\
    &
    \qquad\qquad\qquad\qquad
    \forall \: j \in J, \: \forall \: k \in K_{i}\\
    %%
    \label{eq:GRCTP-E_sumy_j}
    &
    \sum_{h \in H}\sum_{i \in I} y_{i,j}^{h} = 1
    &
    \forall \: j \in J_{e}  &\\
    %%
    \label{eq:GRCTP-E_sumy_i}
    &
    \sum_{h \in H}\sum_{j \in J} y_{i,j}^{h} = 1
    &
    \forall \: i \in I_{e}  &\\
    %%
    \label{eq:GRCTP-E_x}
    &
    x_{i,j}^{h} \geq 0
    &
    \forall \: x \in X  &\\
    %%
    \label{eq:GRCTP-E_y}
    &
    y_{i,j}^{h} \in \{0,1\}
    &
    \forall \: y \in Y &
    %%
  \end{align}
\end{subequations}
