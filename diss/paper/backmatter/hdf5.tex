\chapter{Cyclopts HDF5 Database Layout}\label{app:hdf5}


\subsubsection{Parameter Space}

Both front-end and back-end species record the state of every point in a given
parameter space in a data set called \code{/Species/<species type>/Points},
where \code{<species type>} is either \code{StructuredRequest} or
\code{StructuredSupply}. Each point incorporates both fundamental and instance
parameters as described in \S \ref{method:setup}. 

\subsubsection{Problem Instances}

Problem instances are generated by problem species and are executed by problem
families. Accordingly, both species and families can record information about
instances. Front and back-end exchange species each record two types of
information: details about each arc in an instance and a summary of
species-specific information. The exchange family records information regarding
each of the entities that comprise an instance: nodes, groups of nodes (having
been translated from portfolios), and arcs. Further, aggregate summary
information is also recorded. 

\paragraph{Exchange Species}

Both exchange species record information about each arc in an exchange instance
detailed in Table \ref{tbl:sp_inst_arc}. A parent group for arc data is defined
under each species group. A group for each instance, whose name is the hex
string of the UUID, is defined under the associated arc group. Finally, arc
information associated with each instance is stored as a dataset in that
instance's group. For example, the arc data for a given UUID of a front-end
exchange is located as a dataset in the group
\code{/Species/StructuredRequest/Arcs/<UUID hex>}.

\begin{table}[]
\centering
\label{tbl:sp_inst_arc}
\caption{Data recorded for every arc in an instance.}
\begin{tabular}{|c|c|c|}
\hline
\textbf{Name} & \textbf{Data Type} & \textbf{Description}       \\ \hline
arcid         & 32-byte integer    & ID for an arc              \\ \hline
commod        & 32-byte integer    & ID for a commodity         \\ \hline
pref\_c       & 32-byte float      & Commodity-based preference \\ \hline
pref\_l       & 32-byte float      & Location-based preference  \\ \hline
\end{tabular}
\end{table}

Summary information related to each species is also recorded in a data set for
each species type located in the group \code{/Species/<species
  type>/Summary}. Table \ref{} describes the data recorded for front-end
exchanges, and Table \ref{} describes the data recorded for back-end exchanges.

\TODO{write both tables}

\paragraph{Exchange Family}

The exchange family records information regarding all major constructs in an
exchange: nodes, groups, and arcs. Nodes and group data are recorded in an
aggregate dataset located at \code{/Family/ResourceExchange/ExchangeNodes}. A
summary of the node datastructure is given in Tables \ref{}; the node group
data, located at \code{/Family/ResourceExchange/ExchangeGroups}, is summarized
in Table \ref{}. Arc data is collected in the group
\code{/Family/ResourceExchange/ExchangeArcs}. A dataset per instance UUID is
used because it has been found to be useful in the postprocessing phase. A
summary of the arc dataset structure is provided in Table \ref{}.

\TODO{Table for each}

\subsubsection{Solutions}

For every solution, data is added to the Cyclopts Results dataset. The structure
of the Results dataset is described in Table \ref{}. Problem solutions are
constructed from problem instances, and are thus managed by a problem
family. Aggregate solution information is provided in a family dataset
\code{/Family/ResourceExchange/ExchangeSolutionProperties}, the structure of
which is detailed in Table \ref{}. The full results of each solve, i.e., the
amount of resources flowing across each arc, are recorded in a group specific to
each solution UUID. The structure of the detailed results are provided in Table
\ref{}.

\TODO{Table for each}

\subsubsection{Post Processing}

Given a full set of parameter, instance, and solution data, postprocessing may
be applied to family and species data. The exchange family, front-end species,
and back-end species each contain a \code{PostProcess} dataset. The exchange
family dataset structure is described in Table \ref{}. Both exchange species
share the same dataset structure, outline in Table \ref{}.

\TODO{Table for each}

\subsubsection{Summary Database Layout}

The database hierarchical structure for exchange data, assuming both exchange
species are included, is shown in Figure \ref{}.
