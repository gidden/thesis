% preamble.tex -- packages to include
%
% Wisconsin dissertation template
% Copyright (c) 2008 William C. Benton.  All rights reserved.
%
% This program can redistributed and/or modified under the terms
% of the LaTeX Project Public License Distributed from CTAN
% archives in directory macros/latex/base/lppl.txt; either
% version 1 of the License, or (at your option) any later version.
%
% This program includes other software that is licensed under the
% terms of the LPPL and the Perl Artistic License; see README for details.
%
% You, the user, still hold the copyright to any document you produce
% with this software (like your dissertation).

%% Comment out any of these that you don't want
\usepackage{amssymb}
\usepackage{amsmath}
\usepackage{amsthm}
%\usepackage{theorem}
\usepackage{hyperref}

%%%%% LISTINGS package and setup
\IfFileExists{listings.sty}{%
\usepackage{listings}%
}{}

%% Packages added by Matt Gidden
%%
%% Note, order matters!
%%
\usepackage{cite} % right order for multiple entries in cite
\usepackage{moreverb} % for verbatim snippets of code
\usepackage{fancyvrb}
\usepackage{tabularx} % for tables with line breaks
\usepackage{threeparttable} % for tables with notes
\usepackage[capitalize, noabbrev]{cleveref} % for reference multiple figures
\usepackage{calc} % allows for arithmetic on latex variables
\usepackage{float} % allows for figures to be placed explicitly
\usepackage{algorithm2e} % for algorithms
\usepackage{subfig}
\usepackage{multirow} % combining rows in tables
\usepackage{footnote}
\usepackage{titlesec} % for using \titleformat
\usepackage{slashbox} % for tables with a divided cell, see http://tex.stackexchange.com/questions/7262/diagonally-divided-table-cell
\usepackage{bashful}
\usepackage{xspace}
\usepackage{color}
\definecolor{listinggray}{gray}{0.9}
\definecolor{lbcolor}{rgb}{0.9,0.9,0.9}
\lstset{
    %backgroundcolor=\color{lbcolor},
    language={C++},
    tabsize=4,
    rulecolor=\color{black},
    upquote=true,
    aboveskip={1.5\baselineskip},
    belowskip={1.5\baselineskip},
    columns=fixed,
    extendedchars=true,
    breaklines=true,
    prebreak=\raisebox{0ex}[0ex][0ex]{\ensuremath{\hookleftarrow}},
    frame=single,
    showtabs=false,
    showspaces=false,
    showstringspaces=false,
    basicstyle=\scriptsize\ttfamily\color{black},
    keywordstyle=\color[rgb]{0,0,1.0},
    commentstyle=\color[rgb]{0.133,0.545,0.133},
    stringstyle=\color[rgb]{0.627,0.126,0.941},
    numberstyle=\color[rgb]{0,1,0},
    identifierstyle=\color{black},
    captionpos=t,
}
\lstdefinestyle{BashOutputStyle}{
  basicstyle=\small\ttfamily,
  numbers=none,
  frame=tblr,
  columns=fullflexible,
  backgroundcolor=\color{blue!10},
  linewidth=0.9\linewidth,
  xleftmargin=0.1\linewidth
}
%
% Adding for some table features
\usepackage{array}
%
% Adding package for float barriers
\usepackage{placeins}

%% Custom commands added by Matt Gidden
\setcounter{tocdepth}{2}
\setcounter{secnumdepth}{3}
\theoremstyle{plain}
\newcommand{\Cyclopts}{\textsc{Cyclopts} }
\newcommand{\Cycamore}{\textsc{Cycamore} }
\newcommand{\cyclus}{\textsc{Cyclus}\xspace}
\newcommand{\Cyclus}{\textsc{Cyclus} }
\newcommand{\nucl}[2]{
\ensuremath{^{#1}}\mbox{#2}
}
\newcommand{\horizfig}[2][]{%
  \begin{minipage}{3in}\subfloat[#1]{#2}\end{minipage}}
%% \newcommand{\code}[1]{\lstinline[basicstyle=\ttfamily\color{green!40!black}]|#1|}
\newcommand{\code}[1]{\lstinline[basicstyle=\ttfamily\color{black}]|#1|}
\newcommand{\codeb}[1]{\texttt{#1}}
\newcommand{\units}[1] {\:\text{#1}}%
\newcommand{\citeme}{\textcolor{red}{CITE}\xspace}
\newcommand{\TODO}[1] {{\color{red}\textbf{TODO: #1}}}%
\newcommand{\Reactor}[1]{\texttt{Reactor{#1}}}
\newcommand{\UOXSource}{\texttt{UOX\_Source}}
\newcommand{\MOXSource}{\texttt{MOX\_Source}}
\newcommand{\Enrichment}{\texttt{Enrichment}}
\newcommand{\ffc}{$f_{\text{fc}}$\xspace}
\newcommand{\frx}{$f_{\text{rx}}$\xspace}
\newcommand{\floc}{$f_{\text{loc}}$\xspace}
\newcommand{\dloc}{$\delta_{\text{loc}}$\xspace}
\newcommand{\dreg}{$\delta_{\text{reg}}$\xspace}
\newcommand{\cbc}{Cbc\xspace}
\newcommand{\clp}{Clp\xspace}
% detect beginning of sentences and capitalize appropriately
\sfcode`\.=1001
\sfcode`\?=1001
\sfcode`\!=1001
\sfcode`\:=1001
\newcommand\secref[1]{\ifnum\spacefactor=1001 Section \ref{#1}\else section \ref{#1}\fi}
% this seems like a limitation of sfcode (an error occurs if at the beginnig of
% a paragraph) Accordingly, this can be used manually as
% needed. http://comments.gmane.org/gmane.comp.tex.texhax/17631
\newcommand\Secref[1]{ Section \ref{#1}}


%% You should use natbib
\IfFileExists{natbib.sty}{%
  \usepackage[numbers]{natbib}% added the numbers option from the original WI
                              % thesis template
}{}

%% You probably need appendix, if you want appendices
\IfFileExists{appendix.sty}{%
\usepackage{appendix}%
}{}

%% the spacing in memoir is weird, you'll need to use this
\DisemulatePackage{setspace}
\usepackage[onehalfspacing]{setspace}

%% List setup; the ``hanglist`` environment will allow you to have
%% nicely-typeset enumerated lists (i.e., with the numbers hanging in
%% the margins).  You need at least version 2.1 of enumitem.sty.  If
%% you don't have enumitem installed at all, hanglist will just be an
%% alias for enumerate.
\IfFileExists{enumitem.sty}{%
\usepackage[loadonly]{enumitem}[2007/06/30]%
\newlist{hanglist}{enumerate}{1}% 
\setlist[hanglist]{label=\arabic*.}%
\setlist[hanglist,1]{leftmargin=0pt}%
}{%
\newenvironment{hanglist}{\begin{enumerate}}{\end{enumerate}}%
}

\IfFileExists{mathpartir.sty}{%
\usepackage{mathpartir}%
}{}

%% \newtheorem{thm}{Theorem}[chapter] % reset theorem numbering for each chapter
%% \theoremstyle{definition}
%% \newtheorem{defn}[thm]{Definition} % definition numbers are dependent on theorem numbers
%% \newtheorem{exmp}[thm]{Example} % same for example numbers


%% Get rid of ugly borders around PDF hyperlinks (e.g., for cross-references, bib entries, or URLs)
\hypersetup{pdfborder = 0 0 0}

%% You want microtype.
\IfFileExists{microtype.sty}{%
\usepackage[protrusion=true,expansion=true]{microtype}%
}{}

%\pagestyle{thesisdraft}

% Surround parts of graphics with box
\usepackage{boxedminipage}

%% booktabs (thx to Nate Rosenblum for bringing this beautiful package
%% to my attention)
\IfFileExists{booktabs.sty}{%
\usepackage{booktabs}%
}{}

% This is now the recommended way for checking for PDFLaTeX:
\usepackage{ifpdf}

%% Avoid ugly "Type 3" fonts
\usepackage{lmodern}
\usepackage[LY1]{fontenc}

%% Substitute your favorite serif and sans fonts here....
\IfFileExists{tgpagella.sty}{%
% TeX Gyre pagella, like Palatino
\usepackage{tgpagella}%
}{}

%% overrides the default Latex math font with AMS Euler
%\usepackage[LY1]{eulervm} 

\ifpdf
\usepackage[pdftex]{graphicx}
\else
\usepackage{graphicx}
\fi

\usepackage{makeidx}
\makeindex

{\theoremstyle{plain}
\newtheorem{thm}{Theorem}[chapter]
\newtheorem{cor}[thm]{Corollary}
\newtheorem{define}[thm]{Definition}
\newtheorem{exmpl}[thm]{Example}
}
{\theoremstyle{remark}
\newtheorem{rmk}[thm]{Remark}
}

\newtheoremstyle{customsty1}
{3pt}%
{3pt}%
{}% --- body font
{}% --- indent amount
{\bfseries}% --- Theorem head font
{:}% --- Punctuation after head
{.5em}% --- space after head
{}% --- theorem head spec (can be left empty, meaning 'normal')

% Define 'newtheorems' that use ``customsty1''
{\theoremstyle{customsty1} 
}


%%% NB: the ``deposit'' chapter- and page- styles should conform to UW
%%% requirements.  If you are producing a pretty version of your
%%% dissertation for web use later, you will certainly want to make
%%% your own chapter and page styles.

\makechapterstyle{deposit}{%
  \renewcommand{\chapterheadstart}{}
  \renewcommand{\printchaptername}{}
  \renewcommand{\chapternamenum}{}
  \renewcommand{\printchapternum}{\parbox{2em}{\MakeLowercase{\Large\scshape\thechapter{}}} }
  \renewcommand{\afterchapternum}{}
  \renewcommand{\printchaptertitle}[1]{%
  \raggedright\Large\scshape\MakeLowercase{##1}}
  \renewcommand{\afterchaptertitle}{%
  \vskip\onelineskip \hrule\vskip\onelineskip}
}

\makepagestyle{deposit}
 
\makeatletter
 
\renewcommand{\chaptermark}[1]{\markboth{#1}{}}
\renewcommand{\sectionmark}[1]{\markboth{#1}{}}
 
\makeevenfoot{deposit}{}{}{}
\makeoddfoot{deposit}{}{}{}
\makeevenhead{deposit}{\thepage}{}{}
\makeoddhead{deposit}{}{}{\thepage}
\makeatother

%%% set up page numbering for chapter pages to satisfy UW requirements
%%% NB: You will want to delete until the ``SNIP'' mark if you are
%%% making a ``nice'' copy
\copypagestyle{chapter}{plain}
\makeoddfoot{chapter}{}{}{}
\makeevenhead{chapter}{\thepage}{}{}
\makeoddhead{chapter}{}{}{\thepage}
%%% SNIP

%%% bib nonsense
\makeatletter
\newenvironment{wb-bib}[1]{%
  \chapter*{references}
\ifnobibintoc\else 
\phantomsection 

\addcontentsline{toc}{chapter}{References } 
\fi 
\prebibhook
  \begin{bibitemlist}{#1}}{\end{bibitemlist}\postbibhook}

\AtBeginDocument{%
  \@ifpackageloaded{natbib}{% natbib is loaded
    \addtodef{\endthebibliography}{}{\vskip-\lastskip\postbibhook}
    \@ifpackagewith{natbib}{sectionbib}{% with sectionbib option
      \renewcommand{\bibsection}{\@memb@bsec}}%
      {\renewcommand{\bibsection}{\@memb@bchap}}}%
  {}
  \@ifpackagewith{chapterbib}{sectionbib}{%
    \renewcommand{\sectionbib}[2]{}
    \renewcommand{\bibsection}{\@memb@bsec}}{}
}
\makeatother
