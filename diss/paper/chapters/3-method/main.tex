\chapter{Methodolgy of Experimentation}\label{ch:method}

\section{Overview}

The NFCTP is the first attempt at solving the supply and demand of the nuclear
fuel cycle in a dynamic manner within a NFC simulation. Accordingly, there is no
precedent for either generating exchanges to be solved outside of simulations or
exploring the resulting performance. A novel methodology for exploring the
performance of a general NFCTP solver is described.

The chapter begins with a discussion of the generation of exchanges in \S
\ref{method:setup}. Two types of exchanges are included: one in which reactors
are requesting fuel, and one in which reactors are supplying used fuel. In NFC
parlance, these are called the \textit{Front End} of the fuel cycle and
\textit{Back End} of the fuel cycle. Notably, both of these exchanges can occur
in the same time step. \S \ref{method:setup:split} describes how a full exchange
may be split into multiple exchanges for any given time step. The solvers used
to determine solutions to the DRE instances are then described in \S
\ref{method:setup:solve}.

Generating instances and solving them at a large scale is a difficult
problem. The Cyclopts (Cyclus Optimization Studies) framework was implemented
for this purpose. Cyclopts has both a Python and C layer. The Python layer is
largely responsible for generating exchanges and interfacing with an associated
persistence mechanism. The C layer is linked agaisnt the Cyclus kernel shared
object library and is responsible for calling directly into the kernel's
resource exchange API. \S \ref{method:tools} describes the implementation of
Cyclopts and its varied modes of operation. The structure and design is
described in \S \ref{method:tools:struc}. A persistence mechnism using a
Hierarchical Database Format-5 (HDF5) \cite{hdf5} and associated design studies
is discussed in \S \ref{method:tools:hdf5}. Finally, the use of High Throughput
Computing (HTC), specifically HTCondor \cite{condor-practice}, is addressed in
\S \ref{method:tools:htc}.

\section{Experimental Setup}\label{method:setup}

\subsection{Splitting Exchanges}\label{method:setup:split}

\subsection{Front-End Exchanges}\label{method:setup:front}

\subsection{Back-End Exchanges}\label{method:setup:back}

\subsection{Solvers}\label{method:setup:solve}

\section{Experimental Tools}\label{method:tools}

\subsection{Structure}\label{method:tools:struc}

\subsection{Persistence Mechanisms}\label{method:tools:hdf5}

\subsection{High Throughput Computing}\label{method:tools:htc}

Things \cite{bui_work_2011}.
