
\section{Generating Exchanges}\label{method:setup}

Instances of resource exchanges are required to analyze the effects and
performance of the NFCTP formulation and its solvers. In the absence of large
Cyclus simulations with interesting facility and relationship models, instances
must be generated given some set of rules and parameters. Two distinct
\textit{species} of exchanges are generated, those related to the \textit{front
  end} of the nuclear fuel cycle and those related to the \textit{back end} of
the nuclear fuel cycle. Broadly, the front end of the fuel cycle is concerned
with fueling reactors, and the back end is concerned with either recycling or
disposing of used fuel exiting reactors. Common features to both types of
exchange generation are described in \secref{method:setup:features}.

Previously, \secref{abm:dre}, described the methodology for solving a single
exchange. If a large exchange is separable, i.e., it can be completely separated
into two or more smaller exchanges, sub-exchanges can be solved
independently. \secref{method:setup:split} provides an argument for why it is
valid to split exchange instances into, at minimum, the front and back ends of
the NFC.

Exchange generation, absent full-scale simulation, is a naturally parameterized
process. Some generation parameters are common to any NFCTP instance, and are
described in \secref{method:setup:params}. Specific species may additionally
define their own set of parameters. \secref{method:setup:front} describes the
parameters generation methodology associated with front-end exchanges. \S
\ref{method:setup:back} follows with a similar discussion for back-end
exchanges.  

\subsection{Common Features}\label{method:setup:features}

\subsubsection{Fuel Cycles and Commodities}

Three types of fuel cycles are generated: a once-through fuel cycle, labeled
OT; a plutonium-recycle fuel cycle, labeled MOX; and a plutonium and
thorium-recycle fuel cycle, labeled MOX-ThOX. As fuel cycles increase in
complexity, the number of commodities that exist increases, as shown in Table
\ref{tbl:fc_to_commods}. The commodities are referred to by abbreviation:
Uranium Oxide (UOX), Mixed Plutonium Oxide for Thermal Reactors (TMOX), Mixed
Plutonium Oxide for Fast Reactors (FMOX), Thorium Oxide for Fast Reactors
(FThOX).

\begin{table}[h!]
\centering
\caption{A mapping between fuel cycles to the commodities that exist in each one.}
\label{tbl:fc_to_commods}
\begin{tabular}{|c|c|}
\hline
\textbf{Fuel Cycle}            & \textbf{Commodities} \\ \hline
OT                    & UOX         \\ \hline
\multirow{3}{*}{MOX}  & UOX         \\  
                      & TMOX        \\  
                      & FMOX        \\ \hline
\multirow{4}{*}{ThOX} & UOX         \\  
                      & TMOX        \\  
                      & FMOX        \\  
                      & FThOX       \\ \hline
\end{tabular}
\end{table}

\subsubsection{Reactors}

Reactors are modeled as either thermal or fast reactors. It is necessary to
estimate the amount of fuel exchanged by reactors each time step. Accordingly,
thermal reactors are simplified models of AP-1000 reactors \cite{ARIS}, and fast
reactors are simplified models of BN-600 reactors
\cite{reactors2007experience}. Using the dimensions in Table
\ref{tbl:rx_params}, one can estimate that the AP-1000 core volume is
approximately 12.5 times larger than the BN-600 core.

\begin{table}[h!]
\centering
\caption{Primary Reactor Parameters}
\label{tbl:rx_params}
\begin{tabular}{|c|c|c|c|}
\hline
\textbf{Reactor} & \textbf{Core Height (m)} & \textbf{Core Diameter (m)}
& \textbf{Number of Assemblies} \\ \hline
AP1000  & 4.27            & 3.04              & 157 \\ \hline
BN600   & 0.75            & 2.05              & 369 \\ \hline
\end{tabular}
\end{table}

Reactors operate in a batch mode, where each batch is approximately one quarter
of the reactor core, an assumption which similar to other analyses
\cite{rineiski2011reactivity}. Additionally, a single AP-1000 fuel assembly is
assumed to contain 450 kg of material \cite{kok2009nuclear}. Therefore, a single
batch of thermal reactor fuel is assumed to be

\begin{equation}
   450 \frac{kg}{assembly} * \frac{t}{1000 \: kg} * 157 \frac{assemblies}{core}
   * \frac{1}{4} core = \: \sim 17.6 t.
\end{equation}

\noindent
The amount of fuel required and number of assemblies by each reactor type is
shown in Table \ref{tbl:rx_batch}. The number of assemblies is taken as the
ratio of total number of assemblies and number of batches per core rounded to
the nearest integer. The batch size for the BN600 reactor is estimated by
dividing the AP1000 batch size by the relative core volume.

\begin{table}[h!]
\centering
\caption{Reactor Batch Size}
\label{tbl:rx_batch}
\begin{tabular}{|c|c|c|c|}
\hline
\textbf{Reactor Type} & \textbf{Quantity (t)} & \textbf{Number of Assemblies}
\\ \hline AP1000 & 17.6 & 39 \\ \hline BN600 & 1.41 & 92 \\ \hline
\end{tabular}
\end{table}

The reactors that operate in a given exchange is also a function of the fuel
cycle being modeled. In a OT fuel cycle, only thermal reactors exist. In the MOX
case, fast reactors that prefer MOX-based fuel are added and denoted as FMOX
reactors. Finally, in the MOX-ThOX case, an additional class of fast reactor is
added that prefers ThOX-based fuels and is denoted as FThOX. A summary of
available types of reactors as a function of the fuel cycle being modeled is
shown in Table \ref{tbl:fc_to_rxs}, and a summary of the reactor models used for
each reactor type is shown in Table \ref{tbl:model_to_rxs}. 

\begin{table}[h!]
\centering
\caption{A mapping between fuel cycles to the reactor types that exist in exchange instances.}
\label{tbl:fc_to_rxs}
\begin{tabular}{|c|c|}
\hline
\textbf{Fuel Cycle}            & \textbf{Reactor Types} \\ \hline
OT                    & Thermal         \\ \hline
\multirow{3}{*}{MOX}  & Thermal         \\  
                      & FMOX        \\ \hline
\multirow{4}{*}{ThOX} & Thermal         \\  
                      & FMOX        \\  
                      & FThOX       \\ \hline
\end{tabular}
\end{table}

\begin{table}[h!]
\centering
\caption{A mapping between surrogate reactor models and reactor types.}
\label{tbl:model_to_rxs}
\begin{tabular}{|c|c|}
\hline
\textbf{Reactor Model}            & \textbf{Reactor Types} \\ \hline
AP1000                    & Thermal         \\ \hline
\multirow{2}{*}{BN600}  & FMOX         \\  
                      & FThOX        \\ \hline
\end{tabular}
\end{table}

Reactors may be fueled by different fuel types, i.e., fuel commodities. The set
of commodities that reactors can use is a modeling assumption and a proxy for
how reactors may behave in simulations; this particular reactor-to-commodity
mapping may not be true for other analysts' fuel cycle models. However, it is
appropriate to make certain broad assumptions for such an exploratory study. A
mapping of reactors to acceptable commodities is provided in Table
\ref{tbl:rx_to_commods}. Note that there is still a preference distribution
associated with each reactor-commodity pair as well as constraint coefficient
effects. Accordingly, each reactor-commodity pair provides a unique effect on an
exchange instance. 

\begin{table}[h!]
\centering
\caption{A mapping between reactor types and the commodities allowed to fuel each reactor type.}
\label{tbl:rx_to_commods}
\begin{tabular}{|c|c|}
\hline
\textbf{Reactor Types}            & \textbf{Fuel Commodities} \\ \hline
\multirow{3}{*}{Thermal}                    & UOX         \\ 
                      & TMOX        \\  
                      & FMOX       \\ \hline
\multirow{4}{*}{FMOX}  & UOX         \\  
                      & TMOX        \\ 
                      & TMOX        \\  
                      & FThOX        \\ \hline 
\multirow{4}{*}{FThOX} & UOX         \\  
                     & TMOX        \\ 
                      & TMOX        \\  
                      & FThOX        \\ \hline 
\end{tabular}
\end{table}

Finally, each reactors in the system is representing an individual agent in a
given simulation. Agents are assumed to have some differentiating behavior. In
order to model this effect, each reactor is assigned a random variable, $x \in
[0, 1)$, that represents a unique measure of quality of fuel requested by
  reactors. This stochastic property is expounded upon further in \S
  \ref{method:setup:front} and \ref{method:setup:back}.

\subsubsection{Support Facilities}

In a front-end exchange, fuel suppliers exchange material with reactors. In a
back-end exchange, reprocessing and storage facilities exchange material with
reactors. In either case, facilities that are \text{not} reactors are referred
to as \textit{support} facilities, as they support the reactors which generate
power. Support facilities for front-end exchanges are described in \S
\ref{method:setup:front:sup}, and support facilities for back-end exchanges are
described in \secref{method:setup:back:sup}.

\subsubsection{Preferences}\label{method:setup:features:prefs}

Preferences for all transactions have a default value, $p_{c}(i, j)$, based on
the proposed commodity to be transferred between a supplier, $i$, and consumer,
$j$. However, a large exchange with a small preference distribution results in
problem degeneracy. Further, a primary application for Cyclus is the modeling of
regional and location effects on fuel cycles. Accordingly, a location proxy is
provided for preferences, as shown in Equation \ref{eqn:loc_proxy}, in order to
simulate both location-based preferences and non-degenerate exchange instances.

Preferences can also be a function of facility location. Each facility is
assigned a location value, $\text{loc}_i \in [0, 1)$. The domain is then
  divided evenly into ten regions, where the first region comprises all location
  values in $[0, 0.1)$, \textit{et cetera}. For example, a facility at location
    of $4.6$ is in the fifth region. \dreg and
    \dloc are binary variables which are activated based on the
    parameters described in \secref{method:setup:params}. If
    \dreg is zero, no location-based preferences are used. If
    \dloc is zero, only coarse, region-based preferences are
    used. In both cases, preferences are a function of the Euclidean distance
    between regional and location values. The inverse exponential functional
    form was chosen in order to model a preference gradient that decays as
    distance increases.

\begin{equation}\label{eqn:loc_proxy}
p_{l}(i, j) = \delta_{\text{reg}} 
\frac{\exp(- \lvert \text{reg}_{i} - \text{reg}_{j} \rvert ) + \delta_{\text{loc}}
  \exp(- \lvert \text{loc}_{i} - \text{loc}_{j} \rvert )}
     {1 + \delta_{\text{loc}}}
\end{equation}

The preference for a given arc is then a weighted, linear combination of
location and commodity preferences as shown in Equation
\ref{eqn:total_pref}. The weighting factor, $r_{l, c}$, is a parameter of
exchange generation and described further in \secref{method:setup:params}.

\begin{equation}\label{eqn:total_pref}
p(i, j) = p_{c}(i, j) + r_{l, c} p_{l}(i, j)
\end{equation}

\subsection{Splitting Exchanges}\label{method:setup:split}

A well known simplification of the Multicommodity Transportation Problem occurs
when supply and demand is separate for separate commodities. The large
multicommodity problem can then be decomposed into $n$ single commodity
subproblems, where $n$ is the number of commodities. Each subproblem can be
solved separately from the others.

An analog exists in the NFCTP when the Exchange Graph is \textit{separable}. A
bipartite graph with directed arcs, $A$, consisting of sending nodes, $U$, and
receiving nodes, $V$, is separable if there a partition

\begin{equation}
  A = A_{1} \cup A_{2}
\end{equation}

\begin{equation}
  U = U_{1} \cup U_{2}
\end{equation}

\begin{equation}
  V = V_{1} \cup V_{2}
\end{equation}

\noindent
such that no node in $U_1$ is connected to a node in $V_2$ and no node in $U_2$
is connected to a node in $V_1$. The graph shown in Figure \ref{fig:basic_part} is an
example of a separable bipartite graph.

\begin{figure}
  \begin{center}
    \includegraphics[width=0.55\textwidth]{exchange_part_supreq.pdf}
    \caption{
      \label{fig:basic_part}
      A separable bipartite graph with the partition shown as a red 
      dashed line.}
  \end{center}
\end{figure}

The Exchange Graph of the NFCTP, however, has additional structure in the form
of portfolios and thus has a stricter notion of separability. Specifically, the
partition must also separate the set of supplier portfolios, $S$, and requester
portfolios, $R$, as in Equations \ref{eqn:sup_part} and \ref{eqn:req_part},
respectively.

\begin{equation}\label{eqn:sup_part}
  S = S_{1} \cup S_{2}
\end{equation}

\begin{equation}\label{eqn:req_part}
  R = R_{1} \cup R_{2}
\end{equation}

Figure \ref{fig:port_part} depicts a separable Exchange Graph, for example,
while Figure \ref{fig:port_no_part} shows an Exchange Graph where the underlying
bipartite graph is separable, but full separability is broken by the overlaid
portfolio structure.

\begin{figure}
  \begin{center}
    \includegraphics[width=0.55\textwidth]{exchange_part_port.pdf}
    \caption{
      \label{fig:port_part}
      A separable Exchange Graph with nodes grouped by portfolio and the
      separating partition shown as a red dashed line.}
  \end{center}
\end{figure}

\begin{figure}
  \begin{center}
    \includegraphics[width=0.55\textwidth]{exchange_no_part_port.pdf}
    \caption{
      \label{fig:port_no_part}
      An Exchange Graph with nodes grouped by portfolio that is \textit{not}
      separable because a portfolio crosses the node partition.}
  \end{center}
\end{figure}

The Exchange Graph resulting from the information gathering phase of the DRE
will be minimally separable into front-end and back-end exchanges if two
conditions are true:

\begin{enumerate}
  \item Reactor output commodities can \textit{not} be sent to both other
    reactors and supporting facilities.

  \item Supporting facility output commodities can \textit{not} be sent to both
    other supporting facilities and reactors.
\end{enumerate}

In the first case, separability is broken by a supplier providing bids across a
separating partition. A minimal example is shown in Figure
\ref{fig:no_part_front}. This case can arise if reactors can somehow directly
refuel other reactors. In the NFC domain, such an arrangement only occurs in an
abstraction of a self-recycling system in which there is a dedicated recycling
complex associated with a fast reactor. It is reasonable for a self-recycling
reactor system to be implemented in such a way that it does not participate in
the DRE for self-refueling purposes. Accordingly, this condition is expected to
be met in most use cases of the DRE.

\begin{figure}
  \begin{center}
    \includegraphics[width=0.55\textwidth]{exchange_no_part_front.pdf}
    \caption{
      \label{fig:no_part_front}
      An Exchange Graph where separability broken by a supplier. This occurs in
      NFC modeling if assumption $1$ is broken.}
  \end{center}
\end{figure}

In the second case, separability is broken by a requester requesting commodities
across a separating partition. Again, a minimal example is shown in Figure
\ref{fig:no_part_back}. This case can arise in practice when modeling an NFC
system where both a reactor and a repository compete for some commodity. While
this is a valid modeling case under certain assumptions and simplifications, it
is not very realistic. In general fuel that can be used by a reactor has been
processed differently than material to be sent to a repository. If an instance
of a DRE does not meet this requirement, it will not be able to be subdivided
into smaller instances.

\begin{figure}
  \begin{center}
    \includegraphics[width=0.55\textwidth]{exchange_no_part_back.pdf}
    \caption{
      \label{fig:no_part_back}
      An Exchange Graph where separability broken by a requester. This occurs in
      NFC modeling if assumption $2$ is broken.}
  \end{center}
\end{figure}

Because the majority of fuel cycles analyzed will meet both conditions, most of
DRE instances will be able to be separated into at least two distinct instances
which can solved independently of one another. One instance will be associated
with the front end of the fuel cycle where reactors are requesting fuel. The
other instance will be associated with the back end of the fuel cycle, where
reactors are supplying used fuel. 

Separability is important to this work for two reasons. First, as described in
\secref{intro:prog}, an unmanageable problem instance that is separable can
result in two (or more) managable problem instances. If a truly inseparable
cycle is modeled in practice, and it is found to be an intractable problem,
heuristic solutions can be used. Second, because a given fuel cycle can be
separated into its front-end and back-end components, the remainder of this
works performs analyses on each component independently.

\subsection{Exchange Parameters}\label{method:setup:params}

The generation of exchanges requires a set of parameters. For instance, a
critical parameter is the number of reactors in an exchange. Exchange generation
parameters can be divided into two classifications, \textit{fundamental}
parameters and \textit{instance} parameters. All exchange species share some
fundamental parameters and instance parameters, a discussion of which is the
focus of this section. Species also define their own set of instance parameters
to complete the full set of parameters needed to define an instance of an
exchange.

\subsubsection{Fundamental Parameters}

The fundamental parameters are related to the common features of all instances
described in \secref{method:setup:features}. Each fundamental parameter is a
switch that sets the level of \textit{fidelity} of a given exchange. As such,
they are each denoted as $f_x$, where the $x$ subscript describes the parameter.

The most critical parameter is related to the ``fidelity'' of the fuel cycle
being modeled, \ffc. A value of zero indicates modeling the OT fuel cycle, one
is used for the MOX fuel cycle, and two the ThOX fuel cycle. The
parameter-to-fuel-cycle is summarized in Table \ref{tbl:ffc}. As fuel cycle
fidelity increases, the number of commodities increases, and thus the number of
possible connections between suppliers and consumers that exist increases,
because some entities trade in multiple commodities.

\begin{table}[h!]
\centering
\caption{A mapping between fuel cycles and \ffc values.}
\label{tbl:ffc}
\begin{tabular}{|c|c|}
\hline
\textbf{Fidelity (Fuel Cycle)}            & \textbf{\ffc} \\ \hline
UOX                    & 0         \\ \hline
MOX                    & 1         \\ \hline
ThOX                    & 2         \\ \hline
\end{tabular}
\end{table}

The second parameter is reactor fidelity, \frx. Reactors can make
requests or provide supply based either on their entire batch or for each
assembly in a batch. An $f{\text{rx}}$ value of zero indicates reactors trading
full batches, and a value of one indicates reactors trading individual
assemblies. Trading individual assemblies is of higher fidelity because the
number of possible trades, and thus variables in the NFCTP formulation,
increases by an order of magnitude. The parameter-to-reactor-fidelity mapping is
shown in Table \ref{tbl:frx}.

\begin{table}[h!]
\centering
\caption{A mapping between reactor fidelity and \frx values.}
\label{tbl:frx}
\begin{tabular}{|c|c|}
\hline
\textbf{Fidelity (Reactor)}            & \textbf{\frx} \\ \hline
Batches                    & 0         \\ \hline
Assemblies                    & 1         \\ \hline
\end{tabular}
\end{table}

Finally, the fidelity with with objective value coefficients are generated can
be varied. This parameter is denoted \floc because it governs the degree to
which location is taken into account in Equation \ref{eqn:loc_proxy}. The
mapping between \floc and parameters in Equation \ref{eqn:loc_proxy} is shown in
Table \ref{tbl:floc}. As \floc increases, the size of the distribution of
possible objective coefficient values increases. When \floc is zero, the number
of possible objective coefficient values is equal to the product of the number
of requester types and the number of commodities. Increasing \floc by one, the
total possible values increases by a factor of ten, because there are ten
possible regional-preference values. Finally, when \floc is two, the number of
possible objective values is uncountably infinite \cite{cantor1890ueber}.

\begin{table}[h!]
\centering
\caption{\floc Effects on Objective Coefficient Values in Equation \ref{eqn:loc_proxy}.}
\label{tbl:floc}
\begin{tabular}{|c|c|c|c|}
\hline
\textbf{Fidelity (Location)} & \textbf{\floc} & \textbf{\dreg} 
& \textbf{\dloc} \\ \hline
No region or location data & 0  & 0          & 0 \\ \hline
Region data & 1   & 1          & 0 \\ \hline
Region and location data & 2   & 1          & 0 \\ \hline
\end{tabular}
\end{table}

\subsubsection{Instance Parameters}

Fundamental parameters represent switches that change the notion of the fidelity
of the exchange being generated, for example the difference between a
once-through fuel cycle and a fuel cycle with recycling. Instance parameters, on
the other hand, change the \textit{shape} and \textit{size} of instances in a
given population. In addition to an instance's shape and size, instance
parameters can also affect \textit{coefficient generation}. While fundamental
parameters are related basic modeling assumptions, instance parameters are
related to the specifics of an instance, given those basic modeling
assumptions. Both species of exchange instances share some instance parameters,
namely those related to the population of reactors in a given exchange and
objective coefficient generation.

\paragraph{Reactor Population}

Instances are broadly defined by a parameter representing the number of reactors
that exist in an exchange instance, $n_{rx}$. Next, the split between thermal
and fast reactors is defined by a parameter defining the ratio of thermal
reactors to all reactors in the system, $r_{rx, \text{Th}}$. Assuming $f_{\text{fc}} >
0$, the number of thermal and fast reactors is given by

\begin{equation}
n_{rx, \text{Th}} = r_{rx, \text{Th}} n_{rx},
\end{equation}

\noindent
and 

\begin{equation}
n_{rx, f} = n_{rx} - n_{rx, \text{Th}}.
\end{equation}

\noindent
If \ffc is zero, a OT fuel cycle is modeled, thus $n_{rx}$ is equal to $n_{rx,
  th}$ as there are only thermal reactors in the exchange. If a MOX fuel cycle
is modeled, the number of FMOX reactors, $n_{rx, \text{FMOX}}$, is trivially equal to
the number of fast reactors. However, for a ThOX fuel cycle, i.e.,
$f_{\text{fc}} > 1$, the number of FMOX and FThOX reactors is determined by a
parameter defining the ratio of Thorium-fueled fast reactors to the total
population of fast reactors, $r_{rx, \text{FThOX}}$, such that

\begin{equation}
n_{rx, \text{FThOX}} = r_{rx, \text{FThOX}} n_{rx, f}
\end{equation}

\noindent
and

\begin{equation}
n_{rx, \text{FMOX}} = (1 - r_{rx, \text{FThOX}}) n_{rx, f}.
\end{equation}

\noindent
In the event that the determined number of reactors is non-integral, the value
is rounded to the nearest integer, with an imposed minimum value of unity.

\paragraph{Objective Coefficients}

As shown in Equation \ref{eqn:total_pref}, the value of an objective coefficient
has two components, preference due to a commodity, $p_c$, and
preference due to the relative location between two entities, $p_l$. It is not
obvious to what degree, if any, the relative values of the two components affect
formulation performance. Accordingly, a ratio parameter, $r_{l, c}$, is
introduced to allow for investigating such effects.

\subsubsection{Parameter Summary}

A summary of species-independent parameters is provided in Table
\ref{tbl:global_params}.

\begin{table}[h!]
\centering
\caption{Parameter Description Summary for Species-Independent Parameters.}
\label{tbl:global_params}
\begin{tabularx}{\columnwidth-10pt}{|c|c|X|} % line wraps second column if too long
\hline
Parameter    & Type &
Description
\\ \hline
\ffc     & Fundamental &
The fuel cycle ``fidelity'' of an instance (which fuel cycle is being modeled).
\\ \hline
\frx   & Fundamental &
The reactor fidelity of an instance (whether individual assemblies are modeled
or whole batches are modeled).  
\\ \hline
\floc    & Fundamental &
The location fidelity of an instance (to what degree is facility location
included in objective coefficients).
\\ \hline
$n_{rx}$   & Instance &
The number of reactors in an instance.
\\ \hline
$r_{rx, \text{Th}}$   & Instance &
The ratio of thermal reactors to all reactors in an instance, if appropriate.
\\ \hline
$r_{rx, \text{FThOX}}$ & Instance &
The ratio of ThOX-based fast reactors to all fast reactors, if appropriate.
\\ \hline
$r_{l, c}$ & Instance &
The weight given to location preference with respect to commodity preference.
\\ \hline
\end{tabularx}
\end{table}

\subsection{Front-End Exchanges}\label{method:setup:front}

A front-end exchange is one in which reactors request fuel and supporting
facilities supply fuel resources. Given a specified reactor population, a
supporting facility population is determined, as described in \S
\ref{method:setup:front:sup}. Conceptually, the information gathering procedure
for this exchange begins with the RFB phase where reactors make requests for
commodities with a given quantity and enrichment. Enrichment in this case is a
simple resource quality proxy for an isotopic vector. Supporting facilities are
then polled to provide a response to these requests during the RRFB
phase. Managers of reactors would then adjust preferences based on implemented
strategies. The remainder of this section describes how front-end exchange
generation models the information gathering procedure, starting with the
generation of requests in \secref{method:setup:front:reqgen}, followed by the
generation of supply responses in \secref{method:setup:front:subgen}. The PA
phase is modeled using the location proxy described in \S
\ref{method:setup:features:prefs}. Throughout the discussion on generating
front-end exchanges, instance parameters are defined. A summary of all front-end
specific instance parameters is described in \secref{method:setup:front:sum}.

\subsubsection{Support Facility Population} \label{method:setup:front:sup}

It is assumed that there is a single type of support facility, or supporter, for
each type of commodity used in the fuel cycle. Further, each supporter is paired
with a reactor type, i.e., there is a reactor type which is the \textit{primary
  consumer} of each supporter type. The primary consumer-supplier relationship
is modeled within the formulation by choosing preferences such that there is a
maximum preference for the provided relationship (described in the following
sections). A summary of these relationships is provided in Table
\ref{tbl:commod_to_sup}.

\begin{table}[h!]
\centering
\caption{A mapping between commodities and the supporter type of that commodity.}
\label{tbl:commod_to_sup}
\begin{tabular}{|c|c|c|}
\hline
\textbf{Commodities}            & \textbf{Supporter} & \textbf{Primary Consumer} \\ \hline
UOX                    & UOX  & Thermal       \\ \hline
TMOX                    & TMOX & Thermal        \\ \hline
FMOX                    & FMOX & FMOX        \\ \hline
FThOX                    & FThOX & FThOX        \\ \hline
\end{tabular}
\end{table}

The number of each type of supporter in a front-end exchange instance is a
function of of the number of primary consumers as well as configurable
parameters. Supporter types are divided into two groups: those who primarily
support thermal reactors and those who primarily support fast reactors. The
number of thermal fuel supporters is determined to be the product of the number
of thermal reactors and a ratio parameter, $r_{s, \text{Th}}$, i.e.,

\begin{equation}
n_{s, \text{Th}} = r_{s, \text{Th}} n_{rx, \text{Th}}.
\end{equation}

The number of TMOX supporters, assuming $f_{\text{fc}} > 0$, is then determined
by a parameter defined as the ratio of TMOX to UOX supporters, $r_{s,
  \text{TMOX}, \text{UOX}}$, such that the number of UOX and TMOX supporters is

\begin{equation}
n_{s, \text{UOX}} = \frac{n_{s, \text{Th}}}{1 + r_{s, \text{TMOX}, \text{UOX}}}
\end{equation}

\noindent
and

\begin{equation}
n_{s, \text{TMOX}} = n_{s, \text{Th}} - n_{s, \text{UOX}}.
\end{equation}

The number of fast reactor fuel supporters is determined directly from the number
of associated fast reactors in the exchange using ratio parameters,
$r_{s, \text{FMOX}}$ and $r_{s, \text{FThOX}}$. Assuming $f_{\text{fc}} > 0$, the number of FMOX
supporters is given as

\begin{equation}
n_{s, \text{FMOX}} = r_{s, \text{FMOX}} n_{rx, \text{FMOX}}.
\end{equation}

\noindent
Similarly, assuming $f_{\text{fc}} > 1$, the number of FThOX supporters is given as  

\begin{equation}
n_{s, \text{FThOX}} = r_{s, \text{FThOX}} n_{rx, \text{FThOX}}.
\end{equation}

\subsubsection{Request Generation}\label{method:setup:front:reqgen}

Reactors make \textit{mutual} requests for all commodities that they can consume
as described in Table \ref{tbl:rx_req}. Again, a mutual request set is a group
of requests of which any single request will meet a given demand. When reactors
request a single batch, i.e., when \frx is zero, a single request is made
per commodity. When requesting $n_a$ assemblies, a request is made per assembly
per commodity, with the number of assemblies denoted previously in Table
\ref{tbl:rx_params}. A single request portfolio encompasses all requests, with a
portfolio quantity equal to the reactor's batch size.

It is assumed that fuel is requested at some enrichment level dependent on the
reactor type. Each reactor in an exchange will choose a batch enrichment level
given a uniform distribution. Recycled fuel is modeled as being composed of a
target element oxide and topped up with natural uranium oxide; the mixing ratio
is again based on reactor type. For recycled fuel, the associated enrichment
level describes the enrichment of the fissile isotope in the \textit{target}
element. For example, MOX fuel with 45\% enrichment implies that of the
elemental Plutonium in the mixture, 45\% is comprised of isotopic
\nucl{239}{Pu}. Finally, each reactor has a preference assignment over its set
of consumable commodities.

Thermal reactors can consume UOX fuel as well as both MOX variants. It is
assumed that thermal reactors would prefer to consume thermal MOX fuel in order
to maintain any equilibrium status of the cycle. UOX fuel is next preferred.
Finally ``fast'' MOX is modeled as a type of fuel that is usable by thermal
reactors, i.e., it has thermally-fissile plutonium; however it is assumed that
the property of the plutonium vector is more amenable to fast-spectrum
reactors. Therefore it is least preferential. Preference values for each
commodity are summarized in Table \ref{tbl:rx_req}. A normal operating
enrichment range of $[3.5, 5.5]$ is used for UOX fuel. MOX-based fuels are
assumed to be comprised of 7\% Plutonium-oxide with 93\% Uranium-oxide top up
\cite{bertel2007management} and an enrichment range of $[55, 65]$
\cite{bairiot2003status}. In practice, many reactor concepts restrict the
fraction of an LWR's core that can be made up of MOX fuel rather than UOX fuel
due to a reduced safety margin. Accordingly, a tuneable parameter is added to
the model, $f_{mox}$, which denotes the fraction of a request that can be made
up of MOX-based fuel. This fraction is only relevant if reactors are operating
in assembly mode, i.e., if \frx is unity.

MOX and ThOX fast reactors utilize the same governing request parameters but
have a different preference distribution over commodities. It is assumed that a
Thorium-based fast reactor prefers Thorium-based fast reactor fuel over
MOX-based fast reactor fuel and \textit{vice versa}. Additionally, both fast
reactor types can utilize thermal MOX fuel or medium-enriched UOX, but prefer
fast reactor-based fuels. Preference values for each commodity type and reactor
are summarized in Table \ref{tbl:rx_req}. Both fast reactor types select a UOX
enrichment in $[15, 20]$\cite{bairiot2003status}, with an upper limit set by LEU
legal enrichment limits. All recycled fuels commodities are assigned an
enrichment range of $[55, 65]$ \cite{bairiot2003status} and have a composition
of 20\% of the target element (Plutonium or Thorium), with the given enrichment
of it's primary fissile isotope (\nucl{239}{Pu} or \nucl{233}{U}), and 80\%
Uranium top up \cite{bairiot2003status}. Note that no large-scale fast reactors
have been fueled by Thorium-based fuels. Accordingly, Thorium-related values are
broad generalizations. The purpose of including another fuel type is to expand
on the complexity of possible connections between facilities in a given fuel
cycle, while including somewhat realistic constraining values. The constraining
values used by any individual analyst will vary, perhaps greatly, and thus only
reasonable values are required by this study.

A summary of chosen chosen request parameters based on reactor and commodity
types is shown in Table \ref{tbl:rx_req}.

\begin{table}[h!]
\centering
\caption{A summary of reactor request parameters.}
\label{tbl:rx_req}
\begin{tabularx}{\columnwidth-10pt}{@{} *5{|>{\centering\arraybackslash}X}| @{}}
% line wraps second column if too long
\hline
Reactor Type             & Commodity & Enrichment Range & 
Target Element Fraction (\%), $f_{el}$ & Commodity Preference, $p_c$
\\ \hline
\multirow{3}{*}{Thermal} & 
UOX   & $[3.5, 5.5]$         & 100 & 0.5        \\ \cline{2-5} 
& 
TMOX  & $[55, 65]$         & 7 & 1      \\ \cline{2-5} 
& 
FMOX  & $[55, 65]$         & 7 & 0.1      \\ \hline
\multirow{4}{*}{FMOX}    & 
UOX & $[15, 20]$         & 100  & 0.1     \\ \cline{2-5} 
& 
TMOX & $[55, 65]$         & 20 & 0.5      \\ \cline{2-5} 
& 
FMOX & $[55, 65]$         & 20 & 1      \\ \cline{2-5} 
& 
FThOX & $[55, 65]$         & 20 & 0.25      \\ \hline
\multirow{4}{*}{FThOX}   & 
UOX & $[15, 20]$         & 100 & 0.1      \\ \cline{2-5} 
& 
TMOX & $[55, 65]$         & 20 & 0.25      \\ \cline{2-5} 
& 
FMOX & $[55, 65]$         & 20 & 0.5      \\ \cline{2-5} 
& 
FThOX & $[55, 65]$         & 20 & 1      \\ \hline
\end{tabularx}
\end{table}

\subsubsection{Supply Generation}\label{method:setup:front:subgen}

With all requests known, each supporting supply facility responds to all
requests for their assigned commodity, creating an associated supply node and an
arc between the supply node and request node. Constraint coefficients are
determined for each arc based on the requested enrichment associated with that
arc. Furthermore, a right-hand side (RHS), $b^k_s$, is provided for each
constraint in addition to a coefficient conversion function.

Each supplier has two types of constraints for which coefficients must be
calculated: a \textit{process} constraint and an \textit{inventory} constraint. A
process constraint models a situation in which the amount of supplied fuel is
constrained physically; only so much fuel can be made in one time step. An
inventory constraint models a situation in which a supplier is constrained by
the available material inventory on hand. Both constraints are a function of
requested quantity and fuel enrichment.

\paragraph{UOX Constraints}

A UOX supplying facility is assumed to be constrained by a SWU process
constraint and a natural Uranium inventory constraint. Assuming general
operating parameters, including a tails assay of $0.3$ and a feed assay of
natural Uranium, $0.711$, constraint coefficients can be applied to arcs. The SWU
coefficient conversion function is previously described in Equation
\ref{eqs:enr-swu-beta} while the natural Uranium conversion function is
described in Equation \ref{eqs:enr-prod-beta}. Therefore, for UOX supplying
facilities,

\begin{equation}
\beta^{\text{proc}}_s(\epsilon) = \beta^{\text{SWU}}_s(\epsilon) 
\end{equation}

and

\begin{equation}
\beta^{\text{inv}}_s(\epsilon) = \beta^{\text{NU}}_s(\epsilon). 
\end{equation}

In order to determine a constraining RHS, the proposed Eagle Rock Enrichment
Plant is chosen as a model. It purports to have a SWU capacity of $3.3E6$
Million SWU per year. Accordingly, the process constraint RHS is chosen to be an
approximate monthly value,

\begin{equation}
b^{\text{SWU}}_s = \sim 2.75E5 \frac{\text{SWU}}{\text{month}}.
\end{equation}

Any inventory constraint will be based on the present state of a facility at a
given simulation time step. Therefore, a sufficiently reasonable value must be
provided without actual simulation data. Because two constraints are added,
investigating their relative effects is of interest, which leads to a strategy
for generating an inventory constraining value by deriving it from the process
constraining value. In order to make such comparisons, the two RHS values must
be equivalent in both units and with respect to the expected coefficient values
associated with each constraint. Accordingly, a translation constant is defined
to achieve both aims. The translation , $\tau_s$, constant is taken to be a
ratio of constraint coefficients for the average enrichment of a support
facility's primary consumer, i.e.,

\begin{equation}
\tau_s = \frac{\beta^{\text{inv}}_s(\bar{\epsilon_r})}{\beta^{\text{proc}}_s(\bar{\epsilon_r})}.
\end{equation}

\noindent
Thus, a UOX supporting facility uses a average enrichment, $\bar{\epsilon_r}$,
of $4.5$ because that is the median of the thermal reactor enrichment
range. Further, a ratio coefficient parameter, $r_{inv, proc}$, is added in
order to investigate interesting cases from a formulation point of view. If
$r_{inv, proc} > 1$, then the process constraint RHS is smaller and thus the
process constraint is more likely to be engaged in an feasible solution than the
inventory constraint. On the other hand, if $r_{inv, proc} < 1$, the inventory
constraint is more likely to be engaged. The determination of the inventory RHS
is identical for all supporting facilities and is defined in Equation
\ref{eqn:b_inv_sup}.

\begin{equation}\label{eqn:b_inv_sup}
b^{\text{inv}}_s = 
r_{inv, proc} \tau_s b^{\text{proc}}_s.
\end{equation}


\paragraph{Recycled Commodity Constraints}

Due to the lack of commercially viable, well documented fast reactor fuel
suppliers, a simple linear surrogate model is assumed for an inventory
constraint. The primary inventory of any recycling facility is the amount of
fissile material it has on hand. Therefore, using constants defined in Table
\ref{tbl:rx_req}, the coefficient function conversion function is chosen to be

\begin{equation}
\beta^{\text{inv}}_s(\epsilon) = f_{el} \epsilon. 
\end{equation}

There are many possible process constraints that could be used, such as heat
production or radiotoxicity; however, each of these requires a detailed isotopic
composition to be relevant. Accordingly, a commodity-informed mass throughput
constraint is used. Per the current IAEA practice \cite{heinonen2010}, and
extrapolating the same effect for reprocessing \nucl{233}{U}, a factor of 100 is
added for for Plutonium and Thorium-based commodities. The process constraint
coefficient function is defined as

\begin{equation}
\beta^{\text{proc}}_s = 100 f_{el}. 
\end{equation}

From previous conversations with industry representatives \cite{murraycomm}, a
reasonable size for a processing plant is 800 tonnes per year, which is similar
to the Rokkasho plant in Japan \cite{heinonen2010}. Given the request
parameters defined in Table \ref{tbl:rx_req}, an 800 t Uranium / 8 t Plutonium
facility could service on the order of 2-3 fast reactors or $\sim$2 thermal
reactors with $\frac{1}{3}$ a request as MOX. The yearly process limit is again
translated to a monthly limit, resulting in a constraint RHS value of

\begin{equation}
b^{\text{proc}}_s = \sim 66.7 \frac{\text{t}}{\text{month}}.
\end{equation}

The inventory constraint RHS is determined identically to the UOX case.

\subsubsection{Parameter Summary}\label{method:setup:front:sum}

A summary of front-end exchange species-dependent instance parameters is provided
in Table \ref{tbl:front_params}.

\begin{table}[h!]
\centering
\caption{Parameter Description Summary for Front-End Exchange Instance Parameters.}
\label{tbl:front_params}
\begin{tabularx}{\columnwidth-10pt}{|c|X|} % line wraps second column if too long
\hline
Parameter    & 
Description
\\ \hline
$f_{mox}$     & 
The fraction of thermal reactor requests that can be met with mox fuel.
\\ \hline
$r_{s, \text{Th}}$ & 
The ratio of thermal support facilities to thermal reactors.  
\\ \hline
$r_{s, \text{TMOX}, \text{UOX}}$ & 
The ratio of TMOX to UOX support facilities.
\\ \hline
$r_{s, \text{FMOX}}$ & 
The ratio of FMOX support facilities to FMOX reactors.
\\ \hline
$r_{s, \text{FThOX}}$ & 
The ratio of FThOX support facilities to FThOX reactors.
\\ \hline
$r_{inv, proc}$   & 
The ratio of the inventory RHS to the process RHS.
\\ \hline
\end{tabularx}
\end{table}

\subsection{Back-End Exchanges}\label{method:setup:back}

A back-end exchange models the transfer of used fuel from reactors to supporting
facilities, such as reprocessing facilities and repositories. During the
information gathering process, supporting facilities make requests for
commodities that can either be used directly in the recycling process or need to
be stored, temporarily or permanently. Reactors then respond based on output
fuel to each request during the RRFB phase. Throughout the discussion on
generating back-end exchanges, instance parameters are defined. A summary of all
back-end specific instance parameters is shown in Table \ref{tbl:back_params}.

\subsubsection{Support Facility Population}\label{method:setup:back:sup}

Four classes of supporting facilities are modeled in back-end exchanges: a
thermal fuel recycling facility, a facility that recycles fast MOX fuel, a
facility that recycles fast ThOX fuel, and a repository. As thermal fuel
recycling facilities are the only thermal supporting facilities, the number of
such facilities in a given back-end exchange is trivially $n_{s, \text{Th}}$. As is the
case with front-end exchanges, there is a class of supporting facility for each
fast fuel commodity. The methodology for determining the population of each
facility type is identical to front-end exchanges:

\begin{equation}
n_{s, \text{FMOX}} = r_{s, \text{FMOX}} n_{rx, \text{FMOX}}
\end{equation}

and

\begin{equation}
n_{s, \text{FThOX}} = r_{s, \text{FThOX}} n_{rx, \text{FThOX}}.
\end{equation}

Back-end exchanges include repositories, a facility type not present in
front-end exchanges. A simple ratio parameter, $r_{\text{repo}}$ is applied
based on the total number of other supporting facilities, i.e.,

\begin{equation}
n_{s, \text{repo}} = r_{s, \text{repo}} ( n_{s, \text{Th}} + n_{s, \text{FMOX}} +n_{s, \text{FThOX}} ).
\end{equation}

\subsubsection{Request Generation}

It is assumed that any recycling facility will accept UOX fuel. However, MOX
recycling facilities can not process ThOX-based fuels, and ThOX facilities can
not process MOX-based fuels. Additionally, fast MOX facilities prefer fast MOX
fuel, while thermal facilities prefer thermal MOX fuel. Finally, repositories
can accept all commodities; however, it is a consumer of last resort. The
assigned preference value as a function of commodity type and supporting
facility type, $p_c$ is shown in Table \ref{tbl:sup_to_pref}.

\begin{table}[h!]
\centering
\caption{$p_c$ Value Mapping between Back-End Supporting Facilities and Commodities.}
\label{tbl:sup_to_pref}
\begin{tabular}{|c|c|c|c|c|}
\hline
\backslashbox{Supporting Facility}{Commodity} & UOX & TMOX & FMOX & FThOX \\ \hline
TMOX                & 1     & 1      & 0.5    & N/A     \\ \hline
FMOX                & 0.5   & 1      & 1      & N/A     \\ \hline
FThOX               & 0.3   & N/A    & N/A    & 1       \\ \hline
Repo                & 0.1   & 0.1    & 0.1    & 0.1     \\ \hline
\end{tabular}
\end{table}

A single request for the facility's processing capacity is made for each
commodity. Recycling facilities define their request quantity using the same 800
ton per year limit discussed in \secref{method:setup:front}. Repositories,
however, use a limit based on the Yucca Mountain statutory limit of 77,000 tons
and assuming a 30-year operating lifetime, i.e., period of time in which fuel
can enter the facility. Thus, a repository's monthly request quantity is
determined to be $\sim 215$t.

A fissile quantity constraint is added for each recycling facility. The fissile
constraint models a situation in which recycling facilities have a demand for
fissile material. The amount of fissile material required by recycling
facilities is based on their primary consumer. It is determined to be the
product of the facility's mass constraint and the mean amount of fissile
material in a primary consumer's request per unit mass, as shown in Equation
\ref{eqn:bfiss}. This constraint can be considered as ``recycling facilities
request fissile material quantities as if all reactors in the system are average
primary consumers''.

\begin{equation}\label{eqn:bfiss}
b^{\text{fiss}}_r = \bar{\epsilon} f_{el} b^{\text{mass}}_r.
\end{equation}

\noindent
The fissile constraint coefficient is simply the amount of fissile material for
a given supply, as described in Equation \ref{eqn:betafiss}.

\begin{equation}\label{eqn:betafiss}
\beta^{\text{fiss}}_r(\epsilon) = \epsilon f_{el}.
\end{equation}

\subsubsection{Supply Generation}

A key difference between the front-end and back-end exchanges is that in
front-end exchanges, reactors request fuel, and thus can make a single request
per commodity per assembly. In back-end exchanges, commodities must be assigned
to each assembly. Accordingly, a key parameter in back-end exchanges is the
commodity distribution for assemblies. A normalized uniform distribution
parameter is provided for each reactor type with a value for each commodity type
that reactor can consume as defined in Equation \ref{eqn:assemdist}.

\begin{equation}\label{eqn:assemdist}
\begin{split}
d_{\text{Th}} = & \:
[x_{\text{UOX}}, x_{\text{TMOX}}, x_{\text{FMOX}}], \: x_i \in [0, 1) \\
d_{\text{FMOX}} = & \:
[x_{\text{UOX}}, x_{\text{TMOX}}, x_{\text{FMOX}}, x_{\text{FThOX}}], \: x_i \in [0, 1) \\
d_{\text{FThOX}} = & \:
[x_{\text{UOX}}, x_{\text{TMOX}}, x_{\text{FMOX}}, x_{\text{FThOX}}], \: x_i \in [0, 1) 
\end{split}
\end{equation}

If an exchange is in batch mode, i.e., \frx is zero, then this distribution
acts as a selection distribution, where each $x_i$ represents a probability that
the batch will be of that commodity. If in assembly mode, then commodities are
assigned to each assembly given the relative $x_i$ values.  The assignment of
commodities to number of assemblies for a given reactor type is done by rounding
the product of $x_i$ and the total number of assemblies, starting with the
lowest value of $x_i$. The final assignment is then taken as the difference
between the total number of assemblies and the previously assigned values.

For example, consider a fast reactor with a distribution $d_{\text{Th}} =
[\frac{3}{4}, \frac{1}{4}, 0]$ and number of assemblies $n_a = 39$. The
assembly-commodity breakdown would be calculated as

\begin{gather*}
n_{\text{FMOX}} = \text{round}(x_{\text{FMOX}} n_a) = 0 \\
n_{\text{TMOX}} = \text{round}(x_{\text{TMOX}} n_a) = 10 \\
n_{\text{UOX}} = n_a - n_{\text{TMOX}} - n_{\text{FMOX}} = 29.
\end{gather*}

Once a commodity is assigned either to a single batch or a selection of
assemblies, the remaining supply generation methodology is identical. If \frx is
zero, the following discussion uses the term assembly to mean either an
individual assembly or a batch. That is, a reactor in a back-end exchange has a
single assembly to supply. If \frx is one, then it has $n_a$ assemblies to
supply, where $n_a$ is defined in Table \ref{tbl:rx_batch} for each reactor
type.

In order to assign enrichment values to each assembly, a single random value is
chosen, $x \in [0, 1)$. Each assembly is then assigned an enrichment based on
  the assembly's commodity and enrichment range, as defined in Table
  \ref{tbl:rx_req}. This modeling assumption supports a situation in which, for
  a given batch, equivalent fissile enrichments were used across
  commodities. For example, consider a Thermal reactor with $x$ chosen to be
  $0.55$. All UOX assemblies would be assigned an enrichment value of $4.6$, and
  each MOX-based assembly would be assigned an enrichment value of $60.5$.

A bid portfolio is assigned to each assembly. Given the commodity of each
assembly, a supply response is provided to each supporting facility that requests
that commodity. For example, given a UOX assembly, a reply is sent to each
supporting facility, as all supporting facilities accept UOX, shown in Table
\ref{tbl:sup_to_pref}. The set of supply responses associated with a single
assembly is denoted a \textit{mutual} set. That is, each supply node corresponds
to a single assembly that should not be split between supporting facilities.

\subsubsection{Parameter Summary}

A summary of back-end exchange species-dependent instance parameters is provided
in Table \ref{tbl:back_params}.

\begin{table}[h!]
\centering
\caption{Parameter Description Summary for Back-End Exchange Instance
  Parameters.}
\label{tbl:back_params}
\begin{tabular}{|c|c|}
\hline
Parameter    & 
Description
\\ \hline
$d_{\text{Th}}$     & 
thermal reactor assembly distribution
\\ \hline
$d_{\text{FMOX}}$     & 
fast mox reactor assembly distribution
\\ \hline
$d_{\text{FThOX}}$     & 
fast thox reactor assembly distribution
\\ \hline
$r_{\text{repo}}$     & 
repository to supporting facility ratio
\\ \hline
\end{tabular}
\end{table}
