\chapter{Summary}\label{ch:summary}

This body of work sought to enhance the state of the art in dynamic fuel cycle
simulation. Prior to this effort, most decision making related to a given
simulation was made by a human analyst prior to simulation execution. Further,
even \textit{ex situ}, human-based, decision making was limited to collections
of entities and macroscopic descriptions of commodities. Both of these effects
resulted simulation platforms lacking either physics fidelity, entity relation
fidelity, or both.

The \Cyclus dynamic fuel cycle simulator and this work are inextricably
tied. While many of the concepts and methods described herein may be applied to
any implementation of a non-trivial supply-demand model, the development of this
work was spurred by the need for such methods implemented in \Cyclus. 

Upon its inception, \Cyclus had a variety of goals. An analyst's ability to
choose the level of physical, social, and economic fidelity was of chief
concern. This behavior is supported in \Cyclus through a plug-in framework of
various archetypes. Thus, an analyst could use a high-fidelity reactor archetype
or a low-fidelity archetype, for instance. An equally important concept of
\Cyclus was the ability model a variety of fuel cycles with similar
archetypes. Therefore, the \Cyclus kernel was required to abstract away
fuel-cycle specific behavior. Finally, the developers of \Cyclus also desired to
model regional interaction mechanisms, such as tariffs or other international
trade instruments. In short, it was required to solve the \textit{general case}
of nuclear fuel cycle simulation.

\section{Statement of Work}

The goal of this work was, chiefly, to design, implement, and analyze a
highly-flexible, physics and economics-informed simulation engine. The engine
was split into two primary conceptual categories: entity deployment and entity
interaction. Developing a sophisticated entity interaction mechanism was the
chief focus of the majority of the presented \textit{oeuvre}.

Significant design constraints were placed on the design of a entity interaction
mechanism. First, it must support arbitrary physics and chemical constraints, as
well as general supply-chain constraints, such as inventory and processing
constraints. Further, it must model the competition of resources among entities
for which demand and supply of resources may be fungible. Finally, arbitrary
social phenomenon must be able to be translated to the interaction
framework. The resulting interaction mechanism, termed the Dynamic Resource
Exchange (DRE), was informed chiefly from the fields of supply-chain management,
agent-based modeling, and mathematical programming.

The DRE allows agents to inform both system supply and demand of resources
through a request-bid framework. Physics fidelity is provided to agents in this
framework by utilizing fully specified \texttt{Resource} objects. For example,
nuclear fuel demand can be specified directly by an ideal isotopic vector in a
\texttt{Material} object. Once supply and demand is known, social interaction
models can be applied to affect resource flow-driving mechanisms. For example, a
tariff can be modeled by uniformly reducing preferences of transactions between
agents outside of a given \texttt{Region}. Presently, a cardinal preference
model is used as the flow-driving mechanism.

The DRE is comprised of three layers: a resource layers, with which agents
interact, an exchange layer, and a formulation layer. Supply, demand, and
preferences are defined in the resource layer, for a specific type of
\texttt{Resource} object. The exchange layer provides a general resource
exchange representation, irrespective of a specific object type. The
representation is comprised of a bipartite graph of supply and demand nodes,
supply and demand constraints, and a measure of preference for each proposed
connection between nodes. The DRE can be solved either in the exchange layer or
by translating the exchange into a minimum-cost, network-flow problem, resulting
in the formulation layer. Translation to LPs and MILPs are both supported, where
MILPs are required if entities require individual, quantized resources. Such a
case arises when one would like to model individual reactor assemblies.

After the DRE framework was designed and implemented, it was tested and
analyzed. The full \Cyclus simulator is still nascent with respect to
full-featured archetypes that would utilize DRE-specific features. Therefore,
instances of resource exchanges were required to be generated. The generation of
instances is based on a parameter vector, comprised of fundamental and instance
parameters. Fundamental parameters include which fuel cycle is being modeled and
whether reactors request single batches of fuel or a collection of individual
assemblies of fuel. Instance parameters include the number of reactors being
modeled and the number of support facilities being modeled, among many
others. Exchanges representing the front-end and back-end of the nuclear fuel
cycle were generated separately.

Generated exchanges were solved with the Greedy Heuristic in the exchange layer
and the COIN-OR Clp and Cbc solvers in the formulation layer. 

% user recommendations

\section{Suggested Future Work}

better preference determination of proposed supply (function callback)

\section{Closing Remarks}
