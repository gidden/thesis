\chapter{Summary}\label{ch:summary}

This body of work sought to enhance the state of the art in dynamic fuel cycle
simulation. Prior to this effort, most decision making related to a given
simulation was made by a human analyst prior to simulation execution. Further,
even \textit{ex situ}, human-based, decision making was limited to collections
of entities and macroscopic descriptions of commodities. Both of these effects
resulted simulation platforms lacking either physics fidelity, entity relation
fidelity, or both.

The \Cyclus dynamic fuel cycle simulator and this work are inextricably
tied. While many of the concepts and methods described herein may be applied to
any implementation of a non-trivial supply-demand model, the development of this
work was spurred by the need for such methods implemented in \Cyclus. 

Upon its inception, \Cyclus had a variety of goals. An analyst's ability to
choose the level of physical, social, and economic fidelity was of chief
concern. This behavior is supported in \Cyclus through a plug-in framework of
various archetypes. Thus, an analyst could use a high-fidelity reactor archetype
or a low-fidelity archetype, for instance. An equally important concept of
\Cyclus was the ability model a variety of fuel cycles with similar
archetypes. Therefore, the \Cyclus kernel was required to abstract away
fuel-cycle specific behavior. Finally, the developers of \Cyclus also desired to
model regional interaction mechanisms, such as tariffs or other international
trade instruments. In short, it was required to solve the \textit{general case}
of nuclear fuel cycle simulation.

\section{Statement of Work}

The goal of this work was, chiefly, to design, implement, and analyze a
highly-flexible, physics and economics-informed simulation engine. The engine
was split into two primary conceptual categories: entity deployment and entity
interaction. Developing a sophisticated entity interaction mechanism was the
chief focus of the majority of the presented \textit{oeuvre}.

\subsection{Modeling \& Simulation}

Significant design constraints were placed on the design of a entity interaction
mechanism. First, it must support arbitrary physics and chemical constraints, as
well as general supply-chain constraints, such as inventory and processing
constraints. Further, it must model the competition of resources among entities
for which demand and supply of resources may be fungible. Finally, arbitrary
social phenomenon must be able to be translated to the interaction
framework. The resulting interaction mechanism, termed the Dynamic Resource
Exchange (DRE), was informed chiefly from the fields of supply-chain management,
agent-based modeling, and mathematical programming.

The DRE allows agents to inform both system supply and demand of resources
through a request-bid framework. Physics fidelity is provided to agents in this
framework by utilizing fully specified \texttt{Resource} objects. For example,
nuclear fuel demand can be specified directly by an ideal isotopic vector in a
\texttt{Material} object. Once supply and demand is known, social interaction
models can be applied to affect resource flow-driving mechanisms. For example, a
tariff can be modeled by uniformly reducing preferences of transactions between
agents outside of a given \texttt{Region}. Presently, a cardinal preference
model is used as the flow-driving mechanism.

The DRE is comprised of three layers: a resource layers, with which agents
interact, an exchange layer, and a formulation layer. Supply, demand, and
preferences are defined in the resource layer, for a specific type of
\texttt{Resource} object. The exchange layer provides a general resource
exchange representation, irrespective of a specific object type. The
representation is comprised of a bipartite graph of supply and demand nodes,
supply and demand constraints, and a measure of preference for each proposed
connection between nodes. The DRE can be solved either in the exchange layer or
by translating the exchange into a minimum-cost, network-flow problem, resulting
in the formulation layer. Translation to LPs and MILPs are both supported, where
MILPs are required if entities require individual, quantized resources. Such a
case arises when one would like to model individual reactor assemblies.

\subsection{Experimentation}

After the DRE framework was designed and implemented, it was tested and
analyzed. The full \Cyclus simulator is still nascent with respect to
full-featured archetypes that would utilize DRE-specific features. Therefore,
instances of resource exchanges were required to be generated. The generation of
instances is based on a parameter vector, comprised of fundamental and instance
parameters. Fundamental parameters include which fuel cycle is being modeled and
whether reactors request single batches of fuel or a collection of individual
assemblies of fuel. Instance parameters include the number of reactors being
modeled and the number of support facilities being modeled, among many
others. Exchanges representing the front-end and back-end of the nuclear fuel
cycle were generated separately.

Generated exchanges were solved with the Greedy Heuristic in the exchange layer
and the COIN-OR Clp and Cbc solvers in the formulation layer. A number of key
observations were ascertained by both scaling and stochastic experimental
campaigns. First, the Greedy solver exhibited linear scaling with problem size
(i.e., the number of variables) for all configurations of exchanges, matching
theory. Somewhat surprisingly, the Clp solver was also shown to have efficient
scaling behavior as well. The theoretical complexity of the general case of
simplex method is not known, although it is known to be ``efficient in
practice'' \cite{goldfarb1994complexity}. Further, solving relaxations of LPs is
a fundamental step in branch-and-bound algorithms for MILPs (e.g., those used in
Cbc). Therefore, exploring the behavior of given problem structures as LPs is of
interest. In this work, Clp solves of front-end exchanges were found to have
linear-like scaling, up to problems with ~1 million variables, regardless of
fundamental parameter configuration. Back-end exchanges also experienced
linear-like behavior for smaller problem sizes (~1E4 variables). At larger
problem sizes, especially for instances with high variability in objective
coefficient distribution, scaling becomes much worse than the linear.

Solving MILPs optimally is known to be NP-Hard, and solution times are known to
scale exponentially in the general case. This behavior was observed for the Cbc
solver in both front and back-end exchanges. A solution time limit of three
hours was placed on Cbc calculations, and many cases converged for both exchange
types in the scaling campaign. Threshold problem sizes were observed in each
exchange type and fundamental parameter configuration. Maximum convergence
probability was found in high-fidelity reactor instances for non-once-through
fuel cycles. Low-fidelity reactor, once-through fuel cycle instances were found
to have particularly poor convergence probability. 

Performing comparative solution analyses between the Greedy and Cbc solvers
illuminated the importance of false-arc costs in the MILP formulation of the
NFCTP. Specifically, multiple exchange instances were observed to have better
performance-space results using the heuristic over a full-fledged optimal solve
in cost-space. This behavior was found to be the result of using a large
false-arc unit cost in the NFCTP formulation in conjunction with a relative
optimal value convergence criteria. A relative-value criteria is required in
order to be used in a general simulation framework. Replacing the large cost
with a small cost, however, proved to reverse the observation in both the
scaling and stochastic campaigns. 

Finally, the use of location-based preferences was found to have little or no
effect on Cbc solution times in most cases. However, for back-end exchanges of
MOX and ThOX fuel cycles, cases in which no location-based preferences were
applied were found to have significantly longer solution times than cases in
which they were applied. Therefore, applying a small objective perturbation may
prove a reasonable speed up strategy for some exchange structures.

\subsection{Recommendations}

Through the development of the DRE, users may now apply physical, economic, and
social models to NFC simulation. The choice of solver will largely depend on the
fidelity of the associated models and underlying data. The Greedy solver will
always provide a feasible solution to the given exchange instance, applying any
physical, chemical, or supply-chain constraints. Therefore, if a user has a
low-fidelity economic or social model, then the Greedy solver will likely meet
the users needs.

With higher-fidelity economic and social models, obtaining a optimal solution
becomes paramount. While an LP, and thus Clp, can be used to model
approximations of fuel cycles, to-date user and developer experience has found
the modeling of individual fuel assemblies to be conceptually simpler both to
use and code. In short, having a binary decision regarding a supplied resource
is simpler than managing the acceptance of an arbitrary number of partial
resources, especially in a multi-commodity system. Requiring optimal solutions
and using quantized resource transfers necessitates using a MILP solver, such as
Cbc.

Users can expect exponential solution time scaling when using Cbc. For a
relatively small convergence criteria, i.e., 1\%, most front-end instances with
a reactor population greater than ~150 were found not to converge within a
3-hour time limit. Back-end instances showed better convergence behavior with
problem size. The Cbc solver was found to perform better with higher reactor
fidelity, a promising result for this use case. With a small arc-cost
implemented in the kernel by default, users may find greater speed ups by
loosening the default convergence criteria. However, tuning the specific
convergence criteria and setting a solution-time ceiling will be dependent on
the an individual's use case. Finally, after a preference-perturbation option is
implemented, users may find significant speed ups through its use.

\section{Suggested Future Work}

better preference determination of proposed supply (function callback)

different greedy solvers based on eco/soc models

implementing other solvers

\section{Closing Remarks}
