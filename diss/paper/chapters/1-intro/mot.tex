
\section{Statement of Work}

Deciding how a simulation is structured from an interactions standpoint is a
delicate balance of known necessity and perceived future needs. There are basic
decisions to make, such as modeling material transfer as either discrete or
continuous. Discrete transfers more closely match reality and may provide
insights in that regard, however they require more of their modeling apparatus
due to messaging needs and other structures. More complex decisions include how
one wants to determine connections between facilities, and whether such
connections are assigned statically and incorporated into the simulation
architecture or determined dynamically.

In his conclusions of a MIT benchmarking exercise, Guerin states that
``operation of a fuel cycle model is as much art as science''
\cite{guerin_benchmark_2009}, an opinion that likely stems from this
``freedom''. These simulation-engine decisions comprise the art-related portion
of fuel cycle simulation, but developers have a goal of making these decisions
in as informed a way as possible using domain-level knowledge with respect to
our known and perceived requirements. In general, this work tries to minimize
the sheer number of choices one makes in this regard, instead relying on well
known and well documented practices of computer scientists and systems
engineers. In short, the goal of this work focuses on extending the current
state of the art in nuclear fuel cycle simulation and associated analysis.

To date, no nuclear fuel cycle simulator has been implemented using agent-based
simulation design principles. \Cyclus, the fuel cycle simulator in which this
work is being implemented, was initially developed without a solid simulation
infrastructure design principles. The initial thrust of this work comprises the
development of \Cyclus as an agent-based simulator. In order to do so,
agent-to-agent interaction mechanisms must be defined and designed. Furthermore,
a clear time-stepping procedure must be identified that provides a sufficient
amount of entity-interaction opportunities to agents in a simulation. 

Perhaps the least well-treated aspect in current nuclear fuel cycle simulation
is resource allocation decision making. As stated previously, the vast majority
of current simulators treat this process very simply: staticly connect facility
types \textit{a priori}. The primary thrust of this work is the extention of the
current state of the art by designing and implementing a general framework that
dynamically determines the flow of resources in an arbitrary nuclear fuel cycle. 

Any such mechanism must meet a number of design criteria. First, it must be fuel
cycle agnostic: any possible facility connections must be supported. The
mechanism must take into account the isotopic profiles of the commodities
produced and consumed by agents in a simulation. Any system in which fuel
recycling exists will, by definition, have some supply constraints. Therefore,
the framework must support both the existence capacitated supply and demand as
well as its communication between agents and with the framework. Further,
constraints must be able to be influenced by sophisticated physical, chemical,
and supply chain models. Further extending the state of the art, the framework
must also allow for economic, social, and geographic models to inform the
exchange of resources between simulation entities. Finally, the framework must
support quantized transfers of resources. Nuclear reactors cores in practice are
comprised of individual fuel assemblies. Any framework must support the modeling
of individual fuel assemblies, enabling a high level of simulation detail as
well as nonproliferation analyses.

It is the goal of this corpus of effort to design and implement such a
simulation mechanism.  Once a supply-demand framework is developed, its
performance must be analyzed. Fuel cycle simulation can require varying levels
of computational fidelity. Some scoping studies may wish to sample a large
option space with low fidelity, while others may wish to sample a small option
space with high fidelity. The performance tradeoff between feasible and optimal
solutions to resource flows must be understood. Because the \Cyclus ecosystem is
still nascent, sophisticated agent models have yet to be developed. Accordingly,
a methodology for generating instances of nuclear fuel cycles is required. A
large collection of instances must then be executed with all available
supply-demand solution techniques, those that find optimal solutions and those
that report some best-guess feasible solution.

Upon completion of this work, fuel cycle simulation modelers and analysts will
be provided a robust tool that greatly increases the fidelity and flexibility
with which arbitrary fuel cycles can be modeled. The state of the art of NFC
simulation will be furthered, and novel scenarios that involve sophisticated
interactions such as the competition for resources, dynamic commodity
consumption, and geopolitical relationships can finally be supported.
