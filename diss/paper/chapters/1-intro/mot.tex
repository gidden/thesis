
\section{Statement of Work}

Deciding how a simulation is structured from an interactions standpoint is a
delicate balance of known necessity and perceived future needs. There are basic
decisions to make, such as modeling material transfer as either discrete or
continuous. Discrete transfers more closely match reality and may provide
insights in that regard, however they require more of their modeling apparatus
due to messaging needs and other structures. More complex decisions include how
one wants to determine connections between facilities, and whether such
connections are assigned statically and incorporated into the simulation
architecture or determined dynamically.

In his conclusions of a MIT benchmarking exercise, Guerin states that
``operation of a fuel cycle model is as much art as science''
\cite{guerin_benchmark_2009}, an opinion that likely stems from this
``freedom''. These simulation-engine decisions comprise the art-related portion
of fuel cycle simulation, but developers have a goal of making these decisions
in as informed a way as possible using domain-level knowledge with respect to
our known and perceived requirements. In general, this work tries to minimize
the sheer number of choices one makes in this regard, instead relying on well
known and well documented practices of computer scientists and systems
engineers.

The goal of this work focuses on extending the current state of the art in
nuclear fuel cycle simulation and associated analysis. The primary thrust of the
work is the design and implementation of a sophisticated supply-demand framework
to be used in simulation. The framework must support manifold design
criteria. 

Any system in which recycling exists will, by definition, have some supply
constraints. Therefore, the framework must support capacitated supply and
demand. Further, constraints must be able to be influenced by sophisticated
physical, chemical, and supply chain models. The framework must also allow for
economic, social, and geographic models to inform the exchange of resources
between simulation entities. Finally, the framework must support quantized
transfers of resources. Nuclear reactors cores in practice are comprised of
individual fuel assemblies. Any framework must support the modeling of
individual fuel assemblies, enabling a high level of simulation detail as well
as nonproliferation analyses.

Once a supply-demand framework is designed and implemented, its performance is
analyzed. Because the \Cyclus ecosystem is still nascent, sophisticated agent
models have yet to be developed. Accordingly, a methodology for generating
instances of nuclear fuel cycles is required. A large collection of instances
are then executed with all available supply-demand solution techniques. The
resulting product is a novel supply-demand framework, a collection of
best-practice advice for users of \Cyclus, and associated insights.
