
\section{Motivation}

Deciding how a simulation is structured from an interactions standpoint is a
delicate balance of known necessity and perceived future needs. There are basic
decisions to make, such as modeling material transfer as discrete or
continuous. Discrete transfers more closely match reality and may provide
insights in that regard, however they require more of their modeling apparatus
due to messaging needs and other structures. More complex decisions include how
one wants to determine connections between facilities, and whether such
connections are assigned statically and incorporated into the simulation
architecture or determined dynamically. Guerin's comment
in \S\ref{sec:litrev-benchmarks} stems from this ``freedom''. These
simulation-engine decisions comprise the art-related portion of fuel cycle
simulation, but developers have a goal of making these decisions in as informed
a way as possible using domain-level knowledge with respect to our known and
perceived requirements. In general, this work tries to minimize the sheer number
of choices we make in this regard, instead relying on well known and well
documented practices of computer scientists and systems engineers.

In the absence of supply constraints, aggregated
individual facility behavior and fleet-based models are equivalent. However, any
system in which recycling exists will, by definition, have some supply
constraints.

\subsection{Statement of Work}

The goal of this work was, chiefly, to design, implement, and analyze a
highly-flexible, physics and economics-informed simulation engine. The engine
was split into two primary conceptual categories: entity deployment and entity
interaction. Developing a sophisticated entity interaction mechanism was the
chief focus of the majority of the presented \textit{oeuvre}.
