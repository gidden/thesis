
\section{Tools Used}

\subsection{Modeling \& Simulation}\label{intro:sim}

Simulation and simulation analysis has a rich history and large body of work
that can be leveraged to tackle this problem.

Simulation methodologies can largely be broken down into two groups: discrete
time and discrete event simulation. 

Cyclus has taken a refined discrete time approach.

Cyclus defines an arbitrary time step, requiring consistency among prototype
configurations. Historically, this time step has been one month. Importantly, it
is integral, and unchanging within a single simulation.

A time step is comprised of two types of phases: those in which agents observe
the simulation environment and update their state, and those in which the kernel
queries agents and executes kernel functionality.

The order of phases is:

\begin{itemize}
  \item building phase (kernel)
  \item tick phase (agent)
  \item resource exchange (kernel)
  \item tock phase (agent)
  \item decommissioning phase (kernel)
\end{itemize}

\subsection{Supply Chain Modeling}

\subsection{Mathematical Programming}\label{intro:prog}

These simplifications are important to the computation time required to solve
the resulting problem instance. The general solution technique for LPs is the
Simplex Method, as previously described. Klee and Minty show that in the worst
case, the Simplex Method will execute in exponential time \cite{klee_good_1970},
but in practice it is generally considered very computationally efficient. If
the problem can be simplified to a TP, then the Transportation Simplex Method
can be used \cite{ahuja_network_1993}.

Discussion of P, NP, NP-Hard, NP-Complete, etc.

\subsection{Multicommodity Transportation Problem}\label{intro:mtp}
