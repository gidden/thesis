\chapter{Introduction}\label{ch:intro}

\section{The Nuclear Fuel Cycle}

\subsection{Taxonomy}

\subsection{Technologies}

\section{Nuclear Fuel Cycle Simulation}

The majority of tools developed to date use a system dynamics approach to
modeling the fuel cycle.

System dynamics is good for applications that naturally fit the 'stocks and
flows' approach.

\subsection{Overview}

\subsection{Quasi-Static Models}

\subsection{Dynamic Models}

\section{Tools Used}

\subsection{Modeling \& Simulation}

Simulation and simulation analysis has a rich history and large body of work
that can be leveraged to tackle this problem.

Simulation methodologies can largely be broken down into two groups: discrete
time and discrete event simulation. 

Cyclus has taken a refined discrete time approach.

Cyclus defines an arbitrary time step, requiring consistency among prototype
configurations. Historically, this time step has been one month. Importantly, it
is integral, and unchanging within a single simulation.

A time step is comprised of two types of phases: those in which agents observe
the simulation environment and update their state, and those in which the kernel
queries agents and executes kernel functionality.

The order of phases is:

\begin{itemize}
  \item building phase (kernel)
  \item tick phase (agent)
  \item resource exchange (kernel)
  \item tock phase (agent)
  \item decommissioning phase (kernel)
\end{itemize}

\subsection{Supply Chain Modeling}

\subsection{Mathematical Programming}

\subsection{Multicommodity Transportation Problem}\label{intro:mtp}

\section{Motivation}

\subsection{Cyclus History}

\subsection{Moving Foward}

\subsection{Statement of Work}
