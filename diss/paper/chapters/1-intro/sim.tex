
\section{Nuclear Fuel Cycle Simulation}\label{intro:fcs}
 
Fuel cycle simulation is a field with a variety of actors, including
governments, universities, and international governance
organizations. Accordingly, a variety of modeling strategies have been applied
to the nuclear fuel cycle. Such strategies span a wide range of fidelity, both
at the facility level and the material level. For instance, some simulators
describe reactors by fleet (or types) and solve material balances for the entire
fleet in aggregate \cite{busquim_e_silva_system_2008,yacout_vision_2006} while
others instantiate individual (or discrete) facilities
\cite{schneider_nfcsim:_2005}. Similarly, some simulators make detailed
calculations of fuel depletion due to reactor fluence \cite{boucher_cosi:_2006}
whereas others simply use pre-tabulated values that depend (generally) on burnup
values for thermal reactors and conversion ratios for fast reactors.

There are, broadly, three decision categories that are of concern to fuel cycle
simulation. The first is facility deployment, i.e., how, why, and when certain
facilities are deployed. In the current simulation development environment, the
most common reactor deployment mechanism is allowing a user to define an energy
growth curve and, for each type of reactor in the simulation, a percentage of
that total energy demand to be met by the reactor type. However, the nuclear
fuel cycle is a special case of supply-demand modeling where certain facilities
(e.g., fast reactors) require fuel that has been processed by other facilities
(e.g., thermal reactors). Accordingly, simulation developers must make a choice
regarding the ability for facilities to be built if fuel may not be available
for their use. Certain simulators explicitly disallow this behavior by
determining reactor build decisions based on look-ahead algorithms
\cite{schweitzer_improved_2008}, others explicitly allow it, while still others
offer a hybrid approach that allow a look-ahead function based on a certain
amount of fuel that will eventually be needed over a reactor's lifetime
\cite{van_den_durpel_daness_2009}. The eventual choice of this decision making
process greatly affects simulation outcomes in any scenario in which there is
competition for recycled fuel. Because these simulation tools are built to
analyze the dynamic symbiotic relationship between different reactors in a
cyclical process (e.g., thermal and fast reactors), among other scenarios, this
simulation development decision is arguably very important to simulation
outcomes.

The second simulation design decision category is the level of fidelity with
which to model the physical and chemical processes involved in the nuclear fuel
cycle. Broadly, physical fidelity includes two processes, isotopic decay and
isotopic transmutation due to residency in a reactor. Physical fidelity is an
important concern because fuel cycle simulation measures individual isotopic
masses at each point in the fuel cycle, and the isotopic profiles of those mass
streams change due to physical processes. 

Isotopic decay is important to consider because some isotopes decay on time
scales on the order of or smaller than the simulation time. \nucl{241}{Pu}, for
instance, has a half life of $\sim$14 years. Simulators fall into two camps,
those that include decay and those that do not. Interestingly, the MIT
development team claims that the lack of modeling decay does not affect the
simulation as long as all transuranic isotopes are lumped together
\cite{guerin_impact_2009}. Other codes include isotopic decay in order to inform
output metrics such as repository heat capacity.

Reactor physics, i.e., the process by which the transmuted isotopic profile of
fuel due to reactor residency is determined, is also an important physical
consideration. The rigorous solution of reactor physics equations is an entire
field in nuclear science unto itself, and is thus not normally treated by fuel
cycle simulators. In most cases for the current suite of simulators, some amount
of calculation is performed before a simulation is run, and isotopic profiles
are determined via look-up tables. Some simulators, however, choose to perform
transmutation calculations \textit{in situ}, during the simulation.

The third simulation-level design decision concerns the connections between
facilities and the type of material that flows along those connections. In
general, connections between facilities can either be static or dynamic, and can
either be fleet-based or facility-based. A static connection implies that
material will always flow between two types of facilities, whereas a dynamic
connection implies that a facility's input or output connection may change.  For
those simulators that model fleets of facilities with static connections, the
modeling technique is relatively trivial: a fleet of servicing facilities are
directly connected to their serviced facilities. If more than one type of
facility is being serviced (e.g., TRU-based fuels going to thermal and fast
reactors), then either a user or the simulation engine must define the
percentage of capacity going towards each type of serviced facility
\cite{busquim_e_silva_system_2008}.

Most simulators to date have taken the fleet-based, static-connection approach
to modeling fuel cycles which lacks the ability to be easily extended and
improved upon, a key feature of a research code. This work enables the dynamic
connection approach in the \Cyclus nuclear fuel cycle simulator . Dynamic
connections between facilities allow for more complicated scenarios, e.g.,
scenarios with regional influences or scenarios in which competition for
resources exists, to be modeled. The dynamic exchange of resources, however,
introduces two complications. The first is that a given need, e.g., for fuel,
can be met by multiple commodities. As an example, consider fuel for thermal
reactors. Thermal reactors can be fueled by either uranium oxide (UOX) or mixed
uranium-plutonium oxide (MOX). Furthermore, MOX fuel is composed of plutonium
(and some minor actinides, such as americium) from spent thermal fuel as well as
uranium (the source of which can be depleted enrichment tails, depleted recycled
uranium or natural uranium). The second is that the isotopics comprising fuel
orders are \textit{fungible}. A nuclear reactor generates power by fissioning
nuclei. Whether the fissile nuclei involved is \nucl{235}{U}, \nucl{239}{Pu}, or
\nucl{233}{U} makes little difference from a power-generation standpoint -- each
generates power. However, each is involved with a different nuclear fuel cycle.

Supporting economic and social models is rare among simulators. Only one
simulator purports to have any \textit{in situ} economic decision making
\cite{van_den_durpel_daness_2009}. A single other simulator reports including
any socio-geographic concerns \cite{andrianova_desae_2008}. Any supply-demand
framework that enables< economic, geographic, or other behavior models is a
novel step forward in the realm of computational fuel cycle simulation and
analysis.

The core concept that connects both the design decisions regarding fidelity and
facility connections is the notion of material \textit{quality}. Because of the
nature of nuclear reactors, the simulation of their fuel usage must consider the
isotopic profile of material being produced and consumed. This additional
concern greatly complicates the modeling of fuel cycles and must be taken into
account. Each of the issues addressed by the fidelity decision, decay,
transmutation, and fuel fabrication, are all operations on the isotopic profile
of material. 
