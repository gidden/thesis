\chapter{Experiments \& Results}\label{ch:results}

Given a parameterized framework for generating instances of resource exchanges,
experiments are designed and executed to explore the efficiency and quality of
solutions provided by different solvers. 

\S \ref{results:setup} describes the experimentation apparati, including the
computational tools, solvers and relevant output. Two experiemtnal campaigns
were conducted. A scaling campaign, described in \S \ref{results:scale}, was
performed in order to investigate formulation behavior as a function of problem
size. \S \ref {results:stochastic} then describes the results of a stochastic
campaign.

\section{Experimental Setup}

An experiment consists of a set of resource-exchange graph instances executed
with a collection of solvers. When a solution is found, the solution, the time
required to reach the solution, and the objective value (i.e., the dot-product
of cost and flow vectors) are recorded. Because solution time is a quantity of
interest, all instances in an experiement must be executed on homogenous
architecture. Furthermore, all experiments must be executed on equivalent,
homogenous architecture in order to quantify valid comparisons in solutions
times across experimental campaigns.

Six execution nodes on UW-Madison Advanced Computing Initiative (ACI) HTCondor
system form the homogenous environment used to conduct the experiments herein
described. Each execute node is comprised of an 2.90 GHz eight-core,
sixteen-thread, Intel Xeon E5-2690 \cite{} processor with 128 GB of
RAM. Processor hyperthreading was disabled for the duration of the experimental
campaign to allow comparisons between solution times. 

\subsection{Solvers and Formulations}

The solvers used to determine solutions to the DRE instances are then described
in \S \ref{method:solve}. Three solvers are supported: COIN-CLP, COIN-CBC, and
the Greedy Hueristic described in \S \ref{abm:dre:nfctp:heur}.

\subsection{Parameter Variation}

\subsection{Analysis Metrics}
% discuss what kind of analyses are made, c_pref_flow, rms, etc.

\section{Scalability Campaign}

\subsection{Front-End Exchanges}

\subsubsection{Reference Case}
% Include large and small

\subsubsection{Instance Parameter Variation}
% Include large and small r_l_c

\subsubsection{Convergence Criteria}

\subsection{Back-End Exchanges}

\subsubsection{Reference Case}
% Include large and small

\subsubsection{Instance Parameter Variation}
% Include large and small r_l_c

\subsubsection{Convergence Criteria}

\section{Stochastic Campaign}

\subsection{Front-End Exchanges}
% talk about base parameter vector

\subsubsection{Fundamental Parameter Variation}

\subsubsection{Instance Parameter Variation}

\subsection{Back-End Exchanges}
% talk about base parameter vector

\subsubsection{Fundamental Parameter Variation}

\subsubsection{Instance Parameter Variation}

\section{Observations}
