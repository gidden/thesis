
\section{Dynamic Resource Exchange}\label{abm:dre}

\subsection{Problem Statement}

As a next-generation nuclear fuel cycle simulation framework, Cyclus maintains a
primary goal of modeling flexibility. As facility, institutional, and regional
models are proposed, they should be relatively easily implemented and utilized
in the Cyclus simulation framework. Furthermore, the level of modeling
abstraction for different facilities in a fuel cycle will be different based on
the needs of archetype developer. Any supply-demand resolution framework,
therefore, must be able to support arbitrary facilities. One way to approach
such a problem is to treat facilities as black boxes, clearly defining a
supply-demand communication framework.

As stated previously in \S \ref{abm:abm}, a number of considerations must be
taken into account in such a framework. Supply and demand must be able to be
solved globally at any given time step. Resources must be able to be treated in
a fungible manner. The framework must be able to incorporate arbitrary,
agent-defined constraints. 

In order to address each of these concerns, the concept of a Dynamic Resource
Exchange (DRE) was developed and implemented. That process was motivated by the
following problem statement:

\begin{quote}
    If facilities are treated as individual black boxes and connections between
    facilities are determined dynamically, how does one match suppliers with
    demanders considering quantity and quality-based supply constraints,
    quantity and quality-based demand constraints, supply response to
    quality-based demands, and issues of fungibility?
\end{quote}

\subsection{Information Gathering}\label{abm:dre:info}

Supply-demand determination at any given time step begins with three
\textit{phases}, the terminology of which is influenced from previous supply
chain agent-based modeling work \cite{julka_agent-based_2002}. Importantly, this
information-gathering step is agnostic as to the actual matching algorithm used,
it is concerned only with querying the current status of supply and demand in
the simulation.

The first phase allows consumers of commodities to denote both the quantity of a
commodity they need to consume as well as the target isotopics, or quality, by
\textit{posting} their demand to the market exchange. This posting informs
producers of commodities what is needed by consumers, and is termed the
\textit{Request for Bids} (RFB) phase. Consumers are allowed to over-post, i.e.,
request more quantity than they can actually consume, as long as a corresponding
capacity constraint accompanies this posting. Further, consumers are allowed to
post demand for multiple commodities that may serve to meet the same combine
capacity. For example, consider an LWR that can be filled with MOX or UOX. It
can post a demand for both, but must define a preference over the set of
possible commodities it can consume. Another example is that of an advanced fuel
fabrication facility, i.e., one that fabricates fuel partially from separated
material that has already passed through a reactor. Such a facility can choose
to fill the remaining space in a certain assembly with various types of fertile
material, including depleted uranium from enrichment or reprocessed uranium from
separations. Accordingly, it could demand both commodities as long as it
provides a corresponding constraint with respect to total consumption. At the
completion of the RFB phase, the market exchange will have a set of consumption
portfolios, $P$, where each portfolio consists of a set requests, $R$, a
cardinal preference over the requests, $\alpha_R$, and possibly a set of
constraints over the requests, $C_R$. Each request consists of a quantity,
$q_r$, and a target isotopic vector, $I_r$.

The second phase allows suppliers to \textit{respond} to the set of consumption
portfolios, and is termed the \textit{Response to Request for Bids} (RRFB) phase
(analogous to Julka's Reply to Request for Quote phase
\cite{julka_agent-based_2002}). Each consumption portfolio is comprised of
requests for some set of commodities. Accordingly, for each request, suppliers
of that commodity denote production capacities and an isotopic profile of the
commodity they can provide. Suppliers are allowed to offer the null set of
isotopics as their profile, effectively providing no information. A supplier may
have its production constrained by more than one parameter. For example, a
processing facility may have both a throughput constraint (i.e., it can only
process material at a certain rate) and an inventory constraint (i.e., it can
only hold some total material). Further, the facility could have a constraint on
the quality of material to be processed, e.g., it may be able to handle a
maximum radiotoxicity for any given time step which is a function of both the
quantity of material in processes and the isotopic content of that material. The
formulation provided in \S\ref{sec:gfctp} allows for multiple of such
constraints as long as they are linear functions of the demanded commodity
quantity. At the completion of the RRFB phase the possible connections between
supplier and producer facilities, i.e., the arcs in the graph of the
transportation problem, have been established with specific capacity constraints
defined both by the quantity and quality of commodities that will traverse the
arcs.

The final phase of the information gathering procedure allows consumer
facilities to adjust their set of preferences and for managers of consumer
facilities to affect the consumer's set of preferences, as described in the
remaining sections. Accordingly, the last phase is termed the \textit{Preference
 Adjustment} (PA) phase. Preference adjustments can occur in response to the
set of responses provided by producer facilities. Consider the example of a
reactor facility that requests two fuel types, MOX and UOX. It may get two
responses to its request for MOX, each with different isotopic profiles of the
MOX that can be provided. It can then assign preference values over this set of
potential MOX providers. Another prime example is in the case of repositories. A
repository may have a defined preference of material to accept based upon its
heat load or radiotoxicity, both of which are functions of the quality, or
isotopics, of a material. In certain simulators, limits on fuel entering a
repository are imposed based upon the amount of time that has elapsed since the
fuel has exited a reactor, which can be assessed during this phase. The time
constraint is, in actuality, a constraint on heat load or radiotoxicity (one
must let enough of the fission products decay). A repository could analyze
possible input fuel isotopics and set the arc preference of any that violate a
given rule to 0, effectively eliminating that arc.

\subsection{Exchange Structure}

Upon completion of the information gathering phase, a \textit{bipartite} network
is formed. The network consists of receiving (request) nodes, $U$, and sending
(bid) nodes, $V$. For each request node, $u$, there may be many bid nodes;
however, there is a one-to-one mapping between bid nodes and request nodes. In
other words, a given bid node, $v$, is a unique response to a request node, $v$.

\TODO{Include figure}

There is not a direct analog between the \textit{portfolios} described in \S
\ref{abm:dre:info} and the bipartite graph structure. The notion of
\textit{portfolios} does map onto the formulation, however, and will be
described further in \S \ref{abm:dre:form}.

To ensure a feasible solution, additional nodes are added to both $U$
and $V$. A single bid node is added to $V$ and $N_{p}$ nodes are added to $U$,
where $N_{p}$ is the number of request portfolios. Thus, the total size of node
sets and the total number of arcs in the system are

\begin{equation}
  \left|{U}\right| = \left|{U}\right| + N_{p}
\end{equation}

\begin{equation}
  \left|{V}\right| = \left|{V}\right| + 1
\end{equation}

\begin{equation}
  \left|{A}\right| = \left|{A}\right| + N_{p}
\end{equation}

Each arc in the additional set, $A'$, is unconstrained (any amount of resource
flow is allowed) and is given a preference strictly less than the lowest
preference in the system, i.e.,

\begin{equation}
  p' < \min P.
\end{equation}

Accordingly, any additional arc will only be engaged if no other into a request
node, $u$, have available capacity and if the associated request portfolio is
not satisfied. Arc in $A'$ are referred to as \textit{false arcs}, because they
represent flows that do not exist in final solutions.

\TODO{Include figure}

\subsubsection{Commodities}

During the information gathering step in \S \ref{abm:dre:info}, consumers and
suppliers are queried based on \textit{commodities}. A consumer is allowed to
request multiple commodities, and a supplier is allowed to supply multiple
commodities; however, each possible resource transfer is based on a single
commodity. Accordingly, it is possible to color each arc, given a
commodity-to-color mapping.

For example, consider an exchange with three fuel commodities ($A$, $B$, $C$),
two requesters ($R_1$, $R_2$), and two suppliers ($S_1$, $S_2$) in the following
configuration:

\TODO{Include table}

Given the color map $A$: red, $B$: green, $C$: blue, the resulting exchange
graph can be colored:

\TODO{Include figure}

Importantly, the notion of commodities is critical during the information
gathering step and can be used to group arcs in an exchange graph. It also is
compelling when generating the formulation shown in \S
\ref{abm:dre:form}. However, given the constructed network graph, constraint
structure, and preference structure, the notion of commodities is not necessary
for the exchange graph to be solved.

\subsection{Mathematical Programming Formulation}\label{abm:dre:form}

\subsubsection{Overview}

An instance of supply and demand can be solved in a variety of ways. To solve
the system optimally, however, a formal investigation and solution structure is
needed. This section describes the construction of such a formulation.

The formulation is informed by the supply-demand parameters gathered by the
methodology described in the previous section. The basis for the formulation is
the Multicommodity Transportation Problem described in \S\ref{intro:mtp} with
some departures described in detail below. Two separate formulations are
provided. The first is a strictly linear program (LP) while the second is a
mixed-integer linear program (MILP).

The LP formulation can be solved quickly, but allows split orders, i.e., orders
that are not fully filled. The nuclear fuel cycle deals with bundled orders,
such as nuclear fuel assemblies, thus this modeling paradigm is only an
approximation. The MILP provides a more realistic exchange, but can take much
longer to solver. The following sections will describe both formulations,
starting with the LP formulation and then introducing adaptations required to
formulate the MILP version. Finally, a heuristic solution is described and
initial results are described using a proof-of-principle.

\subsubsection{The Nuclear Fuel Cycle Transportation Problem}

Supply and demand in a nuclear fuel cycle context is inherently a multicommodity
problem. A light water reactor can be fueled by both UOX and MOX fuel, for
instance. How it is fueled is a result both of fuel availability and associated
preferences. Allowing for complex physical and chemical constraints on both
processes and inventories, as well as including economics-based approaches for
determining exchange preferences is a complicated affair. Determining the
optimum solution to such a system is even more complicated. Accordingly,
sophisticated tools in both the operations research and agent based modeling
realms have been leveraged to accomplish the task.

\paragraph{Terminology}

%%% 
\begin{table} [h!]
\centering
\begin{tabularx}{\columnwidth-10pt}{|c|X|} % line wraps second column if too long
\hline
Set         & Description \\
\hline
$H$         & all commodities  \\
$I$         & all producers  \\
$J$         & all consumers  \\
$X$         & the feasible set of flows between producers and consumers  \\
$K_{i}^{h}$  & the set of constraining capacities for 
            producer $i$ of commodity $h$  \\
$H_{j}$     & the set of satisfying commodities for consumer $j$  \\
\hline
\end{tabularx}
\caption{Sets Appearing in the GFCTP-LP Formulation}
\label{tbl:GFCTP-LP-sets}
\end{table}
%%% 

%%% 
\begin{table} [h!]
\centering
\begin{tabularx}{\columnwidth-10pt}{|c|X|} % line wraps second column if too long
\hline
Variable    & Description \\
\hline
$c_{i,j}^{h}$             & the unit cost of commodity $h$ 
                          for producer $i$ and consumer $j$  \\
$x_{i,j}^{h}$             & a decision variable, the flow of commodity $h$ 
                          for producer $i$ and consumer $j$  \\
$q_{j}^{h}$               & the requested quality of commodity $h$ 
                          by consumer $j$  \\
$\beta_{i,k}(q_{j}^{h})$  & a capacity translation function for capacity 
                          constraint $k$ of producer $i$ given $q_{j}^{h}$ \\
$s_{i,k}^{h}$             & a supply capacity of producer $i$ corresponding to 
                          capacity constraint $k$ of commodity $h$ \\
$d_{j}(H_{j})$            & the total demand of consumer $j$ over the set of 
                          satisfying commodities $H_{j}$ \\
\hline
\end{tabularx}
\caption{Variables Appearing in the GFCTP-LP Formulation}
\label{tbl:GFCTP-LP-vars}
\end{table}
%%%

\paragraph{Objective Function}

In any network flow problem, of which transportation problems are a subset, the
cost of transporting commodities is what drives the solution. Accordingly, an
accurate cost function is necessary to determine an accurate solution. Because
the \Cyclus environment is still a nascent simulation platform, accurate pricing
metrics, and what such metrics even are in terms of a centuries-long fuel cycle
simulation, are generally difficult to ascertain, with the current standard source
being the Advanced Fuel Cycle Cost Basis
report \cite{shropshire_advanced_2009}. Accordingly, the cost function is
currently a measure of simulation entity preference, rather than a concrete
representation of cost.

The notion of preference extends the work of Oliver's affinity metric
\cite{oliver_geniusv2:_2009}. The preference metric is generally consumer
centric, i.e., consumers have a preference over the possible commodities that
could meet their demand. For example, a reactor may be able to use UOX or MOX
fuel, but may prefer to use MOX fuel. Such a preference differential allows the
projection of real-world cost into the simulation. Additionally, the managers of
a given facility, which in the \Cyclus simulation environment include its
Institution and Region, also exert an influence over its preference. An obvious
example is the concept of affinities given in \cite{oliver_geniusv2:_2009}. In
Oliver's work, an affinity or preference existed between facilities in
``similar'' institutions in order to drive the trading between institutions as a
simple model of international relations. This idea is expanded upon to cover a
facility's other managers and the commodities themselves. Additionally, a
preference can be delineated between the proposed qualities of the same
commodity from different vendors, e.g. if two vendors of MOX fuel
exist. Finally, the notion of a preference is a positive one, and we require a
notion of cost to solve the minimum-cost formulation of the multicommodity
transportation problem with side constraints. Therefore one must utilize a
translation function.

Formally, we define a preference function, $\alpha_{i,j}(h)$, which is a
cardinal preference ordering over a consumer's satisfying commodity set.

\begin{equation}
\alpha_{i,j}(h) \: \forall i \in I \: \forall h \in H_{j} 
\end{equation}

This ordering is a function both of the consumer, $j$, and producer, $i$. The
dependence on producer encapsulates the relationship effects due to managerial
preferences. We then define a cost translation function, $f$, that operates on
the commodity preference function to produce an appropriate cost.

\begin{equation}
f : \alpha_{i,j}(h) \to c_{i,j}^{h}
\end{equation}

A naive implementation, and perhaps all that is necessary for a
proof-of-principle, is to define f as an inversion operator.

\begin{equation}
f(x) = \frac{1}{x}
\end{equation}

\paragraph{Constraints}
This formulation deviates from the normal MCTP formulation via the expansion of
capacity constraints (Equation \ref{eqs:GFCTP-LP_sup}) and the inclusion of a
constraint allowing multiple commodities that are able to meet the demand of a
producer (Equation \ref{eqs:GFCTP-LP_dem}). The former constraint maintains the
multi-commodity nature of the formulation. 

Under certain conditions, the GFCTP-LP will result in a simpler problem. The
first possible condition is that each consumer could have its demand met by only
one commodity, i.e.,

\begin{equation}\label{eqs:1demand}
  \left|{H_{j}}\right| = 1 \: \forall \: j \in J.
\end{equation}

In such a situation, the GFCTP-LP can be transformed into an analog of the
separable transportation problem as shown in \cite{bertsekas_network_1998}. Such
a condition will effectively allow one to solve $N$ different instances of a
single-commodity problem, where $N$ is the cardinality of $H$. 

The second simplifying condition is if the constraining capacity set has a
cardinality of unity, i.e., 

\begin{equation}\label{eqs:1constraint}
  \left|{K_{i}^{h}}\right| = 1 \: \forall \: i \in I, \: \forall \: h \in H.
\end{equation}

If both Equation \ref{eqs:1constraint} and \ref{eqs:1demand} hold, then the
GFCTP-LP is in fact the a normal Transportation Problem, because the quality
translation function ($\beta_{i,k}(q_{j}^{h})$) translates to a constant at
solution time. 

These simplifications are important to the computation time required to solve
the resulting problem instance. The general solution technique for LPs is the
Simplex Method, as previously described. Klee and Minty show that in the worst
case, the Simplex Method will execute in exponential time \cite{klee_good_1970},
but in practice it is generally considered very computationally efficient. If
the problem can be simplified to a TP, then the Transportation Simplex Method
can be used \cite{ahuja_network_1993}.

\paragraph{Capacity Translation Function and Constraints Example}

The notion of a capacity translation function is something that has been
introduced out of necessity due to the complexity of the GFCTP. Accordingly, an
example will help clarify its purpose. This time can also be used to provide an
example of a producer with multiple capacity constraints for a given commodity.

Take, for example, an enrichment facility. Such a facility produces the
commodity enriched uranium (EU). This facility has two constraints on its
operation for any given time period: the amount of Separative Work Units (SWU)
that it can process, $s_{enr,SWU}$ and the total natural uranium (NU) feed it
has on hand, $s_{enr,NU}$. Note that neither of these capacities are measure
directly in the units of the commodity it produces, i.e., kilograms of enriched
uranium (EU). The set of values for $K_{i}^{h}$ for this facility are:

\begin{equation}\label{eqs:enr-constr-commods}
  K_{enr}^{EU} = \{ \mbox{SWU}, \mbox{NU} \}
\end{equation}

Consider a set of requests for enriched uranium that this facility can possibly
meet. Such requests have, in general, two parameters: $P_{j}$, the total product
quantity (in kilograms), and $\varepsilon_{j}$, the product enrichment (in w/o
\nucl{235}{U}).\footnote{The notation for enrichment, $\varepsilon_{j}$, is chosen over its
normal form, $x_p$, to limit confusion with the LP notation of material flow,
$x^h_{i,j}$.}  For the purposes of this constraint set, the quality of material
in question is its enrichment, i.e.,

\begin{equation}\label{eqs:enr-q-swu}
  q_{j}^{EU} \equiv \varepsilon_{j}.
\end{equation}

These values are set during a prior phase of the overall matching algorithm, and
can therefore be considered constant. Further, let us note that, in general, an
enrichment facility's operation, or rather its capacity, is governed by two
parameters: $\varepsilon_{f,enr}$, the fraction of \nucl{235}{U} in its feed material, and
$\varepsilon_{t,enr}$, the fraction of \nucl{235}{U} in its tails material. These parameters
determine the amount of SWU required to produce some amount of enriched uranium:

\begin{align}
\begin{split}
\label{eqs:swu}
SWU = & \:\: P ( V(\varepsilon_{j}) 
      + \frac{\varepsilon_{j} - \varepsilon_{f,enr}}
               {\varepsilon_{f,enr} - \varepsilon_{t,enr}} V(\varepsilon_{t,enr}) \\
      & - \frac{\varepsilon_{j} - \varepsilon_{t,enr}}
               {\varepsilon_{f,enr} - \varepsilon_{t,enr}} V(\varepsilon_{f,enr}) )
\end{split}
\end{align}

$P$ in Equation \ref{eqs:swu} is the amount of produced enriched uranium, and
$V(x)$ is the value function,

\begin{equation}\label{eqs:value}
  V(x) = (1-2x) \ln \left(\frac{1-x}{x}\right)
\end{equation}

Utilizing the above equations, one can denote the functional forms of the
arguments of this facility's two capacity constraints.

\begin{align}
\label{eqs:enr-prod-beta}
\beta_{enr,NU}(\varepsilon_{j}) = & \:\: \frac{\varepsilon_{j} - \varepsilon_{t,enr}}
                                      {\varepsilon_{f,enr} - \varepsilon_{t,enr}} \\
\begin{split}
\label{eqs:enr-swu-beta}
\beta_{enr,SWU}(\varepsilon_{j}) = & \:\: V(\varepsilon_{j}) \\
                         & + \frac{\varepsilon_{j} - \varepsilon_{f,enr}}
                                  {\varepsilon_{f,enr} - \varepsilon_{t,enr}} V(\varepsilon_{t,enr}) \\
                         & - \frac{\varepsilon_{j} - \varepsilon_{t,enr}}
                                  {\varepsilon_{f,enr} - \varepsilon_{t,enr}} V(\varepsilon_{f,enr})
\end{split}
\end{align}

These constraints correspond to the per-unit requirements for enriched uranium
of natural uranium feed and SWU. Finally, we can form the set of constraint
equations for the enrichment facility by combining
Equations \ref{eqs:GFCTP-LP_sup}, \ref{eqs:enr-q-swu},
\ref{eqs:enr-prod-beta}, and \ref{eqs:enr-swu-beta}.

\begin{align}
\label{eqs:enr-prod-constr}
\sum_{j \in J}\beta_{enr,NU}(\varepsilon_{j}) \: x_{enr,j}^{EU}  & \leq s_{enr,NU} \\
\label{eqs:enr-swu-constr}
\sum_{j \in J}\beta_{enr,SWU}(\varepsilon_{j}) \: x_{enr,j}^{EU} & \leq s_{enr,SWU}
\end{align}

\paragraph{Satisfying Commodity Set Example}

The other departure the GFCTP-LP takes from the normal MCTP formulation is the
location of its multicommodity dependence. As presented above, the
MCTP formulation includes a multicommodity arc capacity constraint, Equation
\ref{eqs:GFCTP-LP_dem}. This constraint models a
situation in which different commodities can satisfy a consumer's demand. There
is no direct analog in the GFCTP, i.e., transportation arcs are assumed separate
for separate commodities.

Take the enrichment facility example, expanding on the previous discussion. Note
that an enrichment facility takes feed uranium and then enriches its \nucl{235}{U}
content. This feed uranium can come from different sources which have different
feed enrichments. In practice, the most likely sources of feed uranium are
natural uranium (NU) or recycled uranium (RU), a product of reprocessing light
water reactor fuel. Recycled uranium may be advantageous to use if it has a
higher weight percent of \nucl{235}{U} than does natural uranium. We can now state the
set the values for $H_{j}$ for this facility:

\begin{equation}\label{eqs:enr-dem-commods}
  H_{enr} = \{ \mbox{NU}, \mbox{RU} \}
\end{equation}

\paragraph{LP Formulation}

Combining the previous discussions, the LP Formulation of the NFCTP can be constructed.

%%% 
\begin{subequations}\label{eqs:GFCTP-LP}
  \begin{align}
    %%
    \min_{z} \:\: & 
    z = \sum_{h \in H}\sum_{i \in I}\sum_{j \in J}c_{i,j}^{h} x_{i,j}^{h} 
    & \label{eqs:GFCTP-LP_obj} \\
    %%
    \text{s.t.} \:\: &
    \sum_{j \in J}\beta_{i,k}(q_{j}^{h}) x_{i,j}^{h} \leq s_{i,k} 
    &
    \: \forall \: k \in K_{i}^{h},  
    \forall \: i \in I, \forall \: h \in H \label{eqs:GFCTP-LP_sup} \\
    %%
    &
    \sum_{i \in I}\sum_{h \in H_{j}} x_{i,j}^{h} \geq d_{j}(H_{j}) 
    & 
    \forall \: j \in J \label{eqs:GFCTP-LP_dem} \\
    %%
    &
    x^h_{i,j} \geq 0
    &
    \forall \: x \in X \label{eqs:GFCTP-LP_x}
    %%
  \end{align}
\end{subequations}
%%% 

\subsubsection{Adaptation for Fuel Assemblies}



\subsection{A Heuristic Solution}

With full simluation domain knowledge of supply and demand, including false
arcs, a feasible solution can be found. By definition a feasible solution is a
\textit{solution} to the possible flow of resources, but not necessarily an
\textit{optimal} solution. Many heuristics may be applied to bipartite graphs
with constrainted flows. A simple \textit{greedy} heuristic is presented here
and implemented. 

\begin{algorithm}[h!]
 \SetAlgoLined
 \KwData{A resource exchange graph.}
 \KwResult{A valid set of resource flows.}
 sort request partitions by average preference\;
 \ForAll{request partitions, $J_r$} {
   sort requests by average preference\;
   matched $\leftarrow$ 0\;        
   \While{matched $\leq q_{J_r}$ and $\exists$ a request} {
     get next request\;
     sort incoming arcs by preference\;
     \While{matched $\leq q_{J_r}$ and $\exists$ an arc} {
       get next arc\;
       remaining $\leftarrow q_{J_r}$ - matched\;
       to\_match $\leftarrow \min \lbrace$remaining, Capacity(arc)$\rbrace$\;
       AddMatch(arc, to\_match)\;
       matched $\leftarrow$ matched + to\_match\;
     }
   }
 }
 \caption{Greedy Exchange Hueristic}\label{alg::greedy}
\end{algorithm}

The capacity of each arc in question is the most constraining value $s$
associated with either node on the arc. For a node $i \in I_b$ and $j \in J_r$,
the constraining value is defined as

\begin{equation}
  s = \min 
        \lbrace 
        \min \lbrace \frac{s_{I_b, k}}{\beta_{i, j, k}} 
        \: \forall k \in K_{I_b} \rbrace, 
        \: \min \lbrace \frac{s_{J_r, k}}{\beta_{i, j, k}} 
        \: \forall k \in K_{J_r} \rbrace
        \rbrace.
\end{equation}

The constraining values of each arc are updated upon declaration of a match in
Algorithm \ref{alg::greedy}.

\subsection{Proof of Principle}

physor paper
