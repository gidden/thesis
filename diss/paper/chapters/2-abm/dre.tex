\section{Dynamic Resource Exchange}\label{abm:dre}

Dynamic Resource Exchange (DRE) is the functional bedrock on which Cyclus
simulations are built. It defines the interaction mechanisms and methodologies
for agents, specifically agents whose archetypes have implemented the
\code{Trader} interface. This section begins by providing a motivating problem
statement in \S \ref{abm:dre:prob}. It then details the methodology for querying
supply and demand during the information gathering phase of the DRE in \S
\ref{abm:dre:info}. The solution phase, in which the defined DRE is translated
into a form of the Multicommodity Transportation Problem (MCTP) and solved, is
then described in \S \ref{abm:dre:fctp}. Finally, two proof of principle
simulations with novel fuel cycle DREs are presented in \S \ref{abm:dre:proof}.

This section represents the culmination of significant previous effort
\cite{gidden_agent-based_2013, gidden_agent-based_2014,
  gidden_agent-based_slc_2013}. What follows constitutes the refinement of
previous descriptions of the DRE methodology with lessons learned from initial
implementation and usage.

\subsection{Problem Statement}\label{abm:dre:prob}

As a next-generation nuclear fuel cycle simulation framework, Cyclus maintains a
primary goal of modeling flexibility. As facility, institutional, and regional
archetypes are proposed, they should be relatively easily implemented and
utilized in the Cyclus simulation framework. Furthermore, the level of modeling
abstraction for different facilities in a fuel cycle will be different based on
the needs of archetype developer. Any supply-demand resolution framework,
therefore, must be able to support arbitrary facilities. 

As stated previously in \S \ref{abm:abm:limits}, a number of considerations must
be taken into account in such a framework. Supply and demand must be able to be
solved globally at any given time step. Therefore, the framework must support an
arbitrary number of facilities. Further, resources must be able to be treated in
a fungible manner. The framework must be able to handle arbitrary resource
definitions and incorporate arbitrary, agent-defined constraints.

In order to address each of these concerns, the concept of a Dynamic Resource
Exchange (DRE) was developed and implemented. That process was motivated by the
following problem statement:

\begin{quote}
    If facilities are treated as individual black boxes and connections between
    facilities are determined dynamically, how does one match suppliers with
    consumers considering quantity and quality-based supply constraints,
    quantity and quality-based demand constraints, supply response to
    quality-based demands, and issues of fungibility?
\end{quote}

\subsection{Information Gathering}\label{abm:dre:info}

The DRE begins at any given time step with three \textit{phases}, the
terminology of which is influenced from previous supply chain agent-based
modeling work \cite{julka_agent-based_2002}. Importantly, this
information-gathering step is agnostic as to the supply-demand matching
algorithm used, it is concerned only with querying the current status of supply
and demand in the simulation. The collective information gathering procedure is
shown in Figure \ref{fig:procedure}.

\begin{figure}
  \begin{center}
    \includegraphics[]{procedure.pdf}
    \caption[]{\label{fig:procedure}
        Schematic illustrating the DRE's information gathering procedure.}
  \end{center}
\end{figure}

The first phase allows consumers of commodities to denote both the quantity of a
commodity they need to consume as well as the target isotopics, or quality, by
\textit{posting} their demand to the market exchange. This posting informs
producers of commodities what is needed by consumers, and is termed the
\textit{Request for Bids} (RFB) phase. Consumers are allowed to over-post, i.e.,
request more quantity than they can actually consume, as long as a corresponding
capacity constraint accompanies this posting. Requests can be denoted as
\textit{exclusive}. An exclusive request is one that must either be met in full
or not at all. Exclusive requests allow the modeling of quantized, packaged
transfers, e.g., fuel assemblies. 

Consumers are allowed to post demand for multiple commodities that may serve to
meet the same combine capacity. For example, consider an LWR that can be filled
with MOX or UOX. It can post a demand for both, but must define a preference
over the set of possible commodities that can be consumed. Such requests are
termed \textit{mutual requests}. Another example is that of an advanced fuel
fabrication facility, i.e., one that fabricates fuel partially from separated
material that has already passed through a reactor. Such a facility can choose
to fill the remaining space in a certain assembly with various types of fertile
material, including depleted uranium from enrichment or reprocessed uranium from
separations. Accordingly, it could demand both commodities as long as it
provides a corresponding constraint with respect to total consumption. A set of
exclusive requests may also be grouped as mutual requests, in which case the set
is termed \textit{mutually exclusive}.

At the completion of the RFB phase, the market exchange will have a set of
request portfolios. Each each portfolio consists of a set requests. Arbitrary
constraints over the set of requests can be provided that are functions of
quantity or quality.  Each request may have an associated preference. For
requests that mutually satisfy a given demand, a preference distribution informs
the solver as to which commodities should be satisfied first, given
constraints. Finally, each request portfolio has a specific quantity associated
with it.

The second phase allows suppliers to \textit{respond} to the set of request
portfolios, and is termed the \textit{Response to Request for Bids} (RRFB) phase
(analogous to Julka's Reply to Request for Quote phase
\cite{julka_agent-based_2002}). Each request portfolio is comprised of requests
for some set of commodities. Accordingly, for each request, suppliers of that
commodity denote production capacities and an isotopic profile of the commodity
they can provide. Suppliers are allowed to offer the null set of isotopics as
their profile, effectively providing no information. Suppliers are also allowed
to denote responses as exclusive, as is done in the RFB phase. Supply responses
can also be grouped into mutual responses, and sets of responses may be mutually
exclusive. This functionality again supports the notion of quantized orders,
e.g., in the case of fuel assemblies. 

A supplier may have its production constrained by more than one parameter. For
example, a processing facility may have both a throughput constraint (i.e., it
can only process material at a certain rate) and an inventory constraint (i.e.,
it can only hold some total material). Further, the facility could have a
constraint on the quality of material to be processed, e.g., it may be able to
handle a maximum radiotoxicity for any given time step which is a function of
both the quantity of material in processes and the isotopic content of that
material. Multiple of such constraints are allowed. At the completion of the
RRFB phase the possible connections between supplier and producer facilities,
i.e., the arcs in the graph of the transportation problem, have been established
with specific capacity constraints defined both by the quantity and quality of
commodities that will traverse the arcs.

The final phase of the information gathering procedure allows consumer
facilities to adjust their set of preferences and for managers of consumer
facilities to affect the consumer's set of preferences. Accordingly, the last
phase is termed the \textit{Preference Adjustment} (PA) phase. By allowing
facility managers, i.e., a facility's institution and region, to also adjust
preferences, socio-economic models are allowed to inform the exchange of
resources. For example, a region can detect a transregional trade between one of
its facilities and a facility in another region. If a tariff model is employed,
the trade preference and be diminished or even removed.

For facilities, preference adjustments occurs in response to the set of
responses provided by producer facilities. Consider the example of a reactor
facility that requests two fuel types, MOX and UOX. It may get two responses to
its request for MOX, each with different isotopic profiles of the MOX that can
be provided. It can then assign preference values over this set of potential MOX
providers. Another prime example is in the case of repositories. A repository
may have a defined preference of material to accept based upon its heat load or
radiotoxicity, both of which are functions of the quality, or isotopics, of a
material. In certain simulators, limits on fuel entering a repository are
imposed based upon the amount of time that has elapsed since the fuel has exited
a reactor, which can be assessed during this phase. The time constraint is, in
actuality, a constraint on heat load or radiotoxicity (one must let enough of
the fission products decay). A repository could analyze possible input fuel
isotopics and set the arc preference of any that violate a given rule to 0,
effectively eliminating that arc.

\subsection{The Nuclear Fuel Cycle Transportation Problem}\label{abm:dre:fctp}

Supply and demand in a nuclear fuel cycle context is inherently a multicommodity
problem. A light water reactor can be fueled by both UOX and MOX fuel, for
instance. How it is fueled is a result both of fuel availability and associated
preferences. Allowing for complex physical and chemical constraints on both
processes and inventories, as well as including economics-based approaches for
determining exchange preferences is a complicated affair. Determining the
optimum solution to such a system is even more complicated. Accordingly,
sophisticated tools in both the operations research and agent based modeling
realms have been leveraged to accomplish the task.

An instance of supply and demand defined by the DRE information gathering step
can be solved in a variety of ways. It can be cast to a constrained, bipartite
network, and any heuristic that provides a feasible solution to such networks
are valid. To solve the system optimally, however, a formal investigation and
solution structure is needed. This section describes the construction of such a
formulation, entitled the \textit{Nuclear Fuel Cycle Transportation Problem}
(NFCTP).

The basis for the formulation is the Multicommodity Transportation Problem
described in \S\ref{intro:prog} with some departures described in detail
below. Two separate formulations are provided. The first is a strictly linear
program (LP) while the second is a mixed-integer linear program (MILP). A
heuristic is also provided that provides a reasonable solution to most simple
problems.

The LP formulation can be solved quickly, but allows split orders. In other
words, the LP formulation solves a relaxation of the defined instance that does
not take into account \textit{exclusive} requests or bids. The nuclear fuel
cycle deals with bundled orders, such as nuclear fuel assemblies, thus this
modeling paradigm is only an approximation. The MILP provides a more realistic
exchange, but can take much longer to solve. 

\subsubsection{Terminology}

Objects and data structures generated in the information gathering procedure are
used in the formal definition of the NFCTP and must be defined.

Each portfolio can be considered separately. The set of supply portfolios is
denoted as $S$ and the set of request portfolios is denoted as $R$. Each supply
portfolio is comprised of $s_M$ supply nodes, and each request portfolio is
comprised of $r_N$ nodes. The set of supply nodes is denoted $I$, and the set
of request nodes is denoted $J$. The total number of supply and request nodes is
then

\begin{equation}
  \left|{I}\right| = \sum_{s \in S} s_M
\end{equation}

and

\begin{equation}
  \left|{J}\right| = \sum_{r \in R} r_N.
\end{equation}

Each portfolio has a set of commodities, $H$, associated with it. These are
denoted $H_s$ for supply portfolios and $H_r$ for request
portfolios. Furthermore, each portfolio has a set of constraints, $K$,
associated with it. Each constraint has a constraining value, $b_s^k$ and
$b_r^k$, respectively. Additionally, each unique combination of portfolio and
constraint has an associated \textit{constraint coefficient conversion
  function}, denoted $\beta_s^k$ for supply portfolios and $\beta_r^k$ for
request portfolios. Each constraint coefficient conversion function takes as an
argument a proposed resource $q_{i,j}$. A clarifying example of the relation
between portfolios, commodities, constraints, and coefficient conversion
functions is provided in \S \ref{abm:dre:fctp:arcs}. Request portfolios are
provided a quantity constraint by default for which coefficients are unity. For
a set of \textit{mutual requests}, $M$, where each request has a request
quantity, $x_m$, the coefficient is defined by the ratio between the the average
request quantity over all mutual requests and $x_m$ 

\begin{equation}
  \beta_{r, m} = \frac{\bar{x_M}}{x_m}.
\end{equation}

The constraint conversion functions are utilized in the NFCTP by applying them
to the proposed resource transfers, creating capacity coefficients.

Coefficients for supply constraints are defined as

\begin{equation}
  a^k_{i, j} = \beta_s^k(q_{i_j}).
\end{equation}

Coefficients for request constraints are defined as

\begin{equation}
  a^k_{j, i} = \beta_r^k(q_{i_j}).
\end{equation}

Finally, for each supply-request node pair, there is an associated preference,
$p_{i, j}$. The set of all preferences is denoted $P$. Similarly, flow between a
node pair is denoted $x_{i, j}$, and the set of all flows is denoted $X$. The
possible flow on an arc is provided an upper bound by the request node quantity,
$\tilde{x_j}$.

\subsubsection{Exchange Graph}

Upon completion of the information gathering phase, a \textit{bipartite} network
is formed. This network is called the \textit{exchange graph}. The network
consists of sending (bid) nodes, $I$, and receiving (request) nodes, $J$. For
each request node, $j$, there may be many bid nodes; however, there is a
one-to-one mapping between bid nodes and request nodes. In other words, a given
bid node, $i$, is a unique response to a request node, $j$. An example of a bare
exchange graph graph is shown in Figure \ref{fig:ex_bare}.

\begin{figure}
  \begin{center}
    \includegraphics[width=0.75\textwidth]{exchange_bare.pdf}
    \caption{A bare example exchange with supply nodes colored orange on left
      and request nodes colored blue on right. As shown, there can be multiple
      supply nodes connected to a request node, but each supply node corresponds
      uniquely to one request node. It is a specific response to that request,
      as outlined in the RRFB phase.}
    \label{fig:ex_bare}
  \end{center}
\end{figure}

In the bipartite graph, portfolios act as partitions that group nodes
together. Node groups share common constraints, and request node groups share a
common notion of satisfiable quantity, i.e., a default mass-based constraint. An
example of a partitioned exchange graph is shown in Figure \ref{fig:ex_groups}.

\begin{figure}
  \begin{center}
    \includegraphics[width=0.75\textwidth]{exchange_groups.pdf}
    \caption{The same exchange shown in Figure \ref{fig:ex_bare} with the
      inclusion of portfolio partitions. In this example, there are three
      suppliers and two consumers. The second consumer has two requests which
      may satisfy its demand. The second supplier can supply the commodities
      requested by both consumers and has provided two bids accordingly.}
    \label{fig:ex_groups}
  \end{center}
\end{figure}

Because of defined constraints, there may not be sufficient supply or demand in
the simulated exchange. To ensure a feasible solution, a false supply source and
a false demand sink are added to the exchange graph. The false source and sink
are unconstrained. Additionally, false nodes are added to each portfolio. Each
false node in a request portfolio is connected to the false supply source, and
each false node in a supply portfolio is connected to the false request
sink. These arcs are denoted as \textit{false arcs}. The preferences given to
each false arc, $p_f$, is defined to be lower than the lowest preference in the
system, $P$.

\begin{equation}\label{eqn:falsepref}
  p_{f} < \min P
\end{equation}

Given the original number of nodes in $I$ and the number of bid portfolios, the
total number of bid nodes including false nodes is

\begin{equation}
  \left|{I_t}\right| = \left|{I}\right| + \left|{S}\right| + 1
\end{equation}

Similarly, the total number of request nodes is

\begin{equation}
  \left|{J_t}\right| = \left|{J}\right| + \left|{R}\right| + 1
\end{equation}

Finally, the total number of arcs is

\begin{equation}
  \left|{A_t}\right| = \left|{A}\right| + \left|{I}\right| + \left|{J}\right|
\end{equation}

Because preferences are defined as in Equation \ref{eqn:falsepref}, any false
arc will only be engaged if no other possible arc can be engage, due to capacity
constraints. If any flow is assigned to false arcs after the exchange graph is
solved, that flow is ignored when initiating transactions. Figure
\ref{fig:ex_false} shows a fully defined exchange graph.

\begin{figure}
  \begin{center}
    \includegraphics[width=0.75\textwidth]{exchange_false.pdf}
    \caption{The same exchange shown in Figure \ref{fig:ex_groups} with the
      inclusion of false arcs. The false supplier and consumer nodes are shown
      with a dashed outline. Similarly, false arcs are dashed. Note that the
      false nodes have no associated portfolio structure -- there are no
      constraints associated with false nodes and arcs. The inclusion of a false
      supplier and consumer guarantees a feasible solution.}
    \label{fig:ex_false}
  \end{center}
\end{figure}

\subsubsection{Arc Properties}\label{abm:dre:fctp:arcs}

The result of the DRE is flow determined along arcs, where arcs connect supply
nodes to request nodes. A number of properties are defined on arcs, namely
commodities, constraint coefficients, and preferences.

\paragraph{Commodities}

During the information gathering step in \S \ref{abm:dre:info}, consumers and
suppliers are queried based on \textit{commodities}. A consumer is allowed to
request multiple commodities, and a supplier is allowed to supply multiple
commodities. However, each possible resource transfer, i.e., each arc, is based
on a single commodity. Accordingly, it is possible to color each arc, given a
commodity-to-color mapping.

For example, consider an exchange similar to that shown in Figure
\ref{fig:ex_groups} with two fuel commodities ($A$, $B$), two requesters ($R_1$,
$R_2$), and two suppliers ($S_1$, $S_2$, $S_3$) in the configuration described
by Tables \ref{tbl:ex_sup} and \ref{tbl:ex_req}.

\begin{table}[h]
\centering
\begin{tabular}{c|c}
Supplier & Commodities \\ \hline
$S_1$             & $A$         \\
$S_2$             & $A$, $B$    \\
$S_3$             & $B$         \\
\end{tabular}
\caption{A mapping from suppliers to commodities supplied.}
\label{tbl:ex_sup}
\end{table}

\begin{table}[h]
\centering
\begin{tabular}{c|c}
Consumer & Commodities \\ \hline
$R_1$             & $A$         \\
$R_2$             & $B$        
\end{tabular}
\caption{A mapping from requesters to commodities requested.}
\label{tbl:ex_req}
\end{table}

Given the color map $A$: red, $B$: green, the resulting exchange
graph can be colored as shown in Figure \ref{fig:ex_groups_color}.

\begin{figure}
  \begin{center}
    \includegraphics[width=0.75\textwidth]{exchange_groups_color.pdf}
    \caption{The same exchange shown in Figure \ref{fig:ex_groups} arcs colored
      by commodity based on Tables \ref{tbl:ex_sup} and \ref{tbl:ex_req}. A red
      arc corresponds to commodity $A$; a green arc corresponds to commodity
      $B$.}
    \label{fig:ex_groups_color}
  \end{center}
\end{figure}

The notion of commodities is critical during the information gathering step and
can be used to partition arcs in an exchange graph. It also is compelling when
generating the formulations shown in \S \ref{abm:dre:lp} and \S
\ref{abm:dre:milp}. However, given the constructed network graph, constraint
structure, and preference structure, the notion of commodities is not necessary
for the exchange graph to be solved.

\paragraph{Constraint Coefficients}

Constraint coefficients are determined for an arc based on the proposed resource
to be transferred along that arc, the requester's constraint translation
functions, and the suppliers constraint translation function. The notion of a
capacity translation function is something that has been introduced out of
necessity due to the complexity of the DRE. An example of supply-based
constraints is provided to help clarify its purpose.

Consider an supplier enrichment facility, $s$, which produces the commodity
enriched uranium (EU). This facility has two constraints on its operation for
any given time period: the amount of Separative Work Units (SWU) that it can
process, $b_{s}^{SWU}$, and the total natural uranium (NU) feed it has on hand.,
$b_{s}^{NU}$. The constraint set for $s$ is then
 
\begin{equation}\label{eqs:enr-constr-commods}
  K_{s} = \{ \mbox{SWU}, \mbox{NU} \}.
\end{equation}

Note that neither of these capacities are measure directly in the units of the
commodity it produces, i.e., kilograms of EU.

Consider a set of requests for enriched uranium that this facility can possibly
meet. Such requests have, in general, two parameters: $P_{j}$, the total product
quantity (in kilograms), and $\varepsilon_{j}$, the product enrichment (in w/o
\nucl{235}{U}).\footnote{The notation for enrichment, $\varepsilon_{j}$, is
  chosen over its normal form, $x_p$, to limit confusion with the notation of
  material flow, $x^h_{i,j}$.}  For the purposes of this constraint set, the
quality of material in question is its enrichment, i.e.,

\begin{equation}\label{eqs:enr-q-swu}
  q_{j} \equiv \varepsilon_{j}.
\end{equation}

These values are set during a prior phase of the overall matching algorithm, and
can therefore be considered constant. Further, let us note that, in general, an
enrichment facility's operation, or rather its capacity, is governed by two
parameters: $\varepsilon_{f,s}$, the fraction of \nucl{235}{U} in its feed
material, and $\varepsilon_{t,s}$, the fraction of \nucl{235}{U} in its tails
material. These parameters determine the amount of SWU required to produce some
amount of enriched uranium:

\begin{align}
\begin{split}
\label{eqs:swu}
SWU = & \:\: P ( V(\varepsilon_{j}) 
      + \frac{\varepsilon_{j} - \varepsilon_{f,s}}
               {\varepsilon_{f,s} - \varepsilon_{t,s}} V(\varepsilon_{t,s}) \\
      & - \frac{\varepsilon_{j} - \varepsilon_{t,s}}
               {\varepsilon_{f,s} - \varepsilon_{t,s}} V(\varepsilon_{f,s}) )
\end{split}
\end{align}

$P$ in Equation \ref{eqs:swu} is the amount of produced enriched uranium, and
$V(x)$ is the value function,

\begin{equation}\label{eqs:value}
  V(x) = (1-2x) \ln \left(\frac{1-x}{x}\right)
\end{equation}

Utilizing the above equations, one can denote the functional forms of the
arguments of this facility's two capacity constraints.

\begin{align}
\label{eqs:enr-prod-beta}
\beta_{s}^{NU}(\varepsilon_{j}) = & \:\: \frac{\varepsilon_{j} - \varepsilon_{t,s}}
                                      {\varepsilon_{f,s} - \varepsilon_{t,s}} \\
\begin{split}
\label{eqs:enr-swu-beta}
\beta_{s}^{SWU}(\varepsilon_{j}) = & \:\: V(\varepsilon_{j}) \\
                         & + \frac{\varepsilon_{j} - \varepsilon_{f,s}}
                                  {\varepsilon_{f,s} - \varepsilon_{t,s}} V(\varepsilon_{t,s}) \\
                         & - \frac{\varepsilon_{j} - \varepsilon_{t,s}}
                                  {\varepsilon_{f,s} - \varepsilon_{t,s}} V(\varepsilon_{f,s})
\end{split}
\end{align}

These constraints correspond to the per-unit requirements for enriched uranium
of natural uranium feed and SWU. Finally, we can form the set of constraint
equations for the enrichment facility by combining Equations
\ref{eqs:enr-q-swu}, \ref{eqs:enr-prod-beta}, and \ref{eqs:enr-swu-beta}.

\begin{align}
\label{eqs:enr-prod-constr}
\sum_{j \in J}\beta_{s}^{NU}(\varepsilon_{j}) \: x_{s,j}  & \leq b_{s}^{NU} \\
\label{eqs:enr-swu-constr}
\sum_{j \in J}\beta_{s}^{SWU}(\varepsilon_{j}) \: x_{s,j} & \leq b_{s}^{SWU}
\end{align}

\paragraph{Preferences \& Costs}

In any network flow problem, of which transportation problems are a subset, the
cost of transporting commodities is what drives the solution. Thus, a cost
function is necessary to determine a solution. Because the \Cyclus environment
is still a nascent simulation platform, accurate pricing metrics, and what such
metrics even are in terms of a centuries-long fuel cycle simulation, are
generally difficult to ascertain, with the current standard source being the
Advanced Fuel Cycle Cost Basis report
\cite{shropshire_advanced_2009}. Accordingly, the cost function is currently a
measure of simulation entity preference, rather than a concrete representation
of cost.

The notion of preference extends the work of Oliver's affinity metric
\cite{oliver_geniusv2:_2009}. The preference metric is generally consumer
centric, i.e., consumers have a preference over the possible commodities that
could meet their demand. For example, a reactor may be able to use UOX or MOX
fuel, but may prefer to use MOX fuel. Such a preference differential allows the
projection of real-world cost into the simulation. Additionally, the managers of
a given facility, which in the \Cyclus simulation environment include its
Institution and Region, also exert an influence over its preference. An obvious
example is the concept of affinities given in \cite{oliver_geniusv2:_2009}. In
Oliver's work, an affinity or preference existed between facilities in
``similar'' institutions in order to drive the trading between institutions as a
simple model of international relations. This idea is expanded upon to cover a
facility's other managers and the commodities themselves. Additionally, a
preference can be delineated between the proposed qualities of the same
commodity from different vendors, e.g. if two vendors of MOX fuel
exist. Finally, the notion of a preference is a positive one, and we require a
notion of cost to solve the minimum-cost formulation of the multicommodity
transportation problem with side constraints. Therefore one must utilize a
translation function.

Formally, a preference function, $p_{i, j}(h)$, is defined which is a cardinal
preference ordering over a consumer's satisfying commodity set. A preference is
assigned to each arc in the NFCTP.

\begin{equation}
p_{i, j}(h) \:\: \forall i \in I  \:\: \forall h \in H_{r} 
\end{equation}

Preference is a function both of the consumer, $j$, and producer, $i$, and the
proposed resource transfer from consumer to producer. The dependence on producer
encapsulates the relationship effects due to managerial preferences, i.e., the
effects of the Preference Adjustment phase described in \S \ref{abm:dre:info}.

A cost translation function, $f$, is defined that operates on the commodity
preference function to produce an appropriate cost for the NFCTP.

\begin{equation}
f : p_{i,j}(h) \to c_{i,j}
\end{equation}

For the purposes of this work, any operator that preserves the preference
monotonicity and cardinal ordering is suitable.  The inversion operator has been
chosen because it preserves required features and also allows for easy
translation from preference to cost as well as translation from cost to
preference.

\begin{equation}
f(x) = \frac{1}{x}
\end{equation}

If cost data and a valid cost assignment methodology is developed in the future,
costs may be used directly, and the preference-to-cost translation may be
ignored.

\subsubsection{Linear Programming Formulation}\label{abm:dre:lp}

Combining the previous discussions, the LP Formulation of the NFCTP, denoted the
NFCTP-LP, can be constructed. In general, the NFCTP is a minimum cost
transportation problem that includes custom constraints as described in previous
sections. Including all of the discussion in the previous sections, the
formulation is straightforward and shown in Equation \ref{eqs:NFCTP-LP}.

%%% 
\begin{subequations}\label{eqs:NFCTP-LP}
  \begin{align}
    %%
    \min_{x} \:\: 
    & 
    z = \sum_{i \in I}\sum_{j \in J}c_{i,j} x_{i,j} 
    & 
    \label{eqs:NFCTP-LP_obj} \\
    %%
    \text{s.t.} \:\: 
    &
    \sum_{i \in I_s} \sum_{j \in J} a^k_{i,j} x_{i,j} \leq b^k_s 
    &
    \: 
    \forall \: k \in K_s, 
    \forall \: s \in S 
    \label{eqs:NFCTP-LP_sup} \\
    %%
    &
    \sum_{j \in J_r} \sum_{i \in I} a^k_{i,j} x_{i,j} \geq b^k_r 
    &
    \: 
    \forall \: k \in K_r,  
    \forall \: r \in R 
    \label{eqs:NFCTP-LP_req} \\
    %%
    &
    x_{i,j} \in [0, \tilde{x_j}]
    &
    \forall \: i \in I, 
    \forall \: j \in J 
    \label{eqs:NFCTP-LP_x}
    %%
  \end{align}
\end{subequations}
%%% 

The variables and sets used to define Equation \ref{eqs:NFCTP-LP} have been
described in detail in previous sections. A short synopsis of the sets used is
provided in Table \ref{tbl:NFCTP-LP-sets}, and a corresponding synopsis of the
variables used is provided in Table \ref{tbl:NFCTP-LP-vars}.

%%% 
\begin{table} [h!]
\centering
\begin{tabularx}{\columnwidth-10pt}{|c|X|} % line wraps second column if too long
\hline
Set         & Description \\
\hline
$S$     & suppliers \\
$R$     & requesters \\
$I$     & all supply nodes \\
$I_s$   & nodes for a supplier $s$ \\
$J$     & all request nodes \\
$J_r$   & nodes for a requester $r$ \\
$K_s$   & constraints for a supplier $s$ \\
$K_r$   & constraints for a requester $r$ \\
$X$         & the feasible set of flows between producers and consumers  \\
\hline
\end{tabularx}
\caption{Sets Appearing in the NFCTP-LP Formulation}
\label{tbl:NFCTP-LP-sets}
\end{table}
%%% 

%%% 
\begin{table} [h!]
\centering
\begin{tabularx}{\columnwidth-10pt}{|c|X|} % line wraps second column if too long
\hline
Variable    & Description \\
\hline
$c_{i,j}$             & the unit cost of flow
                          from producer node $i$ to consumer node $j$  \\
$x_{i,j}$             & a decision variable, the flow 
                          from producer node $i$ to consumer node $j$  \\
$a_{i,j}^k$ & the constraint coefficient for constraint $k$ 
                          on flow between nodes $i$ and $j$  \\
$b_s^k$   & the constraining value for constraint $k$ of supplier $s$ \\
$b_r^k$   & the constraining value for constraint $k$ of requester $r$ \\
$\tilde{x_j}$ & the requested quantity associated with request node $j$ \\
\hline
\end{tabularx}
\caption{Variables Appearing in the NFCTP-LP Formulation}
\label{tbl:NFCTP-LP-vars}
\end{table}
%%%

Notably, a feasible solution to the formulation provided in Equation
\ref{eqs:NFCTP-LP} is guaranteed due to the presence of false arcs. Accordingly,
the DRE using this formulation will never fail within a simulation.

\subsubsection{Mixed Integer Linear Programming Formulation}\label{abm:dre:milp}

The previous linear program (LP) formulation of the Generic Fuel Cycle
Transportation Problem fully describes many of the types of transactions that
arise at any given time step. However, it does not allow the critical case of
reactor fuel orders, which comprise a large amount of material orders within the
simulation context. Specifically, it allows reactor fuel orders to be met by
more than one supplier with an arbitrary amount of the order met by each
supplier. Put another way, the LP formulation does not contain the discrete
material information required to model the transaction of fuel assemblies. In
order to provide this capability of quantizing orders, binary decision variables
must be introduced and integer programming techniques must be utilized to solve
the resulting mixed integer-linear program.

The addition of integer variables changes both the complexity of the formulation
and the complexity of the solution technique. Such a change requires a Mixed
Integer-Linear Program (MILP) formulation and solution via the branch-and-bound
method which solves NP-Hard combinatorial optimization problems. The Linear
Program (LP) version requires the simplex method which is much more efficient.

\paragraph{Binary Variables}

The primary difference between the LP and MILP formulations is the inclusion
binary decision variables $y_{i,j}$. A variable $y_{i,j}$ has a value of 1 if
flow occurs between producer node $i$ and consumer node $j$. If flow occurs, its
quantity will be equal to the equivalent flow upper bound along that arc,
$\tilde{x}_{j}$, which denote the quantity of a quantized order.

Binary variables, representing quantized flow, are directly related to the
notion of \textit{exclusive} bids and requests discussed in \S
\ref{abm:dre:info}. In the MILP formulation, an arc $(i, j)$ is considered
exclusive if either node $i$ or node $j$ was defined as exclusive in the
information gathering phase of the DRE. Accordingly, it is useful to partition
all arcs based on this characteristic. Given the set of arcs $A$, a partition
exists such that $A$ can be separated into exclusive arcs, $A_e$, and
non-exclusive arcs, or arcs that allow partial flow, $A_p$.

\begin{equation}\label{eqs:arc-union}
  A = A_{p} \cup A_{e}
\end{equation}

Similarly, each partition can be further subdivided into partitions based on
supplier and requester. 

\begin{equation}\label{eqs:arc-union}
  A = \bigcup_{r \in R} A_{p_r} \cup A_{e_r}
\end{equation}

\begin{equation}\label{eqs:arc-union}
  A = \bigcup_{s \in S} A_{p_s} \cup A_{e_s}
\end{equation}

\paragraph{Mutually Exclusive Constraints}

\textit{Mutual} requests and responses were described in \S
\ref{abm:dre:info}. These are defined as a set of requests or responses, of
which only one may be satisfied. This is represented in the formulation as a
constraint on the associated variables. Again, if a variable $y_{i,j}$ is set to
$1$, flow is sent along arc $(i, j)$. If it is $0$, no flow occurs. A
\textit{mutually exclusive} constraint simply says that only one arc in a mutual
set may have a value of 1.

The set of mutually satisfying arcs is denoted $M_s$ and $M_r$ for suppliers and
requesters, respectively. The associated constraints are then defined by
Equations \ref{eqs:mutual_sup} and \ref{eqs:mutual_req}.

\begin{equation}\label{eqs:mutual_sup}
  \sum_{(i, j) \in M_{s}} y_{i,j} \leq 1 \: \forall \: s \in S 
\end{equation}

\begin{equation}\label{eqs:mutual_req}
  \sum_{(i, j) \in M_{r}} y_{i,j} \leq 1 \: \forall \: r \in R 
\end{equation}

\paragraph{Formulation}

Using the above arc partition notation allows for a much simpler written
formulation of the MILP that looks quite close to the related LP formulation
shown in Equation \ref{eqs:NFCTP-LP}. The full formulation of the NFCTP is shown
in Equation \ref{eqs:NFCTP}

%%% 
% this could probably be realigned 
\begin{subequations}\label{eqs:NFCTP}
  \begin{align}
    %%
    \min_{x, y} \:\: 
    & 
    z \:\: = 
    \sum_{(i, j) \in A_p} c_{i,j} x_{i,j} 
    \: + 
    \sum_{(i, j) \in A_e} c_{i,j} \tilde{x_j} y_{i,j} 
    & 
    \label{eqs:NFCTP_obj} \\
    %%
    \text{s.t.} \:\: 
    &
    \sum_{(i, j) \in A_{p_s}} a^k_{i,j} x_{i,j}
    \: + 
    \sum_{(i, j) \in A_{e_s}} a^k_{i,j} \tilde{x_j} y_{i,j}
    \leq b^k_s 
    &
    \: 
    \forall \: k \in K_s, 
    \forall \: s \in S 
    \label{eqs:NFCTP_sup} \\
    %%
    &
    \sum_{(i, j) \in M_{s}} y_{i,j} \leq 1 
    &
    \forall \: s \in S 
    \label{eqs:NFCTP_mut_sup} \\
    %%
    &
    \sum_{(i, j) \in A_{p_r}} a^k_{i,j} x_{i,j}
    \: + 
    \sum_{(i, j) \in A_{e_r}} a^k_{i,j} \tilde{x_j} y_{i,j}
    \geq b^k_r 
    &
    \: 
    \forall \: k \in K_r,  
    \forall \: r \in R 
    \label{eqs:NFCTP_req} \\
    %%
    &
    \sum_{(i, j) \in M_{r}} y_{i,j} \leq 1 
    &
    \forall \: r \in R 
    \label{eqs:NFCTP_mut_req} \\
    %%
    &
    x_{i,j} \in [0, \tilde{x_j}]
    &
    \forall \: (i, j) \in A_p
    \label{eqs:NFCTP_x} \\
    %%
    &
    y_{i,j} \in \left\{ 0, 1 \right\}
    &
    \forall \: (i, j) \in A_e
    \label{eqs:NFCTP_y}
    %%
  \end{align}
\end{subequations}
%%% 

The sets and variables involved in Equation \ref{eqs:NFCTP} are described in
Tables \ref{tbl:NFCTP-sets} and \ref{tbl:NFCTP-vars}.

%%% 
\begin{table} [h!]
\centering
\begin{tabularx}{\columnwidth-10pt}{|c|X|} % line wraps second column if too long
\hline
Set         & Description \\
\hline
$S$     & suppliers \\
$R$     & requesters \\
$A_p$     & arcs that allow \textit{partial} flows \\
$A_e$     & \textit{exclusive} flow arcs  \\
$A_{p_s}$     & arcs that allow \textit{partial} flows for supplier $s$ \\
$A_{e_s}$     & \textit{exclusive} flow arcs for supplier $s$ \\
$A_{p_p}$     & arcs that allow \textit{partial} flows for requester $r$ \\
$A_{e_p}$     & \textit{exclusive} flow arcs for requester $r$ \\
$M_s$     & arcs $(i, j)$ associated with \textit{mutually exclusive} supply for supplier $s$ \\
$M_r$     & arcs $(i, j)$ associated with \textit{mutually exclusive} requests for requester $r$ \\
$X$         & the feasible set of flows between producers and consumers  \\
$Y$         & the binary variable set of flows between producers and consumers  \\
\hline
\end{tabularx}
\caption{Sets Appearing in the NFCTP Formulation}
\label{tbl:NFCTP-sets}
\end{table}
%%% 

%%% 
\begin{table} [h!]
\centering
\begin{tabularx}{\columnwidth-10pt}{|c|X|} % line wraps second column if too long
\hline
Variable    & Description \\
\hline
$c_{i,j}$             & the unit cost of flow
                          from producer node $i$ to consumer node $j$  \\
$x_{i,j}$             & a decision variable, the flow 
                          from producer node $i$ to consumer node $j$  \\
$y_{i,j}$             & a decision variable, whether flow exists 
                          from producer node $i$ to consumer node $j$  \\
$a_{i,j}^k$ & the constraint coefficient for constraint $k$ 
                          on flow between nodes $i$ and $j$  \\
$b_s^k$   & the constraining value for constraint $k$ of supplier $s$ \\
$b_r^k$   & the constraining value for constraint $k$ of requester $r$ \\
$\tilde{x_j}$ & the requested quantity associated with request node $j$ \\
\hline
\end{tabularx}
\caption{Variables Appearing in the NFCTP Formulation}
\label{tbl:NFCTP-vars}
\end{table}
%%%

The examples of the various constraints from the previous section also apply
here. The only difference is the notion of the binary variables, $y_{i,j}$,
which act as on/off switch as to whether a consumer's entire requested amount of
a resource is met by a supplier or not.

It should be noted that this advanced formulation adds significant complexity to
the resolution method at every time step. However, simple heuristics exist. A
common heuristic for MILPs is to solve a relaxed version of the problem in the
form of a linear program, and to round values to form an integer solution. A
heuristic used in Cyclus is provided in \S \ref{abm:dre:nfctp:heur}.

Note that each constraint coefficient for binary variables can be rewritten as
Equation \ref{eqs:constr_simple} and each objective coefficient can be rewritten
as Equation \ref{eqs:obj_simple}.

\begin{equation}\label{eqs:constr_simple}
a^{k\prime}_{i,j} = a^k_{i,j} \tilde{x_j}
\end{equation}

\begin{equation}\label{eqs:obj_simple}
c^{\prime}_{i,j} = c_{i,j} \tilde{x_j}
\end{equation}

Using both updated efinitions, a simpler formulation can be written and is shown
in Equation \ref{eqs:NFCTP_simp}.

%%% 
% this could probably be realigned 
\begin{subequations}\label{eqs:NFCTP_simp}
  \begin{align}
    %%
    \min_{x, y} \:\: 
    & 
    z \:\: = 
    \sum_{(i, j) \in A_p} c_{i,j} x_{i,j} 
    \: + 
    \sum_{(i, j) \in A_e} c^{\prime}_{i,j} y_{i,j} 
    & 
    \label{eqs:NFCTP_simp_obj} \\
    %%
    \text{s.t.} \:\: 
    &
    \sum_{(i, j) \in A_{p_s}} a^k_{i,j} x_{i,j}
    \: + 
    \sum_{(i, j) \in A_{e_s}} a^{k\prime}_{i,j} y_{i,j}
    \leq b^k_s 
    &
    \: 
    \forall \: k \in K_s, 
    \forall \: s \in S 
    \label{eqs:NFCTP_simp_sup} \\
    %%
    &
    \sum_{(i, j) \in M_{s}} y_{i,j} \leq 1 
    &
    \forall \: s \in S 
    \label{eqs:NFCTP_simp_mut_sup} \\
    %%
    &
    \sum_{(i, j) \in A_{p_r}} a^k_{i,j} x_{i,j}
    \: + 
    \sum_{(i, j) \in A_{e_r}} a^{k\prime}_{i,j} y_{i,j}
    \geq b^k_r 
    &
    \: 
    \forall \: k \in K_r,  
    \forall \: r \in R 
    \label{eqs:NFCTP_simp_req} \\
    %%
    &
    \sum_{(i, j) \in M_{r}} y_{i,j} \leq 1 
    &
    \forall \: r \in R 
    \label{eqs:NFCTP_simp_mut_req} \\
    %%
    &
    x_{i,j} \in [0, \tilde{x_j}]
    &
    \forall \: (i, j) \in A_p
    \label{eqs:NFCTP_simp_x} \\
    %%
    &
    y_{i,j} \in \left\{ 0, 1 \right\}
    &
    \forall \: (i, j) \in A_e
    \label{eqs:NFCTP_simp_y}
    %%
  \end{align}
\end{subequations}
%%% 

\subsubsection{A Heuristic Solution}\label{abm:dre:nfctp:heur}

With full simulation domain knowledge of supply and demand, including false
arcs, a feasible solution can be found. By definition a feasible solution is a
\textit{solution} to the possible flow of resources, but not necessarily an
\textit{optimal} solution. Many heuristics may be applied to bipartite graphs
with constrained flows. A simple \textit{greedy} heuristic is presented here
and implemented. 

The maximum flow along an arc, $x_{max}$, depends on the constraints associated
with each node on the arc. For nodes $i$ and $j$ belonging to portfolios $s$ and
$r$, respectively, the maximum allowable flow is defined as

\begin{equation}
  x_{max} = \min 
        \lbrace 
        \min \lbrace \frac{b^k_s}{a^k_{i, j}} 
        \: \forall k \in K_s \rbrace, 
        \: \min \lbrace \frac{b^k_r}{a^k_{i, j}} 
        \: \forall k \in K_r \rbrace
        \rbrace.
\end{equation}

The Greedy Exchange Heuristic matches maximum flow along arcs, up to the
requested amount defined by each request portfolio, $q_r$, after having sorted
all arcs. The constraining values of each arc, $b_k$, are updated upon
declaration of a match (via an \code{AddMatch} function) in Algorithm
\ref{alg::greedy}.

\begin{algorithm}[!h]
 \SetAlgoLined
 \KwData{A resource exchange graph with constraints and preferences.}
 \KwResult{A valid set of resource flows.}
 sort request partitions by average preference\;
 \ForAll{$r \in R$} {
   sort requests by average preference\;
   matched $\leftarrow$ 0\;        
   \While{matched $\leq q_r$ and $\exists$ a request} {
     get next request\;
     sort incoming arcs by preference\;
     \While{matched $\leq q_r$ and $\exists$ an arc} {
       get next arc\;
       remaining $\leftarrow q_r$ - matched\;
       to\_match $\leftarrow \min \lbrace$remaining, $x_{max} \rbrace$\;
       \code{AddMatch}(arc, to\_match)\;
       matched $\leftarrow$ matched + to\_match\;
     }
   }
 }
 \caption{Greedy Exchange Heuristic}\label{alg::greedy}
\end{algorithm}

\subsubsection{Departure from the MCTP}

The classic MCTP includes the coloring of flows based on commodity type. For
example, for a commodity, $h$, the unit cost of flow would be $c^h_{i,j}$ rather
than $c_{i, j}$. This is included because multiple commodities can flow along
the same arc in the MCTP. In other words, the node-arc incidence matrix includes
an extra commodity dimension. 

The multicommodity nature of the NFCTP is included in constraints, rather than
arcs. Because each node pairing, $(i, j)$, corresponds to a specific, proposed
resource transfer, it can only have one commodity associated with it. Instead,
the constraint set, $K$, is applied over multiple arcs, where each arc is
assigned its own commodity. 

Take the enrichment facility example, expanding on the previous discussion. Note
that an enrichment facility takes feed uranium and then enriches its
\nucl{235}{U} content. This feed uranium can come from different sources which
have different feed enrichments. In practice, the most likely sources of feed
uranium are natural uranium (NU) or recycled uranium (RU), a product of
reprocessing light water reactor fuel. Recycled uranium may be advantageous to
use if it has a higher weight percent of \nucl{235}{U} than does natural
uranium. We can now state the set the values for $H_{r}$ for this facility:

\begin{equation}\label{eqs:enr-dem-commods}
  H_{r} = \{ \mbox{NU}, \mbox{RU} \}
\end{equation}

One or more constraints would then accompany any requests. For example, one
could constraint total \nucl{235}{U} content needed, which would include both NU
and RU flows.

\subsection{Implementation}\label{abm:dre:impl}

The DRE and its solution framework are implemented in three layers. The first
layer includes information for specific \code{Resource} types. For example, a
\code{Material}-based exchange is used for agents to communicate supply and
demand information regarding \code{Material} objects. The \textit{resource
  layer} is the point of entry and exit of the DRE framework. It is the
agent-facing interface of the DRE: supply and demand is provided to the DRE as
input during the information gathering step, and trades to be executed are
provided to agents as output.

The second layer, called the \textit{exchange layer}, is a
\code{Resource}-agnostic implementation of a specialized bipartite
graph. Supply/demand constructs in the first layer are translated into stateful
objects representing nodes, arcs, constructs that carry constraint information,
\textit{et cetera}. The collection of objects and structures combine to create
an \code{ExchangeGraph}. Any custom, Cyclus-aware solver can be applied to an
\code{ExchangeGraph} to determine a feasible solution to the DRE.

In order to use sophisticated, 3\textsuperscript{rd} party LP and MILP solving
libraries, the \code{ExchangeGraph} must be translated into an appropriate data
structure representing an instance of the NFCTP, resulting in the
\textit{formulation layer}. The Open Solver Interface (OSI) \cite{coinosi} is
used to create the necessary formulation structures, including a constraint
matrix and objective coefficient vector. The NFCTP instance is then solved.

After a feasible, perhaps optimal, solution to the NFCTP is found, whether in
the exchange or formulation layer, the solution is back-translated to the
resource layer. The agents associated with successful supply-demand connections
are informed, and trades of resources between agents are executed. A graphic of
the entire workflow is shown in Figure \ref{fig:dre_impl}.

\begin{figure}
  \begin{center}
    \includegraphics[width=\textwidth]{exchange_xlation.pdf}
    \caption[]{
      \label{fig:dre_impl}
      The full DRE workflow is shown. The information gathering phase results in
      the resource layer. The resource layer is translated to the exchange
      layer; a decision is made whether to continue translation or to directly
      solve, marked by the number $1$. If the exchange is not solved, it is
      translated into an instance of the NFCTP resulting in the formulation
      layer. A choice of solver is made, marked by the number $2$, and the
      instance is solved.  The solution is back-translated through the exchange
      and resource layers. The result is a series of resource trades to be
      executed in the simulation.}
  \end{center}
\end{figure}

\subsubsection{Resource Layer}

The resource layer utilizes \textit{templated} classes in order to reduce the
amount of code required for implementation. Each object is templated on the
concrete \code{Resource} type, e.g., the \code{Material} and \code{Product}
classes. The fundamental data structures in the resource layer reflect the
constructs of the information gathering procedure described in \S
\ref{abm:dre:info}.

In the RFB phase of the DRE, agents populate \codeb{Request\-Portfolio<T>}s with
\code{Request<T>}s and \codeb{Capacity\-Constraint<T>}s. A \code{Request<T>}
defines a desired \code{Resource<T>}, communicating quantity, quality, and
preference. Any number of \codeb{Capacity\-Constraint<T>}s may be added to a
\codeb{Request\-Portfolio<T>}. A \codeb{Capacity\-Constraint<T>} defines a
capacitating value and a conversion function that takes as an argument a
\code{Resource} and returns a value in units of the conversion function. For
\codeb{Request\-Portfolio<T>}s, constraints are assumed to be demand
constraints, i.e., take the form of a greater-than constraint. In the RRFB phase
of the DRE, agents populate \codeb{Bid\-Portfolio<T>}s with \code{Bid<T>}s and
\codeb{Capacity\-Constraint<T>}s. Agents can inspect the population of
\code{Request<T>}s and associated \code{Resource}s. A \code{Bid<T>} targets a
specific \code{Request<T>}, responding with a proposed \code{Resource} to
transfer to the requester. \codeb {Capacity\-Constraint<T>}s are applied to all
\code{Bid<T>}s in a portfolio. For bidders, constraints are assumed to be
less-than constraints. Before continuing, requesting agents and their managers
are allowed to alter the preference associated with each
\code{Request<T>}-\code{Bid<T>} pair in the PA phase of the DRE. When a solution
to the DRE is found, bidders associated with successful
\code{Request<T>}-\code{Bid<T>} pairs are informed, and a trade of the bidder's
\code{Resource} is initiated.

Future work can be focused on providing more features to the DRE
implementation. A natural extension of the present work is to support both kinds
of constraints, greater and less-than, in \code{Portfolio} data
strutures. Additionally, the PA procedure could use a negotiation model that
involves both suppliers and requesters in order to define a final preference for
an arc. Such an extension would allow for more seamless and natural usage of arc
costs in addition to preferences.

\subsubsection{Exchange Layer} 

The exchange layer is constructed by an \code{ExchangeTranslator} object that
translates the resource layer objects into an instance of an
\code{ExchangeGraph}. Request and bid objects are translated to
\code{ExchangeNode}s, and portfolio objects are translated to
\code{ExchangeGroup}s. Constraint coefficient and preference information is
recorded on \code{ExchangeArc}s, which store a reference to a supply
\code{ExchangeNode} and a demand \code{ExchangeNode}. Finally, constraint values
are stored on the appropriate \code{ExchangeGroup} object.

An \code{ExchangeContext} object is tasked with storing a mapping from
\code{Request<T>} and \code{Bid<T>} objects to their associated
\code{ExchangeNode}. Importantly, the exchange layer does \textit{not} depend on
resource type, i.e., the resource type is abstracted away during
translation. Finally, a general solver can be implemented that operates on the
\code{ExchangeGraph}. A solution to the \code{ExchangeGraph} instance is a
mapping from \code{ExchangeArc}s to flow quantities that does not violate the
provided constraints. After a solution is found, it is back-translated to the
resource layer.

\subsubsection{Formulation Layer}

While a solver may operate on the exchange layer, an instance of an
\code{ExchangeGraph} can be translated fully into the NFCTP. Once in an LP or
MILP form, the DRE instance can be solved by sophisticated 3\textsuperscript{rd}
party libraries. In order to interface with a large number of the possible
solvers, including COIN-OR and CPLEX, the COIN-OR Open Solver Interface API
\cite{coinosi} is utilized.

The translation from the exchange layer to formulation layer is managed by the
\code{ProgTranslator} class. A variable in the NFCTP is associated with each
\code{ExchangeArc}, with variable bounds set by request values on
\code{ExchangeNode}s; a binary variable is used if the arc is exclusive,
otherwise a linear variable is used. Capacity coefficients and preference values
defined for \code{ExchangeArc}s are translated into an objective coefficient
vector and constraint matrix. The right-hand-side $b$ constraint vector is
determined by \code{ExchangeGroup} constraining values.

A solution to the NFCTP instance is determined by the identified solver,
assigning values to linear and integer variables. Linear variable values map
directly to assigned resource flow quantity. If a binary variable is set to
unity in a solution, the maximum possible flow value is assigned, analogous to
$\tilde{x_j}$ in the NFCTP formulation. The variable-flow value assignment is
then back-translated into an equivalent \code{ExchangeArc}-flow value assignment
by the \code{ProgTranslator}.

\subsection{Proof of Principle}\label{abm:dre:proof}

In order to demonstrate the correctness of the methodology and implementation,
two test cases were developed and analyzed. These test cases are entire Cyclus
simulations in which the full DRE procedure is executed at each time step. Both
scenarios validate the ability of the DRE to model preferences, preference
adjustment, and unresolved markets. The first scenario is a simulation including
quantity-based constraints and dynamic commodity-based preferences. The second
scenario illustrates a simulation that involves quality constraints and dynamic
quality-based requests. In each scenario, fuel quantities are treated using
arbitrary units without loss of generality. For the purposes of the enrichment
example, a unit of fuel is equivalent to a kilogram.

Each scenario is comprised of archetypes defined in Cycamore
\cite{cycamore2013}. The minimal Institution and Region archetypes are used
because no complicated facility deployment logic is needed. The facility
archetypes used include the \textit{SourceFacility}, \textit{BatchReactor}, and
\textit{EnrichmentFacility}.

Finally, each instance of the DRE is solved using the greedy heuristic described
in \S \ref{abm:dre:nfctp:heur}. In each case, requests and supplies are
\textit{not} exclusive, and thus multiple sources of supply may be matched to a
request. In general, these exchanges are very small and have a unique objective
solution which corresponds to the solution determined by the greedy heuristic.

\subsubsection{Test Cases}\label{abm:dre:proof:cases}

\paragraph{2 Sources, 3 Reactors}

As shown in Figure \ref{fig::fc1}, this scenario includes three
\textit{BatchReactors} and two \textit{SourceFacilities}. The
\textit{BatchReactors}, denoted as \Reactor{1}, \Reactor{2}, and \Reactor{3},
each have a unique fuel preference. One \textit{SourceFacility} supplies MOX
while the other supplies UOX; these are denoted as \MOXSource{} and
\UOXSource{}, respectively. Any reactor may be fueled from MOX or UOX fuel; both
fuel types are \textit{fungible} in this scenario.

\begin{figure}
  \begin{center}
    \includegraphics[width=0.55\textwidth]{fc1.pdf}
    \caption[]{\label{fig::fc1} Schematic illustrating the first fuel cycle
      scenario. The thickness of the arrows represents the preference value and
      the grey color indicates that a material transfer is possible.}
  \end{center}
\end{figure}

In this example case \Reactor{1}, \Reactor{2}, and \Reactor{3} are deployed
sequentially over 3 time steps. Each of these has a full core when built and
requires 1 unit of fresh fuel at each subsequent time step. Both source facilities
have a capacity of 2.5 units each time step.

The simulation begins with the following facilities: \MOXSource{}, \UOXSource{},
and \Reactor{1}. At time step 2, \Reactor{2} is deployed, followed
by \Reactor{3} at time step 3. \Reactor{1} and \Reactor{2} both are given a
stronger preference for MOX requests than \Reactor{3}. At time step
4, \Reactor{1}, \Reactor{2}, and \Reactor{3} all request to refuel with MOX. At
time step 5, \Reactor{1} changes its preference to UOX. Table \ref{table::scen1}
summarizes the reactor preferences as a function of time.

\FloatBarrier
\begin{table}
  \begin{center}
    \caption{\label{table::scen1} 
        Time sequence of reactor preferences and the total MOX requested. The MOX capacity for each time step is 2.5 units.}
    \begin{tabular}{m{1cm}|cc|cc|cc|m{2cm}}
    \toprule
    Time step & \multicolumn{2}{c|}{\Reactor{1}} & \multicolumn{2}{c|}{\Reactor{2}} & \multicolumn{2}{c|}{\Reactor{3}} & Total MOX Requested [units]\\
              & Fuel & Preference & Fuel & Preference & Fuel & Preference  \\
    \midrule
    1         & MOX  & 1.0 & none &     & none &     & 0.0 \\
    2         & MOX  & 1.0 & MOX  & 1.0 & none &     & 1.0 \\
    3         & MOX  & 1.0 & MOX  & 1.0 & MOX  & 0.5 & 2.0 \\
    4         & MOX  & 1.0 & MOX  & 1.0 & MOX  & 0.5 & 3.0 \\
    5         & UOX  & 2.0 & MOX  & 1.0 & MOX  & 0.5 & 2.0 \\
    \bottomrule
    \end{tabular}
  \end{center}
\end{table}
\FloatBarrier

\paragraph{Enrichment, 2 Reactors}

As pictured in Figure \ref{fig::fc2}, this scenario includes
one \textit{EnrichmentFacility} and
two \textit{BatchReactors}. The \textit{EnrichmentFacility}, denoted
as \Enrichment{}, has a designated capacity at each time step. The
two \textit{BatchReactors}, denoted as \Reactor{1} and \Reactor{2}, request a
given amount and quality of enriched uranium upon refueling.

\begin{figure}
  \begin{center}
    \includegraphics[width=0.55\textwidth]{fc2.pdf}
    \caption[]{\label{fig::fc2} Schematic illustrating the second fuel cycle
      scenario. The thickness of the arrows represents the preference value.}
  \end{center}
\end{figure}

The \Enrichment{} facility is constrained by a constant capacity of 10 SWU per
time step. For this entire simulation, trade between \Enrichment{} and
\Reactor{1} is preferred over trade between \Enrichment{} and \Reactor{2} with
preference values of 1.0 and 0.5, respectively. Each reactor requests 1 unit of
enriched uranium at each time step.

Initially, both reactors are present in the simulation and have a full core of
3\% enriched uranium. On time step 1, \Reactor{1} requests uranium enriched to
5\% U-235 while \Reactor{2} requests uranium at a 3\% enrichment level. At time
step 2, \Reactor{1} reduces its enrichment request to
3\%. Table \ref{table::scen2} summarizes the reactor requests as a function of
time.

\FloatBarrier
\begin{table}
  \begin{center}
    \caption{\label{table::scen2}
        Time sequence of reactor preferences and the total SWU requested. The SWU capacity for each time step is 10.}
    \begin{tabular}{m{1cm}|cc|cc|m{2cm}}
    \toprule
    Time step & \multicolumn{2}{c|}{\Reactor{1}} & \multicolumn{2}{c|}{\Reactor{2}} & Total SWU Requested \\
              & Recipe & Preference     &     Recipe & Preference \\
    \midrule
    0         & 3\% U-235 &     & 3\% U-235 &     &  0   \\
    1         & 5\% U-235 & 1.0 & 3\% U-235 & 0.5 & 10.6 \\
    2         & 3\% U-235 & 1.0 & 3\% U-235 & 0.5 &  6.8 \\
    \bottomrule
    \end{tabular}
  \end{center}
\end{table}
\FloatBarrier

\subsubsection{Results}

The cases outlined in \S \ref{abm:dre:proof:cases} have been designed to provide
different conditions at each point in time.  The results for these cases are
discussed below. In each of these cases, the total \Cyclus{} run time was
$\sim$0.1 seconds and the output database size was $\sim$68 kB.

Note that the resource flows in Figures
\ref{fig::2srcs3rxts-t1}-\ref{fig::enr2rxts-t3} have been generated
automatically from \Cyclus{} output using Cyan \cite{Carlsen2014}.  Due to this,
these figures only show facility agents which participated in a resource
exchange. For instance, Figure \ref{fig::2srcs3rxts-t1} does not show the
\UOXSource{} facility, even though it is present in the simulation.

\paragraph{2 Sources, 3 Reactors}

Initially present are the source facilities and \Reactor{1}.  \Reactor{1} has a
preference for accepting MOX fuel over UOX.  The \MOXSource{} capacity of 2.5
units is more than enough to handle the 1 unit of MOX requested by \Reactor{1}.
This matching may be seen in Figure \ref{fig::2srcs3rxts-t1}. The \MOXSource{}
only provides the 1 unit of material requested by \Reactor{1}.  It, correctly,
does not oversupply.

\begin{figure}
  \begin{center}
    \includegraphics[height=3cm]{2_Sources_3_Reactors-t1.pdf}
    \caption[]{\label{fig::2srcs3rxts-t1}Time step 1 for the 2 Sources, 3 Reactors 
        case.}
  \end{center}
\end{figure}

At time step 2 in this simulation, \Reactor{2} is deployed and also requests
fuel with the preference for MOX.  Figure \ref{fig::2srcs3rxts-t2} displays that
the \MOXSource{} indeed has the required capacity to meet the requests of both
of the reactors.  This may seem trivial at first glance but it is important to
emphasize that the resource exchange solver was not altered in any way to handle
both time steps 1 and 2.  Furthermore, the solver on time step 1 had no future
knowledge that \Reactor{2} would be deployed on time step 2.  This is
significantly different than the traditional system dynamics approach.

\begin{figure}
  \begin{center}
    \includegraphics[height=3cm]{2_Sources_3_Reactors-t2.pdf}
    \caption[]{\label{fig::2srcs3rxts-t2}Time step 2 for the 2 Sources, 3 Reactors 
        case.}
  \end{center}
\end{figure}

On time step 3, \Reactor{3} is deployed.  This facility still prefers to accept
MOX fuel over UOX fuel.  At this point, 3 units of MOX are requested (1 unit from
each facility) but the \MOXSource{} may only provide 2.5 units.  Because of
this, the \UOXSource{}, which has been present in the simulation since the
beginning, now enters the exchange to make up for the missing 0.5 units of fuel not
obtainable from the \MOXSource{}. \Reactor{3} is selected to receive the UOX
fuel rather than \Reactor{1} and \Reactor{2}.  These preferences are detailed in
Table \ref{tab::pref-t3}.  In time step, 3 because all agents tie for UOX,
\Reactor{1} and \Reactor{2} tie for MOX, and the \Reactor{3} preference for MOX
is less than the others, \Reactor{3}'s full request for MOX is not met and it
must top-up with UOX.  This situation is displayed in Figure
\ref{fig::2srcs3rxts-t3}.

\begin{table}
  \begin{center}
    \caption{\label{tab::pref-t3}Resource exchange preferences for agents on 
             time steps 3 and 4 for reactors in the 2 sources, 3 reactors case.}
    \begin{tabular}{lcc|cc}
    \toprule
          & \multicolumn{2}{c}{$t=3$} & \multicolumn{2}{c}{$t=4$} \\
    Agent & UOX & MOX & UOX & MOX\\
    \midrule
    \Reactor{1} & 0.0 & 1.0 & 2.0 & 1.0 \\
    \Reactor{2} & 0.0 & 1.0 & 0.0 & 1.0 \\
    \Reactor{3} & 0.0 & 0.5 & 0.0 & 0.5 \\
    \bottomrule
    \end{tabular}
  \end{center}
\end{table}

\begin{figure}
  \begin{center}
    \includegraphics[height=3cm]{2_Sources_3_Reactors-t3.pdf}
    \caption[]{\label{fig::2srcs3rxts-t3}Time step 3 for the 2 Sources, 3 Reactors 
        case.}
  \end{center}
\end{figure}

Finally, on time step 4 the preference of \Reactor{1} for UOX changes from 
0.0 to 2.0.  This alteration causes the tie previously present 
for UOX to be broken.  Furthermore, the value of 2.0 makes this the most 
preferred arc in the system so it is attempted to be satisfied first.  
As may be seen in Figure \ref{fig::2srcs3rxts-t4}, the \UOXSource{} capacity of
2.5 units is more than enough to satisfy the request from \Reactor{1} for 1 unit of UOX.
Since \Reactor{1} does not diminish the capacity of the \MOXSource{} both \Reactor{2} 
and \Reactor{3} are able to obtain their first choice fuel.

\begin{figure}
  \begin{center}
    \includegraphics[height=3cm]{2_Sources_3_Reactors-t4.pdf}
    \caption[]{\label{fig::2srcs3rxts-t4}Time step 4 for the 2 Sources, 3 Reactors 
        case.}
  \end{center}
\end{figure}

This simple 2 source, 3 reactor simulation shows how the resource 
exchange can dynamically and correctly adapt to the both facility deployments and
the preferences that these agents have for requested resources.

\paragraph{Enrichment and 2 Reactors}
\label{subsect::1enr2rxts}

Initially, \Enrichment{}, \Reactor{1}, and \Reactor{2} are all present. The reactors both begin with cores composed of 3\% enriched uranium. 

\begin{figure}
  \begin{center}
    \includegraphics[height=3cm]{1_Enrichment_2_Reactor-t1.pdf}
    \caption[]{\label{fig::enr2rxts-t1}Time step 1 for the Enrichment and 2 Reactors 
        case.}
  \end{center}
\end{figure}

Figure \ref{fig::enr2rxts-t1} shows the result of the resource exchange for time
step 1.  Here the SWU capacity of \Enrichment{} is not sufficient to meet the
requests for 5\% and 3\% enriched fuel simultaneously but is enough to meet
either of them individually.  Therefore, the bid for at least one of the
reactors must be partially unmet.  Due to the preferences, the requests
of \Reactor{1} will be met first.  Thus in
Figure \ref{fig::enr2rxts-t1} \Reactor{1} receives 1 unit of 5\% enriched fuel.
However, \Reactor{2} only receives $\sim$80\% of its request for 3\% enriched
uranium.  This is not enough to run on and so this material is saved for the
future.

\begin{figure}
  \begin{center}
    \includegraphics[height=3cm]{1_Enrichment_2_Reactor-t2.pdf}
    \caption[]{\label{fig::enr2rxts-t2}Time step 2 for the Enrichment and 2 Reactors 
        case.}
  \end{center}
\end{figure}

On time step 2, \Reactor{1} now switches from requesting 5\% enriched fuel to
requesting 3\% enriched fuel.  However, the reactor archetypes are implemented
such that if they have stored fuel from a previous time step, they will instead
only request a mass needed to create 1 unit of fuel.  Therefore, \Reactor{2} only
requests $\sim$0.2 units from \Enrichment{}.  \Reactor{1} continues to request 1 unit
of material and this is met because the SWU capacity constraint is not exceeded.
These resource flows may be seen in Figure \ref{fig::enr2rxts-t2}.

\begin{figure}
  \begin{center}
    \includegraphics[height=3cm]{1_Enrichment_2_Reactor-t3.pdf}
    \caption[]{\label{fig::enr2rxts-t3}Time step 3 for the Enrichment and 2 Reactors 
        case.}
  \end{center}
\end{figure}

No further adjustments of requests were made on time step 3. Thus the results
displayed in Figure \ref{fig::enr2rxts-t3} represent the same dynamic resource
exchange procedure in time step 2.  The key differences here however are that
now the system has returned to a steady state and - unlike in time step 1 - the
SWU capacity of \Enrichment{} is enough to meet the 2 units of 3\% enriched fuel
coming from both reactors.  Thus \Reactor{1} and \Reactor{2} each receive the
kilogram of fuel that they request.  Without further adjustments this system
will continue \emph{ad infinitum}.
