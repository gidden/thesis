
\section{Simulation Principles}\label{abm:sim}

Cyclus is designed to dynamically model the flow of resources and deployment of
facilities in the Nuclear Fuel Cycle (NFC). As such, Cyclus is a
\textit{simulator} which models the NFC as a \textit{system}. System simulation
is a rich field of study, spanning a variety of disciplines, as described in \S
\ref{intro:sim}.

By Law's definition \cite{Law:1999:SMA:554952}, Cyclus is a dynamic,
discrete-event simulation that uses a fixed-increment time advance mechanism. In
general, fixed-increment time advance simulations assume a time step ($\Delta
t$). Further they assume that all events that would happen during a time occur
simultaneously at the end of the time step. This situation can be thought of as
an event-based time advance mechanism, i.e., one that steps from event to event,
that executes all events simultaneously that were supposed to have occurred in
the time step.

A Cyclus simulation models a collection of \textit{entities} which either trade
resources, manage other entities, or perform both actions. The most basic
\textit{entity} in a Cyclus simulation is a Facility. Facilities can be used to
model processes with arbitrary levels of physical fidelity, and can interact
with the simulator and other entities with arbitrary levels of behavioral
fidelity. As such, Cyclus can also be described as an \textit{agent-based model}
(ABM). Accordingly, the \textit{entities} in a given simulation can be
interchangeably referred to as \textit{agents}.

\subsection{Events}

Two key types of events occur in every Cyclus simulation:

\begin{itemize}
\item agent entry into and exit from the simulation
\item the exchange of resources between agents
\end{itemize}

Agent entry and exit events are scheduled by another managing agent, or are
scheduled as an initial condition to the simulation. The managing agent and
managed agent form a parent-child relationship. Upon entering the simulation,
the child entity is constructed and notified of its entry; the parent is then
notified. Upon exiting the simulation, the parent is notified; the child entity
is then notified and desconstructed. Unike many of the simulators described in
\S \ref{intro:fcs}, the Cyclus simulation kernel naturally treats each agent
individually, rather than grouping agents by an attribute and treating
like-facilities in an aggregate manner.

While the determination of supply and demand is complex and described further in
\S \ref{abm:dre}, the execution of resource exchange is rather straightforward
and a primary event in a Cyclus simulation. When an agent's demand for a
resource is matched with another agent's supply of a resource by the Cyclus
kernel, a transfer is initiated. Each transfer is treated as discrete,
individual trade between two agents.

\subsection{Timesteps}

Simulation entities can have arbitrarily complex state which is dependent on the
results of resource exchange and the present discernable status of other agents
in the simulation at a given time step. Furthermore, resource exchange
neccessarily must involve all existing agents in the simulation. Therefore, a
well-defined timestep, incorporating agent entry, exit, resource exchange, and
agent response to system state must be defined. Cyclus implements a timestep
mechanism that deviates slightly from Law's description of fixed-increment time
advance by preserving a specific ordering of \textit{event
  triggers}. Importantly, the following invariant is preserved: \textit{any
  agent that exists in a given time step should experience the entire time step
  execution stack}.

This leads to the following \textit{phases} of time step execution:

\begin{itemize}
\item agents enter simulation (Building Phase)
\item agents respond to current simulation state (Tick Phase)
\item resource exchange execution (Exchange Phase)
\item agents respond to current simulation state (Tock Phase)
\item agents leave simulation (Decommissioning Phase)
\end{itemize}

The Building, Exchange, and Decommissioning phases each include critical,
core-based events, and are called \textit{Kernel} phases. The Tick and Tock
phases do not include core-based events, and instead let agents react to
previous core-based events and inspect core simulation state. Furthermore, they
are periods in which agents can update their own state and are accordingly
considered \textit{Agent} phases. 

Technically, whether agent entry occurs simultaneously with agent exit or not
does not matter from a simulation-mechanics point of view, because the two
phases have a direct ordering. It will, however, from the point of view of
module development. It is simpler to think of an agent entering the simulation
and acting in that time step, rather than entering a simulation at a given time
and taking its first action in the subsequent time step.

In the spirit of Law's definition of a fixed-increment time advance mechanism,
there is an addtiional important invariant: \textit{there is no guaranteed agent
  ordering of within-phase execution}. This invariant allows for:

\begin{itemize}
\item a more cognitively simple process
\item paralellized implementation
\end{itemize}
