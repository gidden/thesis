
\section{Simulation Principles}\label{abm:sim}

As its core functionality, Cyclus seeks to dynamically model the flow of
resources and deployment of facilities in the Nuclear Fuel Cycle (NFC). As such,
Cyclus is a \textit{simulator} which models the NFC as a \textit{system}.

By Law's definition \cite{Law:1999:SMA:554952}, Cyclus is, broadly, a dynamic,
discrete-event simulation that uses a fixed-increment time advance mechanism. In
general, fixed-increment time advance scenarios assume a time step ($\Delta t$),
and assume that all events that would happen during a time occur simultaneously
at the end of the time step. This situation can be thought of as an event-based
time advance mechanism, i.e., one that steps from event to event, that executes
all events simultaneously that were supposed to have occurred in the time step.

Again using Law's definition, Cyclus models a collection of \textit{entities}
which either trade resources, manage other entities, or perform both
actions. The most basic \textit{entity} in a Cyclus simulation is a
Facility. Facilities can be used to model processes with arbitrary levels of
physical fidelity, and can interact with the simulator and other entities with
arbitrary levels of behavioral fidelity. As such, Cyclus can also be described
as an \textit{agent-based model} (ABM). Accordingly, the \textit{entities} in a
given simulation can be interchangeably referred to as \textit{agents}.

\subsection{Events}

Two key types of events occur in every Cyclus simulation:

\begin{itemize}
\item agent entry into and exit from the simulation
\item the exchange of resources between agents
\end{itemize}

Agent entry and exit events are scheduled by another managing agent, or
scheduled as an initial condition to the simulation. Upon entry or exit,
newly-built or about-to-be-decommissioned agents are informed as are their
managing agent. The Cyclus simulation kernel treats each agent individually,
rather than grouping agents by an attribute and treating like-facilities in an
aggregate manner.

While the determination of supply and demand is complex and described further in
\S \ref{abm:dre}, the execution of resource exchange is rather straightforward
and a primary event in a Cyclus simulation. When a agent's demand for a resource
is matched with another matches's supply of a resource by the Cyclus kernel, a
transfer is initiated. Each transfer is treated as discrete, individual trade
between two agents.

Simulation entities can have arbitrarily complex state which is dependent on the
results of resource exchange and the present status of agents in the simulation
at a given time step. Accordingly, methods that allow entities to update state
and schedule new entity entry and exit must occur in response to these events.

\subsection{Timesteps}

Because there is a key event that defines agent interaction in a given time
step, it is necessary to involve all agents in that interaction. Accordingly it
is necessary that there be an ordering between these two key types of events,
deviating slightly from Law's description of fixed-increment time
advance. Specifically, the following invariant is preserved: \textit{any agent
  that exists in a given time step should be included in the resource exchange,
  or, equivalently, experience the entire time step execution stack}.

This leads to the following ordering, or \textit{phases}, of time step
execution:

\begin{itemize}
\item agents enter simulation (Building Phase)
\item agents respond to current simulation state (Tick Phase)
\item resource exchange execution (Exchange Phase)
\item agents respond to current simulation state (Tock Phase)
\item agents leave simulation (Decommissioning Phase)
\end{itemize}

The Building, Exchange, and Decommissioning phases each include critical,
core-based events, and will be called \textit{Kernel} phases. The Tick and Tock
phases do not include core-based events, and instead let agents react to
previous core-based events and inspect core simulation state. Furthermore, they
are periods in which agents can update their own state and are accordingly
considered \textit{Agent} phases. In general, Agent phases \textit{must} bracket
\textit{critical} Kernel phases, of which only the Exchange Phase exists for
now. If another critical core phase is added in the future it must provide a
similar invariant, i.e., that it is bracketed by Agent phases. For example, if a
new phase is added before Exchange, then the time execution stack would follow
as: Building, Tick, \textit{New Kernel Phase}, \textit{New Agent Phase},
Exchange, Tock, Decommission.

Technically, whether agent entry occurs simultaneously with agent exit or not
does not matter from a simulation-mechanics point of view, because the two
phases have a direct ordering. It will, however, from the point of view of
module development. It is simpler cognitively to think of an agent entering the
simulation and acting in that time step, rather than entering a simulation at a
given time and taking its first action in the subsequent time step.

In the spirit of Law's definition of a fixed-increment time advance mechanism,
there is a final important invariant: \textit{there is no guaranteed agent
  ordering of within-phase execution}. This invariant allows for:

\begin{itemize}
\item a more cognitively simple process
\item paralellization
\end{itemize}

Any future addition of phases in the timestep execution stack neccessarily
guarantee the three invariants described above.
