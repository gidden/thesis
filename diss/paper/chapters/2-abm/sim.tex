
\section{Simulation Principles}\label{abm:sim}

Cyclus seeks to dynamically simulate the flow of resources and deployment of
facilities in the nuclear fuel cycle.

The majority of tools developed to date use a system dynamics approach to
modeling the fuel cycle.

System dynamics is good for applications that naturally fit the 'stocks and
flows' approach.

Simulation and simulation analysis has a rich history and large body of work
that can be leveraged to tackle this problem.

Simulation methodologies can largely be broken down into two groups: discrete
time and discrete event simulation. 

Cyclus has taken a refined discrete time approach.

Cyclus defines an arbitrary time step, requiring consistency among prototype
configurations. Historically, this time step has been one month. Importantly, it
is integral, and unchanging within a single simulation.

A time step is comprised of two types of phases: those in which agents observe
the simulation environment and update their state, and those in which the kernel
queries agents and executes kernel functionality.

The order of phases is:

\begin{itemize}
  \item building phase (kernel)
  \item tick phase (agent)
  \item resource exchange (kernel)
  \item tock phase (agent)
  \item decommissioning phase (kernel)
\end{itemize}

Facilities in Cyclus are treated individually. A single facility can have a
supply of resources and a demand for resources. 

Individual resource transfers are treated discretely by Cyclus. When a
facility's demand for a resource is matches with another facility's supply of a
resource by the Cyclus kernel, a transfer is initiated. Each transfer is treated
as discrete, individual trade.
