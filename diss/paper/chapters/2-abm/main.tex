\chapter{Simulation and Agent Based Modeling in Cyclus}\label{ch:abm}

Developing a simulator of any complex system is an involved process requiring a
solid methodological base. A reasonable approach is to define precisely the
simulation framework, including a definition of how time moves forward and what
events can occur. This chapter lays the foundation for the simulation of nuclear
fuel cycles in Cyclus. \secref{abm:sim} begins the discussion by broadly
describing methodological principles on which Cyclus has been developed. The use
of agent-based modeling techniques is presented in \secref{abm:abm}, and an
agent deployment methodology with a proof-of-principle benchmark is
presented. Finally, \secref{abm:dre} treats the most complex simulation
interaction in Cyclus, Dynamic Resource Exchange (DRE), describing its
methodology, discussing its implementation, and presenting a set
proof-of-principle results.


\section{Simulation Principles}\label{abm:sim}

As its core functionality, Cyclus seeks to dynamically model the flow of
resources and deployment of facilities in the Nuclear Fuel Cycle (NFC). As such,
Cyclus is a \textit{simulator} which models the NFC as a \textit{system}.

By Law's definition \cite{Law:1999:SMA:554952}, Cyclus is, broadly, a dynamic,
discrete-event simulation that uses a fixed-increment time advance mechanism. In
general, fixed-increment time advance scenarios assume a time step ($\Delta t$),
and assume that all events that would happen during a time occur simultaneously
at the end of the time step. This situation can be thought of as an event-based
time advance mechanism, i.e., one that steps from event to event, that executes
all events simultaneously that were supposed to have occurred in the time step.

Again using Law's definition, Cyclus models a collection of \textit{entities}
which either trade resources, manage other entities, or perform both
actions. The most basic \textit{entity} in a Cyclus simulation is a
Facility. Facilities can be used to model processes with arbitrary levels of
physical fidelity, and can interact with the simulator and other entities with
arbitrary levels of behavioral fidelity. As such, Cyclus can also be described
as an \textit{agent-based model} (ABM). Accordingly, the \textit{entities} in a
given simulation can be interchangeably referred to as \textit{agents}.

\subsection{Events}

Two key types of events occur in every Cyclus simulation:

\begin{itemize}
\item agent entry into and exit from the simulation
\item the exchange of resources between agents
\end{itemize}

Agent entry and exit events are scheduled by another managing agent, or
scheduled as an initial condition to the simulation. Upon entry or exit,
newly-built or about-to-be-decommissioned agents are informed as are their
managing agent. The Cyclus simulation kernel treats each agent individually,
rather than grouping agents by an attribute and treating like-facilities in an
aggregate manner.

While the determination of supply and demand is complex and described further in
\S \ref{abm:dre}, the execution of resource exchange is rather straightforward
and a primary event in a Cyclus simulation. When a agent's demand for a resource
is matched with another matches's supply of a resource by the Cyclus kernel, a
transfer is initiated. Each transfer is treated as discrete, individual trade
between two agents.

Simulation entities can have arbitrarily complex state which is dependent on the
results of resource exchange and the present status of agents in the simulation
at a given time step. Accordingly, methods that allow entities to update state
and schedule new entity entry and exit must occur in response to these events.

\subsection{Timesteps}

Because there is a key event that defines agent interaction in a given time
step, it is necessary to involve all agents in that interaction. Accordingly it
is necessary that there be an ordering between these two key types of events,
deviating slightly from Law's description of fixed-increment time
advance. Specifically, the following invariant is preserved: \textit{any agent
  that exists in a given time step should be included in the resource exchange,
  or, equivalently, experience the entire time step execution stack}.

This leads to the following ordering, or \textit{phases}, of time step
execution:

\begin{itemize}
\item agents enter simulation (Building Phase)
\item agents respond to current simulation state (Tick Phase)
\item resource exchange execution (Exchange Phase)
\item agents respond to current simulation state (Tock Phase)
\item agents leave simulation (Decommissioning Phase)
\end{itemize}

The Building, Exchange, and Decommissioning phases each include critical,
core-based events, and will be called \textit{Kernel} phases. The Tick and Tock
phases do not include core-based events, and instead let agents react to
previous core-based events and inspect core simulation state. Furthermore, they
are periods in which agents can update their own state and are accordingly
considered \textit{Agent} phases. In general, Agent phases \textit{must} bracket
\textit{critical} Kernel phases, of which only the Exchange Phase exists for
now. If another critical core phase is added in the future it must provide a
similar invariant, i.e., that it is bracketed by Agent phases. For example, if a
new phase is added before Exchange, then the time execution stack would follow
as: Building, Tick, \textit{New Kernel Phase}, \textit{New Agent Phase},
Exchange, Tock, Decommission.

Technically, whether agent entry occurs simultaneously with agent exit or not
does not matter from a simulation-mechanics point of view, because the two
phases have a direct ordering. It will, however, from the point of view of
module development. It is simpler cognitively to think of an agent entering the
simulation and acting in that time step, rather than entering a simulation at a
given time and taking its first action in the subsequent time step.

In the spirit of Law's definition of a fixed-increment time advance mechanism,
there is a final important invariant: \textit{there is no guaranteed agent
  ordering of within-phase execution}. This invariant allows for:

\begin{itemize}
\item a more cognitively simple process
\item paralellization
\end{itemize}

Any future addition of phases in the timestep execution stack neccessarily
guarantee the three invariants described above.



\section{Agents and Agent Based Modeling in Cyclus}\label{abm:abm}

Deciding how a simulation is structured from an interactions standpoint is a
delicate balance of known necessity and perceived future needs. There are basic
decisions to make, such as modeling material transfer as discrete or
continuous. Discrete transfers more closely match reality and may provide
insights in that regard, however they require more of their modeling apparatus
due to messaging needs and other structures. More complex decisions include how
one wants to determine connections between facilities, and whether such
connections are assigned statically and incorporated into the simulation
architecture or determined dynamically. Guerin's comment
in \S\ref{sec:litrev-benchmarks} stems from this ``freedom''. These
simulation-engine decisions comprise the art-related portion of fuel cycle
simulation, but developers have a goal of making these decisions in as informed
a way as possible using domain-level knowledge with respect to our known and
perceived requirements. In general, this work tries to minimize the sheer number
of choices we make in this regard, instead relying on well known and well
documented practices of computer scientists and systems engineers.

The \Cyclus team is also interested in fostering a user and developer ecosystem.
Accordingly, such concerns also drive the simulation architecture design. The
agent-based nature of \Cyclus provides an opportunity to reduce barriers to
entry into the ecosystem. Given a few basic tenets of agent interaction, other
developers should be able to create a new agent to ``plug in'' to the
simulation. Intuitively, a minimal set of behaviors must be defined to
sufficiently inform the simulation infrastructure to run the simulation. This
freedom allows the simulation program introduce agents at run time, effectively
separating the simulation engine's functionality from the agents in the
simulation.

Such a framework provides many benefits. First, there is a clear separation of
concerns. The \Cyclus core is concerned with modeling system dynamics whereas
individual agents are concerned with domain-specific issues. Accordingly,
developers can focus their attention appropriately, focusing either on the core
code base or on agent development. Separating agent development from core
development also allows \Cyclus to remain a viable open-source candidate to
model nuclear fuel cycle dynamics. Because domain-level information is
incorporated into agent libraries which are dynamically loaded at runtime, a
closed-source developer can focus their efforts entirely on developing agent
libraries. Furthermore, developers could participate both privately and
publicly, e.g., adding general capability to the \Cyclus core that is needed for
some functionality without specifying the internals. Such a community paradigm
is shown in Figure \ref{fig:community}.

\begin{figure}[htbp!]
  \begin{center}
    \includegraphics[height=8cm]{./figs/community.png}
  \end{center}
  \caption{The Cyclus Participation Paradigm} 
  \label{fig:community}
\end{figure}

In order to develop and maintain the core code separate from the agent modules,
well-defined interactions must be provided between agents and the \Cyclus
core. The remaining part of the section provides a proposal for such
interactions that allow for a variety facility deployment and supply-demand
matching algorithms to be employed. A description of the market resolution
interface is provided, and basic agent simulation interaction, such as entering
and leaving the simulation is also described.

Moving from a modeling paradigm that does not differentiate between individual
facilities to one that does requires a nuanced approach to determine facility
behavior. In the absence of supply constraints, aggregated individual facility
behavior and fleet-based models are equivalent. However, any system in which
recycling exists will, by definition, have some supply constraints. 

Furthermore, if resource supply and demand depends on more than the quantity of
a resource, moving global management logic becomes difficult. As the complexity
of a quality metric increases, an aggregate approach becomes less desireable as
it loses such detail through aggregation.

In the extreme in which a high level of detail is required in the notion of
resource quality, e.g. tracking an arbitrary number of isotopes, adopting
techniques that allow decision-making based on that level of detail is
desireable. Modeling the nuclear fuel cycle represents such a level of
detail. For example, even in the case of a once-through fuel cycle, many
reactors of the same type (e.g., PWRs), may require different resource qualities
(i.e., Uranium enrichment).

\subsection{Agent Taxonomy}

The Cyclus kernel implements a basic \code{Agent} class that provides the
minimal interface for agents to be built within a simualtion. Furthermore, the
\code{Trader} interface provides a communication layer required for agents to be
included in the exchange of resources. \code{Facility} agents in Cyclus
implement both interfaces, while \code{Institution} and \code{Region} agents
implement only the \code{Agent} interface.

\subsubsection{Facilities}

Facilities in \Cyclus are abstracted to either consumers or suppliers of
commodities, and some may be both. Supplier agents are provided agency by being
able to communicate to the market-resolution mechanism a variety of production
capacity constraints in second phase of the information gathering
methodology. Consumer agents are provided agency by being able to assign
preferences among possible suppliers based on the supplier's quality of
product. Because this agency is encapsulated for each agent, it is possible to
define strategies that can be attached or detached to the agents at
run-time. Such strategies are an example of the Strategy design pattern
\cite{vlissides_design_1995}.

\subsubsection{Institutions}

Institutions in \Cyclus manage a set of facilities. Facility management is
nominally split into two main categories: the commissioning and decommissioning
of facilities and supply-demand association. The goal of including a notion of
institutions is to allow an increased level of detail when investigating
regional-specific scenarios. For example, a consumer facility may prefer to be
supplied by a supplier facility in its instution rather than one associated with
a different institution. Furthermore, there are international governmental
organizations, such as the IAEA, that have proposed managing large fuel cycle
facilities that service many countries in a given global region. A fuel bank is
an example of such a facility. Accordingly, institutions in \Cyclus are able to
augment the preferences of supplier-consumer pairs that have been established in
order to simulate a mutual preference to trade material within an
institution. Of course, situations arise in real life where an institution has
the capability to service its own facilities, but choose to use an outside
provider because of either cost or time constraints. Such a situation is allowed
in this framework as well. It is not clear how such a relationship should be
instantiated and to what degree institutions should be allowed to affect their
managed facilities' preferences. This issue lies squarely in the realm of
simulation design decisions, part of the \textit{art} of
simulation. Accordingly, through the course of research, the possible design
space will be analyzed in order to determine best practices for this type of
design.

\subsubsection{Regions}

Regions in \Cyclus provide the forcing function for simulations by requiring
that certain parameters be met, e.g., power capacity, fuel cycle service
capacity, etc. For example, in the case of nuclear power capacity, a region
knows that it needs additional reactors to be built, but leaves the building of
those reactors to the institutions that operate in the region. The current
GrowthRegion model in the \Cycamore \cite{cycamore2013} code base takes the cost
and nameplate capacity for each facility in a set of acceptable facilities and a
demand gap, and solves a minimum-cost capacity production problem to determine
the number and type of facilities to request. It is important to note here that
this abstraction allows for different deployment algorithms to be tested and
exchanged in the \Cyclus framework without necessitating changes to the
simulation engine, as is the case with other simulators described
in \S\ref{sec:simulators} and is consistent with the types of simulation design
decisions described in
\S\ref{sec:simulators-overview}. 

Regions are further provided agency by their ability to affect preferences
between supplier-consumer facility pairs in the third phase of the market
information gathering algorithm. The ability to perturb arc preferences between
a given supplier and a given consumer allows fuel cycle simulation developers to
model relatively complex interactions at a regional level, such as tariffs and
sanctions. Constraints to cross-border trading can also be applied to the
formulation described in \S\ref{sec:gfctp}. For example, a region could place
constraints on the total amount of a given commodity type that is able to flow
into it or out of it into a different region. Such constraints could applied not
only to bulk quantities of a commodity, but also to the quality of each
commodity. Such a mechanism could be used to model interdiction of
highly-enriched uranium transport, for example.

\subsection{Methods of Agency}

Agency is provided in two primary modes: determining facility deployment and
informing resource exchange mechanisms. 

Facility deployment has, to date, involved some combination of an
\code{Institution} agent, a \code{Facility} agent, and a \code{Region}
agent. \code{Institution} agents represent a simulation entity abstraction that
can deploy \code{Facility} agents. \code{Region} agents represent a simulation
entity abstraction that have a demand for certain commodities that
\code{Facility} agents provide, for example, reactor-like \code{Facility} agents
provide electrical power.

\code{Facility} agents are further provided agency by informing market
mechanisms of thesupply and demand of resource quantity and quality. Cyclus
initially used a very crude interface and algorithm for determining resource
transactions. Individual markets were defined, and initially designed to be
agents much like \code{Facility}, \code{Institution}, and \code{Region}
agents. Many limitations were identified at the time, however, and this approach
was eventually abandoned. An enumeration of the observed limitations is
described further in \S \ref{abm:abm:limits}.

The Dynamic Resource Exchange (DRE), described in detail in \S \ref{abm:dre}, is
the mechanism eventually developed to drive supply-demand transactions. The
primary source of agency is provided to \code{Facility} agents in order to
negotiate the quanity and quality of potential resource
transactions. \code{Region}, \code{Instituion}, and \code{Facility} agents are
then provided agency in the negotiation of preferences of potential
transactions, where preference is a proxy for price.

\subsection{Proof of Principle}\label{abm:abm:proof}

Agents were developed to show an initial proof of principle that fuel cycle
simulation can be implemented using an agent-based modeling methodology.

Facility deployment decision making is the fundamental building block of any
dynamic simulator. By definition, dynamic simulators model the deployment of
facilities and measure the flow of resources in the system over time. 

Furthermore, in the extreme case of unconstrained supply and no competition for
resources, resource exchange decisions can be made arbitrarily. Therefore, only
facility deployment agency is required.

\subsubsection{Benchmark Case}

An initial benchmark test was performed to confirm expected deployment behavior
and basic resource routing. The INPRO Business As Usual (BAU) benchmark for the
once-through fuel cycle was chosen for three reasons. First, it was the simplest
case that demonstrated deployment behavior. Second, no supply or demand
constraints were present, so a basic supply-demand framework would
suffice. Finally, results from two other fuel cycle simulation codes,
specificaly VISION \cite{}, were available for comparison. The INPRO BAU
benchmark identified two cases, high electricity demand and moderate electricity
demand, as shown in Fig. \ref{fig:inpro-demand}. Both cases require that demand
met by a composition of 94\% Light Water Reactors and 6\% Heavy Water Reactors.

\begin{figure}
  \begin{center}
    \includegraphics[height=8cm]{./figs/inpro-demand.pdf}
    \caption{The energy demand specification for the INRPO BAU scenarios.}
    \label{fig:inpro-demand}
  \end{center}  
\end{figure}

The goal of this proof-of-principle benchmark was to showcase the capability for
a developer to generate the required Facility, Institution, and Region
archetypes, and that such archetypes could be deployed in the Cyclus simulation
framework and generate satisfactory results. Metrics including deployment
patterns, natural uranium consumed, and used fuel produced by all reactors were
used to compare results between VISION, GENIUS, and Cyclus.

\subsubsection{Agent Archetypes Developed}

\paragraph{GrowthRegion}

The \code{GrowthRegion} was developed to assist in facility deployment
logic. The \code{GrowthRegion} takes as input a listing of commodities for which
it has a demand. For example, the The \code{GrowthRegion} agents in this
benchmark demand electrical power. The demand for commodities is defined by
symbolic functions. Currently, linear functions, exponential functions, and
piecewise combinations of both are supported. 

At any time step in which there exists a demand gap, i.e., there exists more
demand than supply, a build decision is made. This decision is modeled as the
following integer program:
%%%
\begin{subequations} \label{eqs:growth}
\begin{equation} \label{eq:optBuildObj}
\begin{aligned}
& \min_{n}
& & \sum_{i \in I}n_i*c_i
\end{aligned}
\end{equation}
\begin{equation} \label{eq:optBuildConst}
\begin{aligned}
& \text{s.t.}
& & \sum_{i \in I}n_i*\phi_i \ge \Phi
\end{aligned}
\end{equation}
\begin{equation} \label{eq:optBuildBounds}
\begin{aligned}
& n_i \in [0,\infty) & \forall \:\: i \in I &
\end{aligned}
\end{equation}
\begin{equation} \label{eq:optBuildInt}
\begin{aligned}
& n_i \:\:\: integer & \forall \:\: i \in I &
\end{aligned}
\end{equation}
where $\Phi$ is the unmet demand, $I$ is the set of facilities capable of 
meeting the demand, and, for each facility in $I$, $c_i$ is the cost of building, 
and $\phi_i$ is the nameplate capacity.  Finally, $n_i$ is the optimized number of
facilities to build of type $i$.
\end{subequations}
%%%

\paragraph{ManagerInst}

The \code{ManagerInst} was developed to assist in facility deployment as
well. While the \code{GrowthRegion} determines which facilities \textit{should}
be built, the \code{ManagerInst} determines the set of all possible facilities
that \textit{can} be built. In other word, the \code{ManagerInst} determines the
set of facilities, $I$, shown in in Eqn. \ref{eqs:growth}. Note that the set $I$
can change over time. Once a deployment decision is made, the
\code{GrowthRegion} makes a facility deployment request of the
\code{ManagerInst} which then deploys the chosen facility.

\paragraph{BatchReactor}

While a reactor model existed prior to this work, it did not provide the
functionality to interchange \textit{batches} of fuel. A batch of fuel is a
fraction of a full reactor core that is extracted and then replaced when a
reactor is refuled. While in the extreme, a full core can be replaced, in
pratice a batch size is generally a set fraction of the full core, especially
for LWRs.

The \code{BatchReactor} used in this work had configurable properties as
displayed in Table \ref{tbl:batchrxtr}. The values used based on the defined
INPRO BAU benchmark are described in Table \ref{tbl:inprorxtr}.

\begin{table}[h]
\centering
\begin{tabular}{cc}
Parameter      & Description                     \\ \hline
Process Time   & Active fuel time in the reactor                        \\
Refuel Time    & Time to refuel the reactor                              \\
N Batches      & Number of batches in the reactor                         \\
Batch Size     & Quantity of a batch                                 \\
Power Capacity & Nameplace Capacity for Power                          \\
Power Cost     & Cost to build a new reactor      \\ \hline
\end{tabular}
\caption{Configurable input for the \code{BatchReactor} archetype.}
\label{tbl:batchrxtr}
\end{table}

\begin{table}[h]
\centering
\begin{threeparttable}
\begin{tabular}{ccc}
Parameter      & LWR Value & HWR Value               \\ \hline
Process Time   & 10         & 10                       \\
Refuel Time    & 2          & 2                        \\
N Batches      & 4          & 4                        \\
Batch Size     & 7.87E4     & 1.39e5                   \\
Power Capacity & 1000       & 600                      \\
Power Cost     & 1000*      & 600* \\ \hline
\end{tabular}
\begin{tablenotes}
  \small
  \item (*) Note that the Cost used is arbitrary and set equal to the
  capacity so that a minimum capacity is built per Eqn. \ref{eqs:growth}.
\end{tablenotes}
\caption{Configurable input values for reactors used in the INPRO BAU once-through benchmark.}
\label{tbl:inprorxtr}
\end{threeparttable}
\end{table}

\paragraph{EnrichmentFacility}

The \code{EnrichmentFacility} archetype was developed to provide
enrichment-related output for the simulation, namely the amount of separative
work units (SWU) and natural uranium used during a given time step. For the
INPRO BAU cases, it can be defined quite simply using the values shown in Table
\ref{tbl:inproenr}. The feed assay and product assay, both required for
determinging output metrics, are defined by the isotopic compositions of
resource input and resource output. Resource output is determined by the
requesting reactors.

\begin{table}[h]
\centering
\begin{tabular}{ccc}
Parameter    & Description                      & Values          \\ \hline
Input Recipe & A description of input isotopics & Natural Uranium \\
Tails Assay  & The U-235 assay of tails.        & 0.003          
\end{tabular}
\caption{Configurable input values for the \code{EnrichmentFacility} used in the
  INPRO BAU once-through benchmark.}
\label{tbl:inproenr}
\end{table}

\subsubsection{Results}

% c/Cref is cleveref

In aggregate, Cyclus performed well relative to the other benchmark
codes. Fig. \ref{fig:rxtrs_low} shows the reactor deployment curves for each
simulator for the moderate growth scenario while Fig. \ref{fig:rxtrs_high} shows
reactor deployment for the high scenario. The slight differences are attributed
to VISION's look-ahead functionality which builds the required facilities one
time step after they are needed, whereas Cyclus builds facilities on the
timestep in which they are needed. Cumulative natural uranium utilization curves
for the moderate and high cases are shown in \Cref{fig:nat_u_low,fig:nat_u_high},
and cumulative used fuel inventory curves are shown in
\Cref{fig:used_fuel_low,fig:used_fuel_high}. Slight discrepencies are noted
between Cyclus and VISION. These discrepencies are attributed to the
implementation of core batch recycling in each of the respective codes. Because
the curves show cumulative metrics, the discrepencies grow over time, as
expected.

\begin{figure}
  \begin{center}
    \includegraphics[height=8cm]{./figs/rxtrs_low.pdf}
    \caption{The reactor deployment schedule by reactor type for the moderate demand scenario.}
    \label{fig:rxtrs_low}
  \end{center}  
\end{figure}

\begin{figure}
  \begin{center}
    \includegraphics[height=8cm]{./figs/rxtrs_high.pdf}
    \caption{The reactor deployment schedule by reactor type for the high demand scenario.}
    \label{fig:rxtrs_high}
  \end{center}  
\end{figure}

\begin{figure}
\begin{center}
  \includegraphics[height=8cm]{./figs/nat_u_low.pdf}
  \caption{The total natural uranium used for the moderate demand scenario.}
  \label{fig:nat_u_low}
\end{center}  
\end{figure}

\begin{figure}
\begin{center}
  \includegraphics[height=8cm]{./figs/nat_u_high.pdf}
  \caption{The total natural uranium used for the high demand scenario.}
  \label{fig:nat_u_high}
\end{center}  
\end{figure}

\begin{figure}
  \begin{center}
    \includegraphics[height=8cm]{./figs/used_fuel_low.pdf}
    \caption{The amount of used fuel produced for the moderate demand scenario.}
    \label{fig:used_fuel_low}
  \end{center}  
\end{figure}

\begin{figure}
  \begin{center}
    \includegraphics[height=8cm]{./figs/used_fuel_high.pdf}
    \caption{The amount of used fuel produced for the high demand scenario.}
    \label{fig:used_fuel_high}
  \end{center}  
\end{figure}

\subsection{Multiple Market Limitations}\label{abm:abm:limits}

Cyclus was originally designed to use an addition agent type, the \code{Market}
agent. The \code{Market} agents were envisioned to represent markets for
specific commodities. For example, the simulation described in \S
\ref{abm:abm:proof} used three commodity markets: natural uranium, enriched
uranium fuel, and used fuel. This approach is valid in the absence of supply or
demand constraints, competition, and fungibility. However, the inclusion of any
one of these features requires a much more involved process.

If supply or demand constraints are to be modeled, each
associated \code{Market} agent must have both a corresponding communication
interface and an implementation that accounts for such constraints. While
quantity constraints are not unreasonable to implement and support, quality
constraints are much more difficult. Furthermore, communicating such constraints
is difficult. Whereas the \code{Market} agent can implement a solver algorithm,
constraints are more naturally defined by the trader interacting with the
\code{Market} agent. For example, consider the enriched uranium market used in
\S \ref{abm:abm:proof}. While the simulation used an agent abstraction for an
enrichment facility and fuel fabrication plant, another simulation may wish to
model facilities that downblend HEU, rather than enrich LEU. Such a process will
have different constraints. Importantly, those constraints are a function of the
\code{Facility} agent, not of a \code{Market} agent.

Assuming that supply or demand is constrained either quantitatively or
qualitatively, competition for the resource in question can arise. In fact,
resources constraints are only interesting in the presence of competition. When
competition for resources exist, there must be some mechanism that determines
which transactions are to be executed, i.e., which agents should trade which
resources. Determining supply and demand under competition is a well studied
problem with many possible formulations and solution frameworks.

Fungibility is the property of a good or commodity to be \textit{capable of
  being substituted in place of one another} (CITE
http://www.merriam-webster.com/thesaurus/fungible). For example, a light water
reactor generates power by fissioning nuclei in the thermal energy
spectrum. Whether those nuclei are \nucl{239}{Pu}, \nucl{235}{U}, or
\nucl{233}{U} makes little difference from a power generation perspective. In
other words, those nuclei are \textit{fungible} for light water reactors, given
some safety and cycle length considerations. A similar issue arises from a
suppliers perspective. Consider a MOX fuel supplier and two requesters: a fast
reactor and a thermal reactor. Given the isotopic makeup of Plutonium in the MOX
fuel, such could be potentially be used in either reactor type. Again, Plutonium
in this example is a \textit{fungible} resource. Accordingly, a facility may
demand multiple fungible commodities, which must be accounted for by a given
market clearing mechanism.

The one-market-per-commodity approach does not treat competition, constrained
supply and demand, and fungibility particularly well. Constraints are handled
poorly because constraints are determined by the supplying and demanding
agents. Because the markets are separated, they must either be solved
sequentially or in parallel. If solved sequentially, competition and fungibility
are treated poorly, because information involving multiple commodities is not
taken into account during the solution of a single market. Accordingly, a
solution framework and methodology that incorporates agent querying of supply,
demand, and constraints and resolves markets in parallel is required.



\section{Dynamic Resource Exchange}\label{abm:dre}

\subsection{Problem Statement}

As a next-generation nuclear fuel cycle simulation framework, Cyclus maintains a
primary goal of modeling flexibility. As facility, institutional, and regional
models are proposed, they should be relatively easily implemented and utilized
in the Cyclus simulation framework. Furthermore, the level of modeling
abstraction for different facilities in a fuel cycle will be different based on
the needs of archetype developer. Any supply-demand resolution framework,
therefore, must be able to support arbitrary facilities. One way to approach
such a problem is to treat facilities as black boxes, clearly defining a
supply-demand communication framework.

As stated previously in \S \ref{abm:abm}, a number of considerations must be
taken into account in such a framework. Supply and demand must be able to be
solved globally at any given time step. Resources must be able to be treated in
a fungible manner. The framework must be able to incorporate arbitrary,
agent-defined constraints. 

In order to address each of these concerns, the concept of a Dynamic Resource
Exchange (DRE) was developed and implemented. That process was motivated by the
following problem statement:

\begin{quote}
    If facilities are treated as individual black boxes and connections between
    facilities are determined dynamically, how does one match suppliers with
    demanders considering quantity and quality-based supply constraints,
    quantity and quality-based demand constraints, supply response to
    quality-based demands, and issues of fungibility?
\end{quote}

\subsection{Information Gathering}\label{abm:dre:info}

Supply-demand determination at any given time step begins with three
\textit{phases}, the terminology of which is influenced from previous supply
chain agent-based modeling work \cite{julka_agent-based_2002}. Importantly, this
information-gathering step is agnostic as to the actual matching algorithm used,
it is concerned only with querying the current status of supply and demand in
the simulation.

The first phase allows consumers of commodities to denote both the quantity of a
commodity they need to consume as well as the target isotopics, or quality, by
\textit{posting} their demand to the market exchange. This posting informs
producers of commodities what is needed by consumers, and is termed the
\textit{Request for Bids} (RFB) phase. Consumers are allowed to over-post, i.e.,
request more quantity than they can actually consume, as long as a corresponding
capacity constraint accompanies this posting. Further, consumers are allowed to
post demand for multiple commodities that may serve to meet the same combine
capacity. For example, consider an LWR that can be filled with MOX or UOX. It
can post a demand for both, but must define a preference over the set of
possible commodities it can consume. Another example is that of an advanced fuel
fabrication facility, i.e., one that fabricates fuel partially from separated
material that has already passed through a reactor. Such a facility can choose
to fill the remaining space in a certain assembly with various types of fertile
material, including depleted uranium from enrichment or reprocessed uranium from
separations. Accordingly, it could demand both commodities as long as it
provides a corresponding constraint with respect to total consumption. At the
completion of the RFB phase, the market exchange will have a set of consumption
portfolios, $P$, where each portfolio consists of a set requests, $R$, a
cardinal preference over the requests, $\alpha_R$, and possibly a set of
constraints over the requests, $C_R$. Each request consists of a quantity,
$q_r$, and a target isotopic vector, $I_r$.

The second phase allows suppliers to \textit{respond} to the set of consumption
portfolios, and is termed the \textit{Response to Request for Bids} (RRFB) phase
(analogous to Julka's Reply to Request for Quote phase
\cite{julka_agent-based_2002}). Each consumption portfolio is comprised of
requests for some set of commodities. Accordingly, for each request, suppliers
of that commodity denote production capacities and an isotopic profile of the
commodity they can provide. Suppliers are allowed to offer the null set of
isotopics as their profile, effectively providing no information. A supplier may
have its production constrained by more than one parameter. For example, a
processing facility may have both a throughput constraint (i.e., it can only
process material at a certain rate) and an inventory constraint (i.e., it can
only hold some total material). Further, the facility could have a constraint on
the quality of material to be processed, e.g., it may be able to handle a
maximum radiotoxicity for any given time step which is a function of both the
quantity of material in processes and the isotopic content of that material. The
formulation provided in \S\ref{sec:gfctp} allows for multiple of such
constraints as long as they are linear functions of the demanded commodity
quantity. At the completion of the RRFB phase the possible connections between
supplier and producer facilities, i.e., the arcs in the graph of the
transportation problem, have been established with specific capacity constraints
defined both by the quantity and quality of commodities that will traverse the
arcs.

The final phase of the information gathering procedure allows consumer
facilities to adjust their set of preferences and for managers of consumer
facilities to affect the consumer's set of preferences, as described in the
remaining sections. Accordingly, the last phase is termed the \textit{Preference
 Adjustment} (PA) phase. Preference adjustments can occur in response to the
set of responses provided by producer facilities. Consider the example of a
reactor facility that requests two fuel types, MOX and UOX. It may get two
responses to its request for MOX, each with different isotopic profiles of the
MOX that can be provided. It can then assign preference values over this set of
potential MOX providers. Another prime example is in the case of repositories. A
repository may have a defined preference of material to accept based upon its
heat load or radiotoxicity, both of which are functions of the quality, or
isotopics, of a material. In certain simulators, limits on fuel entering a
repository are imposed based upon the amount of time that has elapsed since the
fuel has exited a reactor, which can be assessed during this phase. The time
constraint is, in actuality, a constraint on heat load or radiotoxicity (one
must let enough of the fission products decay). A repository could analyze
possible input fuel isotopics and set the arc preference of any that violate a
given rule to 0, effectively eliminating that arc.

\subsection{Exchange Structure}

Upon completion of the information gathering phase, a \textit{bipartite} network
is formed. The network consists of receiving (request) nodes, $U$, and sending
(bid) nodes, $V$. For each request node, $u$, there may be many bid nodes;
however, there is a one-to-one mapping between bid nodes and request nodes. In
other words, a given bid node, $v$, is a unique response to a request node, $v$.

\TODO{Include figure}

There is not a direct analog between the \textit{portfolios} described in \S
\ref{abm:dre:info} and the bipartite graph structure. The notion of
\textit{portfolios} does map onto the formulation, however, and will be
described further in \S \ref{abm:dre:form}.

To ensure a feasible solution, additional nodes are added to both $U$
and $V$. A single bid node is added to $V$ and $N_{p}$ nodes are added to $U$,
where $N_{p}$ is the number of request portfolios. Thus, the total size of node
sets and the total number of arcs in the system are

\begin{equation}
  \left|{U}\right| = \left|{U}\right| + N_{p}
\end{equation}

\begin{equation}
  \left|{V}\right| = \left|{V}\right| + 1
\end{equation}

\begin{equation}
  \left|{A}\right| = \left|{A}\right| + N_{p}
\end{equation}

Each arc in the additional set, $A'$, is unconstrained (any amount of resource
flow is allowed) and is given a preference strictly less than the lowest
preference in the system, i.e.,

\begin{equation}
  p' < \min P.
\end{equation}

Accordingly, any additional arc will only be engaged if no other into a request
node, $u$, have available capacity and if the associated request portfolio is
not satisfied. Arc in $A'$ are referred to as \textit{false arcs}, because they
represent flows that do not exist in final solutions.

\TODO{Include figure}

\subsubsection{Commodities}

During the information gathering step in \S \ref{abm:dre:info}, consumers and
suppliers are queried based on \textit{commodities}. A consumer is allowed to
request multiple commodities, and a supplier is allowed to supply multiple
commodities; however, each possible resource transfer is based on a single
commodity. Accordingly, it is possible to color each arc, given a
commodity-to-color mapping.

For example, consider an exchange with three fuel commodities ($A$, $B$, $C$),
two requesters ($R_1$, $R_2$), and two suppliers ($S_1$, $S_2$) in the following
configuration:

\TODO{Include table}

Given the color map $A$: red, $B$: green, $C$: blue, the resulting exchange
graph can be colored:

\TODO{Include figure}

Importantly, the notion of commodities is critical during the information
gathering step and can be used to group arcs in an exchange graph. It also is
compelling when generating the formulation shown in \S
\ref{abm:dre:form}. However, given the constructed network graph, constraint
structure, and preference structure, the notion of commodities is not necessary
for the exchange graph to be solved.

\subsection{Mathematical Programming Formulation}\label{abm:dre:form}

\subsubsection{Overview}

An instance of supply and demand can be solved in a variety of ways. To solve
the system optimally, however, a formal investigation and solution structure is
needed. This section describes the construction of such a formulation.

The formulation is informed by the supply-demand parameters gathered by the
methodology described in the previous section. The basis for the formulation is
the Multicommodity Transportation Problem described in \S\ref{intro:mtp} with
some departures described in detail below. Two separate formulations are
provided. The first is a strictly linear program (LP) while the second is a
mixed-integer linear program (MILP).

The LP formulation can be solved quickly, but allows split orders, i.e., orders
that are not fully filled. The nuclear fuel cycle deals with bundled orders,
such as nuclear fuel assemblies, thus this modeling paradigm is only an
approximation. The MILP provides a more realistic exchange, but can take much
longer to solver. The following sections will describe both formulations,
starting with the LP formulation and then introducing adaptations required to
formulate the MILP version. Finally, a heuristic solution is described and
initial results are described using a proof-of-principle.

\subsubsection{The Nuclear Fuel Cycle Transportation Problem}

Supply and demand in a nuclear fuel cycle context is inherently a multicommodity
problem. A light water reactor can be fueled by both UOX and MOX fuel, for
instance. How it is fueled is a result both of fuel availability and associated
preferences. Allowing for complex physical and chemical constraints on both
processes and inventories, as well as including economics-based approaches for
determining exchange preferences is a complicated affair. Determining the
optimum solution to such a system is even more complicated. Accordingly,
sophisticated tools in both the operations research and agent based modeling
realms have been leveraged to accomplish the task.

\paragraph{Terminology}

%%% 
\begin{table} [h!]
\centering
\begin{tabularx}{\columnwidth-10pt}{|c|X|} % line wraps second column if too long
\hline
Set         & Description \\
\hline
$H$         & all commodities  \\
$I$         & all producers  \\
$J$         & all consumers  \\
$X$         & the feasible set of flows between producers and consumers  \\
$K_{i}^{h}$  & the set of constraining capacities for 
            producer $i$ of commodity $h$  \\
$H_{j}$     & the set of satisfying commodities for consumer $j$  \\
\hline
\end{tabularx}
\caption{Sets Appearing in the GFCTP-LP Formulation}
\label{tbl:GFCTP-LP-sets}
\end{table}
%%% 

%%% 
\begin{table} [h!]
\centering
\begin{tabularx}{\columnwidth-10pt}{|c|X|} % line wraps second column if too long
\hline
Variable    & Description \\
\hline
$c_{i,j}^{h}$             & the unit cost of commodity $h$ 
                          for producer $i$ and consumer $j$  \\
$x_{i,j}^{h}$             & a decision variable, the flow of commodity $h$ 
                          for producer $i$ and consumer $j$  \\
$q_{j}^{h}$               & the requested quality of commodity $h$ 
                          by consumer $j$  \\
$\beta_{i,k}(q_{j}^{h})$  & a capacity translation function for capacity 
                          constraint $k$ of producer $i$ given $q_{j}^{h}$ \\
$s_{i,k}^{h}$             & a supply capacity of producer $i$ corresponding to 
                          capacity constraint $k$ of commodity $h$ \\
$d_{j}(H_{j})$            & the total demand of consumer $j$ over the set of 
                          satisfying commodities $H_{j}$ \\
\hline
\end{tabularx}
\caption{Variables Appearing in the GFCTP-LP Formulation}
\label{tbl:GFCTP-LP-vars}
\end{table}
%%%

\paragraph{Objective Function}

In any network flow problem, of which transportation problems are a subset, the
cost of transporting commodities is what drives the solution. Accordingly, an
accurate cost function is necessary to determine an accurate solution. Because
the \Cyclus environment is still a nascent simulation platform, accurate pricing
metrics, and what such metrics even are in terms of a centuries-long fuel cycle
simulation, are generally difficult to ascertain, with the current standard source
being the Advanced Fuel Cycle Cost Basis
report \cite{shropshire_advanced_2009}. Accordingly, the cost function is
currently a measure of simulation entity preference, rather than a concrete
representation of cost.

The notion of preference extends the work of Oliver's affinity metric
\cite{oliver_geniusv2:_2009}. The preference metric is generally consumer
centric, i.e., consumers have a preference over the possible commodities that
could meet their demand. For example, a reactor may be able to use UOX or MOX
fuel, but may prefer to use MOX fuel. Such a preference differential allows the
projection of real-world cost into the simulation. Additionally, the managers of
a given facility, which in the \Cyclus simulation environment include its
Institution and Region, also exert an influence over its preference. An obvious
example is the concept of affinities given in \cite{oliver_geniusv2:_2009}. In
Oliver's work, an affinity or preference existed between facilities in
``similar'' institutions in order to drive the trading between institutions as a
simple model of international relations. This idea is expanded upon to cover a
facility's other managers and the commodities themselves. Additionally, a
preference can be delineated between the proposed qualities of the same
commodity from different vendors, e.g. if two vendors of MOX fuel
exist. Finally, the notion of a preference is a positive one, and we require a
notion of cost to solve the minimum-cost formulation of the multicommodity
transportation problem with side constraints. Therefore one must utilize a
translation function.

Formally, we define a preference function, $\alpha_{i,j}(h)$, which is a
cardinal preference ordering over a consumer's satisfying commodity set.

\begin{equation}
\alpha_{i,j}(h) \: \forall i \in I \: \forall h \in H_{j} 
\end{equation}

This ordering is a function both of the consumer, $j$, and producer, $i$. The
dependence on producer encapsulates the relationship effects due to managerial
preferences. We then define a cost translation function, $f$, that operates on
the commodity preference function to produce an appropriate cost.

\begin{equation}
f : \alpha_{i,j}(h) \to c_{i,j}^{h}
\end{equation}

A naive implementation, and perhaps all that is necessary for a
proof-of-principle, is to define f as an inversion operator.

\begin{equation}
f(x) = \frac{1}{x}
\end{equation}

\paragraph{Constraints}
This formulation deviates from the normal MCTP formulation via the expansion of
capacity constraints (Equation \ref{eqs:GFCTP-LP_sup}) and the inclusion of a
constraint allowing multiple commodities that are able to meet the demand of a
producer (Equation \ref{eqs:GFCTP-LP_dem}). The former constraint maintains the
multi-commodity nature of the formulation. 

Under certain conditions, the GFCTP-LP will result in a simpler problem. The
first possible condition is that each consumer could have its demand met by only
one commodity, i.e.,

\begin{equation}\label{eqs:1demand}
  \left|{H_{j}}\right| = 1 \: \forall \: j \in J.
\end{equation}

In such a situation, the GFCTP-LP can be transformed into an analog of the
separable transportation problem as shown in \cite{bertsekas_network_1998}. Such
a condition will effectively allow one to solve $N$ different instances of a
single-commodity problem, where $N$ is the cardinality of $H$. 

The second simplifying condition is if the constraining capacity set has a
cardinality of unity, i.e., 

\begin{equation}\label{eqs:1constraint}
  \left|{K_{i}^{h}}\right| = 1 \: \forall \: i \in I, \: \forall \: h \in H.
\end{equation}

If both Equation \ref{eqs:1constraint} and \ref{eqs:1demand} hold, then the
GFCTP-LP is in fact the a normal Transportation Problem, because the quality
translation function ($\beta_{i,k}(q_{j}^{h})$) translates to a constant at
solution time. 

These simplifications are important to the computation time required to solve
the resulting problem instance. The general solution technique for LPs is the
Simplex Method, as previously described. Klee and Minty show that in the worst
case, the Simplex Method will execute in exponential time \cite{klee_good_1970},
but in practice it is generally considered very computationally efficient. If
the problem can be simplified to a TP, then the Transportation Simplex Method
can be used \cite{ahuja_network_1993}.

\paragraph{Capacity Translation Function and Constraints Example}

The notion of a capacity translation function is something that has been
introduced out of necessity due to the complexity of the GFCTP. Accordingly, an
example will help clarify its purpose. This time can also be used to provide an
example of a producer with multiple capacity constraints for a given commodity.

Take, for example, an enrichment facility. Such a facility produces the
commodity enriched uranium (EU). This facility has two constraints on its
operation for any given time period: the amount of Separative Work Units (SWU)
that it can process, $s_{enr,SWU}$ and the total natural uranium (NU) feed it
has on hand, $s_{enr,NU}$. Note that neither of these capacities are measure
directly in the units of the commodity it produces, i.e., kilograms of enriched
uranium (EU). The set of values for $K_{i}^{h}$ for this facility are:

\begin{equation}\label{eqs:enr-constr-commods}
  K_{enr}^{EU} = \{ \mbox{SWU}, \mbox{NU} \}
\end{equation}

Consider a set of requests for enriched uranium that this facility can possibly
meet. Such requests have, in general, two parameters: $P_{j}$, the total product
quantity (in kilograms), and $\varepsilon_{j}$, the product enrichment (in w/o
\nucl{235}{U}).\footnote{The notation for enrichment, $\varepsilon_{j}$, is chosen over its
normal form, $x_p$, to limit confusion with the LP notation of material flow,
$x^h_{i,j}$.}  For the purposes of this constraint set, the quality of material
in question is its enrichment, i.e.,

\begin{equation}\label{eqs:enr-q-swu}
  q_{j}^{EU} \equiv \varepsilon_{j}.
\end{equation}

These values are set during a prior phase of the overall matching algorithm, and
can therefore be considered constant. Further, let us note that, in general, an
enrichment facility's operation, or rather its capacity, is governed by two
parameters: $\varepsilon_{f,enr}$, the fraction of \nucl{235}{U} in its feed material, and
$\varepsilon_{t,enr}$, the fraction of \nucl{235}{U} in its tails material. These parameters
determine the amount of SWU required to produce some amount of enriched uranium:

\begin{align}
\begin{split}
\label{eqs:swu}
SWU = & \:\: P ( V(\varepsilon_{j}) 
      + \frac{\varepsilon_{j} - \varepsilon_{f,enr}}
               {\varepsilon_{f,enr} - \varepsilon_{t,enr}} V(\varepsilon_{t,enr}) \\
      & - \frac{\varepsilon_{j} - \varepsilon_{t,enr}}
               {\varepsilon_{f,enr} - \varepsilon_{t,enr}} V(\varepsilon_{f,enr}) )
\end{split}
\end{align}

$P$ in Equation \ref{eqs:swu} is the amount of produced enriched uranium, and
$V(x)$ is the value function,

\begin{equation}\label{eqs:value}
  V(x) = (1-2x) \ln \left(\frac{1-x}{x}\right)
\end{equation}

Utilizing the above equations, one can denote the functional forms of the
arguments of this facility's two capacity constraints.

\begin{align}
\label{eqs:enr-prod-beta}
\beta_{enr,NU}(\varepsilon_{j}) = & \:\: \frac{\varepsilon_{j} - \varepsilon_{t,enr}}
                                      {\varepsilon_{f,enr} - \varepsilon_{t,enr}} \\
\begin{split}
\label{eqs:enr-swu-beta}
\beta_{enr,SWU}(\varepsilon_{j}) = & \:\: V(\varepsilon_{j}) \\
                         & + \frac{\varepsilon_{j} - \varepsilon_{f,enr}}
                                  {\varepsilon_{f,enr} - \varepsilon_{t,enr}} V(\varepsilon_{t,enr}) \\
                         & - \frac{\varepsilon_{j} - \varepsilon_{t,enr}}
                                  {\varepsilon_{f,enr} - \varepsilon_{t,enr}} V(\varepsilon_{f,enr})
\end{split}
\end{align}

These constraints correspond to the per-unit requirements for enriched uranium
of natural uranium feed and SWU. Finally, we can form the set of constraint
equations for the enrichment facility by combining
Equations \ref{eqs:GFCTP-LP_sup}, \ref{eqs:enr-q-swu},
\ref{eqs:enr-prod-beta}, and \ref{eqs:enr-swu-beta}.

\begin{align}
\label{eqs:enr-prod-constr}
\sum_{j \in J}\beta_{enr,NU}(\varepsilon_{j}) \: x_{enr,j}^{EU}  & \leq s_{enr,NU} \\
\label{eqs:enr-swu-constr}
\sum_{j \in J}\beta_{enr,SWU}(\varepsilon_{j}) \: x_{enr,j}^{EU} & \leq s_{enr,SWU}
\end{align}

\paragraph{Satisfying Commodity Set Example}

The other departure the GFCTP-LP takes from the normal MCTP formulation is the
location of its multicommodity dependence. As presented above, the
MCTP formulation includes a multicommodity arc capacity constraint, Equation
\ref{eqs:GFCTP-LP_dem}. This constraint models a
situation in which different commodities can satisfy a consumer's demand. There
is no direct analog in the GFCTP, i.e., transportation arcs are assumed separate
for separate commodities.

Take the enrichment facility example, expanding on the previous discussion. Note
that an enrichment facility takes feed uranium and then enriches its \nucl{235}{U}
content. This feed uranium can come from different sources which have different
feed enrichments. In practice, the most likely sources of feed uranium are
natural uranium (NU) or recycled uranium (RU), a product of reprocessing light
water reactor fuel. Recycled uranium may be advantageous to use if it has a
higher weight percent of \nucl{235}{U} than does natural uranium. We can now state the
set the values for $H_{j}$ for this facility:

\begin{equation}\label{eqs:enr-dem-commods}
  H_{enr} = \{ \mbox{NU}, \mbox{RU} \}
\end{equation}

\paragraph{LP Formulation}

Combining the previous discussions, the LP Formulation of the NFCTP can be constructed.

%%% 
\begin{subequations}\label{eqs:GFCTP-LP}
  \begin{align}
    %%
    \min_{z} \:\: & 
    z = \sum_{h \in H}\sum_{i \in I}\sum_{j \in J}c_{i,j}^{h} x_{i,j}^{h} 
    & \label{eqs:GFCTP-LP_obj} \\
    %%
    \text{s.t.} \:\: &
    \sum_{j \in J}\beta_{i,k}(q_{j}^{h}) x_{i,j}^{h} \leq s_{i,k} 
    &
    \: \forall \: k \in K_{i}^{h},  
    \forall \: i \in I, \forall \: h \in H \label{eqs:GFCTP-LP_sup} \\
    %%
    &
    \sum_{i \in I}\sum_{h \in H_{j}} x_{i,j}^{h} \geq d_{j}(H_{j}) 
    & 
    \forall \: j \in J \label{eqs:GFCTP-LP_dem} \\
    %%
    &
    x^h_{i,j} \geq 0
    &
    \forall \: x \in X \label{eqs:GFCTP-LP_x}
    %%
  \end{align}
\end{subequations}
%%% 

\subsubsection{Adaptation for Fuel Assemblies}



\subsection{A Heuristic Solution}

With full simluation domain knowledge of supply and demand, including false
arcs, a feasible solution can be found. By definition a feasible solution is a
\textit{solution} to the possible flow of resources, but not necessarily an
\textit{optimal} solution. Many heuristics may be applied to bipartite graphs
with constrainted flows. A simple \textit{greedy} heuristic is presented here
and implemented. 

\begin{algorithm}[h!]
 \SetAlgoLined
 \KwData{A resource exchange graph.}
 \KwResult{A valid set of resource flows.}
 sort request partitions by average preference\;
 \ForAll{request partitions, $J_r$} {
   sort requests by average preference\;
   matched $\leftarrow$ 0\;        
   \While{matched $\leq q_{J_r}$ and $\exists$ a request} {
     get next request\;
     sort incoming arcs by preference\;
     \While{matched $\leq q_{J_r}$ and $\exists$ an arc} {
       get next arc\;
       remaining $\leftarrow q_{J_r}$ - matched\;
       to\_match $\leftarrow \min \lbrace$remaining, Capacity(arc)$\rbrace$\;
       AddMatch(arc, to\_match)\;
       matched $\leftarrow$ matched + to\_match\;
     }
   }
 }
 \caption{Greedy Exchange Hueristic}\label{alg::greedy}
\end{algorithm}

The capacity of each arc in question is the most constraining value $s$
associated with either node on the arc. For a node $i \in I_b$ and $j \in J_r$,
the constraining value is defined as

\begin{equation}
  s = \min 
        \lbrace 
        \min \lbrace \frac{s_{I_b, k}}{\beta_{i, j, k}} 
        \: \forall k \in K_{I_b} \rbrace, 
        \: \min \lbrace \frac{s_{J_r, k}}{\beta_{i, j, k}} 
        \: \forall k \in K_{J_r} \rbrace
        \rbrace.
\end{equation}

The constraining values of each arc are updated upon declaration of a match in
Algorithm \ref{alg::greedy}.

\subsection{Proof of Principle}

physor paper


\section{Summary}

This chapter introduced a novel way to employ agent-based modeling techniques in
the nuclear fuel cycle. \secref{abm:sim} first described how a simulation is
structured, focusing on where agent interactions occur in a given time step. A
discussion of how the notion agency is applied to fuel cycle entities and a
proof-of-principle simulation was shown in \secref{abm:abm}. Finally, a detailed
description of a novel supply-demand, agent-based framework, the DRE, was
presented in \secref{abm:dre}. The DRE is a critical advancement in the realm of
nuclear fuel cycle simulation, enabling arbitrary facility-based constraints,
competition for fungible resources, and the application of socio-economic models.
