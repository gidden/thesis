
\section{Agents and Agent Based Modeling}\label{abm:abm}

Moving from a modeling paradigm that does not differentiate between individual
facilities to one that does requires a nuanced approach to determine facility
behavior. In the absence of supply constraints, aggregated individual facility
behavior and fleet-based models are equivalent. However, any system in which
recycling exists will, by definition, have some supply constraints. 

Furthermore, if resource supply and demand depends on more than the quantity of
a resource, moving global management logic becomes difficult. As the complexity
of a quality metric increases, an aggregate approach becomes less desireable as
it loses such detail through aggregation.

In the extreme in which a high level of detail is required in the notion of
resource quality, e.g. tracking an arbitrary number of isotopes, adopting
techniques that allow decision-making based on that level of detail is
desireable. Modeling the nuclear fuel cycle represents such a level of
detail. For example, even in the case of a once-through fuel cycle, many
reactors of the same type (e.g., PWRs), may require different resource qualities
(i.e., Uranium enrichment).

\subsection{Agent Taxonomy}

(Insert section on RIF from prelim/papers here)

The Cyclus kernel implements a basic \code{Agent} class that provides the
minimal interface for agents to be built within a simualtion. Furthermore, the
\code{Trader} interface provides a communication layer required for agents to be
included in the exchange of resources. \code{Facility} agents in Cyclus
implement both interfaces, while \code{Institution} and \code{Region} agents
implement only the \code{Agent} interface.

\subsection{Methods of Agency}

Agency is provided in two primary modes: determining facility deployment and
informing resource exchange mechanisms. 

Facility deployment has, to date, involved some combination of an
\code{Institution} agent, a \code{Facility} agent, and a \code{Region}
agent. \code{Institution} agents represent a simulation entity abstraction that
can deploy \code{Facility} agents. \code{Region} agents represent a simulation
entity abstraction that have a demand for certain commodities that
\code{Facility} agents provide, for example, reactor-like \code{Facility} agents
provide electrical power.

\code{Facility} agents are further provided agency by informing market
mechanisms of thesupply and demand of resource quantity and quality. Cyclus
initially used a very crude interface and algorithm for determining resource
transactions. Individual markets were defined, and initially designed to be
agents much like \code{Facility}, \code{Institution}, and \code{Region}
agents. Many limitations were identified at the time, however, and this approach
was eventually abandoned. An enumeration of the observed limitations is
described further in \S \ref{abm:abm:limits}.

The Dynamic Resource Exchange (DRE), described in detail in \S \ref{abm:dre}, is
the mechanism eventually developed to drive supply-demand transactions. The
primary source of agency is provided to \code{Facility} agents in order to
negotiate the quanity and quality of potential resource
transactions. \code{Region}, \code{Instituion}, and \code{Facility} agents are
then provided agency in the negotiation of preferences of potential
transactions, where preference is a proxy for price.

\subsection{Proof of Principle}\label{abm:abm:proof}

Agents were developed to show an initial proof of principle that fuel cycle
simulation can be implemented using an agent-based modeling methodology.

Facility deployment decision making is the fundamental building block of any
dynamic simulator. By definition, dynamic simulators model the deployment of
facilities and measure the flow of resources in the system over time. 

Furthermore, in the extreme case of unconstrained supply and no competition for
resources, resource exchange decisions can be made arbitrarily. Therefore, only
facility deployment agency is required.

\subsubsection{Benchmark Case}

(Take from ANS Paper)

\subsubsection{Agents Developed}

(fill in from other descriptions)

\paragraph{GrowthRegion}

\paragraph{ManagerInst}

\paragraph{BatchReactor}

\paragraph{EnrichmentFacility}

\paragraph{Source}

\paragraph{Sink}

\subsubsection{Results}

As designed, facility deployment curves match the required benchmark specification. 

Further, material transaction quantities match VISION values.

Enrichment values match VISION for SWU, however, tails values did not match.

(add more discussion and result images from ANS talk)

\subsection{Multiple Market Limitations}\label{abm:abm:limits}

Cyclus was originally designed to use an addition agent type, the \code{Market}
agent. The \code{Market} agents were envisioned to represent markets for
specific commodities. For example, the simulation described in \S
\ref{abm:abm:proof} used three commodity markets: natural uranium, enriched
uranium fuel, and used fuel. This approach is valid in the absence of supply or
demand constraints, competition, and fungibility. However, the inclusion of any
one of these features requires a much more involved process.

If supply or demand constraints are to be modeled, each
associated \code{Market} agent must have both a corresponding communication
interface and an implementation that accounts for such constraints. While
quantity constraints are not unreasonable to implement and support, quality
constraints are much more difficult. Furthermore, communicating such constraints
is difficult. Whereas the \code{Market} agent can implement a solver algorithm,
constraints are more naturally defined by the trader interacting with the
\code{Market} agent. For example, consider the enriched uranium market used in
\S \ref{abm:abm:proof}. While the simulation used an agent abstraction for an
enrichment facility and fuel fabrication plant, another simulation may wish to
model facilities that downblend HEU, rather than enrich LEU. Such a process will
have different constraints. Importantly, those constraints are a function of the
\code{Facility} agent, not of a \code{Market} agent.

Assuming that supply or demand is constrained either quantitatively or
qualitatively, competition for the resource in question can arise. In fact,
resources constraints are only interesting in the presence of competition. When
competition for resources exist, there must be some mechanism that determines
which transactions are to be executed, i.e., which agents should trade which
resources. Determining supply and demand under competition is a well studied
problem with many possible formulations and solution frameworks.

Fungibility is the property of a good or commodity to be \textit{capable of
  being substituted in place of one another} (CITE
http://www.merriam-webster.com/thesaurus/fungible). For example, a light water
reactor generates power by fissioning nuclei in the thermal energy
spectrum. Whether those nuclei are \nucl{239}{Pu}, \nucl{235}{U}, or
\nucl{233}{U} makes little difference from a power generation perspective. In
other words, those nuclei are \textit{fungible} for light water reactors, given
some safety and cycle length considerations. A similar issue arises from a
suppliers perspective. Consider a MOX fuel supplier and two requesters: a fast
reactor and a thermal reactor. Given the isotopic makeup of Plutonium in the MOX
fuel, such could be potentially be used in either reactor type. Again, Plutonium
in this example is a \textit{fungible} resource. Accordingly, a facility may
demand multiple fungible commodities, which must be accounted for by a given
market clearing mechanism.

The one-market-per-commodity approach does not treat competition, constrained
supply and demand, and fungibility particularly well. Constraints are handled
poorly because constraints are determined by the supplying and demanding
agents. Because the markets are separated, they must either be solved
sequentially or in parallel. If solved sequentially, competition and fungibility
are treated poorly, because information involving multiple commodities is not
taken into account during the solution of a single market. Accordingly, a
solution framework and methodology that incorporates agent querying of supply,
demand, and constraints and resolves markets in parallel is required.
