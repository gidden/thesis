Key components of a novel methodology and implementation of an agent-based,
dynamic nuclear fuel cycle simulator, \Cyclus, are presented. The nuclear fuel
cycle is a complex, physics-dependent supply chain. To date, existing dynamic
simulators have not treated constrained fuel supply, time-dependent,
isotopic-quality based demand, or fuel fungibility particularly well. Utilizing
an agent-based methodology that incorporates sophisticated graph theory and
operations research techniques can overcome these deficiencies. This work
describes a simulation kernel and agents that interact with it, highlighting the
Dynamic Resource Exchange (DRE), the supply-demand framework at the heart of the
kernel. The key agent-DRE interaction mechanisms are described, which enable
complex entity interaction through the use of physics and socio-economic
models. The translation of an exchange instance to a variant of the
Multicommodity Transportation Problem, which can be solved feasibly or
optimally, follows. An extensive investigation of solution performance and
fidelity is then presented. Finally, recommendations for future users of \Cyclus
and the DRE are provided.
