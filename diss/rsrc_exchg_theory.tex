The exchange of resources in Cyclus is designed to create a generic resource
exchange graph that can be solved to determine the matched supply and demand of
resources amongst agents in a simulation.

In general, a resource exchange is made up of \textit{requests} and
\textit{bids}. A request defines a specific resource that is demanded by an
agent in the simulation. A bid defines a corresponding resource that is offered
to meet a given request.

Requests and bids can be mutually constraining, i.e., they can be grouped into
mutually-constraining sets. For example, a supplier agent may respond to many
requests but may only be able to meet a certain fraction of them; accordingly
the bids from this supplier agent can be grouped into a single constraining set.

A given request can have arbitrarily many demand constraints, and a given bid
grouping can have arbitrarily many supply constraints. Furthermore, each
bid-request pair has an associated unit coefficient related to each supply or
demand constraint. Finally, a given request can be \textit{exclusive} or a bid
grouping may be \textit{exclusive}. If a request is exclusive, then it can only
be met by a single bid. If a bid grouping is exclusive, then only one of bid in
its group may be selected. Both of these cases allows for the modeling of
quantized, undivisable trades. Such capability is required to model
naturally-integral objects such as fuel assemblies.
